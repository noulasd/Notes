\documentclass[oneside,a4paper]{article}

%%%%%%%%%%%%%%%%%%%%%%%%%%%%
\usepackage{amsthm}
\usepackage{amsmath}
\usepackage{amssymb}
%%%%%%%%%%%%%%%%%%%%%%%%%%%%%
\usepackage[greek]{babel}
\usepackage[utf8]{inputenc}
\usepackage{mathtools}
\usepackage{blindtext}
\usepackage[T1]{fontenc}
\usepackage{titlesec}
\usepackage{sectsty}
\usepackage{verbatim}
\usepackage{multirow}
\chapternumberfont{\tiny} 
\chaptertitlefont{\Huge}
%ελληνικοι χαρακτηρες σε μαθ pdf utf-8
%%%%%%%%%%%%%%%%%%%%%%%%%%%%%%%%%
\usepackage{tikz-cd}

\usepackage{xcolor}
\usepackage{framed}%frames

\usepackage{array}
\usepackage{pbox}

%%%%%%%%%%%%%%%%%%%%%%%%
\usepackage{tikz}
%%%%%%%%%%%%%%%%%%%%%%%%%%

%%%%%%%%περιθώρια%%%%%%%%%%%%
\usepackage[a4paper,margin=3.5cm]{geometry}


\newtheorem{theorem}{Θεώρημα}
\newtheorem{lemma}{Λήμμα}
\newtheorem{example}{Παράδειγμα}
\newtheorem*{defn}{Ορισμός}
\newtheorem{prop}{Πρόταση}
\newtheorem{cor}{Πόρισμα}

\newcommand {\tl}{\textlatin}
%%%%%%%%%αριθμηση%%%%%%%%%%%%%%
\renewcommand{\theenumi}{\arabic{enumi}}
\renewcommand{\labelenumi}{{\rm(\theenumi)}}
\renewcommand{\labelenumii}{\roman{enumii}) }
%%%%%%%%%%%% New theorems %%%%%%%%%%%%%%%%%%%%%%%%

%%%%%%%%%%%%%%%%%%%%%%%%%%%%%%%%%%%%%%%%%%%%%%%%%%%
\newcommand{\Z}{\mathbb{Z}}
\newcommand{\Q}{\mathbb{Q}}
\newcommand{\Co}{\mathbb{C}}
%%%%%%%%%%%%%%%%%%%%% Document starts %%%%%%%%%%%%
\begin{document}
	
	%%%%%%%%%%%%%%%%%%%%%%%%%%%%%%%%%%%%%%%%%%%%%%%%%%
	\selectlanguage{greek}
	%%%%%%%%%%%%%%%%%%%%%%% Start Roman numbering %%%% vbbnn
	%\pagenumbering{roman}
	%%%%%%%%%%%%%%%%%%%%%%%%%%%%%%%%%%%%%%%%%%%%%%%%%%
	
	\begin{framed}	
		%\vspace{0.3truecm}
		\begin{center}
			\huge Θεωρία Ομάδων 2
		\end{center}
		%\vspace{0.3truecm}
		\begin{center}
			 Λύσεις ασκήσεων από τα φυλλάδια 2 και 3
		\end{center}
		\vspace{0.3truecm}
		\begin{center}
			Ονομ/νο: Νούλας Δημήτριος\\
			ΑΜ: 1112201800377 (προπτυχιακό) \\
			\tl{email}: \tl{dimitriosnoulas@gmail.com}
		\end{center}
		\vspace{0.3truecm}
	\end{framed}
	\vspace*{\fill}
	\begin{center}
	\includegraphics[width=0.5\textwidth]{C:/Users/dimit/Desktop/TeX/uoa_logo}
	%\includegraphics[width=0.5\textwidth]{/mnt/c/Users/dimit/Desktop/TeX/uoa_logo}
	\end{center}
\vspace{1cm}
\pagebreak

\noindent 1) Έστω $X$ ένας $\delta$-υπερβολικός χώρος (με την έννοια του \tl{Rips}), $x,y,x_0 \in X$ και $\gamma = [x,y]$ γεωδαισιακή από το $x$ στο $y$. Τότε $(x,y)_{x_0} \leq d(x_0,\gamma) \leq (x,y)_{x_0} + 2\delta$.

\begin{proof} $ $

	$ $\newline
	Όπως έχει γίνει στην τάξη, έστω $w \in [x,y]: \quad d(x_0,w) = d(x_0, [x,y])$. Μπορούμε να επιλέξουμε τέτοιο $w$ αφού το σύνολο $[x,y]$ είναι συμπαγές. Έχουμε από τριγωνική ανισότητα:

	$$d(x,x_0) \leq d(x,w) + d(x_0,w)$$
	$$d(y,x_0) \leq d(y,w) + d(x_0,w)$$

	$ $\newline
	Και αφού $X$ γεωδαισιακός παίρνουμε ότι $d(x,y) = d(x,w) + d(w,y)$. Άρα:

	$$d(x,x_0) + d(y,x_0) \leq \underset{=d(x,y)}{d(x,w) +  d(y,w)} + 2 d(w,x_0)$$

	$$(x,y)_{x_0} \leq d(w,x_0) = d(x_0,[x,y])$$

	$ $\newline
	Για την άλλη ανισότητα, θέτουμε $$A= \{z \in \gamma: \quad (x,x_0)_z \leq \delta \}$$ και $$B = \{z \in \gamma: \quad (y,x_0)_z \leq \delta\}$$ Ισχυριζόμαστε ότι $A\cup B = \gamma = [x,y]$. Η σχέση $\subseteq$ είναι προφανής από τον ορισμό των $A,B$. 
	
	
	$ $\newline
	Έστω $z \in \gamma$ με $(x,x_0)_z > \delta$. Θα δείξουμε ότι $(y,x_0)_z \leq \delta$. Από την $\delta$-υπερβολικότητα, θα υπάρχει $w \in [x,x_0]\cup[y,x_0]$ με $d(z,w)\leq \delta$. Αν $w\in [x,x_0]$ και εφαρμόσουμε δύο φορές τριγωνική ανισότητα, τότε:
	$$d(z,w)\geq d(z,x_0) - d(w,x_0)$$
	$$d(z,w) \geq d(z,x) - d(w,x)$$ και προσθέτοντας:

	$$2d(z,w) \geq d(z,x_0) + d(z,x) - \left( d(w,x_0) + d(w,x)\right) = d(z,x_0) + d(z,x) - d(x,x_0)$$ αφού ο χώρος είναι γεωδαισιακός. Άρα $d(z,w) > \delta$ το οποίο είναι άτοπο. Συνεπώς $w \in [y,x_0]$. Ωστόσο, τώρα που ισχύει το $w \in [y,x_0]$ η σχέση $(y,x_0)_z > \delta$ θα μας έδινε άτοπο με το ίδιο επιχείρημα όπως παραπάνω. Άρα $(y,x_0)_z \leq \delta$.

	$ $\newline
	Δείξαμε ότι $A\cup B = \gamma$ το οποίο είναι συνεκτικό και τα $A,B$ μη κενά και κλειστά. Άρα αναγκαστικά τα $A,B$ τέμνονται, δηλαδή υπάρχει $z \in \gamma$ με 
	$$(x,x_0)_z \leq \delta$$
	$$(y,x_0)_z \leq \delta$$


	\noindent Επιπλέον, αν κάνουμε τις απλοποιήσεις έχουμε την σχέση $$d(x_0,z) = (x,y)_{x_0} + (x,x_0)_z + (y,x_0)_z$$ Άρα $$d(x_0,\gamma) \leq d(x_0,z) = (x,y)_{x_0} + (x,x_0)_z + (y,x_0)_z \leq (x,y)_{x_0} + 2\delta$$
\end{proof}


\pagebreak


\noindent 2) Αποδείξτε ότι η $H_1(\delta)$-υπερβολικότητα ενός χώρου $X$ είναι ισοδύναμη με την ακόλουθη συνθήκη των $4$-σημείων: $d(x,y) + d(z,x_0) \leq \max \{d(x,z) + d(y,x_0), d(x,x_0) + d(y,z)\} + 2\delta$ για κάθε $x,y,z,x_0 \in X$.

\begin{proof} $ $

    $ $\newline
    Έστω ότι $(y,z)_x \geq \min \{ (x_0,y)_x, (z,x_0)_x\} - \delta$. Υποθέτουμε ότι το ελάχιστο των δύο το συναντάμε στο $(x_0,y)_x$ και ισοδύναμα έχουμε υποθέσει:

    $$d(y,x) + d(z,x_0) \leq d(z,x) + d(x_0,y)$$

    $ $\newline
    Ανοίγουμε την πρώτη σχέση με τα γινόμενα του \tl{Gromov} και με βάση την υπόθεσή μας έχουμε ότι:

    $$ d(y,x) + d(z,x) - d(y,z) + 2\delta \geq d(y,x) + d(x_0,x) - d(x_0,y)$$
    $$\iff d(z,x) + d(x_0,y) + 2\delta \geq d(x_0,x) + d(y,z)$$ και το πρώτο μέλος ισούται με το $\max + 2\delta$, δηλαδή:

    $$\max \{d(z,x) + d(x_0,y), d(y,x) + d(z,x_0)\} + 2\delta \geq d(x_0,x) + d(y,z)$$

    $ $\newline
    Αντίστροφα, δίχως βλάβη γενικότητας υποθέτουμε ότι $d(x,z) + d(y,x_0) \leq d(x,x_0) + d(y,z)$ (το οποίο είναι ισοδύναμο με $(y,x_0)_x \geq (y,z)_x$). Άρα από συνθήκη 4-σημείων έχουμε ότι:
    $$d(x,x_0) + d(y,z) + 2\delta \geq d(x,y) + d(z,x_0)$$ και αλλάζουμε μέλος αυτά που περιέχουν το $z$ καθώς και προσθέτουμε το $d(x,z)$ στα 2 μέλη. Έτσι:

    $$d(x,x_0) + d(x,z) - d(z,x_0) + 2\delta \geq d(x,y) + d(x,z) - d(y,z)$$
    $$\iff (x_0,z)_x + \delta \geq (y,z)_x = \min \{(y,z)_x, (y,x_0)x \}$$
\end{proof}




\pagebreak

\noindent 3) Έστω $X$ ενας $\delta$-υπερβολικός χώρος (με την έννοια του \tl{Rips}). Αποδείξτε (χρησιμοποιώντας ένα επειχείρημα συνεκτικότητας) ότι για κάθε γεωδαισιακό τρίγωνο $T$ με πλευρές $\gamma_1, \gamma_2, \gamma_3$ υπάρχει $x\in X$ έτσι ώστε $d(x,\gamma_i) \leq \delta$ για κάθε $i$.


\begin{proof} $ $

    $ $\newline
    Θέτουμε 
    $$A = \{ w \in \gamma_1: \quad d(w,\gamma_2) \leq \delta\}$$
    $$B = \{w \in \gamma_1: \quad d(w,\gamma_3) \leq \delta\}$$

    $ $\newline
    Από την $\delta$-υπερβολικότητα έχουμε ότι $A\cup B = \gamma_1$, τα οποία $A, B$ είναι κλειστά (αφού έχουμε $\leq$ και η συνάρτηση απόστασης είναι συνεχής). Το $\gamma_1$ είναι συνεκτικό και γράφεται ως ένωση δύο κλειστών, συνεπώς αυτά τα δύο κλειστά τέμνονται, δηλαδή υπάρχει $w \in A\cap B \subseteq \gamma_1$ με:
    $$d(w,\gamma_1) = 0 \leq \delta$$
    $$d(w,\gamma_2) \leq \delta$$
    $$d(w,\gamma_3) \leq \delta$$


\end{proof}


\pagebreak

\noindent 4) Έστω $F(S)$ η ελεύθερη ομάδα με βάση το $S$ και $\Gamma = \Gamma(F(S),S)$ το αντίστοιχο γράφημα \tl{Cayley}. Αποδείξτε ότι για κάθε $1 \neq g \in F(S)$ υπάρχει γεωδαισιακή γραμμή $\gamma : (-\infty, \infty) \rightarrow \Gamma$ (ονομάζεται άξονας του $g$ και συμβολίζεται με $A_g$) επί της οποίας το $g$ δρα με μεταφορές μήκους $\tau_g >0$, δηλαδή για κάθε $t \in \mathbb R$ ισχύει ότι $g \cdot \gamma(t) = \gamma(t+\tau_g)$.

\begin{proof} $ $

	$ $\newline
	Έστω $g \in F(S)$ με $1\neq g$. Τότε το $g$ θα έχει κάποιο μήκος $||g||_S = ||s^{\varepsilon_1}_1 \cdots s^{\varepsilon_n}_n||_S = n$ αν δούμε την παράσταση αυτή του $g$ σαν ανηγμένη λέξη στους γεννήτορες. Θεωρούμε τις γεωδαισιακές $\gamma_{k}:[0,n] \rightarrow \Gamma$ που συνδέουν τα $g^k$ και $g^{k+1}$ μέσω της διαδρομής $w = s^{\varepsilon_1} \cdots s^{\varepsilon_n}_n$, για κάθε $k \in \Z$.


	$ $\newline
	Ορίζουμε μια γεωδαισιακή που ενώνει όλες τις δυνάμεις $g^k$ διασχίζοντας κάθε φορά την ετικέτα $w$ ως εξής:
	$$\gamma : (-\infty, \infty) \longrightarrow \Gamma $$
	$$   [kn, (k+1)n]\ni \quad  t \longmapsto \gamma_k (t - kn)$$ είναι πράγματι γεωδαισιακή, εφόσον οι αποστάσεις μεταξύ τυχαίων δυνάμεων $g^{k},g^{\ell}$ είναι ελάχιστες, αφού η ετικέτα $w$ είναι η συντομότερη διαδρομή για να πολλαπλασιάσουμε με $g$. Αν τώρα θεωρήσουμε τυχαία $\gamma(t_1),\gamma(t_2)$ και υποθέσουμε ότι έχουν ελάχιστη απόσταση με ετικέτα $v$ έξω από την $\gamma$, δηλαδή $d(\gamma(t_1),\gamma(t_2))$ να είναι μικρότερο του μήκους $[\gamma(t_1),\gamma(t_2)]$ πάνω στην $\gamma$, τότε για $t \leq t_1$ θα συναντήσουμε μια κορυφή $g^{k_1}$ και για $t\geq t_2$ θα συναντήσουμε μια κορυφή $g^{k_2}$. Έτσι θα έχουμε (σε μήκη):

	$$[g^{k_1},\gamma(t_1)] \cdot v \cdot [\gamma(t_2), g^{k_2}] < [g^{k_1},\gamma(t_1)] \cdot [\gamma(t_1),\gamma(t_2)] \cdot [\gamma(t_2), g^{k_2}] = [g^{k_1},g^{k_2}] \subseteq \gamma$$ δηλαδή πετύχαμε μικρότερη απόσταση των $g^{k_1},g^{k_2}$ έξω από την γεωδαισιακή τους, το οποίο είναι άτοπο.														
	
	
	$ $\newline
	Άρα θα έχουμε για κάθε $ t \in (-\infty,\infty) \implies t \in [kn,(k+1)]$ για κάποιο $k \in \mathbb{Z}$ (δεν έχουμε πρόβλημα στα άκρα) και άρα:

	$$ g \cdot \gamma(t) = g \cdot \gamma_k (t - kn) = \gamma_{k+1} ( t - (k+1)n) = \gamma(t+n)$$
\end{proof}



\pagebreak

\noindent 5) Έστω $X$ ένας $\delta$-υπερβολικός χώρος (με την έννοια του \tl{Rips}), $x \in X$ και $\gamma: (-\infty, \infty) \rightarrow X$ μια γεωδαισιακή γραμμή. Αν $y$ και $y^{\prime}$ είναι σημεία της γεωδαισιακής $\gamma$ τα οποία επιτυγχάνουν την ελάχιστη απόσταση του $x$ από την $\gamma$ (γιατί υπάρχουν τέτοια σημεία?), δηλαδή $d(x,y) = d(x,y^{\prime}) = d(x,\gamma)$, τότε $d(y,y^{\prime})\leq 4\delta$.

\begin{proof} $ $

	$ $\newline
	Τέτοια σημεία $y,y^{\prime}$ υπάρχουν γιατί το $Im\gamma$ είναι κλειστό αφού η $\gamma$ είναι ισομετρία ως γεωδαισιακή. Αν το $y=y^{\prime}$ δεν έχουμε κάτι να δείξουμε. Αν $y\neq y^{\prime}$ και $d(y,y^{\prime}) > 4\delta$ τότε μπορούμε να υποθέσουμε ότι υπάρχει σημείο $w \in [y,y^{\prime}] \subseteq Im\gamma$ με $d(w,y),d(w,y^{\prime}) > 2\delta$.

	$ $\newline
	Από την υπερβολικότητα του χώρου, στο γεωδαισιακό τρίγωνο που ορίζουν οι $x,y,y^{\prime}$ υπάρχει σημείο $z \in [x,y]\cup[x,y^{\prime}]$ με $d(z,w) \leq \delta$. Υποθέτουμε ότι βρίσκεται πάνω στην $[x,y]$.

	$ $\newline
	Από την τριγωνική $d(w,y) \leq d(w,z) + d(y,z)$ έχουμε $$d(y,z) \geq d(w,y) - d(w,z) \geq d(w,y) - \delta > 2\delta - \delta = \delta$$ δηλαδή
	$$ \delta  < d(y,z) $$
	
	\noindent Έχουμε ότι 
	$$d(x,w) \leq d(z,x) + d(z,w) \leq d(z,x) + \delta < d(z,x) + d(y,z) = d(x,y)$$ Αφού το $z$ ανήκει στην γεωδαισιακή $[x,y]$. Άρα το $w$ που είναι πάνω στην γεωδαισιακή $\gamma$ πετυχαίνει απόσταση μικρότερη από την ελάχιστη, το οποίο είναι άτοπο.


\end{proof}








\pagebreak

\noindent 6) Έστω $X$ ένας $\delta$-υπερβολικός χώρος και $g$ μια ισομετρία του $X$. Αν $\gamma_1,\gamma_2: \mathbb R\rightarrow X$ είναι γεωδαισιακές γραμμές οι οποίες διατηρούνται από την δράση της $g$ και επιπλέον η $g$ δρα σε κάθε μια από αυτές ως μεταφορά θετικού μήκους $\tau_1$ και $\tau_2$ αντίστοιχα, τότε υπάρχει $\varepsilon = \varepsilon(\delta) >0$ έτσι ώστε $\gamma_1 \subseteq N_{\varepsilon}(\gamma_2)$ και $\gamma_2 \subseteq N_{\varepsilon} (\gamma_1)$.

\begin{proof} $ $

	$ $\newline
	Αν θεωρήσουμε δύο κορυφές πάνω σε κάθε γεωδαισιακή και πάρουμε το γεωδαισιακό τετράπλευρο που ορίζουν, τότε από την $\delta$-υπερβολικότητα του χώρου θα υπάρχουν σημεία πάνω στις πλευρές των γεωδαισιακών $\gamma_1,\gamma_2$ που έχουν απόσταση μικρότερη ίση του $2\delta$. Άρα με κατάλληλη μεταφορά των $\gamma_1,\gamma_2$ μπορούμε να υποθέσουμε ότι $\gamma_1(0),\gamma_2(0) \in G$ και $$k:= d(\gamma_1(0),\gamma_2(0)) \leq 2\delta$$


	
	\noindent Για κάθε $m \in \Z$ τώρα έχουμε ότι:
	$$d(\gamma_1(m\tau_1),\gamma_2(m\tau_2)) = k$$

	\noindent Εφόσον: $$k = d(g^m \gamma_1(0),g^m \gamma_2(0)) = d(\gamma_1(m\tau_1),\gamma_2 (m\tau_2))$$

	\noindent Έστω ένα $t \in \mathbb{R}$. Το $t$ θα βρίσκεται σε μοναδικό διάστημα της μορφής $[(m-1)\tau_1, m\tau_1]$ για κάποιο $m \in \Z$.

	$ $\newline
	\noindent Άρα για το τυχόν σημείο $\gamma_1(t)$ έχουμε $d(\gamma_1(t),\gamma(m\tau_1)) \leq \tau_1$. Συνεπώς:

	$$d(\gamma_1(t), Im\gamma_2)\leq d(\gamma_1(t),\gamma_2(m\tau_2)) \leq$$
	$$\leq d(\gamma_1(t), \gamma_1(m\tau_1)) + d(\gamma_1(m\tau_1), \gamma_2(m\tau_2)) \leq \tau_1 + k \leq \tau_1 + 2\delta$$

	\noindent Όμοια δείχνουμε ότι ένα τυχαίο $\gamma_2(t)$ ανήκει στην $2\delta + \tau_2$ περιοχή της $\gamma_1$. Άρα η μία γεωδαισιακή περιέχεται στην $\max \{\tau_1,\tau_2\} + 2\delta$ περιοχή της άλλης.

\end{proof}

\pagebreak

\noindent 7) Έστω $X$ ένας $\delta$-υπερβολικός χώρος και $\gamma_1,\gamma_2 : [0,+\infty) \rightarrow X$ δύο γεωδαισιακές ακτίνες με κοινή αρχή $x_0$. Αν $\lim_{n\rightarrow \infty} (\gamma_1(n),\gamma_2(n))_{x_0} = +\infty$, τότε υπάρχει $M=M(\delta)$ έτσι ώστε $d(\gamma_1(t),\gamma_2(t))\leq M$, για κάθε $t\in[0,\infty)$.


\begin{proof} $ $


	$ $\newline
	Έστω $t \in [0,\infty)$. Εφόσον το γινόμενο \tl{Gromov} απειρίζεται υπάρχει $n \in \mathbb{N}$ τέτοιο ώστε $$t \leq (\gamma_1(n),\gamma_2(n))_{x_0}$$ Θεωρούμε το γεωδαισιακό τρίγωνο που ορίζουν τα σημεία $x_0,\gamma_1(n),\gamma_2(n)$. Αν το απεικονίσουμε μέσω της $\phi$ σε τρίποδο θα έχουμε $$\phi(\gamma_1(t)) = \phi(\gamma_2(t)) = t$$ αφού $d(x_0,\gamma_1(t)) = d(x_0,\gamma_2(t)) = t$ και τα $\gamma_1(t),\gamma_2(t)$ απεικονίζονται στο ίδιο σημείο στο τρίποδο αφού $t \leq (\gamma_1(n),\gamma_2(n))_{x_0}$, καθώς και το γινόμενο \tl{Gromov} είναι το μήκος στο οποίο γίνεται η διακλάδωση στο τρίποδο. Συνεπώς από τον δεύτερο ορισμό υπερβολικότητας έχουμε $$d(\gamma_1(t),\gamma_2(t)) \leq \delta$$ Ουσιαστικά εδώ το γινόμενο \tl{Gromov} μας λέει για πόσο χρόνο μένουν κοντά οι γεωδαισιακές (με κοινή αρχή).
\end{proof}


\pagebreak

\noindent 1) Έστω $\phi : \mathbb{N} \rightarrow \mathbb{N}$ μια απεικόνιση με την ιδιότητα $\phi(m+n) \leq \phi(m) + \phi(n)$ για κάθε ζεύγος $m,n \in \mathbb{N}$. Να δειχθεί ότι υπάρχει στο $\mathbb{R}$ το όριο $\lim_{n\rightarrow \infty} \frac{\phi(n)}{n}$.

\begin{proof} $ $


	$ $\newline
	Θέτουμε $A = \lim \inf_n \frac{\phi(n)}{n}$. Έστω $\varepsilon >0$. Τότε υπάρχει $d \in \mathbb{N}$ με 
	$$\frac{\phi(d)}{d} < A + \varepsilon$$


	\noindent Για κάθε $n \in \mathbb{N}$ θεωρούμε Ευκλείδεια διαίρεση $n = q_n d + r_n, 0\leq r_n < d$. Τότε:
	$$\frac{\phi(n)}{n} \leq \frac{\phi(q_n d)}{n} + \frac{\phi(r)}{n} $$ και γράφοντας το $q_n d = d + d + \ldots + d$ έχουμε:

	$$\frac{\phi(n)}{n} \leq q_n \cdot \frac{\phi(d)}{n} + \frac{\phi(r)}{n} = \frac{q_n d}{n} \cdot \frac{\phi(d)}{d}  + \frac{\phi(r)}{n}$$ Άρα $$\lim\sup_n \frac{\phi(n)}{n} \leq 1 \cdot \frac{\phi(d)}{d} + 0 < A + \varepsilon$$ για το τυχόν $\varepsilon >0$. Άρα το ανώτερο όριο στέκεται κάτω από το κατώτερο όριο και άρα ταυτίζονται. Επιπλέον η ακολουθία $\frac{\phi(n)}{n}$ είναι φραγμένη αφού:
	$$\frac{\phi(n)}{n} = \frac{\phi(1 + 1 + \ldots + 1)}{n} \leq \frac{n \phi(1)}{n} = \phi(1) \in \mathbb{N}$$
\end{proof}


\pagebreak


\noindent 2) Έστω $G$ πεπερασμένα παραγόμενη ομάδα και $S$ πεπερασμένο σύνολο γεννητόρων της $G$. Για κάθε $g \in G$, ορίζουμε $||g|| = ||g||_S = d_S (1,g)$ να είναι η απόσταση του στοιχείου $g$ από το 1 στο γράφημα \tl{Cayley} $\Gamma(G,S)$ της $G$ ως προς το $S$. Τότε $||gh|| \leq ||g|| + ||h||$ για κάθε $g,h \in G$.
\begin{enumerate}
	\item Να δειχθεί ότι υπάρχει το όριο $\lim\limits_{n\rightarrow \infty} \frac{||g^n||}{n}$, το οποίο συμβολίζεται με $\tau_{G,S}(g) = \tau(g)$ και ονομάζεται μήκος μετατόπισης του $g$.
	\item Αν $g$ πεπερασμένης τάξης, τότε $\tau(g) = 0$.
	\item $\tau(g) = \tau(xgx^{-1})$, για κάθε $g,x \in G$ και $\tau(g^m) = |m| \tau(g)$.
	\item $\tau(g) \leq \inf \{ d_S(gv,v)| \quad v \text{ κορυφή του } \Gamma(G,S)\}$.
	\item Αν $G=F$ ελεύθερη και $S$ βάση της $F$, να υπολογιστεί το $\tau(g)$.
	\item Ομοίως, αν $G = \mathbb{Z}^k = \Z \times \cdots \times \Z$ και $S = \{s_1,\ldots,s_k\}$, όπου $s_i$ γεννήτορας του $i$ παράγοντα.
	
\end{enumerate}


\begin{proof}
	\begin{enumerate}
		\item Ορίζουμε $\phi_g :\mathbb{N}\rightarrow \mathbb{N}$ για κάθε $g \in G$ με $\phi_g(n) = ||g^n||$. Τότε
		$$\phi_g (n+m) = ||g^{n+m}|| \leq ||g^n|| + ||g^m|| = \phi_g(n)+\phi_g(m)$$ Άρα το $\tau(g)$ υπάρχει εφόσον η $\phi_g$ ικανοποιεί την προηγούμενη άσκηση.

		\item Αν $o(g) = m < \infty$ τότε θέτουμε $M = \max \{||g||,||g^2||,\ldots,||g^{m-1}||\}$. Τότε για κάθε $n\in\mathbb N$ θα έχουμε $$||g^n|| \leq M$$ και άρα $$\tau(g) = \lim\limits_{n\rightarrow \infty}\frac{||g^n||}{n} \leq \lim\limits_{n\rightarrow \infty}\frac{M}{n} = 0$$
		
		
		\item Έχουμε: $$\frac{||(xgx^{-1})^n||}{n} = \frac{||xg^nx^{-1}||}{n} \leq \frac{||x|| + ||x^{-1}||}{n} + \frac{||g^n||}{n} \rightarrow \tau(g)$$ Άρα $\tau(xgx^{-1}) \leq \tau(g)$. Παίρνουμε το αντίστροφο με τον ίδιο τρόπο στα στοιχεία $h, x^{-1}hx$ όπου $h = xgx^{-1}$. 
		
		$ $\newline
		Με μια αλλαγή μεταβλητής έχουμε ότι: $$\frac{||g^{nm}||}{n|m|} \longrightarrow \tau(g)$$ Επιπλέον $$\frac{||g^{nm}||}{n|m|} = \frac{1}{|m|} \cdot \frac{||\left(g^m\right)^n||}{n} \longrightarrow \frac{1}{|m|} \cdot \tau(g^m)$$ Άρα από μοναδικότητα ορίου
		$$\frac{1}{|m|} \cdot \tau(g^m) = \tau(g)$$ (αν $m=0$ σαφώς ισχύει το ζητούμενο.)

		\item $d_S(gv,v) = d_S(v,gv) = ||v^{-1}gv||$. Αρκεί να δείξουμε ότι το $\tau(g)$ στέκεται κάτω από κάθε $||v^{-1}gv||$. Δείξαμε στην άσκηση 1) ότι το όριο στέκεται κάτω από το $\phi(1)$, $||g||$ στην άσκηση που βρισκόμαστε για το $\tau(g)$. Δηλαδή $$\tau(g) \leq ||g|| \quad \text{ για κάθε } g \in G$$ 
		$$\iff \tau(v^{-1}gv) \leq ||v^{-1}gv|| \quad \text{ για κάθε } v,g \in G$$ Ωστόσο, έχουμε δείξει ότι $\tau(g) = \tau(v^{-1}gv)$. Άρα το $\tau(g)$ στέκεται κάτω από το \tl{infimum} του συνόλου.
	
		\item Γράφουμε την $g$ σαν ανηγμένη λέξη μήκος $n$ στους γεννήτορες και αν στο γινόμενο $g^2$ γίνονται $k$ διαγραφές σημαίνει ότι η $g$ έχει μορφή $vg^{\prime}v^{-1}$ με το μήκος του $v$ να είναι $k$. Συνεπώς $\tau(g) = \tau(vg^{\prime}v^{-1}) = \tau(g^{\prime} = n-2k$.
		
		\item Θα χρησιμοποιήσουμε προσθετικό συμβολισμό, αν το $g$ είναι $g = (m_1 s_1, m_2 s_2 , \ldots , m_k s_k)$ τότε $||g|| = m_1 + m_2 + \ldots +m_k$ και $||ng|| = nm_1 + nm_2 + \ldots nm_k = n\left(m_1 + \ldots m_k\right) = n||g||$ και άρα $\tau(g) = ||g||$.
	\end{enumerate}
\end{proof}

\pagebreak
\noindent 3) Έστω $X$ ένα $\mathbb{R}$-δέντρο, $g$ μια ισομετρία του $X$ και $\tau_g := \inf\{d(gx,x)| x\in X\}$. Αποδείξτε ότι υπάρχει $x_0 \in X$ έτσι ώστε $d(gx_0,x_0) = \tau_g$.
\begin{proof}
	Αν θεωρήσουμε ένα τρίποδο $x,gx,g^2x$ τότε το μέσο $x^{\prime}$ της $[x,gx]$ θα πρέπει να βρίσκεται πριν την διακλάδωση, δηλαδή να είναι μικρότερο του γινόμενου \tl{Gromov} των $gx,g^2$ πάνω στο $x$. Διαφορετικά αν είναι μεγαλύτερο έχουμε το εξής άτοπο με βάση το σχήμα:
	\vspace*{1cm}
	\begin{center}
		\includegraphics[width=0.5\textwidth]{C:/Users/dimit/Desktop/TeX/tripod1}
		%\includegraphics[width=0.5\textwidth]{/mnt/c/Users/dimit/Desktop/TeX/uoa_logo}
		\end{center}
		

	$ $\newline
	Όπου τα δύο γινόμενα \tl{Gromov} είναι ίσα. (Το σχήμα δεν έχει πρόβλημα, αν ο κλάδος του $g^-1x $ ήταν πολύ πιο δεξιά του κλάδου $g^2x$ θα παίρναμε τα γινόμενα $(gx,g^{-1}x)_x, (x,g^2x)_{gx}$ και θα είχαμε ίδια περίπτωση.) Αυτό δεν μπορεί να συμβαίνει καθώς $d(x,x^{\prime}) = d(gx,x^{\prime})$.
	
	$ $\newline
	Είμαστε λοιπόν σε μια περίπτωση όπου:
	\vspace*{1cm}
	\begin{center}
		\includegraphics[width=0.5\textwidth]{C:/Users/dimit/Desktop/TeX/tripod2}
		%\includegraphics[width=0.5\textwidth]{/mnt/c/Users/dimit/Desktop/TeX/uoa_logo}
		\end{center}
	
	$ $\newline
	Αφού $(x,x^{\prime}) = (x^{\prime},gx)$ και $(x,x^{\prime}) = (gx,gx^{\prime})$, το $gx^{\prime}$ δεν είναι ανάμεσα στα $\Lambda, gx$, και επιπλέον αφού η γεωδαισιακή $[x,x^{\prime}]$ είναι μέσα στην $[x,gx]$ τότε η $g[x,x^{\prime}]$ θα είναι μέσα στην $[gx,g^2x]$, δηλαδή το $gx^{\prime}$ πηγαίνει στον πάνω κλάδο. Όμοια κάθε φορά που δράμε το $g$ στο $g^k[x^{\prime},gx^{\prime}]$ έχουμε ότι το πέρας πριν την δράση είναι η αρχή μετά την δράση. Συνεπώς αν τα κολλήσουμε με
	$$A = \bigcup\limits_n g^n [x^{\prime},gx^{\prime}]$$ έχουμε μια γεωδαισιακή που είναι $g$-αναλλοίωτη. Τελικά, για ένα τυχαίο $y \in X$ έχουμε ότι:

	$$d(y,gy) = 2d(y,A) + d(x^{\prime},gx^{\prime})$$ 
	
	
	Εφόσον υποθέσουμε ότι $d(y,A) = d(y,x_1)$ τότε $d(gy,A) = d(gy,gx_1)$ λόγω του $g$-αναλλοίωτου και η απόσταση του $x_1$ από το $gx_1$ είναι ακριβώς $d(x^{\prime},gx^{\prime})$ αφού το $x_1$ είναι πάνω στην $A$. Άρα από την διαδρομή $y,x_1,gx_1,gy$ στο $\mathbb{R}$-δέντρο έχουμε την παραπάνω ισότητα. Δηλαδή $d(x^{\prime},gx) < d(y,gy)$ για τυχαίο $y$ και άρα πετυχαίνουμε το \tl{infimum}.
	

	$ $\newline
	Οι ιδέες σε αυτήν την άσκηση ήταν με συνομιλία με τον συνάδελφο Κωνσταντίνο Γκόλφη σχετικά με την (πιο ισχυρή) πρόταση 1.3 του παρακάτω \tl{paper}:
	
	$ $\newline
	\tl{Marc Culler and John W. Morgan. Group actions on }$\mathbb{R}$-\tl{trees. Proc. London Math. Soc. (3) 55 (1987), no. 3, 571–604.}

\end{proof}

\pagebreak

\noindent 4) Υποθέτουμε ότι η $G$ είναι μια ομάδα η οποία δρα με ισομετρίες επί ενός $\mathbb{R}$-δέντρου $X$. Τότε για κάθε $y \in X$ υπάρχει το όριο $\lim\limits_{n\rightarrow \infty} \frac{d(y,g^ny)}{n}$ και ισούται με $\tau_g$, όπου $\tau_g = \inf\{d(gx,x)| x \in X\}$.

\begin{proof}
	Το ζητούμενο όριο υπάρχει λόγω της άσκησης 1) καθώς η $\phi_y(n) = d(y,g^n)$ ικανοποιεί την ανισότητα:
	$$d(y,g^{n+m}y) \leq d(y,g^ny) + d(g^ny,g^{n+m}y) = d(y,g^ny) + d(y,g^my)$$

	Επιπλέον, βασιζόμενοι στην γεωδαισιακή $A$ που ορίσαμε σαν ένωση στην προηγούμενη άσκηση, αν $d(y,A) = d(y,x_1)$ τότε:
	$$d(y,g^n) = d(y,x_1) + d(x_1,g^nx_1) + d(g^nx_1,g^n y) = 2d(y,A) + d(x_1,g^nx_1) = $$
	$$ = 2d(y,A) + n \cdot d(x^{\prime},gx^{\prime})$$ και άρα
	$$\lim\limits_{n\rightarrow \infty} \frac{d(y,g^ny)}{n} = d(x^{\prime},gx^{\prime}) = \tau_g$$
\end{proof}

\pagebreak

\begin{proof} Αν έχουμε ορίσει τις μετρικές $d_H, d_G$ με κάποια επιλογή μετρικής λέξεις πάνω σε κάποια σύνολα γεννητόρων, εφόσον η ένθεση είναι ισομετρική εμφύτευση υποθέτουμε ότι υπάρχουν $\lambda,\varepsilon$ που εξαρτώνται από την επιλογή των μετρικών με:

	$$\frac{1}{\lambda}d_H(h_1,h_2) - \varepsilon \leq d_G(h_1,h_2) \leq \lambda d_H(h_1,h_2) + \varepsilon$$ Βασιζόμαστε στην προηγούμενη άσκηση και σταθεροποιούμε $h \in H$. Εδώ που έχουμε ομάδες, οι έννοιες των μηκών μετατόπισης των προηγούμενων ασκήσεων ταυτίζονται. Συνεπώς αν το $h$ έχει πεπερασμένη τάξη τότε $\tau_G(h) = \tau_H(h) = 0$ δηλαδή το ζητούμενο ισχύει τετριμμένα. Υποθέτουμε ότι $o(h) = \infty$. Τότε για κάθε $x \in H$:
	$$\tau_G(h) = \lim_{n\rightarrow\infty}\frac{d_G(x,h^nx)}{n}$$ 
	$$\tau_H(h) = \lim_{n\rightarrow\infty}\frac{d_H(x,h^nx)}{n}$$ και άρα 

	$$\frac{1}{\lambda}\tau_H(h) \leq \tau_G(h) \leq \lambda \tau_H(h)$$
\end{proof}
\end{document}


