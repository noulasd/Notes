\documentclass[oneside,a4paper]{article}

%%%%%%%%%%%%%%%%%%%%%%%%%%%%
\usepackage{amsthm}
\usepackage{amsmath}
\usepackage{amssymb}
%%%%%%%%%%%%%%%%%%%%%%%%%%%%%
\usepackage[greek]{babel}
\usepackage[utf8]{inputenc}
\usepackage{mathtools}
\usepackage{blindtext}
\usepackage[T1]{fontenc}
\usepackage{titlesec}
\usepackage{sectsty}
\usepackage{verbatim}
\usepackage{multirow}
\chapternumberfont{\tiny} 
\chaptertitlefont{\Huge}
%ελληνικοι χαρακτηρες σε μαθ pdf utf-8
%%%%%%%%%%%%%%%%%%%%%%%%%%%%%%%%%
\usepackage{tikz-cd}

\usepackage{xcolor}
\usepackage{framed}%frames

\usepackage{array}
\usepackage{pbox}

%%%%%%%%%%%%%%%%%%%%%%%%
\usepackage{tikz}
%%%%%%%%%%%%%%%%%%%%%%%%%%

%%%%%%%%περιθώρια%%%%%%%%%%%%
\usepackage[a4paper,margin=3.5cm]{geometry}


\newtheorem{theorem}{Θεώρημα}
\newtheorem{lemma}{Λήμμα}
\newtheorem{example}{Παράδειγμα}
\newtheorem*{defn}{Ορισμός}
\newtheorem{prop}{Πρόταση}
\newtheorem{cor}{Πόρισμα}

\newcommand {\tl}{\textlatin}
%%%%%%%%%αριθμηση%%%%%%%%%%%%%%
\renewcommand{\theenumi}{\arabic{enumi}}
\renewcommand{\labelenumi}{{\rm(\theenumi)}}
\renewcommand{\labelenumii}{\roman{enumii}) }
%%%%%%%%%%%% New theorems %%%%%%%%%%%%%%%%%%%%%%%%

%%%%%%%%%%%%%%%%%%%%%%%%%%%%%%%%%%%%%%%%%%%%%%%%%%%
\newcommand{\Z}{\mathbb{Z}}
\newcommand{\Q}{\mathbb{Q}}
\newcommand{\Co}{\mathbb{C}}
%%%%%%%%%%%%%%%%%%%%% Document starts %%%%%%%%%%%%
\begin{document}
	
	%%%%%%%%%%%%%%%%%%%%%%%%%%%%%%%%%%%%%%%%%%%%%%%%%%
	\selectlanguage{greek}
	%%%%%%%%%%%%%%%%%%%%%%% Start Roman numbering %%%% vbbnn
	%\pagenumbering{roman}
	%%%%%%%%%%%%%%%%%%%%%%%%%%%%%%%%%%%%%%%%%%%%%%%%%%
	
	\begin{framed}	
		%\vspace{0.3truecm}
		\begin{center}
			\huge Θεωρία Ομάδων 2
		\end{center}
		%\vspace{0.3truecm}
		\begin{center}
			\huge Πρώτο πακέτο Ασκήσεων
		\end{center}
		\vspace{0.3truecm}
		\begin{center}
			Ονομ/νο: Νούλας Δημήτριος\\
			ΑΜ: 1112201800377 (προπτυχιακό) \\
			\tl{email}: \tl{dimitriosnoulas@gmail.com}
		\end{center}
		\vspace{0.3truecm}
	\end{framed}
	\vspace*{\fill}
	\begin{center}
	\includegraphics[width=0.5\textwidth]{C:/Users/dimit/Desktop/TeX/uoa_logo}
	%\includegraphics[width=0.5\textwidth]{/mnt/c/Users/dimit/Desktop/TeX/uoa_logo}
	\end{center}
\vspace{1cm}
\pagebreak


\noindent Άσκηση 1) Αν η $\phi: X \rightarrow Y$ είναι μια σχεδόν ισομετρία μεταξύ μετρικών χώρων, τότε η $\phi$ έχει σχεδόν αντίστροφη, δηλαδή υπάρχει σχεδόν ισομετρία $\psi: Y \rightarrow X$ και $M >0$ έτσι ώστε $d_X(\psi \circ \phi(x),x) \leq M$ και $d_Y (\phi \circ \psi(y),y) \leq M$ για κάθε $x \in X, y \in Y$.

\begin{proof} $ $

$ $\newline
Η $\phi$ είναι σχεδόν ισομετρία από τον $X$ στον $Y$ άρα υπάρχουν $\lambda >0, \varepsilon \geq 0$ με

$$\frac{1}{\lambda}d(x_1,x_2) -\varepsilon \leq d(\phi(x_1),\phi(x_2)) \leq \lambda d(x_1,x_2) + \varepsilon$$
%(σχεδόν εμφύτευση, η λέξη εμφύτευση ισχύει κάπως αν δοκιμάσω να πάρω $\phi (x_1) = \phi(x_2)$ )

$ $\newline
και υπάρχει $K\geq 0$ με $d(\phi(x),y) \leq K$. (Mπορώ να κάνω παραδοχή και να πάρω $R = \max$ από τις τρεις σταθερές )

$ $\newline
Αν $x \in X$ υπάρχει $y \in Y$ με $d(\phi(x),y) \leq K$ δηλαδή $\phi (x) \in N_K(y)$ δηλαδή $x \in \phi^{-1} (N_K(y))$

$ $\newline
Άρα η οικογένεια $\{\phi^{-1} (N_K(y))\}_{y \in Y}$ καλύπτει τον $X$. Έτσι από αξίωμα επιλογής διαλέγουμε $\psi (y) \in \phi^{-1} (N_K(y))$ και έτσι ορίζεται η $\psi$.

$ $\newline
Έχουμε:

$$\phi (\phi^{-1} (N_K(y)) \subseteq N_K(y)$$ δηλαδή 

$$d(\phi \circ \psi (y),y) \leq K$$

$ $\newline
Μετά έχουμε $\psi \circ \phi (x) \in \phi^{-1} (N_K(\phi (x)))$ και άρα $\phi \circ \psi \circ \phi(x) \in \phi (\phi^{-1} (N_K(\phi(x))) \subseteq N_K(\phi(x))$

$ $\newline
Επομένως:


$$d(\psi\circ \phi(x) , x) \leq \lambda d(\phi \circ \psi \circ \phi(x), \phi(x)) + \varepsilon \leq \lambda K + \varepsilon$$

$ $\newline
Άρα θέτω $M = \max \{K,\lambda K + \varepsilon\}$ και έχω τις ζητούμενες συνθήκες. Αρκεί να επιβεβαιωθεί ότι η $\psi$ είναι σχεδόν ισομετρία.

$ $\newline
Εμφύτευση:

$$d(\psi(y_1),\psi(y_2)) \leq \lambda d(\phi \circ \psi (y_1) , \phi \circ \psi (y_2)) + \lambda \varepsilon$$
$$\leq \lambda ( d(\phi\circ\psi(y_1),y_1) + d(y_1,y_2) + d(\phi\circ\psi(y_2),y_2)) + \lambda \varepsilon$$
$$\leq \lambda (2M+\varepsilon) + \lambda d(y_1,y_2)$$
$$\leq \varepsilon^{\prime} + \lambda d(y_1,y_2)$$

$ $\newline
Για την άλλη ανισότητα:


$$d(y_1,y_2) \leq d(y_1,\phi\circ\psi(y_1)) + d(\phi\circ\psi(y_1), \phi\circ\psi(y_2)) + d(y_2,\phi\circ\psi(y_2))$$ άρα 

$$d(\phi\circ\psi (y_1) ,\phi\circ\psi(y_2)) \geq d(y_1,y_2) - d(y_1,\phi\circ\psi (y_1)) - d(y_2,\phi\circ\psi(y_2))$$

$ $\newline
Χρησιμοποιώντας ότι η $\phi$ είναι σχεδόν ισομετρία:

$$d(\psi(y_1),\psi(y_2)) \geq \frac{1}{\lambda} d(\phi\circ\psi(y_1),\phi\circ\psi(y_2)) - \frac{\varepsilon}{\lambda}$$
$$\geq \frac{1}{\lambda} d(y_1,y_2) - \frac{2M + \varepsilon}{\lambda}$$
$$\geq \frac{1}{\lambda} d(y_1,y_2) - \varepsilon^{\prime}$$

$ $\newline
όπου θέσαμε $\varepsilon^{\prime} = \max \{\frac{2M+\varepsilon}{\lambda}, \lambda(2M + \varepsilon)\}$

$ $\newline
Σχεδόν επί:


$$d(\psi(y),x) \leq d(\psi(y),\psi \circ \phi(x)) + d(x,\psi\circ \phi(x))$$
$$\leq \lambda d(y,\phi(x)) + \lambda \varepsilon + M$$
$$\leq \lambda K + \lambda \varepsilon + M$$
$$\leq  \lambda M + \lambda \varepsilon + M = C$$ άρα σχεδόν επί.

\end{proof}

\pagebreak 


\noindent Άσκηση 2) Αν $X$ και $Y$ είναι μετρικοί χώροι γράφουμε $X \underset{qi}{\sim} Y$ αν ο $X$ είναι σχεδόν ισομετρικός με τον $Y$. Αποδείξτε ότι η σχέση $X \underset{qi}{\sim} Y$ είναι σχέση ισοδυναμίας.

\begin{proof} $ $

$ $\newline
Η ταυτοτική απεικόνιση του $X$ είναι σχεδόν ισομετρία και άρα $X\underset{qi}{\sim} X$.

$ $\newline
Στην άσκηση 1 δείξαμε ότι αν $X\underset{qi}{\sim} Y$ τότε $Y \underset{qi}{\sim} X$ .

$ $\newline
Έστω $f: X \rightarrow Y$ και $g:Y\rightarrow Z$ $(\lambda,\varepsilon)$-σχεδόν ισομετρίες. Μπορούμε να θεωρήσουμε ίδιο $\lambda = \max \{\lambda_1,\lambda_2\}$ και όμοια $\varepsilon = \max \{\varepsilon_1,\varepsilon_2\}$ . Μάλιστα θα μπορούσαμε να θεωρήσουμε μια σταθερά $\max \{\varepsilon,\lambda \}$ για διευκολία στις πράξεις.

$ $\newline
Ισχυρισμός: $gf : X \rightarrow Z$ σχεδόν ισομετρία.

$$d(gf(x_1),gf(x_2)) \leq \lambda d(f(x_1),f(x_2)) + \varepsilon \leq \lambda \left(\lambda d(x_1,x_2) +  \varepsilon \right) + \varepsilon = \lambda^2 d(x_1,x_2) + \lambda \varepsilon + \varepsilon$$

$$d(gf(x_1),gf(x_2)) \geq \frac1{\lambda} d(f(x_1),f(x_2)) - \varepsilon \geq  \frac1{\lambda} \left( \frac1{\lambda} d(x_1,x_2) - \varepsilon \right)  - \varepsilon  = \frac1{\lambda^2} d(x_1,x_2) - (\varepsilon + \frac{\varepsilon}{\lambda})$$

$ $\newline
θέτουμε $k = \max \{\varepsilon + \frac{\varepsilon}{\lambda}, \lambda \varepsilon + \varepsilon\}$ και τότε $gf : X \rightarrow Z$ είναι $(\lambda^2,k)$ - σχεδόν εμφύτευση.

$ $\newline
Θεωρούμε τώρα ότι $f,g$ είναι σχεδόν επί με κοινή σταθερά $M \geq 0$

$ $\newline
Έστω $z \in Z$, υπάρχει $y \in Y$ έτσι ώστε $d(g(y),z) \leq M$ και για το $y$ υπάρχει $x \in X$ ώστε $d(f(x),y) \leq M$


$$d(z,gf(x)) \leq d(z,g(y)) + d(g(y),gf(x)) \leq M + \lambda d(y,f(x)) + \varepsilon \leq M + \lambda M + \varepsilon$$

$ $\newline
άρα και $gf :X\rightarrow Z$ σχεδόν επί.


\end{proof}

\pagebreak

\noindent Άσκηση 3) Έστω $H$ μια υποομάδα μιας πεπερασμένα παραγόμενης ομάδας $G$. Τότε η ένθεση $H\xhookrightarrow{} G$ είναι σχεδόν ισομετρία να και μόνο αν η $H$ είναι πεπερασμένου δείκτη στη $G$. Γενικότερα,

\begin{proof} $ $

$ $\newline
Αν η $H$ έχει πεπερασμένο δείκτη, τότε έχουμε δείξει στην θεωρία ότι η δράση της $H$ (περιορισμός της δράσης της $G$) στο $\Gamma(G,S)$ για οποιοδήποτε σύνολο γεννητόρων της $G$ (πεπερασμένο από υπόθεση) ικανοποιεί τις συνθήκες του θεμελιώδους θεωρήματος της γεωμετρικής θεωρίας ομάδων. Άρα έχουμε ότι $H \underset{qi}{\sim} \Gamma(G,S)$ και ειδικότερα, η $H$ είναι και αυτή πεπερασμένα παραγόμενη. Τώρα έχουμε δείξει στις προηγούμενες ασκήσεις ότι η σχέση σχεδόν ισομετρίας επάγει σχέση ισοδυναμίας στην κλάση των μετρικών χώρων και άρα αφού
$$G \underset{qi}{\sim}\Gamma(G,S)$$ έχουμε $H \underset{qi}{\sim} G$.

$ $\newline
Εδώ δεν δείξαμε ότι η ένθεση είναι σχεδόν ισομετρία, αλλά από την απόδειξη του θεωρήματος παίρνουμε ότι $i \circ \phi : H \rightarrow G \rightarrow \Gamma(G,S)$ είναι σχεδόν ισομετρία και η $\phi$ είναι σχεδόν ισομετρία. Με βάση τις προηγούμενες ασκήσεις, η $\phi$ έχει σχεδόν ισομετρική αντίστροφη, άρα και η $i$ είναι σχεδόν ισομετρία.

$ $\newline
Αντίστροφα, έστω ότι $H \underset{qi}{\sim} G$ , δηλαδή η ένθεση είναι σχεδόν ισομετρία και έστω ένα τυχαίο σύμπλοκο $Hg$ με $g \not\in H$.

$ $\newline
Έχουμε ότι η $i: H \xhookrightarrow{}G$ είναι σχεδόν επί και άρα υπάρχει $M \geq 0$ έτσι ώστε για κάθε $x \in G$ να υπάρχει $h \in H$ με $d(x,i(h)) = d(x,h) \leq M$. Άρα υπάρχει $h \in H$ με $d(g,h) \leq M$ και $d(g,h) = ||h^{-1}g||_S$ όπου $S$ είναι ένα πεπερασμένο σύνολο γεννητόρων της $G$. Επιπλέον, έχουμε ότι το σύνολο των $\{x \in G: \quad ||x||_S \leq M\}$ είναι πεπερασμένο, αφού η $G$ είναι πεπερασμένα παραγόμενη, δηλαδή έχουμε ένα πεπερασμένο αλφάβητο και άρα μπορούμε να φτιάξουμε πεπερασμένες λέξεις κάτω από ένα μήκος. Γράφουμε το στοιχείο $h^{-1}g$ ως

$$h^{-1}g = s^{\varepsilon_1}_1 \cdots s^{\varepsilon_n}_n, \quad s_i \in S, \varepsilon_i \in \{\pm 1\} $$ και έτσι:

$$Hg = H (hh^{-1} g) = H(h^{-1}g) = H(s^{\varepsilon_1}_1 \cdots s^{\varepsilon_n}_n)$$ δηλαδή το τυχόν σύμπλοκο γράφεται με συγκεκριμένο αντιπρόσωπο με μήκος μικρότερο του $M$, οι οποίοι είναι πεπερασμένοι. Άρα έχουμε πεπερασμένα το πλήθος σύμπλοκα και άρα $[G:H]< \infty$.


%$ $\newline
%Τότε υπάρχουν αντιπρόσωποι $(g_i)_{i \in I}$ με την ομάδα $G$ να είναι ξένη ένωση των συμπλόκων $g_i H$. Ισχυριζόμαστε ότι υπάρχουν αντιπρόσωποι με αυθαίρετη απόσταση από την $H$. Πράγματι, αν υπάρχει κάποιο φράγμα $M$ έτσι ώστε $d(g_i,h) \leq M$ για κάθε $i \in I$ και $h \in H$ έχουμε δηλαδή $||g^{-1}_i h|| \leq M$ για όλους τους άπειρους αντιπροσώπους και για κάθε $h \in H$. Ωστόσο, το σύνολο των $g \in G$ με $||g|| \leq M$ είναι πεπερασμένο αφού η $G$ είναι πεπερασμένα παραγόμενη. Άρα δεν μπορεί να υπάρχει τέτοιο φράγμα. Αυτό έρχεται σε αντίθεση καθώς έχουμε υποθέσει ότι η $i : H \xhookrightarrow{} G$ είναι σχεδόν επί, δηλαδή υπάρχει $M \geq 0$ ώστε για κάθε $g \in G$ υπάρχει $h \in H$ τέτοιο ώστε $d(g,i(h)) = d(g,h) \leq M$. Διαλέγουμε ως $g$ έναν αντιπρόσωπο με απόσταση μεγαλύτερη από $M$ από το $H$ και έχουμε άτοπο.

\end{proof}




\pagebreak

\noindent Άσκηση 4) Έστω $\phi : G_1 \rightarrow G_2$ ομομορφισμός μεταξύ πεπερασμένα παραγόμενων ομάδων. Δείξτε ότι αν η $\phi$ είναι σχεδόν ισομετρική εμφύτευση, τότε ο πυρήνας $ker\phi$ είναι πεπερασμένος και ότι η $\phi$ είναι σχεδόν ισομετρία αν και μόνο αν ο πυρήνας $ker\phi$ είναι πεπερασμένος και η εικόνα $im\phi$ πεπερασμένου δείκτη στην $G_2$. Ιδιαιτέρως, αν $N$ πεπερασμένη κανονική υποομάδα μιας πεπερασμένα παραγόμενης ομάδας $G$, τότε $G\underset{qi}{\sim} G/N$.

\begin{proof} $ $


$ $\newline
Έστω $g \in ker\phi$ . Τότε $||g|| =d(g,1) \leq \lambda(\phi(g),\phi(1)) + \varepsilon  = \lambda d(1_{G_2},1_{G_2}) + \lambda \varepsilon = \lambda\varepsilon$

$ $\newline
Αφού η ομάδα $G_1$ είναι πεπερασμένα παραγόμενη έχουμε πεπερασμένα στοιχεία με μήκος κάτω από ένα συγκεκριμένο φράγμα. Άρα αν $\phi$ ισομετρική εμφύτευση έπεται ότι ο πυρήνας $ker\phi$ είναι πεπερασμένος.

$ $\newline
Αν τώρα η $\phi$ είναι και σχεδόν επί, θεωρούμε ως προς άτοπο ότι η εικόνα $Im\phi$ έχει άπειρο δείκτη και δουλεύουμε με το ίδιο επιχείρημα με την άσκηση 3, 


$ $\newline
Αντίστροφα, υποθέτουμε ότι $ker\phi$ πεπερασμένος και $Im\phi$ πεπερασμένου δείκτη.


$ $\newline
Αν $S_2$ ένα σύνολο γεννητόρων της $G_2$, Θεωρούμε αντιπρόσωπους για τα δεξιά σύμπλοκα $(Im\phi) y_1 ,\ldots (Im\phi) y_n$ και θέτουμε $M = \max \{ ||y_i||_{S_2}: i=1,\ldots,n \}$

$ $\newline
Έστω $y \in G_2$, τότε το $y$ θα ανήκει σε ένα μοναδικό σύμπλοκο $(Im\phi) y_i$ έτσι ώστε $y y^{-1}_i \in Im\phi$, δηλαδή υπάρχει $x \in G_1$ με $\phi(x) = yy^{-1}_i$

$ $\newline
Συνεπώς $d(\phi(x),y) = d(yy^{-1}_i,y) = ||y^{-1}_i||_{S_2} \leq M$

$ $\newline
άρα $\phi$ σχεδόν επί.

$ $\newline
Αν $S_1$ ένα σύνολο γεννητόρων της $G_1$, τότε εφόσον ο πυρήνας είναι πεπερασμένος, υπάρχει τουλάχιστον ένας γεννήτορας $s \in S_1$ με $\phi(s) \neq 1$, δηλαδή αν θέσουμε $\lambda = \max \{ ||\phi(s)||_{S_1} : s \in S_1 \}$ τότε $\lambda \geq 1$.  Θεωρούμε $g_1, g_2 \in G_1$. Αν το στοιχείο $g^{-1}_2 g_1$ γράφεται ως γινόμενο $n = d(g_1,g_2)$ γεννητόρων από το $S_1$ τότε το $\phi(g^{-1}_2 g_1)$ θα έχει μια παράσταση ως 

$$\phi(g^{-1}_2 g_1) = \phi (s^{\varepsilon_1}_1 \cdots s^{\varepsilon_{m}}_m) = \phi (s^{\varepsilon_1}_1) \cdots \phi (s^{\varepsilon_{m}}_m)$$
με $m\leq n$ όπου έχουμε διώξει τους γεννήτορες που φτιάχνον στοιχεία του πεπερασμένου πυρήνα $ker\phi$ και θέτουμε $\varepsilon = n-m$.

$ $\newline
Άρα με τις ιδιότητες του ομομορφισμού έχουμε:

$$d(\phi(g_1),\phi(g_2)) = ||\phi(g^{-1}_2 g_1)|| \leq m \cdot \lambda  \leq n \cdot \lambda = \lambda d(g_1,g_2) \leq \lambda d(g_1,g_2) + \varepsilon$$ αφού έχουμε μήκη $m$ στοιχείων με μήκος κάτω από $\lambda$. Όμοια κάθε ένα από αυτά τα στοιχεία $\phi(s^{\varepsilon_i}_i)$ έχει μήκος πάνω από $1$ αφού έχουμε διώξει τα στοιχεία του πυρήνα, άρα:

$$ ||\phi(g^{-1}_2 g_1)|| \geq m \cdot 1  = n \cdot 1 - \varepsilon \geq n \frac{1}{\lambda} - \varepsilon = \frac{1}{\lambda} d(g_1,g_2) -\varepsilon$$ άρα $\phi$ σχεδόν ισομετρία.

$ $\newline
Έστω $N$ πεπερασμένη κανονική υποομάδα μιας πεπερασμένα παραγόμενης $G$. Τότε η φυσική προβολή $\pi : G\rightarrow G/N$ είναι επιμορφισμός, άρα και σχεδόν επί. Έχουμε και το πεπερασμένο του $[G:Im\pi] = 1$ αλλά το επιμορφισμός μας αρκεί. Επιπλέον $ker\pi = N$ πεπερασμένο και άρα έχουμε και το σχεδόν εμφύτευση. Συνεπώς $G \underset{qi}{\sim} G/N$.

\end{proof}




\pagebreak

\noindent Άσκηση 5) Έστω $T_n$ το δέντρο του οποίου κάθε κορυφή είναι άκρο ακριβώς $n$ το πλήθος (γεωμετρικών) ακμών (κάθε ακμή θεωρούμε ότι έχει μήκος 1). Δείξτε απ ευθείας (δηλ. χωρίς να κάνετε χρήση του $F_4 \underset{qi}{\sim} F_3$ ) ότι $T_4 \underset{qi}{\sim} T_3$.

\begin{proof} $ $


$ $\newline
Χρωματίζουμε με τρία χρώματα κόκκινο, πράσινο και μπλε κάθε ακμή του δέντρου $T_3$ έτσι ώστε σε κάθε κορυφή να βρίσκεται μια ακμή από κάθε χρώμα. Ορίζουμε μια $\phi : T_3 \rightarrow T_4$ με την οποία κολλάμε τις άκρες κάθε ακμής κόκκινου χρώματος, δηλαδή στέλνουμε ολόκληρη την ακμή σε ένα σημείο. Έτσι, κάθε κορυφή θα έχει πλέον 4 ακμές, δύο μπλε και δύο πράσινες και άρα είναι το δέντρο $T_4$.

$ $\newline
Από τον ορισμό, η $\phi$ είναι επί και άρα και σχεδόν επί.

$ $\newline
Έστω $x_1,x_2 \in T_3$. Οι αποστάσεις στο $T_4$ δεν γίνονται μεγαλύτερες αλλά μόνο μικραίνουν στην περίπτωση που περνάμε στο $[x_1,x_2] \subseteq T_3$ πάνω από ακμή με κόκκινο χρώμα. Φυσικά, δεν έχουμε κύκλους στο δέντρο και το $[x_1,x_2]$ είναι η γεωδαισιακή των δύο σημείων. Άρα

$$d(\phi(x_1),\phi(x_2)) \leq d(x_1,x_2) \leq 2d(x_1,x_2) + 1$$

$ $\newline
Τώρα για ένα τυχαίο (γεωδαισιακό) μονοπάτι $[\phi(x_1),\phi(x_2)]$ στο $T_4$ αυτό στην πιο εξαντλητική περίπτωση προέρχεται από μονοπάτι $[x_1,x_2]$ το οποίο έχει ακέραιο μήκος, δηλαδή τα $x_1,x_2$ είναι κορυφές, καθώς και πριν την κάθε μπλε ή πράσινη ακμή που μεταφέρεται στο $T_4$ να περάσαμε από κόκκινη ακμή. Μαζί με την μία έξτρα κόκκινη ακμή που μπορεί να διασχίσαμε στο τέλος του $[x_1,x_2]$ παίρνουμε ότι:

$$d(x_1,x_2) \leq 2d(\phi(x_1),\phi(x_2)) + 1$$ και άρα

$$\frac{1}{2} d(x_1,x_2) -1 \leq \frac12 d(x_1,x_2) -\frac12 \leq d(\phi(x_1),\phi(x_2))$$ άρα με την $\phi$ έχουμε $T_3\underset{qi}{\sim} T_4$.
\end{proof}

\end{document}