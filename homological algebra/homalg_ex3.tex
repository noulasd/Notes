\documentclass{article}
\usepackage[english,greek]{babel}
\usepackage[utf8x]{inputenc}
\usepackage{amsfonts, amsmath, amssymb}
\usepackage{mathtools}
\usepackage{enumitem}

\usepackage{tikz-cd}

\newcommand{\Z}{\mathbb{Z}}

\title{\textbf{Ομολογική Άλγεβρα και Κατηγορίες} \\ 3η Ομάδα Ασκήσεων}

\author{Νούλας Δημήτριος \\ 1112201800377}

\begin{document}
\maketitle


\begin{enumerate}
	
	\item Η ομάδα $A$ είναι διαιρετή και έχει υποομάδα την $kerf$ συνεπώς η ομάδα πηλίκο $A /kerf \cong imf$ είναι διαιρετή. Άρα η ομάδα $imf$ είναι επιπλέον ένα εμφυτευτικό πρότυπο αφού είναι διαιρετή.

		Ισοδύναμα κάθε μονομορφισμός $imf \xhookrightarrow{} M$ διασπάται.

		Θεωρούμε την βραχεία ακριβή ακολουθία:
		$$ 0 \rightarrow imf \xhookrightarrow{i} B \xrightarrow{p} cokerf \rightarrow 0$$

		Έχουμε ότι ο μονομορφισμός $i$ είναι διασπώμενος $\iff$ η βραχεία ακριβής ακολουθία είναι διασπώμενη $\iff$ ο επιμορφισμός $p$ είναι διασπώμενος.
	$ $\newline
	

	\item Αν $Ext_{R}^1 (M,N)=0$ για κάθε $N$ $R$-πρότυπο, δηλαδή είναι η τετριμμένη ομάδα, λόγω αμφιμονοσήμαντης αντιστοιχίας έχουμε ότι και το $ext(M,N)$ είναι σύνολο με ένα στοιχείο.

		Δηλαδή κάθε επέκταση $0 \rightarrow N \rightarrow M^{\prime} \rightarrow M \rightarrow 0$ είναι ισοδύναμη με την τετριμμένη επέκταση:
		$$0 \rightarrow N \rightarrow M\oplus N \rightarrow M \rightarrow 0$$
		To οποίο είναι ισοδύναμο με το ότι κάθε επιμορφισμός $M^{\prime} \rightarrow M$ διασπάται. Aυτό με την σειρά του είναι ισοδύναμο ότι το $M$ είναι προβολικό.
$ $\newline


	\item Θα δείξουμε αρχικά ότι $n \cdot \Z / m\Z = (d\Z ) / (m \Z)$ όπου $d = $μκδ$(m,n)$. Πράγματι, έστω $n [z]_m$. Αν $n=dk$ τότε:
		$$n [z]_m = dk[z]_m = [dkz]_m \in (d\Z ) / (m\Z )$$

		Αντίστροφα, έστω $[dz]_m \in (d\Z ) / (m\Z)$. Υπάρχουν $a,b \in \Z$ τέτοια ώστε $am+bn=d$. Συνεπώς:
		$$[dz]_m = [(am + bn)z]_m = [bnz]_m = n [bz]_m \in n \cdot \Z / m\Z$$

		Επιπλέον, στην θεωρία έχει δειχθεί ότι $Ext_{\Z}^1 (\Z_n, B) = B / nB$. Άρα:
		$$Ext_{\Z}^1 (\Z_n , \Z_m) = \Z_m / n \Z_m = \frac{\Z / m\Z}{ n \cdot \Z / m\Z} = \frac{\Z / m \Z}{d\Z / m\Z} \cong \Z / d\Z $$
		χρησιμοποιώντας και το 3ο θεώρημα ισομορφισμών. Άρα πράγματι η ομάδα $Ext_{\Z}^1(\Z_m , \Z_n)$ είναι κυκλική τάξης $d$.
		$ $\newline

	\item Σταθεροποιούμε $n \in \mathbb{N}$ και ορίζουμε $K_n = \{ a \in A : na=0\}$. Ισχύει ότι $Hom_{\Z} (K_n , B) = 0$. Πράγματι, αν $f: K_n \rightarrow B$ προσθετική τότε:
		$$0 = f(nk) = nf(k)$$
		Δηλαδή το στοιχείο $f(k) \in B$ έχει πεπερασμένη τάξη $n$ και η ομάδα $B$ είναι ελευθέρας στρέψης, συνεπώς $f(k) = 0$ για κάθε $k \in K_n$.

		Έχουμε την ακριβή ακολουθία:

		\begin{center}
			\begin{tikzcd}
0 \arrow[r] & K_n \arrow[r, "i", hook] & A \arrow[r, "\cdot n"] & A \arrow[r] & 0
\end{tikzcd}
		\end{center}

		όπου έχουμε τον πολλαπλασιασμό με $n$. Είναι πράγματι ακριβής σε κάθε θέση αφού:

		\begin{itemize}
			\item $im 0 = keri = 0$
			\item $imi = K_n = kern$
			\item Εφόσον η $A$ είναι διαιρετή, $imn = A = ker(A\rightarrow 0)$
		\end{itemize}

		Έχουμε την μακριά ακριβής ακολουθία συνομολογίας:

		\begin{center}
			\begin{tikzcd}
\ldots \arrow[r] & {Ext_{\Z}^0 (K_n ,B)} \arrow[r] &   {Ext_{\Z}^1 (A,B)} \arrow[r] &   {Ext_{\Z}^1 (A,B)} \arrow[r] & \ldots
\end{tikzcd}
		\end{center}

		δηλαδή παίρνουμε την ακριβή ακολουθία:

		$$0 \rightarrow Ext_{\Z}^1 (A,B) \xrightarrow{n} Ext_{\Z}^1 (A,B)$$

		Έπεται ότι η δεξιά απεικόνιση είναι 1-1. Αυτό σημαίνει ότι αν έχουμε ένα στοιχείο $x$ της ομάδας $Ext_{Z}^1 (A,B)$ με πεπερασμένη τάξη $n$, δηλαδή $nx=0$ συνεπάγεται ότι $x=0$. Αυτό συμβαίνει για το τυχόν $n \in \mathbb{N}$, άρα η ομάδα $Ext_{\Z}^1 (A,B)$ είναι ελεύθερη στρέψεως.
$ $\newline

\item $a)$ Θεωρούμε ως προς άτοπο τον ισομορφισμό $f: \Z_9 \rightarrow \Z_9$ ώστε το παρακάτω διάγραμμα να είναι μεταθετικό

	\begin{center}
		\begin{tikzcd}
0 \arrow[r] & \Z_3 \arrow[r, "i"] \arrow[d, no head, Rightarrow] & \Z_9 \arrow[r, "p"] \arrow[d, "f"] & \Z_3 \arrow[r] \arrow[d, no head, Rightarrow] & 0 \\
0 \arrow[r] & Z_3 \arrow[r, "-i"]                                & \Z_9 \arrow[r, "p"]                & Z_3 \arrow[r]                                 & 0
\end{tikzcd}
	\end{center}

	Για να οριστεί πλήρως ο ισομορφισμός $f$ χρειάζεται να απεικονίσουμε έναν γεννήτορα σε έναν γεννήτορα. Έστω $f([1]_9 ) = [x]_9$ με μκδ$(x,9)=1$.
	
	Λόγω της μεταθετικότητας του δεύτερου τετραγώνου έχουμε:
	$$(pf)[n]_9 = [n]_3$$ για κάθε $[n]_9 \in Z_9$.

	ωστόσο $(pf)[n]_9 = n(pf) [1]_9 = n \cdot p\left( f[1]_9 \right) = n p[x]_9 = [nx]_3$. Δηλαδή, για κάθε $n \in \Z$ έχουμε ότι:
	$$nx = n \quad mod 3$$ και άρα
	$$ x = 1 \quad mod 3$$

Από την μεταθετικότητα του πρώτου τετραγώνου προκύπτει ότι:
$$f[3n]_9 = [-3n]_9$$

Συνεπώς $[-3]_9 = f[3]_9 = 3f[1]_9 = [3x]_9$. Δηλαδή έχουμε τις συνεπαγωγές:

	$$3x = -3 \quad mod 9$$
	$$3(x+1) = 0 \quad mod 9 $$
	$$x+1 = 0 \quad mod 3 $$
	$$x = 2 \quad mod 3$$
και άρα καταλήξαμε σε άτοπο.
$ $\newline

$b)$ Γνωρίζουμε ότι το $\Z_9 $ δεν είναι ισόμορφο με το $\Z_3 \oplus \Z_3$, συνεπώς καμία από τις δύο επεκτάσεις δεν είναι ισοδύναμη με την τετριμμένη. Έχουμε δηλαδή ότι το σύνολο $ext(\Z_3, \Z_3)$ έχει τουλάχιστον 3 κλάσες ισοδυναμίας για στοιχεία. Αρκεί να δείξουμε ότι έχει ακριβώς 3. 

Λόγω αμφιμονοσήμαντης αντιστοιχίας με την ομάδα $Ext_{\Z}^1 (\Z_3 , \Z_3)$, αρκεί να δείξουμε ότι αυτή είναι (κυκλική) τάξης 3. Πράγματι:

$$Ext_{\Z}^1 (\Z_3 , \Z_3) = \Z_3 / 3 \Z_3 \cong \Z_3 / \{0\} \cong \Z_3$$$ $\newline



\item $a)$ Ορίζουμε την $\widetilde{f} : ext(M,N) \rightarrow ext(M^{\prime}, N)$ ως εξής:
	$$[e] \mapsto [e^{\prime}]$$
	Πρέπει να δείξουμε ότι είναι καλά ορισμένη απεικόνιση, δηλαδή δεν εξαρτάται από τον αντιπρόσωπο της κλάσης.

	Αν $[e] = [\varepsilon]$ δηλαδή έχουμε ισομορφισμό $g: L_1 \rightarrow L_2$ ώστε το παρακάτω διάγραμμα να είναι μεταθετικό:

	\begin{center}
		\begin{tikzcd}
			e:0 \arrow[r] & N \arrow[r] \arrow[d, no head, Rightarrow] & L_1 \arrow[r, "p_1"] \arrow[d, "g"] & M \arrow[r] \arrow[d, no head, Rightarrow] & 0 \\
\varepsilon :0 \arrow[r] & N \arrow[r]                                & L_2 \arrow[r, "p_2"]                & M \arrow[r]                                & 0
\end{tikzcd}
	\end{center}

	Θα πρέπει να έχουμε ισομορφισμό $\widetilde{g} : L^{\prime}_1 \rightarrow L^{\prime}_2$ και μεταθετικό διάγραμμα:

	\begin{center}
		\begin{tikzcd}
0 \arrow[r] & N \arrow[r] \arrow[d, no head, Rightarrow] & L^{\prime}_1 \arrow[r, "p^{\prime}_1"] \arrow[d, "\widetilde{g}"] & M^{\prime} \arrow[r] \arrow[d, no head, Rightarrow] & 0 \\
0 \arrow[r] & N \arrow[r]                                & L^{\prime}_2 \arrow[r, "p^{\prime}_2"]                            & M^{\prime} \arrow[r]                                & 0
\end{tikzcd}
	\end{center}

	όπου:
	$$L^{\prime}_1 = \{ (m^{\prime}, l) \in M^{\prime} \oplus L_1 : \quad f(m^{\prime}) = p_1 (l) \in M^{\prime} \}$$
	$$L^{\prime}_2 = \{ (m^{\prime}, l) \in M^{\prime} \oplus L_2 : \quad f(m^{\prime}) = p_2 (l) \in M^{\prime} \} $$


	Ορίζουμε την $\widetilde{g} : L^{\prime}_1 \rightarrow L^{\prime}_2 $ ως εξής:
	$$(m^{\prime}, l) \longmapsto (m^{\prime}, g(l))$$

	H $\widetilde{g}$ είναι καλά ορισμένη αφού $m^{\prime} \in M^{\prime}, g(l) \in L_2$ και ισχύει $f(m^{\prime} ) = p_1 (l) = p_2 (g(l))$ από την μεταθετικότητα του πρώτου διαγράμματος. Αρκεί να δείξουμε ότι η $\widetilde{g}$ είναι ισομορφισμός $R$-προτύπων και να ελέγξουμε την μεταθετικότητα του δεύτερου διαγράμματος.

	Δείχνουμε τα εξής:

	\begin{itemize}
		\item $R$-γραμμικότητα:

			$$\widetilde{g} \left( r\left( m^{\prime}, l \right) \right) = \widetilde{g} \left( rm^{\prime} , rl \right)  = \left( rm^{\prime}, g\left( rl \right)  \right) = r \left( m^{\prime}, g(l) \right) =  r\widetilde{g}\left( m^{\prime} , l \right)   $$

			$$ \widetilde{g} \left( (m^{\prime}_1 , l_1 ) + (m^{\prime}_2, l_2 ) \right) = \widetilde{g} \left( m^{\prime}_1 + m^{\prime}_2 , l_1 + l_2  \right) = \left( m^{\prime}_1 + m^{\prime}_2 , g(l_1 + l_2) \right) = $$
		$$	\left( m^{\prime}_1, g(l_1 ) \right) + \left( m^{\prime}_2 , g(l_2 ) \right)  = \widetilde{g} \left( m^{\prime}_1, l \right) + \widetilde{g} \left( m^{\prime}_2 , l_2 \right)   $$
		$ $\newline
		
	\item 1-1:

		Αν $(m^{\prime}, l) \mapsto (0_{M^{\prime}} , 0_{L_2}) = (m^{\prime}, g(l)$ τότε $m^{\prime}=0$ και $g(l) = 0 \implies l = 0$ αφού $g$ ισομορφισμός.
$ $\newline

	\item Επί:

		Έστω $(m^{\prime}, l) \in L_2$. Αφού $g$ ισομορφισμός, υπάρχει $l^{\prime} \in L_1$ τέτοιο ώστε $g(l^{\prime}) = l$. Άρα:
		$$\widetilde{g} (m^{\prime}, l^{\prime} ) = (m^{\prime}, g(l^{\prime}) = (m^{\prime}, l)$$


			\end{itemize}
$ $\newline

			Αν $h_1 , h_2 : N \rightarrow L_1 , L_2 $ και $h^{\prime}_1 = (0, h_1) : N \rightarrow L^{\prime}_1, h^{\prime}_2 = (0,h_2) : N \rightarrow L^{\prime}_2$

			\begin{center}
				\begin{tikzcd}
0 \arrow[r] & N \arrow[r, "h^{\prime}_1"] \arrow[d, no head, Rightarrow] & L^{\prime}_1 \arrow[r, "p^{\prime}_1"] \arrow[d, "\widetilde{g}"] & M^{\prime} \arrow[r] \arrow[d, no head, Rightarrow] & 0 \\
0 \arrow[r] & N \arrow[r, "h^{\prime}_2"]                                & L^{\prime}_2 \arrow[r, "p^{\prime}_2"]                            & M^{\prime} \arrow[r]                                & 0 \\
            & n \arrow[dd, no head, Rightarrow] \arrow[r]                & {(0,h_1 (n))} \arrow[d]                                           &                                                     &   \\
            &                                                            & {(0,g(h_1(n))} \arrow[d, no head, Rightarrow]                     & g h_1 = h_2                                         &   \\
            & n \arrow[r]                                                & {(0,h_2 (n))}                                                     &                                                     &   \\
            & {(m^{\prime} , l)} \arrow[r] \arrow[d]                     & m^{\prime} \arrow[d, no head, Rightarrow]                         &                                                     &   \\
            & {(m^{\prime} , g(l))} \arrow[r]                            & m^{\prime}                                                        &                                                     &
\end{tikzcd}
			\end{center}

			άρα και τα δύο τετράγωνα είναι μεταθετικά. Συνεπώς $\widetilde{f}$ καλά ορισμένη απεικόνιση.
$ $\newline




$b)$ Αν έχουμε $I$ εμφυτευτικό και την βραχεία ακριβή ακολουθία:
$$0 \rightarrow N \xrightarrow{i} I \xrightarrow{p} Q \rightarrow 0$$

Τότε:
$$Ext_{R}^1 (M,N) \cong coker[Hom_R (M,I) \xrightarrow{p_*} Hom_R {M,Q}] $$
Συνεπώς αν $g \in Hom_R (M,Q)$ συμβολίζουμε με $[g]$ την κλάση  $g + imp_* $. Θεωρούμε αυτά τα $[g]$ ως στοιχεία του $Ext_R (M,N)$ (μέσω ισομορφισμού).

Πηγαίνοντας το $[g]$ δεξιά και κάτω στο διάγραμμα παίρνουμε:
$$[g] \rightarrow \lambda_{M,N} [g] = [0 \rightarrow N \xrightarrow{i^{\prime}_1 }  X_1 \xrightarrow{p^{\prime}_1} M \rightarrow 0 ] \rightarrow \widetilde{f}(\lambda_{M,N} [g] ) = $$
$$[0 \rightarrow N \xrightarrow{i^{\prime \prime}_1 }   X^{\prime}_1   \xrightarrow{p^{\prime \prime }_1 } M^{\prime} \rightarrow 0 ]  $$

Ομοίως πηγαίνοντάς το κάτω και δεξιά έχουμε:
$$[g] \rightarrow f^* ([g]) = [gf] \rightarrow \lambda_{M^{\prime} , N}[gf] = $$
$$[0 \rightarrow N \xrightarrow{ i^{\prime}_2} X_2 \xrightarrow{p^{\prime}_2 } M^{\prime} \rightarrow 0 ]$$

όπου:
\begin{gather*}
	X_1 = \{ (x,y) \in M \oplus I : \quad g(x) = p(y) \} \\
	X_2 = \{ (x,y) \in M^{\prime} \oplus I : \quad (gf)(x) = p(y) \} \\
	X^{\prime}_1 = \{ (m^{\prime}, x_1) \in M^{\prime} \oplus X_1: \quad f(m^{\prime}) = p^{\prime}_1 (x_1) \in M \}
\end{gather*}
\begin{gather*}
	i^{\prime}_1 (n) = (0,n) \in X_1 \\
	i^{ \prime \prime}_1 (n) = (0 , i^{\prime}_1 (n)) \in X^{\prime}_1 \\
	i^{\prime}_2 (n) = (0,n) \in X_2
\end{gather*}
\begin{gather*}
	p^{\prime}_1( x,y) = x \in M \\
	p^{\prime \prime}_1 (m^{\prime} , x_1) = m^{\prime} \in M^{\prime} \\
	p^{\prime}_2 (m^{\prime} , y ) = m^{\prime} \in M^{\prime}
\end{gather*}
	
Για να είναι το διάγραμμα μεταθετικό, δηλαδή να συμπίπτουν οι δύο κλάσεις πρέπει να υπάρχει ισομορφισμός $h: X^{\prime}_1 \rightarrow X_2$ ώστε το παρακάτω διάγραμμα να είναι μεταθετικό:

\begin{center}
\begin{tikzcd}
0 \arrow[r] & N \arrow[r, "i^{\prime\prime}_1"] \arrow[d, no head, Rightarrow] & X^{\prime}_1 \arrow[d, "h"] \arrow[r, "p^{\prime \prime}_1"] & M^{\prime} \arrow[d, no head, Rightarrow] \arrow[r] & 0 \\
0 \arrow[r] & N \arrow[r, "i^{\prime}_1"]                                      & X_2 \arrow[r, "p^{\prime}_2"]                                & M^{\prime} \arrow[r]                                & 0
\end{tikzcd}
\end{center}

Ορίζουμε την απεικόνιση $h : X^{\prime}_1 \rightarrow X_2$ ως εξής:
$$(m^{\prime},x_1) \longmapsto (m^{\prime}, y)$$
όπου $x_1 = (x,y) \in M \oplus I$. Ο $h$ είναι καλά ορισμένος εφόσον έχουμε:
$$(m^{\prime}, x_1) \in X^{\prime}_1 \implies f(m^{\prime}) = p^{\prime}_1 (x_1) = x \in M$$
$$x_1 = (x,y) \in X_1 \implies g(x) = p(y)$$

και άρα $(gf)(m^{\prime}) = g[f(m^{\prime})] = g(x) = p(y)$ δηλαδή ισχύει ότι $(m^{\prime} , y) \in X_2$.

Στην συνέχεια δείχνουμε ότι είναι ισομορφισμός $R$-προτύπων

\begin{itemize}
	\item $R$-γραμμικότητα:

		$$r(m^{\prime}, x_1) = (rm^{\prime},rx_1) \mapsto (rm^{\prime} , ry) = r (m^{\prime}, y)\quad\text{αφού } rx_1= (rx,ry)$$
		$$(m^{\prime}_1, x^1_1) + (m^{\prime}_2, x^2_1) = (m^{\prime}_1 + m^{\prime}_2 , x^1_1 + x^2_1) \mapsto (m^{\prime}_1 + m^{\prime}_2 , y^1 + y^2) = (m^{\prime}_1, y^1) + (m^{\prime}_2, y^2) $$
		εφόσον $x^1_1 = (x^1, y^1) $ και $x^2_1 = (x^2 , y^2)$.
		$ $\newline
	
	\item 1-1:

		Αν $χ_1=(χ,y)$ και $(m^{\prime}, x_1) \mapsto (0_{M^{\prime}} , 0_I) = (m^{\prime}, y)$ τότε $m^{\prime} = 0, y = 0$ και μένει να δειχτεί ότι $x = 0$. Πράγματι, από την προσθετικότητα της $f$:

		$$0 = f(0) = f(m^{\prime}) = p^{\prime}_1 (x_1) = x$$
		$ $\newline

	\item Επί:

		Έστω $(m^{\prime},y) \in X_2$. Έχουμε ότι $g(f(m^{\prime})) = p(y)$. Από αυτό συνεπάγεται ότι το στοιχείο $(f(m^{\prime}), y)$ ανήκει στο $X_1$. Έτσι έχουμε ότι το στοιχείο $(m^{\prime} , x_1)$ ανήκει στο $X^{\prime}_1$ αφού:
		$$p^{\prime}_1 (x_1) = p^{\prime}_1 (f(m^{\prime}), y) = f(m^{\prime}) $$
		Επιπλέον έχουμε ότι $h(m^{\prime}, x_1) = (m^{\prime}, y)$.
\end{itemize}

Μένει να επαληθευτεί η μεταθετικότητα των δύο τετραγώνων. Πράγματι:

\begin{center}
	\begin{tikzcd}
n \arrow[r, "i^{\prime\prime}_1"] \arrow[d, no head, Rightarrow] & {(0,i^{\prime}_1 (n))} \arrow[r, "h"] & {h(0,i^{\prime}_1 (n)) } \\
n \arrow[r, "i^{\prime}_2"]                                      & {(0,n)}                               &                          \\
{i^{\prime}_1 (n) = (0,n)}                                       & {h(0,i^{\prime}_1 (n)) = (0,n)}       &
\end{tikzcd}
\end{center}
$ $\newline

\begin{center}
\begin{tikzcd}
{(m^{\prime}, x_1 )} \arrow[r, "p^{\prime \prime}_1"] \arrow[d, "h"'] & m^{\prime} \arrow[d, no head, Rightarrow] \\
{(m^{\prime} , y)} \arrow[r, "p^{\prime}_2"]                          & m^{\prime}                                \\
{x_1 = (x,y)}                                                         &
\end{tikzcd}
\end{center}
\end{enumerate}
\end{document}
