Από κάθε $J_1$ αφαιρώ ανοικτό διάστημα με μέσο το μέσο του $J_1$ και μήκος $a \cdot m(J_1)$ δηλαδή μήκος:

$$\frac{\frac{1-a}{2} - a \frac{1-a}{2}}{2} = \frac{(1-a)^2}{2^2} $$

\begin{align*}
m(F_1)  & = 1-a \\
m(F_2)  & = 1-a -a(1-a) = (1-a)^2 \\
& \vdots \\
m(F_n) &  = (1-a)^n
\end{align*}

$ $\newline
Οπότε κάθε διάστημα του $F_n$ έχει μήκος $\frac{(1-a)^n}{2^n}$. Ορίζουμε $$\mathcal{F} = \bigcap\limits_{n=0}^{\infty} F_n$$ το $\mathcal{F}$ θα έχει τις ιδιότητες του $C$.

$ $\newline
2) Γιατί το ίδιο $a$?

$ $\newline
Θεωρώ $a_n \in (0,1), n=1,2,\ldots$. Ξεκινάμε με $F_0$ ως το $[0,1]$ και το σπάμε σε τρία κομμάτια θέτοντας $F_1$ τα δύο ακριανά ξεχνώντας το μεσαίο κομμάτι $a_1$. Έτσι $m(F_1) = 1-a_1$. Επαγωγικά σπάμε το κάθε κομμάτι του $F_n$ σε τρία και διώχνουμε τα μεσαία και αυτό που μένει είναι το $F_{n+1}$. Έτσι $m(F_2) = (1-a_1) - 2 a_2 m(J_1) = (1-a_2)(1-a_1)$ και κάθε υποδιάστημα $J_2$ του $F_2$ έχει μήκος $\frac{(1-a_1)(1-a_2)}{2^2}$.

$ $\newline
Επαγωγικά $$m(F_n) = \prod\limits_{i=1}^n(1-a_i)$$

$ $\newline
Ορίζουμε $\mathcal{F} = \bigcap\limits_{n=0}^{\infty}$, έχει τις ίδιες ιδιότητες εκτός από το μέτρο.

$ $\newline
($F_n$ φθίνουσα, $m(F_0) = 1 < \infty$. Ισχύει $m(F) = \lim_{n\rightarrow \infty} m(F_n) = \lim_{n \rightarrow \infty} \prod (1-a_i) = \prod\limits_{i=1}^{\infty}(1-a_i)$)

$ $\newline
Ισχυρισμός: Για κάθε $B$ στο $(0,1)$ υπάρχουν $a_i \in (0,1)$ τέτοια ώστε το γινόμενο $\prod\limits_{i=1}^{\infty} (1-a_i)$ να τείνει στο $B$. (Άρα για κάθε $B$ στο (0,1) υπάρχει σύνολο τύπου {\eng Cantor} με $m(F) = B$)

$ $\newline
Ισοδύναμα: θέλω $γ >1$ να το γράψω σαν γινόμενο $\prod\limits_{i=1}^{\infty} δ_i, δ_i>1$.

$ $\newline
Θεωρώ $γ_n < γ_{n+1} < \ldots$ και $γ_n$ να τείνει στο $γ$.

$$γ_n = \frac{γ_n}{γ_{n-1}} \cdot \frac{γ_{n-1}}{γ_{n-2}} \cdots \frac{γ_2}{γ_1} \cdot γ_1$$

$ $\newline
Ορίζω $δ_1 = γ_1, δ_i = \frac{γ_i}{γ_{i-1}}$. Τότε $δ_i > 0$ και $δ_1 \cdot δ_2 \cdots δ_n = γ_n$ που τείνει στο $γ$.

$ $\newline
Εδώ $γ = \frac{1}{B}$ και άρα $\frac{1}{B} = \prod δ_i$ δηλαδή $B = \prod \frac{1}{δ_i}$.

$ $\newline
Βρίσκω $a_i$ τέτοια ώστε $\frac{1}{δ_i} = 1-a_i$.


$ $\newline
3) Υπάρχει $E\subseteq [0,1]$ τέτοιο ώστε για κάθε διάστημα $I\subseteq [0,1], 0 < m(E\cap I) < m(I)$. Τα σύνολα $F$ που φτιάξαμε ως τώρα δεν έχουν αυτή την ιδιότητα. Αν $I$ υποδιάστημα κομματιού που αφαίρεσα, τότε $F \cap I = \emptyset$.

\begin{problem} Να κατασκευαστεί $E \subseteq [0,1]$ με την ιδιότητα: για κάθε υποδιάστημα $I$ του $[0,1]$ τέτοιο ώστε $0< m(E\cap I) < m(I)$
\end{problem}

$ $\newline
Γνωστό: για κάθε $0<β<1$ μπορούμε να φτιάξουμε σύνολο τύπου {\eng Cantor} στο $[0,1]$ με μέτρο $1-β$.

$ $\newline
Βήμα 1: Για κάποιο $b_1 \in (0,1)$ φτιάχνω $C_1$ τύπου {\eng Cantor} με $m(C_1) = 1-b_1$. Γράφουμε $D_1 = [0,1]\setminus C_1 = \bigcup\limits_{j=1}^{\infty} I^1_j$, όπου $I^1_j$ ανοικτά και ξένα διαστήματα.

$ $\newline
Βήμα 2: Για κάθε $j$ ορίζω $\overline{I^1_j}$ το κλειστό διάστημα που προκύπτει από το $I^1_j$ αν προσθέσω τα άκρα του και φτιάχνω σύνολο τύπου {\eng Cantor} $C^2_j$ μέσα στο $\overline{I}^1_j$ τέτοιο ώστε $m(C^2) = (1-b_2) m(\overline{I}^1_j)$ για σταθερό $b_2 \in (0,1)$. Ορίζω $C^2 = C^1 \cup \left( \bigcup\limits_{j=1}^{\infty}C^2_j\right)$ (τομή μέτρου 0 άρα σχεδόν ξένα).

$ $\newline
Γράφουμε $D_2 = [0,1]\setminus C_2 = \bigcup\limits_{j=1}{\infty}$ όπου $I^2_j$ ξένα ανοικτά διαστήματα.

$ $\newline
Το $C^2$ είναι {\eng Borel} ($F_{\sigma}$ σύνολο) άρα μετρήσιμο και $m(C^2) = m(C^1) + m(\bigcup_{j=1}{\infty} C^2_j) = (1-b_1) + b_1(1-b_2) = 1-b_1 b_2$.

$ $\newline
Βήμα 3: Γενικά, έχω μια ακολουθία $(b_i)$ στο $(0,1)$ και στο $n$-οστό βήμα έχω φτιάξει σύνολο $F_{\sigma}$ το $C^n$ τέτοιο ώστε:

\begin{enumerate}
    \item $m(C^n) = 1-b_1 b_2 \cdots b_n$
    \item $D^n = [0,1]\setminus C^n = \bigcup\limits_{j=1}{\infty} I^n_j$ ξένα ανοικτά διαστήματα
\end{enumerate}

$ $\newline
Από τα $I^n_j$ προκύπτει $C^{n+1}_j \rightarrow C^{n+1} = C^n \cup \left( \bigcup\limits_{j=1}^{\infty} C^{n+1}_j \right)$. Συνεχίζω επ άπειρον και ορίζω $C = \bigcup\limits_{n=1}^{\infty}C^n$.

$ $\newline
Τότε έχουμε:

\begin{enumerate}
    \item $m(C) =  \lim_{n\rightarrow \infty} m(C^n) = 1 - \prod\limits_{j=1}^{\infty} b_j$. Επιλέγω τα $b_j$ έτσι ώστε $\prod\limits_{j=1}^{\infty} b_j = \frac{1}{2}$. Οπότε $m(C) = \frac{1}{2}$.
    \item Αν $I$ είναι οποιοδήποτε υποδιάστημα του $[0,1]$, τότε υπάρχουν $n$ και $j$ τέτοια ώστε $I^j_n \subseteq I$.
    \item Έστω $I$ υποδιάστημα του $[0,1]$. Θέλω να δείξω ότι $0< m(E\cap I) < m(I)$. Από το προηγούμενο, υπάρχει $I^j_n \subseteq I$. Στο $n+1$ βήμα, μέσα στο $\overline{I^n_j}$ φτιάχνω κάποια $C^{n+1}_j$ με $m(C^{n+1}_j) = (1-b_{n+1}) m(I^n_j)$ και συνεχίζω φτιάχνοντας κάποια $C^{n+2}_j$ μέσα στα $\overline{I}^{n+1}_j \ldots$.
    \noindent Επειδή ουσιαστικά το κομμάτι του $E (=C)$ που περιέχεται στο $I^n_j$ είναι σύνολο όμοιο με το $E$ (με το $\overline{I}^n_j$ στον ρόλο του $[0,1]$) έχουμε $m(E \cap \overline{I}^n_j) = (1-b_{n+1}b_{n+2} \cdots)m(\overline{I}^n_j)$. Δηλαδή $m(E\cap I^n_j) = (1-\frac{1}{2b_1\cdots b_n}) m(I^n_j) < 1$. Και $m(I^n_j\setminus E) = \frac{1}{2b_1\cdots b_n} m(I^n_j)$. Τότε $m(E\cap I) \geq m(E\cap I^n_j) > 0$ και $m(I\setminus E) \geq m(I^n_j\setminus E) > 0$. Άρα $0< m(E\cap I) < m(I)$.
\end{enumerate}

$ $\newline
Απόδειξη του 2): Έστω $I$ υποδιάστημα του $[0,1]$. Υποθέτουμε ότι κανένα $I^n_j$ δεν περιέχεται στο $I$. Για $n=1$: Κανένα $I^1_j$ δεν περιέχεται στο $I$. Υπάρχει $j$ τέτοιο ώστε $I^1_j \cap I \neq \varnothing$. Αλλιώς το $I \subseteq [0,1] \setminus \cup I^1_j = C^1$ άτοπο γιατί το $C^1$ δεν περιέχει διαστήματα.

%εδώ το χάνω λίγο με το σχήμα στα διαστήματα
$ $\newline
Ισχύει ένα από τα τρία: ??. Έστω ότι έχω ??. Με την ίδια λογική υπάρχει ?? που τέμνει το $A$. Το $I^1_k$ είναι ξένο προς το $I^1_j$ και δεν μπορεί να περιέχεται στο $A \subseteq J$. Με την ίδια λογική, υπάρχει $I^1_l$ που τέμνει το $B$ (ξένο προς τα $I^1_j, I^1_k$) άρα περιέχεται στο $B \subseteq I$ άτοπο. Δηλαδή, με τις υποθέσεις μου υπάρχει $I^1_j \supseteq I$.

$ $\newline
Για $n=2$: Περιορίζομαι στο $I^1_j$ και κοιτάω το $C^2_j$ που κατασκευάστηκε στο $\overline{I}^1_j$. Με την ίδια λογική υπάρχει $I^2_j \supseteq I$. Επαγωγικά βρίσκω $I^1_{j1} \supset I^2_{j2} \supset \ldots \supset I$. Αν δείξω ότι $max_j (m(I^n_j))$ τείνει στο $0$ θα έχω άτοπο, γιατί $m(I)$ είναι γνήσια μεγαλύτερο του $0$. Αλλά επιπλέον μπορώ να πετύχω $max_j m(I^1_j) \leq \frac{b_1}{3}$.

$ $\newline
Έστω $\delta \in (0,1)$. Ακολουθούμε την κατασκευή του συνόλου {\eng Cantor}, αλλά στο $n$-οστό βήμα αφαιρούμε διαστήματα μήκους $\delta/3$. Συνολικά αφαιρούμε:

$$\frac{\delta}{3} + 2\frac{\delta}{3^2} + 2^2 \frac{\delta}{3^3} + \ldots + 2^n \frac{\delta}{3^{n+1}} + \ldots = \frac{\delta}{3} \left( 1+ \frac{2}{3} + \left(\frac{2}{3}\right)^2 + \ldots \right) = \frac{\delta}{3} \cdot \frac{1}{1-2/3} = \delta$$

$ $\newline
Δηλαδή το $C$ που προκύπτει έχει μέτρο $1-\delta$. Τότε στο $1$: 

\begin{align*}
\max_j [m(I^1_j)] & = \frac{b_1}{3} \\
\max_j [m(I^2_j)] & = \frac{b_2}{3} \frac{b_1}{3} \\
&  \vdots \\
\max_j [m(I^n_j)] & = \frac{b_1 \cdots b_n}{3^n} < \frac{1}{3^n} \rightarrow 0
\end{align*}

\pagebreak

\section*{Η συνάρτηση {\eng Cantor-Legesgue}}

\noindent Έστω $C$ το κλασικό σύνολο του {\eng Cantor}.

\begin{theorem} Υπάρχει $F: [0,1]\rightarrow [0,1]$ συνεχής. αύξουσα, επί και σταθερή σε κάθε $I_j$, όπου $I_j$ τα ανοιχτά διαστήματα του $[0,1] \setminus C$.
\end{theorem}

$ $\newline
$F(C) = [0,1]$ άρα και $C$ υπεραριθμήσιμο.

\begin{proof} Θα ορίσουμε ακολουθία συναρτήσεων $f_n : [0,1] \rightarrow [0,1]$, $f_n$ συνεχείς, αύξουσες και $f_n \overset{\text{ομ}}{\rightarrow} F$.
    
    $ $\newline
    Βήμα 1: $f_1  \equiv \frac{1}{2}$ στο $[\frac{1}{3}, \frac{2}{3}]$ και γραμμική στα $[0,\frac13), (\frac23,1]$ και $f_1 (0) = 0, f_1(1) = 1$.

    $ $\newline
    Βήμα 2: 
    $$f_2(x) = \begin{cases}
        \frac12, &  x \in [\frac13, \frac23] \\
        \frac14, &  x \in [\frac19, \frac29] \\
        \frac34, &  x \in [\frac79, \frac89] \\
        0, & x=0 \\
        1, & x=1
    \end{cases}$$
    
    $ $\newline
    Γενικά το συμπλήρωμα του $C^n$ αποτελείται από $1+2+2^2 + \ldots + 2^n = 2^n -1 $ ανοικτά διαστήματα. Τα $I^n_1, \ldots, I^n_{2^n-1}$ από αριστερά προς τα δεξιά.

    $ $\newline
    Ορίζω: $f_n(x) = \frac{j}{2^n}$ αν $x \in I^n_j, j = 1,2,\ldots 2^n-1$ και $f_n(0) = 0, f_n(1) = 1$ και επεκτείνω γραμμικά.

    $ $\newline
    Παρατήρηση: Το $[\frac13, \frac23]$ για $n=1 \rightarrow I^1_1$, για $n=2 \rightarrow I^2_2$, για $n=3 \rightarrow I^3_4$, για $n = 4 \rightarrow I^4_8, \ldots n \rightarrow I^n_{2^n-1}$. Γενικά $I^n_j \rightarrow I^{n+1}_{2j}$.

    $ $\newline
    Κάθε $f_n$ είναι αύξουσα και συνεχής (το επί το έχουμε αφού $f_n(0)= 0 $ και $f_n (1)= 1$). Ισχυριζόμαστε ότι $|f_n (x) - f_{n+1}(x) | \leq \frac{1}{2^n}$.
    
    $ $\newline
    Αν $x \in I^n_j$ για κάποιο $j$ τότε $f_n(x) = f_{n+1}(x) = \frac{j}{2^n}$ και άρα $|f_n(x) - f_{n+1}(x)| = 0$.

    $ $\newline
    Έστω ότι $x \in A_j$ τέτοιο ώστε $I^n_j \subseteq A_j \subseteq I^n_{j+1}$. Τότε:

    $$\frac{j}{2^n} \leq f_n(x) \leq \frac{j+1}{2^n}$$ και 

    $$\frac{2j}{2^{n+1}} = \frac{j}{2^n} \leq f_{n+1} (x) \leq \frac{2j + 2}{2^{n+1}} = \frac{j}{2^n} + \frac{1}{2^n}$$ επομένως $|f_n(x) - f_{n+1}(x)| \leq \frac{j+1}{2^n} - \frac{j}{2^n} = \frac{1}{2^n}$. Άρα από κριτήριο {\eng Weierstrass} η σειρά $\sum (f_{n+1} - f_n)$ συγκλίνει ομοιόμορφα στο $[0,1]$, αν και μόνο αν $f_n \overset{\text{ομ}}{\rightarrow} F$ στο $[0,1]$.

\end{proof}

$ $\newline
Αυτή η $F$ θα λέγεται συνάρτηση του {\eng Legesgue}. Έχουμε ότι κάθε $f_n$ είναι αύξουσα και συνεχής, στέλνει το $0$ στο $0$ και το $1$ στο $1$, άρα ισχύουν και τα ίδια για την $F$ μέσω της ομοιόμορφης σύγκλισης.

$ $\newline
Σε κάθε διάστημα του συμπληρώματος του $C$ η $F$ είναι σταθερή, γιατί τελικά η κάθε $f_n$ είναι σταθερή (με την ίδια πάντα τιμή).

$ $\newline
Οι τιμές της $F$ στο $[0,1]\setminus C$ είναι αριθμήσιμες το πλήθος (και μάλιστα πιάνονται στο $C$). Δηλαδή $F(C) = [0,1]$.

$ $\newline
Ειδικότερα, ??$card(C) = card([0,1]) = 2^{\mathbb{N}}$.


\begin{enumerate} 
    \item Κάθε $x \in [0,1]$ έχει τριαδικό ανάπτυγμα δηλαδή γράφεται $x = \sum\limits_{j=1}^{\infty} \frac{a_j}{3^j}$. (Μερικοί $x \in [0,1]$ έχουν δύο αναπτύγματα, οι $\frac{p}{3^m}$.) 
    \item  $x \in C$ αν και μόνο αν $a_j = 0$ ή $2$. Αν
    $$ x = \frac{p}{3^m} = \frac{a_1}{3} + \frac{a_2}{3^2} + \ldots + \frac{a_m}{3^m{}}$$ τότε αν $a_m \neq 1$ κρατάμε αυτό το ανάπτυγμα ενώ αν $a_m = 1$ θεωρούμε το
    $$x = \frac{a_1}{3} + \frac{a_2}{3^2} + \ldots + \frac{a_{m-1}}{3^{m-1}} + \frac{0}{3^m} + \frac{2}{3^m} + \frac{2}{3^{m+2}} + \ldots $$
    άρα έτσι έχω μοναδικό ανάπτυγμα.

    \item 
    $$F\left(\sum\limits_{j=1}^{\infty} \frac{a_j}{3^j}\right) = \sum\limits_{j=1}^{\infty} \left(\frac{a_j}{2}\right) \frac{1}{2^j}$$ για $x \in C$, δηλαδή $a_j = 0$ ή $2$ και έτσι $\frac{a_j}{2} = 0$ ή $1$.
\end{enumerate}

\pagebreak
\section*{Ασκήσεις}
???
%δεν έχει εκφωνήσεις, μάλλον είναι λύσεις από τις αριθμημένες ασκήσεις του Γιαννόπουλου;;;;

$ $\newline
19) $N \subseteq \mathbb{R},\quad  m(N)= 0, \quad  Z = \{x^2: \quad x\in N\}$
\begin{enumerate}
    \item $Z = f(N)$ όπου $f(x) = x^2$.
    \item Μπορώ να υποθέσω ότι $N \subseteq [-A,A]$ για κάποιο $A >0$. Έστω ότι το ξέρω για φραγμένα σύνολα. Ορίζω $N_n = N\cap [-n,n]$. Τότε $N = \bigcup\limits_{n=1}^{\infty}$ και ξέρω ότι $m(f(N_n)) = 0$ γιατί $N_n \subseteq [-n,n]$ και άρα $m(N_n) \leq m(N) = 0$. Τώρα $Z = f(N) = f\left(\bigcup N_n\right) = \bigcup f(N_n) \ldots$.
    \item Η $f(x)$ δεν είναι {\eng Lipschitz} συνεχής στο $[-A,A]$, για κάθε $A>0$ $f(x)- f(y) = |x+y||x-y|\leq 2A|x-y|$, για κάθε $x,y \in [-A,A]$.
    \item Αν $m(N) = 0, N\subseteq[-A,A]$ και $f: [-A,A] \rightarrow \mathbb{R}$ {\eng Lipschitz} συνεχής, τότε $m(f(N)) = 0$.
\end{enumerate}

$ $\newline
16) Θεωρώ δεδομένο το $18$ και τα $N_r$ έχουν την ιδιότητα: $N_r - N_r$ δεν περιέχει διάστημα. (Στο $N - N: x-y = 0$ ή άρρητος.)


\pagebreak

\section*{Μετρήσιμες Συναρτήσεις}

$ $\newline
Έστω $X,Y$ σύνολα και $f:X \rightarrow Y$. Ορίζω $f^{-1}: \mathcal{P}(Y) \rightarrow \mathcal{P}(X)$ με $f^{-1}(E) = \{x \in X: f(x) \in E\}$. Ισχύουν:
\begin{enumerate}
    \item $f^{-1}\left(\bigcup\limits_{i=1}^{\infty} E_i \right) = \bigcup\limits_{i=1}^{\infty} f^{-1}(E_i)$.
    \item $f^{-1}\left(\bigcap\limits_{i=1}^{\infty} E_i \right) = \bigcap\limits_{i=1}^{\infty} f^{-1}(E_i)$.
    \item $f^{-1}(Y\setminus E) = X\setminus f^{-1}(E)$.
\end{enumerate}


για κάθε $E_i, i \in I$ και $E$ στο $\mathcal{P}(Y)$.


\begin{lemma} Αν $\mathcal{N}$ είναι σ-άλγεβρα στο $Y$, τότε η $f^{-1}(\mathcal{N}) = \{f^{-1}(E): \quad E\in \mathcal{N}\}$ είναι σ-άλγεβρα στο $X$.
\end{lemma}

\begin{proof}

$ $\newline
Έστω $A \in f^{-1}(\mathcal{N})$. Τότε $X\setminus A = X\setminus f^{-1}(E), E\in \mathcal{N} = f^{-1}(\underset{\in \mathcal{N}}{Y\setminus E}) \in f^{-1}(\mathcal{N})$.

$ $\newline
Αν $A_i = f^{-1}(E_i), E_i \in \mathcal{N} \quad i=1,2,\ldots$ τότε
$$\bigcup\limits_{i=1}^{\infty} A_i = \bigcup\limits_{i=1}^{\infty} f^{-1}(E_i) = f^{-1} \left(\bigcup\limits_{i=1}^{\infty} E_i \right) \in f^{-1} (\mathcal{N}) $$
\end{proof}



%% τέλος σελίδας 60 %%



%%%%% Proofread χαμος με δείκτες στις παραλλαγές του Cantor
%%%%% Γραμμή 195, εκφωνήσεις ασκήσεων
%%%%% Γραμμή 180, cardinality του συνόλου C είναι το συνεχές και όχι >=
%%%%% Σχήματα στην σελίδα 55, γραμμή 98
