\documentclass{amsbook}
\usepackage[english,greek]{babel}
\usepackage{alphabeta}
\usepackage{amsmath}
\RequirePackage{amsthm}
\RequirePackage{amssymb}
\usepackage[top=2.6cm,bottom=2.6cm,left=2.2cm,right=2.2cm,a4paper]{geometry}
%\usepackage[top=4cm,bottom=4cm,a4paper]{geometry}
\usepackage{graphicx}
\graphicspath{ {./images/} }
\usepackage{xcolor}
%%%%%%%%συντομευσεις%%%%%%%%%%
\newcommand {\lt}{\latintext}
\newcommand {\tl}{\textlatin}
%%%%%%%%%αριθμηση%%%%%%%%%%%%%%
\renewcommand{\theenumi}{\arabic{enumi}}
\renewcommand{\labelenumi}{{\rm(\theenumi)}}
\renewcommand{\labelenumii}{\roman{enumii}) }

\newtheorem*{problem}{Άσκηση}
%%%%%% Document starts %%%%%%%%%%%%



\begin{document}
	%%%%%%%%%%%%%%%%%%%%%%%%%%%%%%
	%\selectlanguage{greek}
	%%%%%%%%%%%%%%%%%%%%%%%%%%%%%%
	
\begin{center}
		\textbf{Πιαθανότητες σε πολλές μεταβλητές 2021-22} 
\end{center}

\vspace{1cm}
Όνομα: Μιχαήλ

Επώνυμο: Αξαρλής

ΑΜ: 

Ημ/ία

Παρητηρήσεις
\newpage

\begin{problem}[1]
Έστω \((X,d,\mu,)\) μετρικός χώρος πιθανότητας και \(\phi :(X,d) \to (Y,\sigma)
\lt{Lipschitz}\) με συνάρτηση με σταθερά 1.\\
Θεωρούμε το μέτρο πιθανότητας \(\mu_{\phi}\) στην \(\mathcal{B}(Y)\) που ορίζεται 
από την 
\[\mu_{\phi}(A) = \mu(\phi^{-1}(A)), \quad A \in \mathcal{B}(Y)\]
Αποδείξτε ότι, για κάθε \(t >0\),
\[\alpha_{\mu_{\phi}}(t) \leq \alpha_{\mu}(t)\]
\end{problem}

\begin{proof}
    Έστω \(t>0\).
    Από τον ορισμό των \(\alpha_{\mu_{\phi}}(t), \alpha_{\mu}(t)\)
    αρκεί για κάθε \(A \in \mathcal{B}(Y)\) με \(\mu_{\phi}(A) = \mu(\phi^{-1}(A)) \geq \frac{1}{2}\)
    να δείξουμε ότι \(1-\mu(\phi^{-1}(A_t)) \leq 1-\mu(\phi^{-1}(A)_t)
    \iff \mu(\phi^{-1}(A)_t) \leq \mu(\phi^{-1}(A_t))\). \\
    Για αυτό αρκεί \(\phi^{-1}(A)_t \subseteq \phi^{-1}(A_t)\)\\
    Αφού \(\phi\) 1-\lt{Lipschitz} έχουμε 
    \[dist(x,\phi^{-1}(A)) \geq dist(\phi(x),\phi(\phi^{-1}(A))) \geq 
    dist(\phi(x),A)\]
    Άρα \(dist(x,\phi^{-1}(A)) <t \Rightarrow dist(\phi(x),A) <t\)\\
    Οπότε 
    \[\phi^{-1}(A)_t = \{x \in X : dist(x,\phi^{-1}(A)) <t\} \subseteq
    \{x \in X: dist(\phi(x),A) <t\} = \{x \in X: \phi(x) \in A_t\} = \phi^{-1}(A_t)
    \]
\end{proof}

\begin{problem}[2]
Έστω $(X,d,\mu )$ μετρικός χώρος πιθανότητας και
έστω $\alpha_{\mu }$ η συνάρτηση συγκέντρωσης του $\mu $. Υποθέτουμε
ότι για κάποιο $\varepsilon \in (0,1)$ και για κάποιο $t>0$ ισχύει
$\alpha_{\mu }(t)<\varepsilon $. Αποδείξτε ότι: αν $A \in \mathcal{B}(X)$ και $\mu (A)\geq \varepsilon $, τότε $1-\mu (A_{t+r})\leq
\alpha_{\mu }(r)$ για κάθε $r>0$.    
\end{problem}
\begin{proof}
Έστω \(r > 0, \ A \in \mathcal{B}(X) \text{ με } \mu(A) \geq \varepsilon\). \\
Αν \(\varepsilon \leq \frac{1}{2}\) τότε 
\[(A_t)_r \subseteq A_{t+r} \Rightarrow \mu((A_t)_r) \leq \mu(A_{t+r})
\Rightarrow 1-\mu(A_{t+r}) \leq 1-\mu((A_t)_r) \leq \alpha_{\mu}(r) \]
Αφού \(1-\mu(A_t) \leq \alpha_{\mu}(t) < \varepsilon \Rightarrow 
\mu(A_t) > 1-\varepsilon \geq \frac{1}{2}\)\\
Αν \(\varepsilon > \frac{1}{2}\) τότε 
\[A_r \subseteq A_{t+r} \Rightarrow 1-\mu(A_{t+r}) \leq 1-\mu(A_r) \leq \alpha_{\mu}(r)\]
Αφού \(\mu(A) \geq \varepsilon > \frac{1}{2}\).\\
Σε κάθε περίπτωση έχεουμε το ζητούσμενο αποτέλεσμα.  
\end{proof}


\end{document}