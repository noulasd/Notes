$46)$ Έστω $ X = Y  = [0,1], \quad \mathcal M = \mathcal N = \mathcal B [0,1]$, $\mu = $ μέτρο {\eng Lebesgue}, $\nu =$ μέτρο απαρίθμησης (δεν είναι σ-πεπερασμένο, το $[0,1]$ δεν γράφεται ως αριθμήσιμη ένωση πεπερασμένων συνόλων γιατί τότε θα ήταν αριθμήσιμο.)

$ $\newline
Έστω $D = \{(x,x), \quad 0\leq x \leq 1\}$ και $f = X_D$. Ισχύει ότι $\int X_D d(\mu \times \nu) = (\mu \times \nu) (D)$?

$$\int\limits_{0}^1 \left( \int\limits_{0}^1 \underset{ = 0 \text{ αν } x\neq y }{X_D (x,y)} d\mu(x) \right) d\nu(y) = \int \limits_0^1 0 d\nu(y) = 0$$

$$\int\limits_{0}^1 \left( \int \limits_0^1 f(x,y) d\nu(y)\right) d\mu(x) = \int\limits_0^1 1 d\mu(x) = 1 \neq 0$$ με $f(x,y) = 0$ αν $y\neq x$ και $1$ αν $x=y$.

$ $\newline
Για το $(\mu \times \nu)(D)$ (μπορεί και να μην είναι μετρήσιμο αλλά μπορώ να βρω σε μέτρο)

% σε μέτρο; σελ 143
$$(\mu \times \nu)(D) = \inf \left\{ \sum\limits_{j=1}^{\infty} \mu (A_j) \nu (B_j), \quad D\subseteq A_j \times B_j \right\}$$

$ $\newline
Έστω $(x,x) \in D$ και $(x,x) \in A_j \times B_j$ δηλαδή $A_j \cap B_j \neq \varnothing$.

$$(\mu \times \nu) (D) = \inf \left\{ \sum\limits_{j=1}^{\infty} \mu (A_j) \nu (B_j), \quad D\subseteq A_j \times B_j, \quad A_j \cap B_j \neq \varnothing \right\} = $$

$$ = \inf \left\{ \sum\limits_{j=1}^{\infty} \mu(A_j) \nu(A_j), \quad D\subseteq A_j \times A_j \right\}$$ γιατί $D\cap ( A_j \times B_j) = D\cap \left( (A_j \cap B_j) \times (A_j \cap B_j) \right)$. Αν $D \subseteq \cup A_j \times A_j$ τότε $[0,1] = \cup_j A_j$, υπάρχει $j_0$ τέτοιο ώστε $\mu (A_{j_0}) \geq 0$ και άρα $\nu (A_{j_0}) = \infty$. Άρα υπάρχει $j$ τέτοιο ώστε $\mu(A_j) \nu(A_j) =\infty$. Άρα $(\mu \times nu)(D) = \infty$.

\pagebreak
\section{Θεώρημα {\eng Fubini}}

%L^1 καλλιγραφικό ή όχι, δηλαδή κλάσεις ισοδυναμίας ή όχι
\begin{theorem} Έστω $(X, \mathcal M, \mu), (Y,\mathcal N, \nu)$ χώροι $\sigma$-πεπερασμένου μέτρου . Έστω $f \in L^1 (\mu \times \nu)$. Τότε:
    \begin{enumerate}
    \item Η $f_x \in L^1(\nu)$ για $\mu$-σχεδόν όλα τα $x \in X$. Η $f^y \in L^1(\mu)$ για $\nu$-σχεδόν όλα τα $y \in Y$.

    \item Οι $g(x) = \int f_x d\nu, \quad h(y) = \int f^y d\mu$ είναι στους $L^1(\mu)$ και $L^1(\nu)$ αντίστοιχα.

    \item $\int f d(\mu \times \nu) = \int\limits_X g(x) d\mu(x) = \int\limits_Y h(y) d\nu(y)$.
    \end{enumerate}
\end{theorem}

\begin{proof} $ $

    $ $\newline
    Γράφουμε $f = f^+ - f^-$. Επειδή $f \in L^1 (\mu \times \nu)$ έχω ότι $\int f^+ d(\mu \times \nu), \int f^- d(\mu \times \nu) < \infty$. Παρατηρώ ότι αν $f,g,h : X\times Y \rightarrow \mathbb R$ και $f = g-h$ τότε $f_x = g_x - h_x$ και $f^y = g^y - h^y$ (προφανές). Άρα $(f^+)_x - (f^-)_x = f_x$ και $(f^+)^y - (f^-)^y = f^y$. Θεωρώ την $f^+ \in L^1 (\mu \times \nu)$. Από το {\eng Tonelli}, οι $(f^+)_x, (f^+)^y$ είναι μετρήσιμες για κάθε $x,y$ και:
    
    $$ \int f^+ d(\mu \times \nu) = \int\limits_X \left( \int\limits_Y \underset{= g_1(x)}{(f^+)_x} d\nu \right) d\mu = \int_Y \left( \int\limits_X \underset{=h_1(y)}{(f^+)^y} d\mu \right)  d\nu < \infty$$

    $ $\newline
    Αφού $\int\limits_X g_1(x) d\mu < \infty$ έπεται ότι $g_1(x) < \infty$ $\mu$-σχεδόν παντού και άρα $(f^+)_x \in L^1 (\nu)$ $\mu$-σχεδόν παντού. Όμοια $(f^+)^y \in L^1 (\mu)$ $\nu$-σχεδόν παντού. Επίσης, $\int g_1 d\mu = \int h_1 d\nu < \infty$ και άρα $g_1 \in L^1(\mu)$ και $h_1 \in L^1(\nu)$. Όμοια $g_2 = \int (f^-)_x \in L^1(\mu)$ και $h_2(y) = \int (f^-)^y \in L^1 (\nu)$. Καθώς και 
    $$\int g_2 d\mu = \int h_2 d\nu = \int f^- d(\mu\times \nu)$$

    $ $\newline
    Αφαιρώντας κατά μέλη, παίρνω ότι 
    $$g_1(x) - g_2(x) = \int \left((f^+)_x - (f^-)_x  \right) d\nu = \int f_x d\nu \in L^1(\mu)$$
    $$h_1(y) - h_2(y) = \int \left((f^+)^y - (f^-)^y \right) d\mu = \int f^y d\mu \in L^1(\nu)$$ καθώς και 

    $$ \int f d(\mu \times \nu) = \int \left(f^+ - f^-\right) d(\mu \times \nu) = \int (g_1 - g_2) d\mu = \int \left(\int f_x d\nu\right) d\mu = $$
    $$ = \int (h_1 - h_2)d\nu  = \int \left( \int f^y d\mu \right) d\nu$$
\end{proof}

\subsection{Τρόπος Εφαρμογής}

$ $\newline
Δίνεται $f :X \times Y \rightarrow \mathbb R$ και θέλω να υπολογίσω το $\int fd(\mu \times \nu)$.
\begin{itemize}
    \item Δείχνω ότι $f \in L^1 (\mu \times \nu)$. Συνήθως δείχνω ότι ισοδύναμα $|f| \in L^1 (\mu \times \nu)$. Από {\eng Tonelli} θα έχουμε:
    $$ \int |f| d(\mu \times \nu) = \int\limits_X \left( \int\limits_Y |f(x,y)| d\nu(y)\right) d\mu(x) < \infty $$

    \item Αν αυτό ισχύει, τότε το θεώρημα {\eng Fubini} μου επιτρέπει να υπολογίσω το $\int f d(\mu \times \nu)$ σαν $\int\limits_X \left(\int\limits_Y f_x \right)$ ή $\int\limits_{Y} \left(\int\limits_X f^y \right)$.
    
\end{itemize}

\pagebreak
\section{Το $n$-διάστατο μέτρο και ολοκήρωμα {\eng Lebesgue}}

$ $\newline
Θεωρώ το μέτρο {\eng Lebesgue} $m$ στο $\mathbb R$ (ορισμένο σε μια κάση $\mathcal L \supset \mathbb B (\mathbb{R}))$.

\begin{definition}
    Θεωρώ τον $(\mathbb{R}, \mathbb{B}(\mathbb{R}),m)$ $n$ φορές ($n \geq 2$). Ορίζεται η $\sigma$-άλγεβρα γινόμενο $ \mathbb{B}(\mathbb{R}) \underset{ n \text{ φορές }}{ \otimes \cdots \otimes} \mathbb{B}(\mathbb{R}) = \mathbb B (\mathbb{R}^n)$ με την Ευκλείδεια μετρική και το μέτρο γινόμενο $m \times \cdots \times m$ στην $\mathbb B (\mathbb{R}^n)$. Το τελικό μέτρο $(m^n)$ θέλουμε να είναι πλήρες. Ορίζουμε $m^n$ να είναι η πλήρωση του $m \times \cdots \times m$. Η πλήρωση ενός $\mu$ ορισμένου στην $\mathcal M$ είναι το $\bar{\mu}$ που ορίζεται στην $\bar{\mathcal M} = \{ E\cup A: \quad E\in \mathcal M \text{ και υπάρχει } F \in \mathcal M \text{ τέτοιο ώστε } A \subseteq F \text{ με } \mu (F) = 0 \}$ από την σχέση $\bar{\mu}(E\cup A) = \mu(A)$.
\end{definition}


\begin{remark}Το γινόμενο δύο πλήρων μέτρων δεν είναι σχεδόν ποτέ πλήρες (άρα κάποια διαδικασία πλήρωσης στον ορισμό του $m^n$ είναι απαραίτητη)
\end{remark}

$ $\newline
Έστω $(X,\mathcal M, \mu), (Y,\mathcal N, \nu)$ πλήρεις χώροι μέτρου και έστω ότι υπάρχει $A \neq \varnothing, A\subseteq X$ με $\mu(A) = 0$ και $\mathcal N subset \mathcal P (Y)$ (δηλαδή υπάρχει $B \subseteq Y$ μη μετρήσιμο, για παράδειγμα $X= Y = \mathbb{R}, \mathcal M = \mathcal N = \mathcal L, \mu = \nu = m$). Τότε $(X\times Y, \mathcal M \otimes \mathcal N, \mu \times \nu)$ δεν είναι πλήρης. Θέλω $E \subseteq X\times Y, E\subseteq F \in \mathcal M \otimes \mathcal N$ με $\mu(F) = 0$ αλλά $E \not\in \mathcal M \otimes \mathcal N$.

$ $\newline
Το $A \times B = E \subseteq A\times Y$ και $\mu \times \nu(A\times Y) = 0$ αφού $A \in \mathcal M, Y \in \mathcal N$ και άρα $A \times Y \in \mathcal M \otimes \mathcal N$ και $\mu \times \nu (A\times Y)  = \mu (A) \nu(Y) = 0$. Επιπλέον, $E \not\in \mathcal M \otimes \mathcal N$ γιατί αν ίσχυε θα είχαμε για κάθε $x \in X$ το $E_x \in \mathcal N$. Αφού $A \neq \varnothing$, για $x \in A$ έχω $E_x = B \in \mathcal N$ άτοπο.

$ $\newline Συμβολίζω με $m = m^n$ το $n$-διάστατο μέτρο {\eng Lebesgue} στον $\mathbb R^n$ και $\mathcal L^n$ την κλάση των {\eng Lebesgue} μετρήσιμων υποσυνόλων του $\mathbb R^n$.

\begin{theorem}
    Έστω $E \in \mathcal L^n$. Τότε:
    \begin{enumerate}
        \item $m(E) = \inf \{ m(U):\quad U \text{ ανοικτό με } U \supseteq E \} = \sup \{ m(K): \quad K \text{ συμπαγές με } K\subseteq E\}$.
        \item $ E = A\cup N_1 = B \setminus N_2$ όπου $m(N_1) = m(N_2) = 0$, $A$ είναι $F_{\sigma}$-σύνολο, $B$ είναι $G_{\delta}$-σύνολο.
        \item Αν $m(E) < \infty$, για κάθε $\varepsilon >0$ υπάρχει πεπερασμένη οικογένεια ${R_1,\ldots, R_N}$ ορθογωνίων που είναι γινόμενα διαστημάτων του $\mathbb{R}$ με $m(E \triangle \bigcup\limits_{i=1}^N) <\infty$.
        
    \end{enumerate}
\end{theorem}

\begin{theorem}
    Αν  $f\in \mathcal{L}^n(m)$, τότε για κάθε $\varepsilon >0$ υπάρχει $\phi = \sum\limits_{j=1}^N a_j X_{R_j}$, $R_j$ γινόμενο διαστημάτων τέτοιο ώστε $\int |f-\phi|dm < \varepsilon$. Επίσης, υπάρχει συνεχής $g$ που μηδενίζεται έξω από ένα φραγμένο σύνολο τέτοιο ώστε $\int |f-g|dm <\varepsilon$.

\end{theorem}

\begin{theorem} Έστω $a \in \mathbb{R}^n$. Ορίζω $T_a (x) = x+a$.
    \begin{enumerate}
        \item Αν $E \in \mathcal{L}^n$, τότε $T_a (E) \in \mathcal{L}^n$ και $m(T_a(E)) = m(E)$.
        \item Αν $f : \mathbb R^n \rightarrow \mathbb{R}$ {\eng Lebesgue} μετρήσιμη, τότε η $(f \circ T_a)(x) = f(x+a)$ είναι {\eng Lebesgue} μετρήσιμη. Αν επιπλέον $f \geq 0$ ή $f \in \mathcal L^1(m)$, τότε $\int f dm = \int (f\circ T_a) dm$.
        
    \end{enumerate}
\end{theorem}

\begin{theorem}
    Έστω $T \in GL(n,\mathbb{R})$ (δηλαδή αντιστρέψιμος γραμμικός μετασχηματισμός). Τότε
    \begin{enumerate}
        \item Αν $E \in \mathcal{L}^n$, τότε $T(E) \in \mathcal L^n$ και $m(T(E)) = |\det T| \cdot m(E)$.
        \item Αν $f: \mathbb{R}^n \rightarrow \mathbb{R}$ {\eng Lebesgue} μετρήσιμη, τότε η $f\circ T$ είναι {\eng Lebesgue} μετρήσιμη και αν $f\geq 0$ ή $f \in \mathcal L^1 (m)$, τότε 
        $$\int f dm = |\det T| \int (f\circ T) dm$$
    \end{enumerate}
\end{theorem}


\begin{proof} $ $

    1) $E = B \cup N$ με $B \in \mathbb B ( \mathbb R^n )$ και $N \subseteq F \in \mathbb B (\mathbb R^n)$ με $m(F) = 0$.
    $$m(B) = \inf \{ \sum m(F_j): \quad B \subseteq \cup F_j, \quad F_j \text{ ορθογώνια }\}$$

    $ $\newline
    Επειδή $N\subseteq F$ και $F \in \mathbb B (\mathbb R^n)$ υπάρχουν $F^{\prime}_j, F \subseteq \cup F^{\prime}_j$ και $\sum m(F^{\prime}_j) < \frac{\varepsilon}{2}$. Παίρνοντας τα $F_j, F^{\prime}_j$ μαζί έχω ορθογώνια $A_j$ τέτοια ώστε $E \subseteq A_j$ και $\sum m(A_j) \leq m(E_j) + \varepsilon$.

    $ $\newline
    Κάθε $A_j = A_{j1} \times A_{j2} \times \ldots \times A_{jn}$ με $A_{ji} \in \mathbb B(\mathbb R)$ για κάθε $i=1,\ldots,n$.

    $ $\newline
    Ξέρω ότι $m(\cup A_j) \leq \sum m(A_j) \leq m(E) + \varepsilon$ και $\cup A_j \supseteq E$.

    $ $\newline
    Για κάθε $j \in \mathbb N$ και $k=1,\ldots,n$ και $\delta >0$ μπορώ να βρω ανοικτό $U_{jk} \supseteq A_{jk}$ τέτοιο ώστε $μ(U_{jk}) \leq m(A_{jk}) + \delta$ (αφού ισχύει στο $\mathbb{R})$, τότε 
    $$\underset{=U_j \text{ ανοικτό }}{ U_{j1} \times \ldots \times U_{jn}} \supseteq A_{j1} \times \ldots \times A_{jn}$$ και $m(U_j) = m(U_{j1})\cdots m(U_{jn}) \leq \prod \left( m(A_{jk}) + \frac{\varepsilon}{2^j}\right) = m(A_j) + \frac{\varepsilon}{2^j}$ αν $\delta$ αρκετά μικρό.

    $ $\newline
    Τώρα αν $U = \cup U_j \supseteq \cup A_j \supseteq E$, $U$ ανοικτό και $m(U) \leq \sum m(U_j) \leq \sum m(A_j) + \sum\frac{\varepsilon}{2^j} \leq m(E) + 2\varepsilon$.


    $ $\newline
    2) Υποθέτω ότι $m(E) <\infty$ και $E$ φραγμένο. Θεωρώ το $\bar E\setminus E$ και βρίσκω από το 1) $U$ ανοικτό τέτοιο ώστε $U \supseteq \bar{E}\setminus E$ και $m(U) \leq m(\bar{E}\setminus E) + \varepsilon$. Ορίζω $K = \bar{E}\setminus U \subseteq E$ και $m(K) \geq m(E)-\varepsilon$.
\end{proof}

\begin{proof} $ $

    $ $\newline
    1) Υποθέτω ότι $f: \mathbb R^n \rightarrow \mathbb R$ {\eng Borel} μετρήσιμη. Η $T$ είναι συνεχής σαν γραμμικός μετασχηματισμός, άρα η $f\circ T$ είναι {\eng Borel} μετρήσιμη.

    $$(f\circ T)^{-1}(B) = T^{-1}\left( f^{-1}(B)\right) \in \mathbb B (\mathbb{R})$$
    
    $ $\newline
    2) Αν η $\int f dm = |detT| \int f\circ T dm$ ισχύει για $T,S \in GL(n,\mathbb R)$ και κάθε {\eng Borel} μετρήσιμη $f \in L^1 (m)$ ή $f\geq 0$ τότε ισχύει και για τον $T\circ S \in GL(,\mathbb R)$
    $$\int f dm = |\det T| \int (f\circ T) dm$$
    $$\int f dm = |\det T||\det S| \int (f\circ T) \circ S dm$$
    $$\int f dm = det(T\circ S) \int f\circ(T\circ S)dm$$

    3) Ορίζω $T_1$ = κλάση γραμμικών απεικονίσεων της μορφής $T_1(x_1,\ldots x_n) = (x_1,\ldots, tx_j,\ldots ,x_n)$ με $\det T_1 = t$, $t\neq 0$. $T_2$ = κλάση των γραμμικών απεικονίσεων της μορφής $T_2(x_1,\ldots,x_n) = (x_1,\ldots,x_j +tx_k,\ldots,x_n)$ με $t \in \mathbb R, k\neq j$ και $|\det T_2| = 1$. $T_3 = $ κλάση των γραμμικών απεικονίσεων της μορφής $T_3 (x_1,\ldots,x_j,\ldots,x_k,\ldots,x_n) = (x_1,\ldots,x_k,\ldots,x_j,\ldots,x_n)$ με $|\det T_3| = 1$.

    $ $\newline
    Ισχύει ότι κάθε $T \in GL(n,\mathbb R)$ γράφεται $T=S_N \circ S_{N-1} \circ \ldots \circ S_1$ όπου $S_i \in T_1$ ή $T_2$ ή $T_3$.

    $ $\newline
    Αρκεί λοιπόν να δείξουμε ότι για κάθε $f: \mathbb R^n \rightarrow \mathbb R$ {\eng Borel} μετρήσιμη και κάθε $T_1,T_2,T_3$ γραμμικό μετασχηματισμό ισχύει ότι
    $$\int f dm = |\det T| \int (f\circ T)dm$$

    $ $\newline
    Για $j=2, T \in T_1$

    $$ \int\limits_{\mathbb R^2} f(x,ty) dm \overset{ Fubini }{=} \int\limits_{\mathbb R}\left( \int\limits_{\mathbb{R}} f(x,ty) dy\right)dx = \int\limits_{\mathbb{R}} \left( \frac{1}{|t|} \int\limits_{\mathbb{R}} f(x,y)dy\right) dx = \frac{1}{|t|} \int fdm $$

    Αρκεί να το δείξω για μια διάσταση. Για $n=2, T\in T_3$

    $$\int\limits_{\mathbb R^2} f(y,x) dm = \int\limits_{\mathbb R^2} f(x,y)dm$$

    $$\int\limits_{\mathbb{R}} \left( \int\limits_{\mathbb{R}} f(y,x) dx\right) dy \text{ ταυτολογία}$$
    %δεν κατάλαβα τι παίζει εδώ.


    4.1) Αν $E$-{\eng Borel}, τότε $T(E)$ θα είναι {\eng Borel}. Αν $f = X_{T(E)}$ και $\int X_{T(E)} (x) dx = |\det T| \int X_{T(E)} (T_x) dx$ τότε $m(T(E)) = |\det T| m(E)$. Ειδικότερα, αν $F$ είναι {\eng Borel} με $m(F) = 0$ τότε $m(T(F)) = 0$, επομένως $E \subseteq F$, $E$ μετρήσιμο, τότε $T(E) \subseteq T(F)$ και άρα $T(E) = T(B) \cup T(N)$. Δηλαδή $m(T(E)) = m(T(B)) + m(T(N)) = |\det B| m(B) + 0 = |\det B| m(E)$.
    
\end{proof}

%γενικά δεν αποδεικνύονται όλα σε αυτό το σημείο

\pagebreak
\section{ Χώροι με νόρμα}

$ $\newline
$X$ γραμμικός χώρος πάνω από το $\mathbb R$. Συμβολισμός: αν $x \in X, \mathbb R_x = \{ tx: t\in \mathbb{R}\}$. Αν $M,N$ υπόχωροι του $X$, τότε:

$$M+N = \{x+y: \quad x\in M, y \in N\}$$

$ $\newline
Νόρμα στον $X$ είναι μια συνάρτηση $||\cdot || : X \rightarrow [0,+\infty)$ τέτοια ώστε:

\begin{enumerate}
    \item $\forall x,y \in X: \quad ||x+y|| \leq ||x|| + ||y||$
    \item $\forall t \in\mathbb R, \forall x \in X: \quad ||tx|| = |t| \cdot ||x||$
    \item $||x|| = 0 \iff x=0$
\end{enumerate}

$ $\newline
Αν ισχύουν μόνο τα 1),2) λέμε ότι έχουμε ημινόρμα. (Από το 2) $x=0 \implies ||x||=0$, αλλα μπορεί να υπάρχουν $x\neq 0$ με $||x|| = 0$.)

$ $\newline
Κάθε νόρμα επάγει μια μετρική στον $X$ ως εξής: $p(x,y) = ||x-y||$. Είναι μετρική, η οποία έχει τις επιπλέον ιδιότητες: $p(x+z,y+z) = p(x,y)$ και $p(tx,ty) = |t| p(x,y)$.

$ $\newline
Αν $B(x,r) = \{y \in X: \quad ||y-x|| < r\}$ τότε $B(x,r) = x + B(0,r)$ και $B(0,tr) = tB(0,r), t>0$.

$ $\newline
Δύο νόρμες $||\cdot||_1$ και $||\cdot||_2$ στον ίδιο γραμμικό χώρο $X$ λέγονται ισοδύναμες, αν υπάρχουν σταθερές $c_1,c_2 >0$ τέτοιες ώστε $c_1 ||x||_1 \leq ||x||2 \leq c_2 ||x||_1$ για κάθε $x \in X$. Τότε οι αντίστοιχες μετρικές είναι ισοδύναμες και έχουμε τα ίδια ανοικτά σύνολα στους δύο μετρήσιμους χώρους και τις ίδιες ακολουθίες {\eng Cauchy}.

\begin{definition}
    Ένας χώρος με νόρμα λέγεται χώρος {\eng Banach}, αν είναι πλήρης ως προς τη μετρική $\rho$, που επάγεται από την νόρμα $||\cdot ||$ (Δηλαδή αν κάθε ακολουθία {\eng Cauchy} $\{x_n\}$ στον $X$ ως προς την $\rho$, συγκλίνει σε κάποιο $x \in X$).
\end{definition}

$ $\newline
Αν $x_n \in X$ λέμε ότι η $\sum x_n$ συγκλίνει στο $x \in X$ αν $\sum\limits_{n=1}^N x_n \overset{ N\rightarrow \infty}{\longrightarrow } x$. Λέμε ότι συγκλίνει απολύτως αν $\sum\limits_{n=1}^{\infty} ||x_n|| < \infty$.

\subsection{ Κριτήριο Πληρότητας}

$ $\newline
Έστω $(X,||\cdot||)$ χώρος με νόρμα. Τότε, ο $X$ είναι πλήρης αν και μόνο αν κάθε απολύτως συγκλίνουσα σειρά στον $X$ συγκλίνει.

\begin{proof} $ $

    $ $\newline
    Έστω $X$ πλήρης και έστω ότι $\sum\limits_{n=1}^{\infty} ||x_n|| < \infty$. Ορίζουμε $S_N = \sum\limits_{n=1}^N x_n$. Αν $N > M$ τότε:
    $$|| S_N - S_M || = || \sum\limits_{n=M+1}^N x_n || \leq \sum\limits_{n = M+1}^N < \varepsilon$$ απο τριγωνική ανισότητα, για τυχόν $\varepsilon$ αν $M,N \geq n_0 (\varepsilon)$. Άρα $\{S_n\}$ {\eng Cauchy} και αφού $X$ πλήρης, έπεται ότι υπάρχει $x \in X: S_N \rightarrow x$, δηλαδή $\sum\limits_{n=1}^{\infty} = x$.

    $ $\newline
    Αντίστροφα, έστω $\{x_n\}$ ακολουθία {\eng Cauchy} στον $X$. Αρκεί να δείξουμε ότι έχει υπακολουθία που συγκλίνει σε κάποιο $x \in X$. Τότε θα έχουμε και $x_n \rightarrow x$ (απόδειξη όπως στο $\mathbb R$). Επειδή η $\{x_n\}$ είναι {\eng Cauchy}, μπορώ να βρω $n_1 < n_2 < \ldots < n_k <\ldots$ τέτοια ώστε:

    $$|| \underset{ = y_k }{x_{n_{k+1}} - x_{n_k}} || \leq \frac1{2^k} \text{ από ορισμό } Cauchy$$

    Έχουμε $\sum\limits_{k=1}^{\infty} < \infty$. Άρα υπάρχει $x$ τέτοιο ώστε $\sum\limits_{k=1}^{\infty} y_κ$ συγκλίνει, δηλαδή $\sum\limits_{k=1}^s \underset{s\rightarrow \infty}{\longrightarrow } x$. Δηλαδή $x_{n_{s+1}}-x_{n_1} \rightarrow x$ και άρα $x_{n_s} \rightarrow x_{n_1} + x = x^{\prime}$.
\end{proof}

$ $\newline
Παράδειγμα: $L^1(\mu) = $ οι ολοκληρώσιμες $f : X \rightarrow \mathbb R$ όπου ταυτίζουμε τις $f,g \in L^1(\mu)$ αν $f=g$ $\mu$-σχεδόν παντού. Ορίζουμε $||f||_1 = \int |f| d\mu$ (φαίνεται εύκολα ότι είναι νόρμα).

$$\rho (f,g) = \int |f-g|d\mu$$ δηλαδή:
$$f_n \overset{L^1(\mu)}{\longrightarrow } f \iff \int |f_n - f| d\mu \rightarrow 0$$

\begin{protash}
    Ο $L^1(\mu)$ είναι πλήρης.
\end{protash}

\begin{proof} $ $

    $ $\newline
    Θα δείξουμε ότι ο $L^1(\mu)$ ικανοποιεί το κριτήριο πληρότητας. Έστω $\{f_n\}$ στον $L^1(\mu)$ τέτοιο ώστε $$\sum\limits_{n=1}^{\infty} ||f_n||_1 = \sum\limits_{n=1}^{\infty} \int |f_n| d\mu < \infty$$ Έπεται ότι η $f = \sum\limits_{n=1}^{\infty} f_n$ ορίζεται $\mu$-σχεδόν παντού και $\int f = \sum\limits_{n=1}^{\infty} \int f_n \quad (f \in L^{1}(\mu))$. Θέλουμε να δείξουμε ότι $||f - \sum\limits_{n=1}^{N} f_n||_1 \overset{N \rightarrow \infty}{ \longrightarrow} 0$. Έχουμε:

    $$|| f - \sum\limits_{n=1}^N f_n|| = ||\sum\limits_{n = N+1}^{\infty} || = \int | \sum\limits_{n = N+1}^{\infty} f_n| d\mu \leq$$
    $$\leq \int \sum\limits_{n= N+1}^{\infty} |f_n| d\mu = \sum\limits_{n = N+1}^{\infty} \int |f_n| d\mu \overset{N \rightarrow \infty}{\longrightarrow} 0$$
\end{proof}

\pagebreak

\section{Φραγμένοι Γραμμικοί Τελεστές}

$ $\newline
Έστω $X,Y$ χώροι με νόρμα. Μια συνάρτηση $T: X \rightarrow Y$ λέγεται γραμμικός τελεστής αν $T(\lambda x + \mu y) = \lambda T(x) + \mu T(y)$ για κάθε $x,y \in X$ και $\lambda,\mu \in \mathbb R$. Ο $T$ λέγεται φραγμένος αν υπάρχει $C \geq 0$ τέτοιος ώστε για κάθε $x  \in X$ να ισχύει $||Tx||_Y \leq C ||x||_X$.

$ $\newline
Έχουμε ότι $||T(\lambda x) || = |\lambda| ||Tx|| \rightarrow \infty$ αν $Tx \neq 0, |\lambda|\rightarrow \infty$. Δηλαδή ο μόνος γραμμικός τελεστής που έχει την ιδιότητα $||Tx|| \leq M$ για κάθε $x \in X$ είναι ο μηδενικός.

$ $\newline
Φραγμένος τελεστής είναι φραγμένη συνάρτηση στα φραγμένα σύνολα.

\begin{protash}
    Έστω γραμμικός τελεστής $T: X \rightarrow Y$ ($X,Y$ χώροι με νόρμα). Τα ακόλουθα είναι ισοδύναμα:
    \begin{enumerate}
    \item ο $T$ είναι συνεχής.
    \item ο $T$ είναι συνεχής στο $0$.
    \item ο $T$ είναι φραγμένος.
    \end{enumerate}
\end{protash}

\begin{proof} $ $

    $ $\newline
    1) $\implies$ 2) είναι προφανές.

    $ $\newline
    2) $\implies$ 3) Για $\varepsilon = 1 >0$ υπάρχει $\delta >0$ τέτοιο ώστε $||x||< \delta$ έτσι ώστε $||Tx|| <1$. Έστω $x \in X, x\neq 0$, τότε:

    $$\left|\left| \frac{\delta}{2} \cdot \frac{x}{||x||} \right|\right| = \frac{\delta}{2||x||} \cdot ||x|| = \frac{\delta}{2} < \delta$$ Άρα 
    $$|| T(\frac{\delta}{2} \frac{x}{||x||})|| <1$$ Δηλαδή $$\frac{\delta}{2 ||x||_X} ||Tx||_Y <1$$ και άρα 
    $$||Tx||_Y < \frac{2}{\delta} ||x||_X$$

    
    3) $\implies$ 1) Θα δείξουμε ότι αν $x_n \rightarrow x$, τότε $Tx_n \rightarrow Tx$. Υπάρχει από υπόθεση $C>0$ τέτοιο ώστε για κάθε $z$
    $$|| Tz || \leq C ||z||$$ Άρα 
    $$||Tx_n - Tx|| = ||T(x_n - x)|| \leq C ||x_n - x|| \rightarrow 0$$ από την υπόθεση που κάναμε.

\end{proof}

$ $\newline
Γράφουμε $L(X,Y)$ για το σύνολο των φραγμένων γραμμικών τελεστών $T: X \rightarrow Y$.

\begin{definition}
    Νόρμα ενός $T \in L(X,Y)$ ορίζεται: 

    $$||T|| = \inf \{ C \geq 0: \quad \forall x \in X \quad ||Tx|| \leq C ||x||\}$$
    $$=\sup \{ ||Tx||: \quad ||x||=1\}$$
    $$=\sup \{||Tx||: \quad ||x||\leq 1\}$$
    $$\sup \{ \frac{||Tx||}{||x||}: \quad x\neq 0\}$$
\end{definition}

\begin{protash}
    Η απεικόνιση $T \mapsto ||T||$ είναι νόρμα στον $L(X,Y)$ (που είναι γραμμικός χώρος). Επιπλέον, $T \in L(X,Y), S \in L(Y,Z)$ τότε $S\circ T \in L(X,Z)$ και $|| S\circ T ||\leq ||S|| \cdot ||T||$. Ειδικότερα, αν $T,S \in L(X,X)$ τότε $S\circ T \in L(X,X)$ και $||S\circ T|| \leq ||S|| \cdot ||T||$. (Ο $L(X,X)$ είναι μια {\eng Banach} άλγεβρα.)

\end{protash}

\begin{theorem}Ο $L(X,Y)$ είναι πλήρης αν και μόνο αν ο $Y$ είναι πλήρης.
\end{theorem}

\begin{proof} Έστω $Y$ πλήρης και έστω $\{T_n\}$ {\eng Cauchy} ακολουθία στον $L(X,Y)$. Δηλαδή $\forall \varepsilon >0$ υπάρχει $n_0 (\varepsilon)$ τέτοιο ώστε για κάθε $N,M \geq n_0$ να ισχύει $||T_N - T_M|| < \varepsilon$. Άρα για κάθε $\varepsilon >0$ υπάρχει $n_0(\varepsilon)$ τέτοιο ώστε για κάθε $N,M\geq n_0$ και για κάθε $x \in X$ να ισχύει:
    $$||T_N x - T_M x || \leq \varepsilon ||x||$$ (Βασική ανισότητα $||Tx|| \leq ||T|| \cdot ||x||$)

    $ $\newline
    Για κάθε $x$ χωριστά ( σταθεροποιώ το $x$) η $\{Tn x\}$ είναι ακολουθία {\eng Cauchy} στον $Y$ από την παραπάνω σχέση. Αφού ο $Y$ είναι πλήρης υπάρχει το $Tx := \lim_{n\rightarrow \infty} T_n (x)$. Έτσι ορίζεται μια απεικόνιση $T:X \rightarrow Y$ η οποία (εύκολα) είναι γραμμικός τελεστής. Μένει να δείξουμε ότι 1) ο $T$ είναι φραγμένος και 2) $||Tn-T||\rightarrow 0$.

    $ $\newline
    Για το 1) σταθεροποιώ το $m$ και το $x$ και αφήνω το $n\rightarrow \infty$. (Η $||\cdot ||$ είναι πάντα συνεχής συνάρτηση)

    $ $\newline
    Τότε $||Tx - Tm(x)|| = \lim_{n\rightarrow \infty}||T_n x - T_m x|| \leq \varepsilon ||x||$. Άρα $||Tx|| \leq ||T_m x|| + \varepsilon ||x|| \leq ||Tm||\cdot ||x|| + \varepsilon ||x|| = (||T_m|| + \varepsilon) ||x||$.

    $ $\newline
    Δηλαδή, για κάθε $\varepsilon >0$ υπάρχει $n_0$ τέτοιο ώστε για κάθε $m\geq n_0$ και για κάθε $x \in X$:
    $$||Tx|| \leq \left( ||T_m|| + \varepsilon\right)||x|| \implies$$
    $$||Tx|| \leq (\liminf_m ||T_m||) ||x||$$ Δηλαδή
    $$||T|| \leq \liminf_m ||Tm||$$


    2) Είδαμε ότι για κάθε $\varepsilon >0$ υπάρχει $n_0$ τέτοιο ώστε για κάθε $m\geq n_0$ και $x \in X$ ισχύει $||Tx-T_m(x) || \leq \varepsilon ||x||$. Δηλαδή $T_m \rightarrow T$ στον $L(X,Y)$ και $||T-T_m|| \leq \varepsilon$.

    $ $\newline
    Το αντίστροφο δεν είναι στα πλαίσια του μαθήματος. %??? να την γράψουμε από αλλού;
\end{proof}

\pagebreak

\section{Δυϊκός Χώρος του $X$}

$ $\newline
Έστω $X$ χώρος με νόρμα. Γραμμικό συναρτησοειδές στον $X$ λέμε κάθε γραμμική απεικόνιση $f: X \rightarrow \mathbb R$. Ο $\mathbb{R}$ είναι χώρος με νόρμα την απόλυτη τιμή $|\cdot |$ και είναι χώρος {\eng Banach}. Ο δυϊκός χώρος του $X$ είναι ο $X^* = L(X,\mathbb R)$, δηλαδή ο χώρος των φραγμένων γραμμικών συναρτησοειδών. Η νόρμα στον $X^*$ είναι η 
$$||f||_* = \sup \{ |f(x)|: \quad ||x|| = 1\}$$ Ο $X^*$ είναι χώρος {\eng Banach} (αφού ο $\mathbb R$ είναι πλήρης)

\begin{definition}
    Έστω $X$ γραμμικός χώρος. Μια $p : X \rightarrow \mathbb R$ λέγεται υπογραμμικό συναρτησοειδές αν:
    \begin{enumerate}
        \item $p(x+y) \leq p(x) + p(y)$ για κάθε $x,y \in X$
        \item $p(\lambda x) = \lambda p(x)$ αν $x \in X, \lambda \geq 0$.
    \end{enumerate}
\end{definition}

$ $\newline
Παράδειγμα: Κάθε υπονόρμα είναι υπογραμμικό συναρτησοειδές.

\begin{theorem}[{\eng Hahn-Banach}]
    Έστω $X$ γραμμικός χώρος πάνω από το $\mathbb R$, $p: X \rightarrow \mathbb R$ υπογραμμικό συναρτησοειδές, $M$ γραμμικός υπόχωρος του $X$ και $f: M \rightarrow \mathbb R$ γραμμικό συναρτησοειδές τέτοιο ώστε για κάθε $x \in M$ έχουμε $f(x) \leq p(x)$. Τότε υπάρχει γραμμικό συναρτησοειδές $F:X \rightarrow \mathbb R$ τέτοιο ώστε $F|_M = f$ και $F(x) \leq p(x)$ για κάθε $x \in X$.
\end{theorem}

\begin{proof} $ $

    $ $\newline
    Έστω $x \in X\setminus M$. Θεωρώ τον $M_1=  M + \mathbb R_x = \{ y + \lambda x: y \in M, \lambda \in \mathbb R\}$. Θα δείξουμε ότι υπάρχει επέκταση $g$ του $f$ στον $M_1$ τέτοια ώστε $g(y+\lambda x) \leq p(y+\lambda x)$ για κάθε $y \in M, \lambda \in \mathbb R$. Επειδή το $g$ θα είναι γραμμικό, θα πρέπει $g(y+\lambda x) = g(y) + \lambda g(x) = f(y) + \lambda \theta$, όπου $\theta = g(x)$.

    $ $\newline
    Παρατήρηση: Αν $y_1,y_2 \in M$, τότε
    $$f(y_1) + f(y_2) = f(y_1+y_2) \leq p(y_1+y_2) = p(y_1+x + y_2 - x) \leq p(y_1 - x) + p(y_2 + x)$$
    $$f(y_1) - p(y_1-x) \leq p(y_2 + x) - f(y_2)$$


    %υπόλοιπη απόδειξη σελ 161
    
\end{proof}