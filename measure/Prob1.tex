\documentclass[a4paper,11pt,draft]{article} %papersize,fontsize,
\usepackage{graphicx} %to include photos
\usepackage{tikz} %draw graphs
\usepackage{tikz-cd} %draw graphs with tikzcd

%%__Languages__%%
\usepackage[utf8]{inputenc} %encoding
\usepackage[english,greek]{babel} %languages
\usepackage{alphabeta} %instead of \alpha use Greek char α

\newcommand{\eng}{\selectlanguage{english}} %to swap
\newcommand{\grk}{\selectlanguage{greek}} %to swap
%\newcommand{\ker}{\operatorname{ker}}
\newcommand{\ann}{\operatorname{ann}}
\newcommand{\im}{\operatorname{im}}
\newcommand{\Spec}{\operatorname{Spec}}
\newcommand{\Ass}{\operatorname{Ass}}
\newcommand{\Ann}{\operatorname{Ann}}
%%_%%



%%___Colorsshortcut__%%
\usepackage{xcolor}

\newcommand{\rd}{\textcolor{red}}
\newcommand{\gr}{\textcolor{green}}
\newcommand{\bl}{\textcolor{blue}}
\newcommand{\pink}{\textcolor{magenta}}
%%__%%


%%__Theorems__%%
\usepackage{amsmath} %align enviroment and small stuff
\usepackage{amssymb} %lots of symbols
\usepackage{amsthm} % theorem structures 

%these are for books or articles with sections
%\newtheorem{thm}{Theorem}[section]
%\newtheorem{cor}[thm]{Corollary}
%\newtheoren{lem}[thm]{Lemma}



\renewcommand\qedsymbol{$\blacksquare$} % \begin{proof} 
%\renewcommand\qedsymbol{\pink{$ \heartsuit$}} % \begin{proof} 
%%_%%


\newtheorem*{lemma}{Λήμμα}
\newtheorem*{porisma}{Πόρισμα}

\newtheorem{problem}{Άσκηση}
\newtheorem*{protash}{Πρόταση}
\theoremstyle{definition}
\newtheorem*{definition}{Ορισμός}
\theoremstyle{plain}
\newtheorem*{theorem}{Θεώρημα}
\theoremstyle{remark}
\newtheorem*{remark}{Παρατήρηση}

%%_improvedsymbols__%%
\usepackage{xfrac} % for sfrac
\let\emptyset\varnothing % nicer empty set
\usepackage{faktor}
%%__%% % all other userpackages

%\title{Μεταθετική άλγεβρα - 6η εργασία}
%\author{Θελξινόη Λουκίδου - 1112201700118}
\date{}

\linespread{1.3} %distance between lines
\usepackage[top=50pt,bottom=50pt,left=55pt,right=55pt]{geometry}
\setlength{\parindent}{0ex}

\begin{document}

\begin{problem}[1]
    Έστω \((X,d,\mu,)\) μετρικός χώρος πιθανότητας και \(\phi :(X,d) \to (Y,\sigma)
    \lt{Lipschitz}\) με συνάρτηση με σταθερά 1.\\
    Θεωρούμε το μέτρο πιθανότητας \(\mu_{\phi}\) στην \(\mathcal{B}(Y)\) που ορίζεται 
    από την 
    \[\mu_{\phi}(A) = \mu(\phi^{-1}(A)), \quad A \in \mathcal{B}(Y)\]
    Αποδείξτε ότι, για κάθε \(t >0\),
    \[\alpha_{\mu_{\phi}}(t) \leq \alpha_{\mu}(t)\]
    \end{problem}
    
    \begin{proof}
        Έστω \(t>0\).
        Από τον ορισμό των \(\alpha_{\mu_{\phi}}(t), \alpha_{\mu}(t)\)
        αρκεί για κάθε \(A \in \mathcal{B}(Y)\) με \(\mu_{\phi}(A) = \mu(\phi^{-1}(A)) \geq \frac{1}{2}\)
        να δείξουμε ότι \(1-\mu(\phi^{-1}(A_t)) \leq 1-\mu(\phi^{-1}(A)_t)
        \iff \mu(\phi^{-1}(A)_t) \leq \mu(\phi^{-1}(A_t))\). \\
        Για αυτό αρκεί \(\phi^{-1}(A)_t \subseteq \phi^{-1}(A_t)\)\\
        Αφού \(\phi\) 1-\lt{Lipschitz} έχουμε 
        \[dist(x,\phi^{-1}(A)) \geq dist(\phi(x),\phi(\phi^{-1}(A))) \geq 
        dist(\phi(x),A)\]
        Άρα \(dist(x,\phi^{-1}(A)) <t \Rightarrow dist(\phi(x),A) <t\)\\
        Οπότε 
        \[\phi^{-1}(A)_t = \{x \in X : dist(x,\phi^{-1}(A)) <t\} \subseteq
        \{x \in X: dist(\phi(x),A) <t\} = \{x \in X: \phi(x) \in A_t\} = \phi^{-1}(A_t)
        \]
    \end{proof}
    
    \begin{problem}[2]
    Έστω $(X,d,\mu )$ μετρικός χώρος πιθανότητας και
    έστω $\alpha_{\mu }$ η συνάρτηση συγκέντρωσης του $\mu $. Υποθέτουμε
    ότι για κάποιο $\varepsilon \in (0,1)$ και για κάποιο $t>0$ ισχύει
    $\alpha_{\mu }(t)<\varepsilon $. Αποδείξτε ότι: αν $A \in \mathcal{B}(X)$ και $\mu (A)\geq \varepsilon $, τότε $1-\mu (A_{t+r})\leq
    \alpha_{\mu }(r)$ για κάθε $r>0$.    
    \end{problem}
    \begin{proof}
    Έστω \(r > 0, \ A \in \mathcal{B}(X) \text{ με } \mu(A) \geq \varepsilon\). \\
    Αν \(\varepsilon \leq \frac{1}{2}\) τότε 
    \[(A_t)_r \subseteq A_{t+r} \Rightarrow \mu((A_t)_r) \leq \mu(A_{t+r})
    \Rightarrow 1-\mu(A_{t+r}) \leq 1-\mu((A_t)_r) \leq \alpha_{\mu}(r) \]
    Αφού \(1-\mu(A_t) \leq \alpha_{\mu}(t) < \varepsilon \Rightarrow 
    \mu(A_t) > 1-\varepsilon \geq \frac{1}{2}\)\\
    Αν \(\varepsilon > \frac{1}{2}\) τότε 
    \[A_r \subseteq A_{t+r} \Rightarrow 1-\mu(A_{t+r}) \leq 1-\mu(A_r) \leq \alpha_{\mu}(r)\]
    Αφού \(\mu(A) \geq \varepsilon > \frac{1}{2}\).\\
    Σε κάθε περίπτωση έχεουμε το ζητούσμενο αποτέλεσμα.  
    \end{proof}





\end{document}