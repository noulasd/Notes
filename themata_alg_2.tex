\documentclass[oneside,a4paper]{article}

%%%%%%%%%%%%%%%%%%%%%%%%%%%%
\usepackage{amsthm}
\usepackage{amsmath}
\usepackage{amssymb}
%%%%%%%%%%%%%%%%%%%%%%%%%%%%%
\usepackage[greek]{babel}
\usepackage[utf8]{inputenc}
\usepackage{mathtools}
\usepackage{blindtext}
\usepackage[T1]{fontenc}
\usepackage{titlesec}
\usepackage{sectsty}
\usepackage{verbatim}
\usepackage{multirow}
\chapternumberfont{\tiny} 
\chaptertitlefont{\Huge}
%ελληνικοι χαρακτηρες σε μαθ pdf utf-8
%%%%%%%%%%%%%%%%%%%%%%%%%%%%%%%%%
\usepackage{tikz-cd}

\usepackage{xcolor}
\usepackage{framed}%frames

\usepackage{array}
\usepackage{pbox}

%%%%%%%%%%%%%%%%%%%%%%%%
\usepackage{tikz}
%%%%%%%%%%%%%%%%%%%%%%%%%%

%%%%%%%%περιθώρια%%%%%%%%%%%%
\usepackage[a4paper,margin=3.5cm]{geometry}


%%%%%%%%συντομευσεις%%%%%%%%%%
\newtheorem{theorem}{Θεώρημα}
\newtheorem{lemma}{Λήμμα}
\newtheorem{example}{Παράδειγμα}
\newtheorem*{defn}{Ορισμός}
\newtheorem{prop}{Πρόταση}
\newtheorem{cor}{Πόρισμα}

\newcommand {\tl}{\textlatin}
%%%%%%%%%αριθμηση%%%%%%%%%%%%%%
\renewcommand{\theenumi}{\arabic{enumi}}
\renewcommand{\labelenumi}{{\rm(\theenumi)}}
\renewcommand{\labelenumii}{\roman{enumii}) }
%%%%%%%%%%%% New theorems %%%%%%%%%%%%%%%%%%%%%%%%

%%%%%%%%%%%%%%%%%%%%%%%%%%%%%%%%%%%%%%%%%%%%%%%%%%%
\newcommand{\Z}{\mathbb{Z}}
\newcommand{\Q}{\mathbb{Q}}
\newcommand{\Co}{\mathbb{C}}
%%%%%%%%%%%%%%%%%%%%% Document starts %%%%%%%%%%%%
\begin{document}
	
	%%%%%%%%%%%%%%%%%%%%%%%%%%%%%%%%%%%%%%%%%%%%%%%%%%
	\selectlanguage{greek}
	%%%%%%%%%%%%%%%%%%%%%%% Start Roman numbering %%%% vbbnn
	%\pagenumbering{roman}
	%%%%%%%%%%%%%%%%%%%%%%%%%%%%%%%%%%%%%%%%%%%%%%%%%%
	
	\begin{framed}	
		%\vspace{0.3truecm}
		\begin{center}
			\huge Θέματα Άλγεβρας και Γεωμετρίας \tl{II}
		\end{center}
		%\vspace{0.3truecm}
		\begin{center}
			\huge $P$-αδικοί αριθμοί και το λήμμα του \tl{Hensel}
		\end{center}
		\vspace{0.3truecm}
		\begin{center}
			Ονομ/νο: Νούλας Δημήτριος\\
			ΑΜ: 1112201800377\\
			\tl{email}: \tl{dimitriosnoulas@gmail.com} \\
			\vspace{0.1cm}
			
		\end{center}
		\vspace{0.3truecm}
	\end{framed}
	\vspace*{\fill}
	\begin{center}
	\includegraphics[width=0.5\textwidth]{C:/Users/dimit/Desktop/TeX/uoa_logo}
	\end{center}
\vspace{1cm}
\pagebreak

να βάλω $I_n$ με το $det$ στην εξίσωση
δακτύλιοι μεταθετικοί με μονάδα!!!
η σχέση με \tl{adjoint} για το τέχνασμα της ορίζουσας ισχύει και για πίνακες $A \in M_n(R)$ για τυχαίο $R$ δακτύλιο μεταθετικό με μονάδα!!!
\noindent Γνώσεις από θεωρία \tl{Galois} και μεταθετική άλγεβρα 



\noindent Έστω $R\subseteq S$ δύο δακτύλιοι και $r \in S$. Θα λέμε ότι το $r$ είναι ακέραιο υπεράνω του $R$ αν υπάρχει μονικό πολυώνυμο με συντελεστές από το $R$ έτσι ώστε $$r^n + a_{n-1}r^{n-1} + \ldots + a_1 r + a_0 = 0$$ Κάθε $r \in R$ είναι ακέραιο υπεράνω του $R$. Aν έχουμε επιπλέον ότι το $R$ είναι σώμα και το $S$ επέκτασή του, τότε το $r \in S$ είναι ακέραιο υπεράνω του $R$ αν και μόνο αν είναι αλγεβρικό υπεράνω του $R$. Ουσιαστικά, θέλουμε να μεταφέρουμε τον ορισμό του αλγεβρικού στοιχείου και σε δακτυλίους και απαιτούμε το πολυώνυμο να είναι μονικό. Αυτό είναι αναγκαίο καθώς και σε πολλά επιχειρήματα στην θεωρία \tl{Galois}, όπως στην μοναδικότητα του αναγώγου πολυωνύμου, πολλαπλασιάζαμε με τον αντίστροφο συντελεστή του μεγιστοβαθμίου. Σε δακτυλίους φυσικά μπορεί ένας συντελεστής να μην έχει αντίστροφο και για αυτό απαιτούμε το πολυώνυμο να είναι μονικό.

\begin{prop}
	Έστω $R \subseteq S$ δακτύλιος και $x \in S$. Τα ακόλουθα είναι ισοδύναμα:
	\begin{enumerate}
		\item Το $x$ είναι ακέραιο υπεράνω του $R$.
		\item Το $R[x]$ είναι πεπερασμένα παραγόμενο $R$-πρότυπο.
		\item Το $R[x]$ περιέχεται σε έναν υποδακτύλιο $C$ του $S$, το οποίο $C$ είναι πεπερασμένα παραγόμενο $R$-πρότυπο.  
	\end{enumerate}
	\vspace*{0.1cm}
	\begin{proof} $ $
		
		$ $\newline
		$(1)\implies (2)$ Αν έχουμε ότι $$x^n + a_{n-1} x^{n-1} + \ldots + a_1 x + a_0 = 0$$ με $a_i \in R$ από την υπόθεση, τότε αρκεί να δείξουμε ότι κάθε δύναμη του $x$ παράγετεαι από τα $1,x,x^2,\ldots,x^{n-1}$. Έτσι θα παράγεται από αυτά και κάθε πολυώνυμο του $R[x]$. Πράγματι, από την παραπάνω σχέση έχουμε ότι για κάθε $r\geq 0$ ισχύει η ισότητα: $$x^{n+r} = - \left(a_{n-1}x^{n+r-1} + \ldots + a_0 x^r \right)$$ και άρα το $R[x]$ παράγεται από τα $1,x,\ldots,x^{n-1}$.

		$ $\newline
		$(2) \implies (3)$ Παίρνουμε $C = R[x]$.

		$ $\newline
		$(3) \implies (1)$ Θα χρησιμοποιήσουμε κάτι που αναφέρεται στην βιβλιογραφία ως το τέχνασμα της ορίζουσας. Έστω ότι $C = (y_1,y_2,\ldots,y_n)$ το πεπερασμένα παραγόμενο $R$-πρότυπο. Έχουμε ότι $R[x] \subseteq C$ και άρα $xy_i \in C$ για κάθε $i=1,\ldots,n$. Έτσι, για κάθε $i=1,\ldots,n$ μπορούμε να γράψουμε ότι: $$xy_i = \sum\limits_{j=1}^n a_{ij}y_j \iff 
		\sum\limits_{j=1}^n \left(\delta_{ij}x - a_{ij}\right)y_j = 0$$ όπου $\delta_{ij}$ είναι το δέλτα του \tl{Kronecker} που παίρνει την τιμή $1$ αν $i=j$ και $0$ διαφορετικά. Αν δούμε τα παραπάνω σε μορφή πίνακα $A$ με στοιχεία $x-a_{ii}$ στην διαγώνιο και $-a_{ij}$ στις άλλες θέσεις, καθώς και $y = (y_1,\ldots,y_n)^T$ θα έχουμε ότι:
		$$A y = 0$$ στην οποία σχέση πολλαπλασιάζουμε από αριστερά με τον προσαρτημένο πίνακα $adj(A)$ και παίρνουμε ότι $$det(A) y = 0$$ Δηλαδή, το $det(A)$ μηδενίζει όλο το $C$ αφού μηδενίζει κάθε $y_i$ και άρα $det(A) \cdot 1_C = 0$. Άρα $det(A)=0$ και αν αναπτύξουμε την ορίζουσα θα έχουμε ένα μονικό πολυώνυμο του $x$ να είναι ίσο με το μηδενικό, εφόσον το $x^n$ θα προκύψει μόνο όταν πολλαπλασιαστούν τα στοιχεία της διαγωνίου και θα έχει συντελεστή $1$ και οι υπόλοιποι συντελεστές θα είναι πράξεις των $a_{ij}$ δηλαδή στοιχεία του $R$.
	\end{proof}
\end{prop}

\begin{cor}
	Αν $x_1,\ldots,x_n \in S$ ακέραια υπεράνω του δακτυλίου $R \subseteq S$, τότε το $R[x_1,\ldots,x_n]$ είναι πεπερασμένα παραγόμενο $R$-πρότυπο.
\end{cor}
\vspace*{0.1cm}
\begin{proof} $ $

	$ $\newline
	Θα το δείξουμε εφαρμόζοντας επαγωγή στο $n$. Για $n=1$ έχουμε την απάντηση από το $(2)$ της προηγούμενης πρότασης. Για $n>1$, έχουμε $$R[x_1,x_2,\ldots,x_n] = A [x_n]$$ όπου $A = R[x_1,\ldots,x_{n-1}]$ και από την επαγωγική υπόθεση, το $A$ είναι πεπερασμένα παραγόμενο $R$-πρότυπο. Χρησιμοποιώντας πάλι το $(2)$ της προηγούμενης πρότασης, το $A[x_n]$ είναι πεπερασμένα παραγόμενο $A$-πρότυπο. Από αυτό έπεται ότι είναι και πεπερασμένα παραγόμενο $R$-πρότυπο από το γνωστό επιχείρημα πύργων, καθώς αν το $A[x_n]$ παράγεται από τα $a_i$ ως $A$-πρότυπο και το $A$ παράγεται από τα $b_j$ ως $R$-πρότυπο, τότε το $A[x_n]$ παράγεται από τα $a_i b_j$ ως $R$-πρότυπο.
\end{proof}

\begin{cor}
	Έστω $R,S$ δακτύλιοι με $R\subseteq S$ και $C$ το σύνολο των $r \in S$ που είναι ακέραια υπεράνω του $R$, τότε το $C$ είναι υποδακτύλιος του $S$ που περιέχει το $R$. 
\end{cor}
\vspace*{0.1cm}
\begin{proof} $ $

	$ $\newline
	Έστω $x,y \in C$. Τότε από το προηγούμενο πόρισμα έπεται ότι το $R[x,y]$ είναι πεπερασμένα παραγόμενο $R$-πρότυπο.  Τα $R[x+y],R[x-y],R[xy]$ περιέχονται στο $R[x,y]$ που είναι πεπερασμένα παραγόμενο $R$-πρότυπο και περιέχεται στο $S$ αφού $x,y \in S$. Άρα από το $(3)$ της πρότασης $1$ τα $x\pm y,xy$ είναι ακέραια υπεράνω του $R$.

\end{proof}
\begin{defn}[Ακέραια Θήκη] 
	Έστω $R,S$ δακτύλιοι με $R \subseteq S$. Τότε ο δακτύλιος $C$ του προηγούμενου πορίσματος ονομάζεται ακέραια θήκη (ή κλειστότητα) του $R$ στο $S$. Επιπλέον, λέμε ότι το $R$ είναι ακέραια κλειστό στο $S$ αν $C=R$.
\end{defn}
%ιδιότητες ακέραιας κλειστότητας αν χρειαστούν παρακάτω είναι τα πορίσματα 5.4 , 5.5 Atiyah Macdonald σελ 60


\begin{defn}[Περιοχή \tl{Dedekind}]
	Μια ακέραια περιοχή $R$ την ονομάζουμε περιοχή του \tl{Dedekind} αν:
	\begin{enumerate}
		\item Κάθε ιδεώδες είναι πεπερασμένα παραγόμενο.
		\item Κάθε πρώτο ιδεώδες είναι και μεγιστικό.
		\item Το $R$ είναι ακέραια κλειστό στο σώμα πηλίκων του. Δηλαδή, αν $K$ είναι το σώμα πηλίκων του και $a/b \in K$ είναι μια ρίζα ενός μονικού πολυωνύμου με συντελεστές από το $R$, τότε $a/b \in R$ και έτσι $b|a$ στο $R$.
	\end{enumerate}
\end{defn}

\begin{defn}[Σώμα Αριθμών]
	Ένα σώμα αριθμών ή αλλιώς αλγεβρικό σώμα αριθμών είναι μια πεπερασμένη επέκταση $K$ του σώματος $\Q$. Ως πεπερασμένη επέκταση, υπενθυμίζουμε ότι από το θεώρημα πρωταρχικού στοιχείου στην θεωρία \tl{Galois} υπάρχει κάποιο $\theta \in K$ έτσι ώστε $K=\Q(\theta)$ εφόσον βρισκόμαστε σε χαρακτηριστική σώματος $0$.
\end{defn}


\noindent Έστω ένα σώμα αριθμών $K=\Q(\theta)$ βαθμού $n = [K:\Q]$. Μπορούμε να δούμε το $K$ μέσω ισομορφισμού ως $\Q[x]/(f)$ όπου $f$ είναι το ελάχιστο πολυώνυμο του $\theta \in K$. Καθώς το $\mathbb{C}$ είναι αλγεβρικά κλειστό και το $f$ είναι ανάγωγο, αυτό έχει $n$ διακεκριμένες μιγαδικές ρίζες. Κάθε τέτοια ρίζα $z_i \in \mathbb{C}$ επάγει έναν ομομορφισμό $\Q[x] \rightarrow \mathbb{C}$ με $g(x) \mapsto g(z_i)$ με πυρήνα το ιδεώδες που παράγεται από το $f$. Έτσι έχουμε $n$ μονομορφισμούς του $K\simeq \Q[x]/(f) \xhookrightarrow{} \mathbb{C}$. Αυτοί είναι και όλοι οι μονομορφισμοί $K\xhookrightarrow{} \mathbb{C}$, εφόσον κάθε τέτοιος μονομορφισμός θα κρατάει σταθερό το $\Q$ και στην ουσία μιλάμε για τα $n$ στοιχεία της ομάδας $Gal(K/\Q)$ με την μόνη διαφορά ότι δεν τα βλέπουμε πλέον σαν $\Q$-αυτομορφισμούς του $K$ και αλλάζουμε το πεδίο τιμών να είναι το $\mathbb{C}$. 

\noindent Με βάση τους παραπάνω μονομορφισμούς, θα ορίσουμε την διακρίνουσα βάσης ενός σώματος αριθμών.

\begin{defn}[Διακρίνουσα Βάσης]
	Έστω $K=\Q(\theta)$ και $a_1,\ldots,a_n$ μια βάση του ως $\Q$-διανυσματικός χώρος. Θεωρούμε τους μονομορφισμούς $\sigma_i : K \xhookrightarrow{} \mathbb{C}$ για $i=1,\ldots,n$. Ορίζουμε τον πίνακα:
	$$(\sigma_i(a_j)) = \begin{pmatrix}
		\sigma_1 (a_1) & \sigma_1 (a_2) & \cdots & \sigma_1 (a_n) \\
		 & \cdots & & \\
		 \sigma_n (a_1) & \sigma_n(a_2) & \cdots & \sigma_n(a_n)
	\end{pmatrix}$$ και η διακρίνουσα της βάσης θα ονομάζεται ο μιγαδικός αριθμός $\Delta(a_1,\ldots,a_n) = (det(\sigma_i(a_j)))^2$.
\end{defn}

\begin{prop}
	Κάθε διακρίνουσα βάσης αριθμητικού σώματος είναι μη μηδενικός ρητός αριθμός.
\end{prop}
\begin{proof}
	Έστω $K=\Q(\theta)$ σώμα αριθμών και $\theta_1,\ldots,\theta_n$ οι μιγαδικές ρίζες του ελαχίστου πολυωνύμου $Irr(\theta,\Q)$. Δεν έχουμε κάποια πολλαπλή ρίζα καθώς τα ανάγωγα πολυώνυμα με συντελεστές υποσώματα του $\mathbb{C}$ έχουν μόνο απλές ρίζες αν χρησιμοποιήσουμε το κριτήριο της παραγώγου και την χαρακτηριστική $0$. Υπενθυμίζουμε ότι μια βάση του $K$ ως $\Q$-διανυσματικού χώρου είναι η $\{1,\theta,\theta^2,\ldots,\theta^{n-1}\}$. Έχουμε την διακρίνουσα αυτής της βάσης να είναι 
	$$\Delta(1,\theta,\ldots,\theta^{n-1}) = det(\theta^j_i)^2, \quad i=1,\ldots,n, \quad j=1,\ldots,n-1$$ και η ορίζουσα $$det(\theta^j_i) = det \begin{pmatrix}
	1 & \theta_1 & \theta^2_1 & \cdots & \theta^{n-1}_1 \\
	 & & \cdots & & \\
	 1 & \theta_n & \theta^2_n & \cdots & \theta^{n-1}_n
	\end{pmatrix}$$ είναι γνωστή ως ορίζουσα \tl{Vandermode}. Άρα $$\Delta(1,\theta,\ldots,\theta^{n-1}) = \prod\limits_{r<s} (\theta_r - \theta_s)^2$$ Το δεξί μέλος είναι συμμετρικό ως προς τα $\theta_i$ και άρα από το θεμελιώδες θεώρημα συμμετρικών πολυωνύμων είναι πολυώνυμο στα στοιχειώδη συμμετρικά πολυώνυμα $e_1 = \theta_1 + \ldots + \theta_n, e_2 = \theta_1 \theta_2 + \ldots + \theta_1 \theta_3 + \ldots + \theta_{n-1}\theta_n,\ldots, e_n = \theta_1 \cdots \theta_n$. Από τουυς τύπους του \tl{Vieta} έχουμε ότι $e_i = \pm p_{n-i}$ όπου $Irr(\theta,\Q) = p_0 + p_1 x + \ldots + p_{n-1}x^{n-1} + x^n$. Οι συντελεστές $p_i$ ανήκουν ήδη στο $\Q$ και άρα $\Delta(1,\theta,\ldots,\theta^{n-1}) \in Q$. Έχουμε ότι $\Delta \neq 0$ εφόσον $\theta_i \neq \theta_s$ αφού δεν έχουμε διπλές ρίζες και άρα δεν συναντάμε το $0$ στο γινόμενο.

	$ $\newline
	Θα δείξουμε τώρα τον ισχυρισμό της πρότασης και για τυχαία βάση του $K$. Έστω $a_1,\ldots,a_n$ τέτοια βάση. Γράφουμε $$a_k = \sum\limits_{j}b_{kj}\theta^j, \quad b_{kj} \in Q, k=1,\ldots,n$$ και έτσι έχουμε $$\Delta(a_1,\ldots,a_n) = det(\sigma_i(a_k))^2 = det\left(\sum\limits_{j} b_{kj}\theta^j\right)^2 = $$
	$$= det(b_{kj})^2 det(\theta^j_i)^2 = det(b_{kj})^2 \Delta(1,\theta,\ldots,\theta^{n-1})$$

	$ $\newline
	Ισχύει ότι $det(b_{kj}) \neq 0$ γιατί αυτός ο πίνακας είναι πίνακας αλλαγής βάσης διανυσματικού χώρου. Έχουμε ότι $\Delta(1,\theta,\ldots,\theta^{n-1})\neq 0$ και άρα $\Delta(a_1,\ldots,a_n) \neq 0$. Τέλος, έχουμε ότι $$det(b_{kj})^2,\Delta(1,\theta,\ldots,\theta^{n-1})\in Q \implies \Delta(a_1,\ldots,a_n) \in \Q$$


\end{proof}


\noindent Αυτή η απόδειξη μας δίνει κάτι επιπλέον, με το επιχείρημα που χρησιμοποιήθηκε στο τέλος της έχουμε ότι για δύο τυχαίες βάσεις $A,A^{\prime}$ του $K$ ως $\Q$-διανυσματικού χώρου, αν $B$ είναι ο πίνακας μετάβασης από την $A$ στην $A^{\prime}$ τότε $$\Delta(A^{\prime}) = (detB)^2 \Delta(A)$$
\begin{defn}[Αλγεβρικοί Ακέραιοι]
	Έστω $F$ ένα σώμα αριθμών. Ο δακτύλιος των αλγεβρικών ακεραίων του $F$ είναι η ακέραια θήκη του $\Z$ στο $F$ και θα συμβολίζεται με $\mathcal{O}_F$.
	(να δω Μαλιάκα)
\end{defn}


\begin{prop}
	Έστω $a \in K$ όπου το $K$ είναι σώμα αριθμών. Τότε το $a$ είναι αλγεβρικός ακέραιος αν και μόνο αν $Irr(a,\Q) \in \Z [x]$, δηλαδή οι συντελεστές του ελαχίστου πολυωνύμου του $a$ υπεράνω του $\Q$ είναι ακέραιοι.
\end{prop}

\begin{proof}
	%9.3.1 Μαλιάκας σελ 166
\end{proof}
\begin{cor}
	Έστω $K$ σώμα αριθμών. Τότε $$\mathcal{O}_K \cap \Q = \Z$$
\end{cor}
\begin{proof}
	%9.3.2
\end{proof}

\begin{prop}
	Έστω $a_1,\ldots,a_n$ βάση του σώματος αριθμών $K$ ως $Q$-διανυσματικού χώρου. Αν $a_1,\ldots,a_n \in \mathcal{O}_K$, τότε $\Delta(a_1,\ldots,a_n) \in \Z$.
\end{prop}
\begin{proof}
	%9.3.3 Μαλιάκας σελ 167
\end{proof}

\begin{prop}
	Έστω $K$ σώμα αριθμών και $a \in K$. Τότε υπάρχει $m\in \Z-\{0\}$ έτσι ώστε $ma \in \mathcal{O}_K$. Ως συνέπεια, $K= \Q(\theta)$ με $\theta \in \mathcal{O}_K$.
\end{prop}

\begin{proof}
	%9.3.4 Μαλιάκας σελ 167
\end{proof}

\begin{theorem} Έστω $K$ ένα σώμα αριθμών βαθμού $n$. Τότε το $\Z$-πρότυπο $\mathcal{O}_K$ είναι ελεύθερο τάξης $n$.
\end{theorem}

\begin{proof}
	%9.3.5 Μαλιάκας σελ 168
\end{proof}

\begin{cor}
	Έστω $K$ ένα σώμα αριθμών. Τότε ο δακτύλιος $\mathcal{O}_K$ των αλγεβρικών ακεραίων του $K$ είναι δακτύλιος της \tl{Noether}.
\end{cor}
\begin{proof}
	%9.3.6 169
\end{proof}


\begin{prop} Έστω $K$ ένα σώμα αριθμών. Τότε το $K$ είναι το σώμα πηλίκων του $\mathcal{O}_K$.
\end{prop}

\begin{proof}
	Έστω $b \in K$. Το $b$ είναι αλγεβρικό υπεράνω του $\Q$ και άρα υπάρχει πολυώνυμο τέτοιο ώστε 
	$$q_n b^n + \ldots + q_1 b + q_0 = 0, \quad q_i \in \Q$$ και κάνοντας απαλοιφή των παρονομαστών παίρνουμε
	$$a_n b^n + \ldots + a_1 b + a_0 = 0, \quad a_i \in \Z$$ την οποία σχέση την πολλαπλασιάζουμε με $a^{n-1}_n$ και έχουμε ένα μονικό πολυώνυμο με ρίζα το $a_n b$.

	$$(a_n b)^n + a_{n-1}(a_n b)^{n-1} + \ldots + a_1 a^{n-2}_{n}(a_n b) + a^{n-1}_n a_0 = 0$$ συνεπώς το $a_n b \in K$ είναι ακέραιο υπεράνω του $\Z$, άρα $a_n b \in \mathcal{O}_K$. Επίσης $a_n \in \Z \subseteq \mathcal{O}_K$ αφού το $\mathcal{O}_K$ είναι η ακέραια θήκη του $\Z$ στο $K$. Άρα έχουμε ότι $$b = \frac{a_n b}{a_n}, \quad a_n b, a_n \in \mathcal{O}_K$$ δηλαδή το $K$ περιέχεται στο σώμα πηλίκων του $\mathcal{O}_K$. Ωστόσο, το σώμα πηλίκων του $\mathcal{O}_K$ είναι το μικρότερο σώμα που το περιέχει. Άρα το σώμα πηλίκων ταυτίζεται με το $K$.
\end{proof}

\begin{prop}
	Έστω $K$ ένα σώμα αριθμών, τότε ο δακτύλιος $\mathcal{O}_K$ είναι ακέραια κλειστός στο σώμα πηλίκων του. Ειδικότερα, ο δακτύλιος $\overline{\Z}$ όλων των αλγεβρικών ακεραίων είναι ακέραια κλειστός στο σώμα πηλίκων του.
\end{prop}
\begin{proof}
	%βίντεο 8.2 μαλιάκα, πρόταση 3 με S-άλγεβρα
	Δείχνουμε ότι το $\overline{Z}$ είναι ακέραια κλειστό στο σώμα πηλίκων του. Έστω $c$ ένα στοιχείο στο σώμα πηλίκων του $\overline{Z}$ που είναι ακέραιο υπεράνω του $\overline{Z}$. Τότε υπάρχει μονικό πολυώνυμο $$f(x) = x^n + a_{n-1}x^{n-1} + \ldots + a_1 x + a_0$$ με $a_i \in \overline{\Z}$ και $f(c) = 0$. Τα $a_i$ ανήκουν και στο σώμα αριθμών $\mathcal{O}_F$ με $F=\Q(a_0,a_1,\ldots,a_{n-1})$ και έχουμε δείξει ότι το $\mathcal{O}_F$ είναι πεπερασμένα παραγόμενο $\Z$-πρότυπο, άρα το $\Z[a_0,\ldots,a_{n-1}]$ είναι πεπερασμένα παραγόμενο $\Z$-πρότυπο. Καθώς $f(c) = 0$, μπορούμε να γράψουμε το $c^n$ ως έναν $\Z[a_0,\ldots,a_{n-1}]$-γραμμικό συνδυασμό των $c^i$ για τα $i<n$. Άρα και ο δακτύλιος $Z[a_0,a_1,\ldots,a_{n-1},c]$ είναι πεπερασμένα παραγόμενος ως $Z$-πρότυπο. Εφόσον ο δακτύλιος $\Z$ είναι της \tl{Noether} ???? και το $\Z$-υποπρότυπο $\Z [c]$ είναι πεπερασμένα παραγόμενο. Άρα από πρόταση ?? το $c$ είναι ακέραιο υπεράνω του $\Z$.

	%(!) not dodgy https://wstein.org/papers/ant/html/node17.html πρέπει να δείξω ότι το σώμα πηλίκων του O_K είναι το ίδιο το K
	%https://math.stackexchange.com/questions/236825/how-to-show-an-algebraic-number-field-is-the-field-of-fractions-of-its-ring-of-i

	ΠΡΟΣΟΧΗ πρέπει να δείξω σώμα πηλίκων αλγεβρικών ακεραίων είναι το $\overline{Q}$??????????????
	$ $\newline
	Τώρα για τυχαίο σώμα αριθμών $K$, έχουμε δείξει ότι το $K$ είναι το σώμα πηλίκων του $\mathcal{O}_K$. Έστω $c \in K$ ακέραιο υπεράνω του $\mathcal{O}_K$. Δηλαδή υπάρχει μονικό πολυώνυμο $f(x) \in \mathcal{O}_K$ με $f(c)=0$. Επιπλέον $\mathcal{O}_K [x] \subseteq \overline{\Z}[x]$ και άρα αν επαναλάβουμε το παραπάνω επιχείρημα για το $\overline{Z}$ έχουμε ότι $c \in \overline{Z}$. Άρα $c \in K\cap \overline{\Z} = \mathcal{O}_K$.
\end{proof}

\begin{theorem} Έστω $K$ ένα σώμα αριθμών. Τότε το $\mathcal{O}_K$ είναι περιοχή του \tl{Dedekind}.
\end{theorem}

\begin{proof}
	Έχουμε ήδη αποδείξει ότι το $\mathcal{O}_K$ είναι δακτύλιος της \tl{Noether} και ότι είναι ακέραια κλειστό στο σώμα πηλίκων του. Μένει να δείξουμε ότι κάθε πρώτο ιδεώδες είναι και μεγιστικό. Έστω $\mathfrak{p}$ ένα μη τετριμμένο πρώτο ιδεώδες του $\mathcal{O}_K$. Από τον ομομορφισμό δακτυλίων $\Z \xhookrightarrow{i} \mathcal{O}_K$ έχουμε ότι το $\mathfrak{p} \cap \Z$ είναι πρώτο ιδεώδες του $\Z$ ως η συστολή ενός πρώτου ιδεωδούς $i^{-1}(\mathfrak{p})$. Επιπλέον, δεν είναι το πρώτο ιδεώδες $(0)$ του $\Z$ καθώς αν $x \in \mathfrak{p}$ τότε το $x$ ως ακέραιο στοιχείο υπεράνω του $\Z$ είναι ρίζα κάποιου μονικού πολυωνύμου ελαχίστου βαθμού, δηλαδή 
	$$x^n + a_{n-1}x^{n-1} + \ldots + a_1 x + a_0 = 0, \quad a_i \in \Z$$ και ο ελάχιστος βαθμός του πολυωνύμου μας εξασφαλίζει ότι $a_0 \neq 0$. Συνεπώς το $a_0$ ανήκει και στο $\mathfrak{p}$ αφού $x \in \mathfrak{p}$ και αυτό γράφεται ως
	$$a_0 = - (x^n + \ldots + a_1 x)$$ Άρα $a_0 \in \mathfrak{p}\cap \Z \neq (0)$. Έτσι, έχουμε ότι το $\mathfrak{p}\cap \Z$ είναι ίσο με το ιδεώδες $(p)$ για κάποιον πρώτο $p$.


	!!!!νδο αυτο το ελαχίστου βαθμού είναι το $Irr(x,\Q)$ με λήμμα \tl{Gauss}!!!!
	
	$ $\newline
	Έχουμε την σύνθεση ομομορφισμών $$\Z \xhookrightarrow{i} \mathcal{O}_K \overset{\pi}{\rightarrow} \mathcal{O}_K / \mathfrak{p}$$ η οποία έχει πυρήνα $ker(\pi \circ i) = \mathfrak{p} \cap \Z = (p)$. Έτσι έχουμε έναν μονομορφισμό δακτυλίων $$\Z / (p) \simeq Im(\pi \circ i ) \xhookrightarrow{} \mathcal{O}_K/\mathfrak{p}$$
	
	$ $\newline
	Έστω τώρα ένα μη μηδενικό $\overline{a} \in \mathcal{O}_K/\mathfrak{p}$. Μπορύμε να θεωρήσουμε το ελάχιστο πολυώνυμο $m(x) = Irr(\overline{a},\Z/(p))$ που θα έχει μη μηδενικό σταθερό όρο $m_0$. Έχουμε ότι $m(\overline{a}) = 0$ και αν πολλαπλασιάσουμε αυτή τη σχέση με $- m^{-1}_0 \in \Z/(p)$ έχουμε ότι $$1 = (\overline{a})\left((\overline{a})^{n-1} + m_{n-1}(\overline{a})^{n-2} + \ldots + m_1\right)(-m^{-1}_0)$$ και άρα το $\overline{a}$ είναι αντιστρέψιμο στοχείο, δηλαδή το $\mathcal{O}_K/\mathfrak{p}$ είναι σώμα. Ισοδύναμα, το $\mathfrak{p}$ είναι μεγιστικό. 
\end{proof}

\begin{theorem}Κάθε ιδεώδες του $\mathcal{O}_F$ έχει μοναδική παραγοντοποίηση σε πρώτα ιδεώδη.
\end{theorem}
\begin{proof}
	υπάρχουν ενδιάμεσες προτάσεις που χρειάζονται
	%https://crypto.stanford.edu/pbc/notes/numberfield/unique.html
	%https://faculty.math.illinois.edu/~r-ash/Ant/AntChapter3.pdf
\end{proof}

\begin{defn}
	Έστω $R$ μια περιοχή και $K$ το σώμα πηλίκων της. Ένα $R$-υποπρότυπο $M$ του $K$ λέγεται κλασματικό ιδεώδες του $R$ αν υπάρχει $x \in R, x\neq 0$ τέτοιο ώστε $xM \subseteq R$. 
\end{defn}
Ουσιαστικά τα ιδεώδη με την συνήθη έννοια είναι κλασματικά ιδεώδη με $x=1$. Οποιοδήποτε στοιχείο $u$ στο σώμα πηλίκων $K$ παράγει ένα κλασματικό ιδεώδες $(u)$ το οποίο το λέμε κύριο κλασματικό ιδεώδες επεκτείνωντας τα συνήθη κύρια ιδεώδη.
%Αν $M$ είναι ένα κλασματικό ιδεώδες τότε το σύνολο των $x \in K$ με $x M \subseteq R$ συμβολίζεται ως ο μεταφορέας $(R:M)$.


Για μια περιοχή $R$, κάθε πεπερασμένα παραγόμενο $R$-υποπρότυπο $M$ του σώματος πηλίκων $K$ του $R$ είναι ένα κλασματικό ιδεώδες
(\tl{Macdonald 96}) μιλάω και για αντιστρέψιμο κλασματικό ιδεώδες

Τώρα για ένα σώμα αριθμών $F$, το $F$ είναι το σώμα πηλίκων του $\mathcal{O}_F$ και έτσι τα κλασματικά ιδεώδη του $F$ είναι τα μη μηδενικά πεπερασμένα παραγόμενα $\mathcal{O}_F$-υποπρότυπα του $F$. Αυτά ορίζουν μια ομάδα $\mathcal{I}_F$ με πράξη το γινόμενο κλασματικών ιδεωδών, όπως έχει οριστεί για τα συνήθη ιδεώδη δηλ \dots και για ένα $\mathfrak{a}$ κλασματικό ιδεώδες του $F$ έχουμε ότι $$\mathfrak{a}^{-1} = \{x \in F: \quad x\mathfrak{a} \subseteq \mathcal{O}_F\}.$$


(απόδειξη/πρόταση) Τα κύρια κλασματικά ιδεώδη του $F$ αποτελούν μια κανονική υποομάδα του $\mathcal{I}_F$ που συμβολίζεται με $\mathcal{P}_F$. Η ομάδα πηλίκο $\mathcal{I}_F / \mathcal{P}_F$ λέγεται \tl{ideal class group???} του $F$.

Αναφορά σε αποτέλεσμα: Αυτή η ομάδα είναι πεπερασμένη για κάθε αριμητικό σώμα $F$ και η τάξη της λέγεται ο αριθμός της κλάσης του $F$.


Έστω μια πεπερασμένη επέκταση $K/F$ αλγεβρικών αριθμητικών σωμάτων. Θεωρούμε το ιδεώδες $\mathfrak{p} \mathcal{O}_K$ όπου το $\mathfrak{p}$ είναι μη τετριμμένο πρώτο ιδεώδες του $\mathcal{O}_F$. Εφόσον έχουμε μοναδική παραγοντοποίηση ιδεωδών στο $\mathcal{O}_K$ τότε έχουμε $$\mathfrak{p} \mathcal{O}_K = \mathfrak{P}^{e_1}_1 \cdots \mathfrak{P}^{e_n}_n$$
όπου τα $\mathfrak{P}_i$ είναι διακεκριμένα πρώτα ιδεώδη του $\mathcal{O}_K$ και $n = n(\mathfrak{p})$ μαζί με $e_i$ θετικοί ακέραιοι. Ονομάζουμε το $e_i$ δείκτη διακλάδωσης (\tl{ramification index}) του $\mathfrak{P}_i/\mathfrak{p}$. Αν η αρχική επέκταση $K/F$ είναι \tl{Galois} τότε η ομάδα \tl{Galois} μεταθέτει τα $\mathfrak{P}_i$ μεταβατικά, (?δρα μεταβατικα?) έτσι ώστε $e_1 = \ldots = e_n = e$.


Καθώς τα πρώτα ιδεώδη είναι μεγιστικά σε μια περιοχή του \tl{Dedekind}, οι δακτύλιοι πηλίκο $\mathcal{O}_K/\mathfrak{P}_j$ και $\mathcal{O}_F/\mathfrak{p}$ είναι σώματα που τα ονομάζουμε σώματα υπολοίπων. Είναι πεπερασμένα σώματα χαρακτηριστικής $p$ με $\mathfrak{p}\cap \Z = p \Z$. Μπορούμε να δούμε το σώμα $\mathcal{O}_F/\mathfrak{p}$ ως υπόσωμα του $\mathcal{O}_K/\mathfrak{P}_i$. Έτσι ο βαθμός επέκτασης των σωμάτων υπολοίπων θα συμβολίζεται:
$$f(\mathfrak{P}_i / \mathfrak{p}) = \left[\mathcal{O}_K/\mathfrak{P}_j : \mathcal{O}_F/\mathfrak{p}\right]$$


(απόδειξη/πρόταση) Αν η επέκταση $K/F$ είναι \tl{Galois} τότε $f(\mathfrak{P}_1 / \mathfrak{p}) = f(\mathfrak{P}_2 / \mathfrak{p}) = \ldots = f(\mathfrak{P}_n / \mathfrak{p}) = f$

Γενικότερα έχουμε $$\sum\limits_{i=1}^n e(\mathfrak{P}_i/\mathfrak{p}) f(\mathfrak{P}_i / \mathfrak{p}) = [K:F]$$ και όταν η επέκταση των σωμάτων αριθμών είναι \tl{Galois} τότε $[K:F] = efn$

Αν $K/F$ είναι μια επέκταση αριθμητικών σωμάτων, θα λέμε ότι το πρώτο ιδεώδες $\mathfrak{p}$ είναι αδιακλάδωτο στην επέκταση $K/F$ αν $e(\mathfrak{P}_i / \mathfrak{p}) = 1$ για κάθε $i$. Θα λέμε επίσης ότι το $\mathfrak{p}$ είναι τελείως διακλαδωμένο στην επέκταση $K/F$ αν υπάρχει μοναδικό πρώτο ιδεώδες $\mathfrak{P}$ πάνω από το $\mathfrak{p}$ και $e(\mathfrak{P}/\mathfrak{p}) = [K:F]$. Επιπλέον, το $\mathfrak{p}$ θα λέμε ότι είναι αδρανή στην επέκταση $K/F$ αν το $\mathfrak{p}\mathcal{O}_K$ είναι πρώτο ιδεώδες του $\mathcal{O}_K$ και επιπλέον το $\mathfrak{p}$ διασπάται πλήρως στην επέκταση αν $n = [K:F]$.
%%%προσοχή στο "πάνω από το p" Εννοούμε με βάση την διαίρεση ιδεωδών και όχι περιέχεσθαι. Ότι είναι πάνω σαν πρώτο ιδεώδες του O_K και το p ως πρώτο ιδεώδες του 


Για δοσμένη επέκταση $K/F$ αριθμητικών σωμάτων και πρώτο ιδεώδες $\mathfrak{p}$ του $\mathcal{O}_F$, ένας τρόπος για να βρούμε παραγοντοποίηση βασίζεται στο ακόλουθο θεώρημα:

\begin{theorem}[\tl{Dedekind-Kummer}]
	Έστω $K/F$ μια επέκταση αριθμητικών σωμάτων και υποθέτουμε ότι $\mathcal{O}_K = \mathcal{O}_F [a]$. Έστω $f(x) = Irr(a,F)$ το ελάχιστο πολυώνυμο του $a$ υπεράνω του $F$ και έστω $\mathfrak{p}$ ένα πρώτο ιδεώδες του $\mathcal{O}_F$. Θέτουμε $\mathbb{F}_{\mathfrak{p}} = \mathcal{O}_F / \mathfrak{p}$ και συμβολίζουμε την εικόνα του $f(x)$ στο $\mathbb{F}_{\mathfrak{p}}$ με $\overline{f(x)}$, δηλαδή παίρνουμε νέο πολυώνυμο από το $f(x)$ με τους συντελεστές του $mod\mathfrak{p}$. Υποθέτουμε ότι στο $\mathbb{F}_{\mathfrak{p}}[x]$ εχουμε παραγοντοποίηση του $\overline{f(x)}$:
	$$\overline{f(x)} = \overline{p_1(x)}^{e_1} \cdots \overline{p_n(x)}^{e_n}$$ όπου τα $\overline{p_i}$ είναι διακεκριμένα ανάγωγα πολυώνυμα του $\mathbb{F}_{\mathfrak{p}}[x]$. Για κάθε $i$, έστω $p_i(x)$ ένα μονικό πολυώνυμο του $\mathcal{O}_F$ που αντιστοιχεί στο $\overline{p_i(x)}$ αν πάρουμε τους συντελεστές του $mod\mathfrak{p}$. Έστω $\mathfrak{P}_i$ το ιδεώδες του $\mathcal{O}_K$ που παράγεται από τα $\mathfrak{p}$ και $p_i(a)$. Τότε $$\mathfrak{p} \mathcal{O}_K = \mathfrak{P}^{e_1}_1 \cdots \mathfrak{P}^{e_n}_n$$ με τα $\mathfrak{P}_i$ να είναι διακεκριμένα πρώτα ιδεώδη του $\mathcal{O}_K$.

	\begin{proof}
		%pdf dedekind kummer proof + example
	\end{proof}
\end{theorem}

!!Διακρίνουσα γενικά για στοιχεία και βάση αριθμητικών σωμάτων από Μαλιάκα και \tl{notes} (και \tl{Trace} αν χρειάζεται)

%Θεώρημα 9.1.2 Μαλιάκα, θ. πρωτεύον στοιχείου.
%9.2 σελ 161 μιλάω για μονομορφισμούς προς το C
% Ορισμός διακρίνουσας number field over Q εξαρτάται από την βάση
%παράδειγμα ίσως;

Όπως ορίσαμε την διακρίνουσα για μια βάση ενός αριθμητικού σώματος υπεράνω του $\Q$ θα επεκτείνουμε την έννοια και για πεπερασμένες επεκτάσεις αριθμητικών σωμάτων. Αν έχουμε δηλαδή μια πεπερασμένη επέκταση $K/F$ αριθμητικών σωμάτων με $\{v_1,\ldots,v_n\}$ μια βάση του $K$ ως $F$-διανυσματικού χώρου, τότε ορίζουμε την διακρίνουσα της συγκεκριμένης βάσης να είναι ο μιαγαδικός αριθμός 
$$\Delta(v_1,v_2,\ldots,v_n) = det[\sigma_i (v_j)]^2$$ όπου αντί για μονομορφισμούς από το $K\rightarrow \mathbb{C}$ που κρατάνε σταθερό το $\Q$ έχουμε $\sigma_1,\ldots , \sigma_n:$ $ K \xhookrightarrow{} F^{\text{\tl{alg}}}$ μονομορφισμούς που κρατάνε σταθερό το $F$
%μπορώ και εδώ C αντί για κλειστότητα του F?? τι διαφορά έχει, καλύτερα safe με κλειστότητα.

Η σχέση για τις διακρίνουσες από διαφορετικές βάσεις του $K$ υπεράνω του $F$ δίνεται με βάση τον πίνακα αλλαγής βάσης μεταξύ τους. Αν $A$ είναι ο $n\times n$ πίνακας αλλαγής βάσης, δηλαδή $(w_1,\ldots,w_n)^T = A (v_1,\ldots,v_n)^T$, τότε $$\Delta(w_1,\ldots,w_n) = (detA)^2 \Delta(v_1,\ldots,v_n)$$
%παρατήρηση 9.2.4 Μαλιάκα αλλιώς marcus number fields


(απόδειξη/ πρόταση/ ορισμός \tl{vandermode})Στην περίπτωση που $K = F(a)$ με $[K:F]=n$, ο πίνακας $[\sigma_i (a^{j-1})]$ (?) είναι πίνακας \tl{Vandermode}, δηλαδή
$$\Delta(1,a,\ldots,a^{n-1}) = \prod\limits_{1\leq i < j \leq n} \left(\sigma_i (a) - \sigma_j (a)\right)^2$$


%%%Norm???? από σημειώσεις εισαγωγικές ή αλλιώς number fields marcus
Πιο συγκεκριμένα, αν $\mathcal{O}_K = \mathcal{O}_F [a]$ και $f(x)$ είναι το έλαχιστο πολυώνυμο του $a$ υπεράνω του $F$ τότε $$N_{K/F} (f^{\prime}(a)) = (-1)^{\frac{n(n-1)}2}\Delta(1,a,\ldots,a^{n-1})$$

Με την σχέση του πίνακα αλλαγής βάσης έχουμε ότι οι διακρίνουσες δεν είναι απαραίτητα ίσες για διαφορετικές βάσεις. Άρα αν θέλουμε να ορίσουμε την διακρίνουσα μιας επέκτασης αριθμητικών σωμάτων $K/F$ τότε πρέπει να το κάνουμε για όλες τις δυνατές βάσεις του $K$ ως $F$-διανυσματικού χώρου. Θα ορίσουμε ένα πρότυπο που τις περιέχει.  

Έστω $M$ ένα μη τετριμμένο $\mathcal{O}_F$-πρότυπο που περιέχει μια $F$-βάση του $K$. Ορίζουμε ως $d(M)$ το $\mathcal{O}_F$-πρότυπο που παράγεται από όλα τα στοιχεία $\Delta (v_1,\ldots,v_n)$ για κάθε $\{v_1,\ldots,v_n\}$ $F$-βάση του $K$ που περιέχεται στο $M$.

(Πρόταση/απόδειξη) Αν το $M$ είναι κλασματικό ιδεώδες του $K$ τότε το $d(M)$ είναι κλασματικό ιδεώδες του $F$.
(Πρόταση/απόδειξη) Αν το $M$ είναι ελεύθερο $\mathcal{O}_F$ πρότυπο, δηλαδή $$M = \bigoplus\limits_{i=1}^n w_i \mathcal{O}_F = \bigoplus\limits_{i=1}^n (w_i)$$ τότε $$d(M) = \Delta (w_1,\ldots,w_n) \mathcal{O}_F = (\Delta (w_1,\ldots,w_n))$$


%https://proofwiki.org/wiki/Definition:Integral_Ideal εδώ λέει integers??? integral ideal τα fractional ideals που το στοιχείο μπροστά περιέχεται στο ιδεώδες??? και άρα xI = I???

(σχετική) διακρίνουσα της επέκτασης $K/F$ είναι $d_{K/F} = d(\mathcal{O}_K)$ όπου θεωρούμε το $\mathcal{O}_K$ ως πεπερασμένα παραγόμενο $\mathcal{O}_F$-πρότυπο.  Έπεται (?) ότι το $d_{K/F}$ είναι ακέραιο ιδεώδες του $\mathcal{O}_F$.



(απόλυτη) διακρίνουσα του $K$ θα είναι $d_K = d_{K/\Q}$

To $\mathcal{O}_K$ είναι ελεύθερο $\Z$-πρότυπο τάξης $n = [K:\Q]$, οπότε το $d_K = d_{K/\Q}$ είναι (κύριο) ιδεώδες του $\Z$ που παράγεται από το $\Delta(v_1,\ldots,v_n)$ όπου $\{v_1,\ldots,v_n\}$ είναι οποιαδήποτε $\Z$-βάση του $\mathcal{O}_K$.





οι διακρίνουσες όπως θα δούμε μεταφέρουν πληροφορία για το ποιοι πρώτοι (εννοεί πρώτα ιδεώδη??? εν τέλει συμπίπτουν με ιδεώδη που παράγονται από πρώτους??) διακλαδώνονται σε μια επέκταση
Πρόταση: Έστω ένα μη τετριμμένο πρώτο ιδεώδες $\mathfrak{p}$ του $\mathcal{O}_F$. Έχουμε ότι το $\mathfrak{p}$ είναι διακλαδωμένο στην επέκταση $K/F$ αν και μόνο αν $\mathfrak{p} | d_{K/F}$.
%τι σημαίνει εδώ η διαίρεση ιδεωδών? σχέση περιέχεσθαι(όχι) ή διαίρεση όπως ορίζεται με τον πολμό αλλά με ιδεώδη?
%πάει με την παραγοντοποίηση σε πρώτους
%https://math.stackexchange.com/questions/2975463/what-does-it-mean-for-an-ideal-in-the-ring-of-integers-to-divide-another-ideal-i
%το ίδιο είναι βασικά και με την κλασική διαίρεση

Εφόσον έχουμε ορίσει την νόρμα για στοιχεία (?) θα ορίσουμε την νόρμα ενός κλασματικού ιδεωδούς.
Έστω $K/F$ μια πεπερασμένη επέκταση αριθμητικών σωμάτων. Έστω $\mathfrak{p}$ ένα πρώτο ιδεώδες του $\mathcal{O}_F$ και $\mathfrak{P}$ ένα πρώτο ιδεώδες του $\mathcal{O}_K$ που διαιρεί το $\mathfrak{p} \mathcal{O}_K$.
Ορίζουμε την νόρμα του $\mathfrak{P}$ να είναι
$$N_{K/F}(\mathfrak{P}) = \mathfrak{p}^{f(\mathfrak{P}/\mathfrak{p})}$$

και επεκτείνουμε την έννοια σε τυχαία κλασματικά ιδεώδη του $K$ ως εξής:
$$N_{K/F}(\mathfrak{P}^{a_1}_1 \mathfrak{P}^{a_2}_2 \cdots \mathfrak{P}^{a_t}_t) = N_{K/F}(\mathfrak{P}_1)^{a_1} N_{K/F}(\mathfrak{P}_2)^{a_2} \cdots N_{K/F}(\mathfrak{P}_t)^{a_t}$$
%%%Ερώτηση: μπορώ να πω το πρώτο ως N_{K/F}(\mathfrak{p}\mathcal{O}_K) για κάποιο \mathfrak{p} πρώτο ιδεώδες του \mathcal{O}_F??????


%%%%Σχετικά με norm και trace για στοιχεία, μιλάω για τους μονομορφισμούς στο C ενός number field και τα ορίζω με βάση αυτούς. Για τους μονομορφισμούς παίρνω από Μαλιάκα και μετά πάω με Marcus
%https://en.wikipedia.org/wiki/Field_norm 

%%% https://en.wikipedia.org/wiki/Ideal_norm
%%% το ideal norm είναι η γενίκευση που θέλω του Field Norm

Έτσι η νόρμα ενός κλασματικού ιδεωδούς του $K$ είναι ένα κλασματικό ιδεώδες του $F$.


(πρόταση) Αν τώρα η επέκταση $K/F$ είναι \tl{Galois} τότε $$N_{K/F} (\mathfrak{U}) \mathcal{O}_K =\prod\limits_{\sigma \in Gal(K/F)} \sigma( \mathfrak{U})$$
%ΤΟ U είναι fractional ideal του K?
και έστω $a \in K$, τότε $N_{K/F}(a \mathcal{O}_K) = N_{K/F}(a)\mathcal{O}_F$ όπου $N_{K/F}(a)$ είναι η συνήθης νόρμα του στοιχείου $a$.

Επιπλέον, αν $F\subseteq E \subseteq K$ είναι σώματα αριθμών, τότε (Πρόταση)
$N_{K/F} = N_{E/F} \circ N_{K/E}$

Όταν $F = \Q$ τότε $N_{K/\Q} (\mathfrak{U}) = a \Z$ για κάποιο $a \in \Q$. 


Έστω μια \tl{Galois} επέκταση $K/F$ σωμάτων αριθμών με ομάδα \tl{Galois} $G$, έστω ένα μη τετριμμένο πρώτο ιδεώδες $\mathfrak{p}$ του $\mathcal{O}_F$ και ένα πρώτο ιδεώδες $\mathfrak{P}$ του $\mathcal{O}_K$ με $\mathfrak{P}|\mathfrak{p}\mathcal{O}_K$. Ορίζουμε την ομάδα διάσπασης:
%decomposition group, υπάρχει σε alg number theory
$$Z(\mathfrak{P}/\mathfrak{p}) = \{\sigma \in G: \sigma(\mathfrak{P}) = \mathfrak{P}\}$$


Η $Z(\mathfrak{P}/\mathfrak{p})$ δρα στο πεπερασμένο σώμα $\mathbb{F} = \mathcal{O}_K/\mathfrak{P}$ και σταθεροποιεί κατά σημείο το $\mathcal{O}_F/\mathfrak{p}$, άρα υπάρχχει ένας φυσικός ομομορφισμός ομάδων:
$$Z(\mathfrak{P}/\mathfrak{p}) \longrightarrow Gal(\mathbb{F}_{\mathfrak{P}}/\mathbb{F}_{\mathfrak{p}})$$

%από αλγ θεωρία αριθμών.
\begin{theorem} Έστω $K/F$ μια επέκταση \tl{Galois} σωμάτων αριθμών με \tl{Galois} ομάδα $G$. Έστω $\mathfrak{p}$ ένα μη τετριμμένο ιδεώδες του $\mathcal{O}_F$. τότε
	\begin{enumerate}
		\item Η $G$ δρα μεταβατικά στο σύνολο των πρώτων ιδεωδών $\mathfrak{P}$ του $\mathcal{O}_K$ τα οποία διαιρούν το $\mathfrak{p}\mathcal{O}_K$ και έτσι $$[G:Z(\mathfrak{P}/\mathfrak{p})] = \#\{\text{ πρώτα ιδεώδη } \mathfrak{P} \text{ του } \mathcal{O}_K: \quad \mathfrak{P}|\mathfrak{p}\mathcal{O}_K\} = n = n(\mathfrak{p})$$ όπου $n$ είναι το πλήθος των διακεκριμένων πρώτων $\mathfrak{P}$ που διαιρούν το $\mathfrak{p}\mathcal{O}_K$. Επιπλέον, αν $\mathfrak{P},\mathfrak{P}^{\prime}$ είναι πρώτα ιδεώδη του $\mathcal{O}_K$ που διαιρούν το $\mathfrak{p}\mathcal{O}_K$, τότε οι ομάδες $Z(\mathfrak{P}/\mathfrak{p})$ και $Z(\mathfrak{P}/\mathfrak{p})$ είναι $G$-συγυζείς.
		\item $N_{K/\Q}(\mathfrak{p}) = |\mathbb{F}_{\mathfrak{p}}|, N_{K/\Q}(\mathfrak{P}) = |\mathbb{F}_{\mathfrak{P}}|$ και η ομάδα $Gal(\mathbb{F}_{\mathfrak{P}}/ \mathbb{F}_{\mathfrak{p}})$ είναι κυκλική, παράγεται από τον αυτομορφισμό του \tl{Frobenius} $\phi_{\mathfrak{p}} : x \mapsto x^{N_{K/Q}(\mathfrak{p})}$.
		\item Ο ομομορφισμός ομάδων $Z(\mathfrak{P}/\mathfrak{p}) \rightarrow Gal(\mathbb{F}_{\mathfrak{P}}/ \mathbb{F}_{\mathfrak{p}})$ είναι επιμορφισμός. Ο πυρήνας του θα ονομάζεται υποομάδα αδράνειας και θα συμβολίζεται με $T(\mathfrak{P}/\mathfrak{p})$. Ετσι θα έχουμε ότι $[Z(\mathfrak{P}/\mathfrak{p}): T(\mathfrak{P}/\mathfrak{p})] = f$ και ότι η ομάδα $T(\mathfrak{P}/\mathfrak{p})$ έχει τάξη $e$. 
	\end{enumerate}
\end{theorem}

\begin{proof}
	%alg nt
\end{proof}

Έστω ότι η επέκταση σωμάτων αριθμών $K/F$ είναι \tl{Galois}. Τότε μέσω της αντιστοιχίας \tl{Galois} οι ομάδες διάσπασης και αδράνειας αντιστοιχούν σε ενδιάμεσα σώματα της επέκτασης τα οποία ονομάζουμε σώμα διάσπασης και σώμα αδράνειας αντίστοιχα. Έστω $K_Z$ και $K_T$ να είναι τα σταθερά σώματα των $Z(\mathfrak{P}/\mathfrak{p})$ και $T(\mathfrak{P}/\mathfrak{p})$ αντίστοιχα. Για μια αβελιανή επέκταση, η παραγοντοποίηση των ιδεωδών που παράγονται από το $\mathfrak{p}$ σε αυτά τα ενδιάμεσα σώματα δίνεται από το ακόλουθο θεώρημα:

\begin{theorem}[\tl{Layer}]
	Έστω $\mathfrak{p}$ ένα μη τετριμμένο πρώτο ιδεώδες του $\mathcal{O}_F$, όπου $K/F$ είναι μια αβελιανή επέκταση σωμάτων αριθμών. Τότε το $\mathfrak{p}$ διασπάται πλήρως στην επέκταση $K_Z / F$. Τα πρώτα ιδεώδη (?)πάνω(?) από το $\mathfrak{p}$ παραμένουν αδρανή στην επέκταση $K_Z / K_T$ και διακλαδώνονται πλήρως στην επέκταση $K/K_T$.

	%%τι εννοεί above \mathfrak{p} ??? γίνεται να υπάρχουν πρώτα ιδεώδη που περιέχουν το p αν το O_F είναι dedekind και άρα κάθε πρώτο μεγιστικό? μήπως εννοεί τα πρώτα \mathfrak{P} στο O_K που διαρούν το pO_K????
	\end{theorem}
\begin{proof}
	%alg nt
\end{proof}

Αν τώρα $e(\mathfrak{P}/\mathfrak{p}) = 1$ τότε από τον φυσικό ομομορφισμό του $(3)$ του θεωρήματος \tl{ref}? έχουμε τον ισομορφισμό $Z(\mathfrak{P}/\mathfrak{p}) \simeq Gal(\mathbb{F}_{\mathfrak{P}}/ \mathbb{F}_{\mathfrak{p}})$ κυκλικών ομάδων τάξης $f = f(\mathfrak{P}/\mathfrak{p})$. Η ομάδα \tl{Galois} των σωμάτων υπολοίπων παράγεται από τον αυτομορφισμό του \tl{Frobenius} $\phi_{\mathfrak{p}}$ και έτσι υπάρχει μοναδικό $\sigma \in Z(\mathfrak{P}/\mathfrak{p})$ που αντιστοιχεί στο $\phi_{\mathfrak{p}}$ κάτω από τον φυσικό ισομορφισμό. Έχουμε $Z(\mathfrak{P}/\mathfrak{p}) = <\sigma >$. Το στοιχείο $\sigma$ λέγεται το στοιχείο του \tl{Frobenius} στο $\mathfrak{B}$. Το συμβολίζουμε με $$\sigma = \left(\frac{\mathfrak{P}}{K/F}\right) = \left(\mathfrak{P}, K/F \right)$$

\begin{prop}
	Έστω $K/F$ μια επέκταση \tl{Galois} σωμάτων αριθμών, $\mathfrak{p}$ ένα μη τετριμμένο πρώτο ιδεώδες που είναι αδιακλάδωτο στην επέκταση $K/F$ και $\mathfrak{P}$ ένα πρώτο ιδεώδες του $\mathcal{O}_K$ με $\mathfrak{P}|\mathfrak{p}\mathcal{O}_K$. Τότε το στοιχείο του \tl{Frobenius} στο $\mathfrak{P}$ είναι το μοναδικό στοιχείο $\sigma \in Gal(K/F)$ που ικανοποιεί την σχέση $$\sigma (a) = a^{N_{K/Q}(\mathfrak{p})} \quad (mod \mathfrak{P}) $$ για κάθε $a \in \mathcal{O}_K$.
	 \end{prop}
\begin{proof}
	%class field theory
\end{proof}


Αν υποθέσουμε επιπλέον ότι η ομάδα \tl{Galois} της επέκτασης είναι αβελιανή, τότε από το $(1)$ του θεωρήματος \tl{ref}? παίρνουμε ότι η ομάδα $Z(\mathfrak{P}/\mathfrak{p})$ εξαρτάται μόνο το $\mathfrak{p}$ και έτσι την γράφουμε ως $Z(\mathfrak{p})$. Αν το $\mathfrak{p}$ είναι αδιακλάδωτο στην επέκταση $K/F$ τότε με βάση την πρόταση που ακολουθεί, δείχνουμε ότι το στοιχείο του \tl{Frobenius} στο $\mathfrak{P}$ εξαρτάται μόνο από το $\mathfrak{p}$. Σε αυτή τη περίπτωση, ονομάζουμε το στοιχείο ως τον αυτομορφισμό του \tl{Artin} για το $\mathfrak{p}$ και το συμβολίζουμε με $$\left(\frac{\mathfrak{p}}{K/F}\right) = (\mathfrak{p},K/F)$$ Επιπλέον, ορίζουμε την απεικόνιση 
$$\{\text{ πρώτα ιδεώδη του } \mathcal{O}_F \text{ που είναι αδιακλάδωτα στην επέκταση } K/F\} \longrightarrow G$$
$$\mathfrak{p} \longmapsto \sigma_{\mathfrak{p}} = \left(\frac{\mathfrak{p}}{K/F}\right)$$




\begin{prop}
	Έστω $K/F$ μια αβελιανή επέκταση σωμάτων αριθμών, $\mathfrak{p}$ ένα μη τετριμμένο πρώτο ιδεώδες του $\mathcal{O}_F$ που είναι αδιακλάδωτο στην επέκταση $K/F$ και $\mathfrak{P}$ ένα πρώτο ιδεώδες του $\mathcal{O}_K$ με $\mathfrak{P}| \mathfrak{p} \mathcal{O}_K$. Τότε το $\sigma = \left(\frac{\mathfrak{P}}{K/F}\right)$ δεν εξαρτάται από την επιλογή του πρώτου $\mathfrak{P}$ πάνω από το $\mathfrak{p}$.
\end{prop}

\begin{proof}
	%class field theory
\end{proof}

\begin{prop}
	Έστω $K/F$ μια αβελιανή επέκταση με ομάδα \tl{Galois} $G$ και ένα ενδιάμεσο σώμα $L$ (άρα και οι επεκτάσεις $L/F$ και $K/L$ είναι αβελιανές). Έστω $\mathfrak{p}$ ένα πρώτο ιδεώδες του $\mathcal{O}_F$ που είναι αδιακλάδωτο στην επέκταση $K/F$. Τότε τα στοιχεία $\left(\frac{\mathfrak{p}}{L/F}\right)$ και $\left(\frac{\mathfrak{p}}{K/F}\right)$ είναι ορισμένα και ισχύει ότι $$\left(\frac{\mathfrak{p}}{L/F}\right) = \left(\frac{\mathfrak{p}}{K/F}\right)\bigg|_L$$ 
\end{prop}

\begin{proof}%class field theory
\end{proof}

%exercise 1???

%example 1??
%μάλλον χρειάζεται το παράδειγμα για την επόμενη πρόταση
\begin{prop}
	Αν $\zeta,\zeta^{\prime}$ είναι $m$-οστές ρίζες της μονάδας στο $K$ και $\mathfrak{P}|p\Z$ είναι αδιακλάδωτο με $\zeta = \zeta^{\prime} (mod \mathfrak{P})$, τότε $\zeta = \zeta^{\prime}$.
\end{prop}
\begin{proof}
	%class field theory
\end{proof}

τώρα με βάση τον αυτοομορφισμό του \tl{Artin} θα δείξουμε το ακόλουθο αποτέλεσμα για τους πρώτους που διασπώνται πλήρως σε ενδιάμεσες επεκτάσεις κυκλοτομικών σωμάτων.


%%%%%%%%%%%%%%%%%%%%%%%%%%%%%%%%%%%%%%%%%%%%%%%%%%%%%%
Κεντρικό Θεώρημα:

\begin{theorem}
	Έστω $K$ υπόσωμα του $\Q (\zeta_m)$. Υπενθυμίζουμε ότι η ομάδα $Gal(\Q (\zeta_m)/ \Q)$ είναι ισόμορφη με την πολλαπλασιαστική ομάδα του δακτυλίου $\Z / m \Z$. Έστω $H$ η υποομάδα της $\left(\Z / m \Z \right)^x$ που αντιστοιχεί μέσω του ισομορφισμού στην ομάδα $Gal(\Q(\zeta_m)/ K)$. Τότε οι πρώτοι αριθμοί $p$ οι οποίοι δεν διαιρούν το $m$ και διασπώνται πλήρως στην επέκταση $K/ \Q$ είναι ακριβώς εκείνοι για τους οποίους ισχύει $ p \text{ \tl{mod} } m \in H$.
\end{theorem}
\begin{proof}
	%class field theory
\end{proof}

παραδείγματα!!!!!

%%%% Dirichlets unit theorem????%%%%%
\end{document}