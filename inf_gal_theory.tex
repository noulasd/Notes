
\documentclass[oneside,a4paper]{article}

%%%%%%%%%%%%%%%%%%%%%%%%%%%%
\usepackage{amsthm}
\usepackage{amsmath}
\usepackage{amssymb}
%%%%%%%%%%%%%%%%%%%%%%%%%%%%%
\usepackage[greek]{babel}
\usepackage[utf8]{inputenc}

\usepackage{blindtext}
\usepackage[T1]{fontenc}
\usepackage{titlesec}
\usepackage{sectsty}
\usepackage{verbatim}
\usepackage{multirow}
\chapternumberfont{\tiny} 
\chaptertitlefont{\Huge}
%ελληνικοι χαρακτηρες σε μαθ pdf utf-8
%%%%%%%%%%%%%%%%%%%%%%%%%%%%%%%%%
\usepackage{tikz-cd}

\usepackage{xcolor}
\usepackage{framed}%frames

\usepackage{array}
\usepackage{pbox}

%%%%%%%%%%%%%%%%%%%%%%%%
\usepackage{tikz}
%%%%%%%%%%%%%%%%%%%%%%%%%%

%%%%%%%%περιθώρια%%%%%%%%%%%%
\usepackage[a4paper,margin=3.5cm]{geometry}


%%%%%%%%συντομευσεις%%%%%%%%%%
\newtheorem{theorem}{Θεώρημα}
\newtheorem{lemma}{Λήμμα}
\newtheorem{example}{Παράδειγμα}
\newtheorem*{defn}{Ορισμός}
\newtheorem{prop}{Πρόταση}
\newtheorem{cor}{Πόρισμα}

\newcommand {\tl}{\textlatin}
%%%%%%%%%αριθμηση%%%%%%%%%%%%%%
\renewcommand{\theenumi}{\arabic{enumi}}
\renewcommand{\labelenumi}{{\rm(\theenumi)}}
\renewcommand{\labelenumii}{\roman{enumii}) }
%%%%%%%%%%%% New theorems %%%%%%%%%%%%%%%%%%%%%%%%

%%%%%%%%%%%%%%%%%%%%%%%%%%%%%%%%%%%%%%%%%%%%%%%%%%%
\newcommand{\Z}{\mathbb{Z}}
\newcommand{\Q}{\mathbb{Q}}
\newcommand{\Co}{\mathbb{C}}
%%%%%%%%%%%%%%%%%%%%% Document starts %%%%%%%%%%%%
\begin{document}
	
	%%%%%%%%%%%%%%%%%%%%%%%%%%%%%%%%%%%%%%%%%%%%%%%%%%
	\selectlanguage{greek}
	%%%%%%%%%%%%%%%%%%%%%%% Start Roman numbering %%%% vbbnn
	%\pagenumbering{roman}
	%%%%%%%%%%%%%%%%%%%%%%%%%%%%%%%%%%%%%%%%%%%%%%%%%%
	
	\begin{framed}	
		%\vspace{0.3truecm}
		\begin{center}
			\huge Θέματα Άλγεβρας και Γεωμετρίας \tl{I}
		\end{center}
		%\vspace{0.3truecm}
		\begin{center}
			\huge Άπειρη Θεωρία \tl{Galois}.
		\end{center}
		\vspace{0.3truecm}
		\begin{center}
			Ονομ/νο: Νούλας Δημήτριος\\
			ΑΜ: \\
			\tl{email}: \tl{}
		\end{center}
		\vspace{0.3truecm}
	\end{framed}
\vspace{1cm}
\pagebreak
\tableofcontents
\pagebreak
\section{Κλασική Θεωρία \tl{Galois}}
\vspace{1cm}
% proposition 3.28!

\begin{defn}
	Έστω $F\subseteq K$ εγκλεισμός σωμάτων. Το $K$ θα λέγεται επέκταση του $F$ και θα συμβολίζεται με $K/F$.
\end{defn}

Παρατηρούμε ότι με τις πράξεις:
$$K\times K \longrightarrow K \quad\quad F \times K \longrightarrow K$$
$$(x,y) \longmapsto x+y \quad\quad (\lambda, x) \longmapsto \lambda x$$

To $K$ είναι διανυσματικός χώρος υπεράνω του $F$ και συμβολίζουμε την διάσταση του με $[K:F]$. Αν $[K:F]<\infty$ θα λέμε ότι η επέκταση $K/F$ είναι πεπερασμένη.

$ $\newline
Για παράδειγμα, έχουμε $[\mathbb C : \mathbb R] = 2$ και αν $K = \Z_2 [x]/(x^2 + x + 1)$ τότε $[K:\Z_2] = 2$.

\begin{defn}
	Έστω $K/F$ και $a \in K$. Το $a$ θα λέγεται αλγεβρικό υπεράνω του $F$ αν υπάρχει $f(x) \in F[x]$ τέτοιο ώστε $f(a)=0$. Αν αυτό ισχύει για κάθε $a \in K$ τότε λέμε ότι η επέκταση είναι αλγεβρική.
\end{defn}

\begin{prop}
	Έστω $K/F$ με $[K:F]<\infty$. Τότε $K/F$ αλγεβρική.
\end{prop}

\begin{proof}
\end{proof}

\begin{theorem}[Κανόνας Πύργων]
	Άν $K/E$ και $E/F$ πεπερασμένες επεκτάσεις τότε $K/F$ πεπερασμένη και επιπλέον:
	$$[K:F]=[K:E][E:F]$$
\end{theorem}
\begin{proof}
\end{proof}

\begin{defn} Έστω $K/F$. Ονομάζουμε αλγεβρική κλειστότητα του $F$ στο $K$ το σώμα:
	$$\overline F = \{a \in K: \text{ το } a \text{ είναι αλγεβρικό υπεράνω του } F\}$$ 
\end{defn} %υπαρξη

\begin{defn} Έστω $K/F$ και $S\subseteq K$ με $F(S)$ συμβολίζουμε την τομή όλων των υποσωμάτων του $K$ που περιέχουν το $F$ και το $S$. To $F(S)$ είναι σώμα και $F\subseteq F(S) \subseteq K$. Αν $S=\{a_1, \ldots, a_n\}$ τότε γράφουμε $F(a_1, \ldots, a_n)$.

\end{defn}

\begin{prop} Έστω $K/F$ και $a_1,\ldots a_n \in K$. Έχουμε ότι το $F(a_1, \ldots , a_n)$  είναι ίσο με:
	$$ \left\{\frac{f(a_1,\ldots,a_n)}{g(a_1,\ldots, a_n)} \in K: \quad f(x_1,\ldots,x_n), g(x_1,\ldots,x_n) \in F[x_1,\ldots,x_n] , g(a_1,\ldots,a_n)\neq 0\right\}$$
\end{prop}

\begin{proof}
\end{proof}

Έστω $a$ αλγεβρικό στοιχέιο υπεράνω κάποιου σώματος $F$. Θεωρούμε το πολυώνυμο $p(x) \in F[x]$ ελαχίστου βαθμού που έχει ρίζα το $a$. Εφόσον το $F$ είναι σώμα μπορούμε να υποθέσουμε ότι το $p$ είναι μονικό καθώς μπορούμε να το πολλαπλασιάσουμε με τον αντίστροφο του μεγιστοβάθμιου συντελεστή. Επιπλέον, αυτό το πολυώνυμο είναι μοναδικό. Αν δεν είναι και έχουμε $p(x), g(x)$ με αυτές τις ιδιότητες, τότε το $p(x)-g(x) \in F[x]$ θα έχει ρίζα το $a$ και θα είναι βαθμού μικρότερου του $p(x)$ δηλαδή θα είναι βαθμού 0 από την υπόθεσή μας.
\begin{defn} Για ένα $a$ που ανήκει σε κάποια επέκταση και είναι αλγεβρικό στοιχέιο υπεράνω ενός σώματος $F$ λέμε το παραπάνω πολυώνυμο $p(x)$ ως ελάχιστο πολυώνυμο του $a$ υπεράνω του $F$ και το συμβολίζουμε με $Irr(a,F)$. 
\end{defn}

\begin{prop}
	%prop 1.15
	Έστω $K/F$ με $a$ αλγεβρικό στοιχείο υπεράνω του $F$. Τότε:
	\begin{enumerate}
		\item $Irr(a,F)$ είναι ανάγωγο στο $F[x]$.
		\item Αν $g(x) \in F[x]$ τότε $g(a) = 0 \iff Irr(a,F) | g(x)$.
		\item Αν $n = deg(Irr(a,F))$ τότε τα $1,a,a^2,\ldots,a^{n-1}$ αποτελούν βάση του $F(a)$ υπεράνω του $F$.
	\end{enumerate}
\end{prop}
\begin{proof}
\end{proof}

\begin{cor}  $[F(a): F] < \infty$ αν και μόνο αν το $a$ είναι αλγεβρικό υπεράνω του $F$.
\end{cor}

\begin{proof}
\end{proof}


\begin{theorem}Αν $K/E$ και $E/F$ είναι αλγεβρικές επεκτάσεις τότε και η επέκταση $K/F$ είναι αλγεβρική.
\end{theorem}

\begin{proof}
\end{proof}

\subsection{Αυτομορφισμοί}
\vspace{1cm}

Έστω $K$ ένα σώμα. Ένας ισομορφισμός δακτυλιών $K \rightarrow K$ ονομάζεται αυτομορφισμός του $K$ και η ομάδα των αυτομορφισμών με πράξη την σύνθεση συμβολίζεται με $Aut(K)$. Καθώς ασχολούμαστε με επεκτάσεις πρέπει να θεωρήσουμε απεικονίσεις επεκτάσεων. Έστω $K,L$ επεκτάσεις ενός σώματος $F$. Ένας $F$-ομομορφισμός $\tau : K \rightarrow L$ είναι ένας ομομορφισμός δακτυλίων τέτοιος ώστε $\tau (a) = a$ για κάθε $a \in F$. Δηλαδή, $\tau |_F = id_F$. Αν o $\tau$ είναι 1-1 και επί τότε λέγεται $F$-ισομορφισμός. Αν επιπλέον $K=L$, τότε λέγεται $F$-αυτομορφισμός του $K$.

\begin{defn}[Ομάδα \tl{Galois}]
	Έστω $K/F$. Ορίζουμε $Gal(K/F)$ να είναι οι $F$-αυτομορφισμοί του $K$ με πράξη την σύνθεση.
\end{defn}

\begin{lemma}
	%lemma 2.2
	Έστω $K=F(X)$ με $X \subseteq K$. Αν $\sigma , \tau \in Gal(K/F)$ με $\sigma|_X = \tau|_X $ τότε $\sigma = \tau$. Συνεπώς, οι $F$-αυτομορφισμοί του $K$ καθορίζονται πλήρως από τις εικόνες τους στο σύνολο $X$ που επισυνάπτουμε.
\end{lemma}

\begin{proof}
	%σελ 16
\end{proof}



\begin{lemma}
	%2.3 lemma
	Έστω $\tau : K \rightarrow L$ ένας $F$-ομομορφισμός και $a \in K$ αλγεβρικό υπεράνω του $F$. Αν $f(x) \in F[x]$ με $f(a)=0$ τότε $f(\tau(a)) = 0$. Δηλαδή το $\tau$ μεταθέτει τις ρίζες του $Irr(a,F)$. Συνεπώς $Irr(a,F) = Irr(\tau(a),F)$.
\end{lemma}

\begin{cor}
	Αν $K/F$ επέκταση με $[K:F]<\infty$ τότε $|Gal(K/F)| <\infty$.
\end{cor}
\begin{proof}
\end{proof}

\begin{defn}[Σταθερό σώμα]
	Έστω $K/F$ και $S \subseteq Aut(K)$. Τότε το σύνολο:
	$$F^S = \{a \in K: \quad \tau(a) = a \quad\forall \tau \in S\}$$
	λέγεται σταθερό σώμα του $S$ και είναι πράγματι σώμα και μάλιστα υπόσωμα του $K$.
\end{defn}

Λέμε ένα σώμα $L$ τέτοιο ώστε $F \subseteq L \subseteq K$ ενδιάμεση επέκταση της $K/F$ ή αλλιώς $K/L/F$. Αν $S\subseteq Gal(K/F)$ τότε $F^S$ είναι ενδιάμεση επέκταση της $K/F$.

\begin{lemma}
	%2.9 sel 18
	\label{duo ennia}
	Έστω $K$ ένα σώμα. Τότε:
	\begin{enumerate}
		\item Αν $L_1 \subseteq L_2$ υποσώματα του $K$ τότε $Gal(K/L_2) \subseteq Gal(K/L_1)$.
		\item Αν $L\subseteq K$ τότε $L \subseteq F^{Gal(K/L)}$.
		\item Αν $S_1 \subseteq S_2$ υποσύνολα του $Aut(K)$ τότε $F^{S_2} \subseteq F^{S_1}$.
		\item Αν $S \subseteq Aut(K)$ τότε $S\subseteq Gal(K/F^S)$.
		\item Αν $L = F^S$ για κάποιο $S\subseteq Aut(K)$ τότε $L = F^{Gal(K/F)}$.
		\item Αν $H = Gal(K/L)$ για κάποιο σώμα $L \subseteq K$ τότε $H = Gal(K/F^H)$.
	\end{enumerate}
\end{lemma}
\begin{proof}
\end{proof}


\begin{cor} Αν $K/F$ τότε υπάρχει 1-1 αντιστοιχία που αλλάζει την φορά μεταξύ των υποομάδων της $Gal(K/F)$ της μορφής $Gal(K/L)$ για κάποιο υπόσωμα $L$ του $K$ που περιέχει το $F$ και των υποσωμάτων του $K$ που περιέχουν το $F$ και είναι της μορφής $F^S$ για κάποιο $S \subseteq Aut(K)$. Η αντιστοιχία δίνεται από την απεικόνιση $L \mapsto Gal(K/L)$ και την αντίστροφή της $H \mapsto F^H$
\end{cor}
\begin{proof}
	%sel 19
\end{proof}

\begin{prop}
	%2.13 apodeiksh apo maliaka
	Εστω $K/F$ πεπερασμένη επέκταση. Τότε $|Gal(K/F)| \leq [K:F]$
\end{prop}

\begin{proof}
\end{proof}

\begin{prop}
	%prop 2.14 apodeikhs apo maliaka
	\label{duo dekatessera}
	Έστω $G$ πεπερασμένη ομάδα αυτομορφισμών του $K$ με $F^G$. Τότε $|G| = [K:F]$ και $G = Gal(K/F)$.
\end{prop}

\begin{defn}[Επέκταση \tl{Galois}]
	Έστω $K/F$ αλγεβρική επέκταση. Τότε λέμε ότι η $K/F$ είναι \tl{Galois} αν $F = F^{Gal(K/F)}$.
\end{defn}

\begin{cor}
	Έστω $K/F$ πεπερασμένη επέκταση. Τότε $K/F$ είναι \tl{Galois} αν και μόνο αν $|Gal(K/F)| = [K:F]$.
\end{cor}
\begin{proof}
	%sel 22
\end{proof}

\begin{cor} Έστω $K/F$ και $a\in K$ αλγεβρικό υπεράνω του $F$. Τότε $|Gal(F(a)/F)|$ είναι ο αριθμός των διακεκριμένων ριζών του $Irr(a,F)$ στο $F(a)$. Συνεπώς, $F(a)/F$ είναι \tl{Galois} αν και μόνο αν $Irr(a,F)$ έχει διακεκριμένες ρίζες όσες είναι ο βαθμός του.
\end{cor}
\begin{proof}
\end{proof}

\begin{example} Η επέκταση $\Q (\sqrt[3]{2}) / \Q$ δεν είναι \tl{Galois}. To $Irr(\sqrt[3]{2} , \Q ) = x^3 -2$ έχει 3 διακεκριμένες ρίζες και $[\Q (\sqrt[3]{2}) : \Q] = 3$ αλλά μόνο μια ανήκει στο πραγματικό σώμα $\Q (\sqrt[3]{2})$, δηλαδή $|Gal(\Q (\sqrt[3]{2}) / \Q)| = 1$.
	
	Αν επισυνάψουμε και την $\omega = e^{2\pi i /3}$, έχουμε $x^3 - 2 = (x-\sqrt[3]{2})(x-\sqrt[3]{2}\omega ) (x-\sqrt[3]{2} \omega^2)$. Όλες οι ρίζες ανήκουν στο $\Q (\sqrt[3]{2}, \omega)$ και εφόσον $Irr(\sqrt[3]{2}, \Q) = x^3 - 2$ και $Irr(\omega , \Q) = x^2 + x + 1$ έχουμε από κανόνα των πύργων ότι $[\Q (\sqrt[3]{2} , \omega) : \Q] = 6$.
	Ταυτόχρονα έχουμε τους 6 ισομορφισμούς:

	\begin{center}
		\begin{tabular}{|c|c|c |} 
		\hline
		  & $\sqrt[3]{2}$ & $\omega$  \\ 
		\hline
		$\sigma_1=id$ & $\sqrt[3]{2}$ & $\omega$ \\ 
		
		$\sigma_2$ & $\omega\sqrt[3]{2}$ & $\omega$  \\
		
		$\sigma_3$ & $\sqrt[3]{2}$ & $\omega^2$  \\
		
		$\sigma_4$ & $\omega\sqrt[3]{2}$ & $\omega^2$  \\
		
		$\sigma_5$ &$\omega^2\sqrt[3]{2}$ & $\omega$  \\
		
		$\sigma_6$ &$\omega^2\sqrt[3]{2}$ & $\omega^2$ \\
		\hline
	   \end{tabular}
	   \end{center}

	επομένως $|Gal(\Q (\sqrt[3]{2}, \omega)/ \Q )| = [\Q (\sqrt[3]{2} , \omega): \Q]$
	και άρα αυτή η επέκταση είναι \tl{Galois}.
\end{example}

\subsection{Κανονικές Επεκτάσεις}

Υπενθυμίζουμε εδώ ότι, με βάση τον αλγόριθμο διαίρεσης, $a \in F$ είναι ρίζα του $f(x) \in F[x]$ αν και μόνο αν $x-a | f(x)$.

\begin{lemma}
	%3.1
	Έστω $f(x) \in F[x]$. Τότε το πολυώνυμο $f$ έχει το πολύ $deg(f)$ ρίζες σε οποιαδήποτε επέκταση του $F$.
\end{lemma}

\begin{proof}
\end{proof}

\begin{defn} Έστω $K/F$ και $f(x) \in F[x]$. Λέμε ότι το $f$ διασπάται πλήρως στο $K$ αν υπάρχουν $a_1 , \ldots a_n \in K$ και $a \in F$ τέτοια ώστε:
	$$f(x) = a (x-a_1)\cdots (x-a_n) \in K[x]$$
\end{defn}

\begin{defn}[Σώμα ριζών]
	Έστω $K/F$ και $f(x) \in F[x]$. Λέμε ότι το $K$ είναι σώμα ριζών του $f(x) \in F[x]$ αν το $f$ διασπάται πλήρως στο $K$ και $K=F(a_1 , \ldots, a_n)$ όπου $a_1 ,\ldots ,a_n$ είναι οι ρίζες του $f$. Μπορούμε να λέμε ότι το $K$ είναι και σώμα ριζών ενός συνόλου πολυωνύμων αν καθένα από αυτά διασπάται πλήρως στο $K$ και $K=F(X)$ όπου $X$ οι ρίζες των πολυωνύμων.
\end{defn}

Aν $X = \{f_1 , \ldots f_n\}$ ουσιαστικά μιλάμε για το σώμα ριζών του πολυωνύμου $f = f_1 \cdots f_n$.

\begin{theorem} [Ύπαρξη ρίζας σε επέκταση]
\end{theorem}

\begin{proof}
	%θεώρημα 3.3 σελ 28
\end{proof}

\begin{theorem}[Ύπαρξη σώματος ριζών για πεπερασμένα πολυώνυμα]
\end{theorem}

\begin{proof}
\end{proof}

\begin{lemma} 
	\label{tria deka}
	Αν $K$ είναι σώμα τότε τα επόμενα είναι ισοδύναμα:
	\begin{enumerate}
		\item Δεν υπάρχουν αλγεβρικές επεκτάσεις του $K$ εκτός από το ίδιο το $K$.
		\item Δεν υπάρχουν πεπερασμένες επεκτάσεις του $K$ εκτός από το ίδιο το $K$.
		\item Αν $L$ είναι επέκταση του $K$, τότε $K = \{a \in L: a \text{ αλγεβρικό υπεράνω του } K \}$.
		\item Κάθε $f(x) \in K[x]$ διασπάται πλήρως στο $K$.
		\item Κάθε $f(x) \in K[x]$ έχει ρίζα στο $K$. 
	\end{enumerate}
\end{lemma}

\begin{proof}
\end{proof}

\begin{defn} Αν κάποιο $K$ ικανοποιεί κάποια από τις συνθήκες του λήμματος $\ref{tria deka}$ τότε λέμε ότι το $K$ είναι αλγεβρικά κλειστό. Αν $K/F$ είναι αλγεβρική επέκταση και το $K$ είναι αλγεβρικά κλειστό, λέμε ότι το $K$ είναι αλγεβρική κλειστότητα του $F$.
\end{defn}

\begin{lemma}
	Αν $K/F$ αλγεβρική επέκταση, τότε $|K| \leq max\{|F| , |\mathbb N|\}$.
\end{lemma}
\begin{proof}
\end{proof}


\begin{theorem}[Ύπαρξη αλγεβρικής κλειστότητας]
	Έστω $F$ ένα σώμα. Τότε υπάρχει αλγεβρική κλειστότητα του $F$
\end{theorem}
\begin{proof}
\end{proof}


\begin{cor}[Υπαρξη σώματος ριζών]
\end{cor}


\begin{lemma} Έστω $\sigma : F \rightarrow F^{\prime}$ ένας ισομορφισμός σωμάτων. Έστω $f(x) \in F[x]$ ανάγωγο και $a$ μια ρίζα του σε κάποια επέκταση $K/F$. Επιπλέον, έστω $a^{\prime}$ μια ρίζα του $\sigma(f)$ σε μια επέκταση $K^{\prime} / F^{\prime}$. Τότε, υπάρχει ισομορφισμός $\tau : F(a) \rightarrow F^{\prime} ( a ^{\prime})$ με $\tau (a) = a^{\prime}$ και $\tau|_F = \sigma$.
\end{lemma}

\begin{proof}% 3.17
\end{proof}

\begin{lemma} Έστω $\sigma : F \rightarrow F^{\prime}$ ένας ισομορφισμός σωμάτων και οι επεκτάσεις $K/F$ και $K^{\prime} / F^{\prime}$. Υποθέτουμε ότι το $K$ είναι σώμα ριζών μιας οικογένειας $\{f_i\}$ υπεράνω του $F$ και ότι $\tau : K \rightarrow K^{\prime}$ είναι ένας ομομορφισμός με $\tau|_F = \sigma$. Τότε το $\tau(K)$ είναι σώμα ριζών της οικογένειας $\{\sigma(f_i) \}$ υπεράνω του $F^{\prime}$.
\end{lemma}

\begin{proof}
\end{proof}

\begin{theorem} %3.19
	Έστω $\sigma : F \rightarrow F^{\prime}$ ενας ισομορφισμός σωμάτων και θεωρούμε τα πολυώνυμα $f(x) \in F[x] , \sigma(f) \in F^{\prime}[x]$. Έστω $K$ το σώμα ριζών του $f$ υπεράνω του $F$ και $K^{\prime}$ το σώμα ριζών του $\sigma (f)$ υπεράνω του $F^{\prime}$. Τότε υπάρχει ισομορφισμός $\tau: K \rightarrow K^{\prime}$ με $\tau|_F = \sigma$. Επιπλέον, αν $a \in K$ και το $a^{\prime} \in K^{\prime}$ είναι ρίζα του $\sigma (Irr(a,F))$ στο $K^{\prime}$ τότε ο ισομορφισμός $\tau$ μπορεί να επιλεχθεί έτσι ώστε $\tau(a) = a^{\prime}$. 
\end{theorem}

\begin{proof}
\end{proof}


\begin{theorem} [Θεώρημα Επέκτασης Ισομορφισμών]
	Έστω $\sigma : F \rightarrow F^{\prime}$ ένας ισομορφισμός σωμάτων. Έστω $S = \{f_i (x)\}$ ένα σύνολο πολυωνύμων με συντελεστές από το $F$ και $S^{\prime} = \{\sigma (f_i)\}$. Έστω $K$ ένα σώμα ριζών του $S$ υπεράνω του $F$ και $K^{\prime}$ ένα σώμα ριζών του $S^{\prime}$ υπεράνω του $F^{\prime}$. Τότε υπάρχει ισομορφισμός $\tau : K \rightarrow K^{\prime}$ με $\tau|_F = \sigma$. Επιπλέον, αν $a \in K$ και το $a^{\prime}$ είναι οποιαδήποτε ρίζα του $\sigma(Irr(a,F))$ στο $K^{\prime}$ τότε ο ισομορφισμός $\tau$ μπορεί να επιλεγεί έτσι ώστε $\tau(a) = a^{\prime}$.
\end{theorem}

\begin{proof}
\end{proof}

\begin{cor} Έστω $F$ ένα σώμα και $S\subseteq F[x]$. Τότε οποιαδήποτε δύο σώματα ριζών του $S$ είναι ισόμορφα μέσω ενός $F$-ισομορφισμού. Ειδικότερα το ίδιο ισχύει για αλγεβρικές κλειστότητες.
\end{cor}

\begin{proof}
\end{proof}

\begin{cor}Έστω $F$ ένα σώμα και $N$ μια αλγεβρική κλειστότητα του $F$. Αν $K$ είναι αλγεβρική επέκταση του $F$ τότε το $K$ είναι ισόμορφο με ένα υπόσωμα του $N$.
\end{cor}


\begin{proof}
\end{proof}


\begin{defn}[Κανονική Επέκταση] Έστω $K/F$. Τότε λέμε ότι η επέκταση $K/F$ είναι κανονική αν το $K$ είναι σώμα ριζών για κάποιο σύνολο πολυωνύμων υπεράνω του $F$.
\end{defn}

\begin{prop}
	\label{exa3.25}
	Έστω $K/L/F$ και $K/F$ κανονική επέκταση. Τότε η επέκταση $K/L$ είναι κανονική.
\end{prop}
\begin{proof}
	Το $K$ είναι σώμα ριζών ενός συνόλου πολυωνύμων $S \subseteq F[x]$. Δηλαδή, το $K$ είναι το $F$ με επισύναψη των ριζών των πολυωνύμων του $S$. Λόγω της επέκτασης $K/L/F$ το $K$ θα έιναι το $L$ με επισύναψη τις ρίζες των πολυωνύμων του $S$ (πιθανότατα κάποιες από αυτές θα ανήκουν ήδη στο $L$). Άρα το $K$ είναι σώμα ριζών του $S$ υπεράνω του $L$ και συνεπώς κανονική επέκταση του $L$.
\end{proof}

\begin{prop}% 3.28
	\label{3.28}
	Έστω $K/F$ αλγεβρική. Τότε τα ακόλουθα είνα ισοδύναμα:
	\begin{enumerate}
		\item $K/F$ κανονική επέκταση.
		\item Αν $M$ είναι αλγεβρική κλειστότητα του $K$ και αν $\tau : K \rightarrow M$ είναι ένας $F$-ισομορφισμός, τότε $\tau(K) = K$. %!!!!!
		\item Αν $N/K/L/F$ και $\sigma : L \rightarrow N$ είναι ένας $F$-ομομορφισμός, τότε $\sigma(L) \subseteq K$ και υπάρχει $\tau \in Gal(K/F)$ με $\tau|_L = \sigma$.
		\item Για κάθε ανάγωγο $f(x) \in F[x]$, αν $f$ έχει ρίζα στο $K$ τότε διασπάται πλήρως στο $K$.
	\end{enumerate}
\end{prop}

\begin{proof}
\end{proof}

\subsection{Διαχωρίσιμες Επεκτάσεις}
\vspace{1cm}

\begin{defn}[Διαχωρίσιμο πολυώνυμο]
	Έστω $F$ ένα σώμα. Ένα ανάγωγο πολυώνυμο $f(x) \in F[x]$ είναι διαχωρίσιμο υπεράνω του $F$ αν οι ρίζες του είναι απλές σε οποιοδήποτε σώμα ριζών. Ένα πολυώνυμο $g(x) \in F[x]$ είναι διαχωρίσιμο υπεράνω του $F$ αν όλοι οι ανάγωγοι παράγοντες του είναι διαχωρίσιμοι υπεράνω του $F$.
\end{defn}
\begin{example}
	Τα πολυώνυμα $x^2 - 2$ και $(x-1)^5$ είναι διαχωρίσιμα υπεράνω του $\Q$. Για να βρούμε μη διαχωρίσιμο πολυώνυμο πρέπει να κοιτάξουμε σε σώματα θετικής χαρακτηριστικής λόγω αποτελέσματος που θα ακολουθήσει. Έστω σώμα $F$ χαρακτηριστικής $p$ και ένα στοιχείο $a \in F \setminus F^p$. Έστω $y$ ρίζα του $x^p - a$ σε κάποια επέκταση $L/F$. Έχουμε ότι η χαρακτηριστικη του $L$ θα είναι $p$ και επομένως $(x-y)^p = x^p - y^p = x^p - a$ αφού $a = y^p$ στο $L[x]$. Από αυτή τη σχέση, αν το $x^p -a$ δεν είναι ανάγωγο στο $F$ τότε κάποιος παράγοντας $(x-y)^r , 1\leq r < p$ θα ανήκει στο $F[x]$. Έστω λοιπόν:
	$$x^p - a = g(x) h(x) \quad g(x),h(x) \in F[x] \text{ μη σταθερά}$$
	αν δούμε την εξίσωση στο μεγάλο σώμα $L$ έχουμε:
	$$(x-y)^p = g(x)h(x)$$
	δηλαδή, εφόσον έχουμε ότι το $L$ είναι σώμα και άρα το $L[x]$ περιοχή μοναδικής παραγοντοποίησης, $g(x) = (x-y)^r$ και $h(x) = (x-y)^{p-r}$ για κάποιο $1\leq r < p$. Καθώς $g(x) \in F[x]$ τότε ο συντελεστής $-ry$ του $x^{r-1}$ ανήκει στο $F$. Ωστόσο, το $r\neq 0_F$ από την επιλογή του άρα $y \in F$. Αυτό είναι άτοπο καθώς $a = y^p \not\in F^p$. Άρα το $x^p - a$ είναι ανάγωγο στο $F[x]$. Όπως είδαμε παραπάνω, αν έχει ρίζα $y$ σε επέκταση το $x^p -a$ θα γράφεται ως $(x-y)^p$ και άρα δεν είναι διαχωρίσιμο υπεράνω του $F$. 
\end{example}

\begin{lemma}
	\label{tessera tria}
	Έστω $f(x)$ και $g(x)$ πολυώνυμα υπεράνω ενός σώματος $F$. Τότε:
	\begin{enumerate}
		\item Αν το $f$ έχει μόνο απλές ρίζες σε οποιοδήποτε σώμα ριζών, τότε είναι διαχωρίσιμο.
		\item Αν $g(x) | f(x)$ και το $f$ είναι διαχωρίσιμο υπεράνω του $F$, τότε και το $g$ είναι διαχωρίσιμο υπεράνω του $F$.
		\item  Αν $f_1 , \ldots f_n$ είναι διαχωρίσιμα πολυώνυμα υπεράνω του $F$ τότε το γινόμενο τους είναι διαχωρίσιμο υπεράνω του $F$
		\item  Αν το $f$ είναι διαχωρίσιμο υπεράνω του $F$, τότε το $F$ είναι διαχωρίσιμο υπεράνω οποιασδήποτε επέκτασης του $F$.
	\end{enumerate}

\end{lemma}
\begin{proof}
\end{proof}
%sel 40


\begin{prop}Έστω $f(x) \in F[x]$ μη σταθερό πολυώνυμο. Τότε το $f$ έχει μόνο απλές ρίζες σε ένα σώμα ριζών αν και μόνο αν $(f,f^{\prime})=1$ στο $F[x]$, όπου $f^{\prime}$ είναι η τυπική παράγωγος του πολυωνύμου $f$.

\end{prop}

\begin{proof}
\end{proof}

\begin{prop}Έστω $f(x) \in F[x]$ ανάγωγο. Τότε:
	\begin{enumerate} 
		\item Αν η χαρακτηριστική του $F$ είναι 0, τότε το $f$ είναι διαχωρίσιμο υπεράνω του $F$. Αν η χαρακτηριστική είναι $p > 0$ τότε το $f$ είναι διαχωρίσιμο αν και μόνο αν $f^{\prime} \neq 0$ και αυτό συμβαίνει μόνο όταν $f(x) \not\in F[x^p]$.
		\item Αν η χαρακτηριστική του $F$ ειναι $p>0$, τότε $f(x) = g(x^{p^m})$ για κάποιον ακέραιο $m\geq 0$ και κάποιο $g(x) \in F[x]$ το οποίο είναι ανάγωγο και διαχωρίσιμο υπεράνω του $F$.
	\end{enumerate}
\end{prop}
\begin{proof}
\end{proof}


\noindent Τώρα θα επεκτείνουμε την ιδέα της διαχωρισιμότητας στα στοιχεία ενός σώματος και γενικότερα στις επεκτάσεις.

\begin{defn}[Διαχωρίσιμο Στοιχείο και Επέκταση]
	Έστω $K/F$ επέκταση και $a \in K$. Τότε το $a$ είναι διαχωρίσιμο υπεράνω του $F$ αν το $Irr(a,F)$ είναι διαχωρίσιμο υπεράνω του $F$. Αν αυτό ισχύει για κάθε $a \in K$ λέμε την επέκταση $K/F$ διαχωρίσιμη.
\end{defn}

\noindent Με βάση τα παραπάνω, Θα δώσουμε στην συνέχεια έναν χρήσιμο χαρακτηρισμό για τις επεκτάσεις \tl{Galois}.

\begin{theorem}
	Έστω $K/F$ αλγεβρική επέκταση. Τότε τα ακόλουθα είναι ισοδύναμα:
	\begin{enumerate}
		\item η επέκταση $K/F$ είναι \tl{Galois}.
		\item η επέκταση $K/F$ είναι κανονική και διαχωρίσιμη.
		\item το $K$ είναι σώμα ριζών ενός συνόλου διαχωρίσιμων πολυωνύμων υπεράνω του $F$.
	\end{enumerate}
\end{theorem}
\begin{proof}
	%sel 42
\end{proof}

\begin{cor} Έστω $L/F$ πεπερασμένη επέκταση. Τότε:
	\begin{enumerate}
		\item το $L$ είναι διαχωρίσιμο υπεράνω του $F$ αν και μόνο αν περιέχεται σε μια \tl{Galois} επέκταση του $F$.
		\item Αν $L = F(a_1, \ldots , a_n)$ με τα $a_i$ να είναι διαχωρίσιμα υπεράνω του $F$ τότε το $L$ είναι διαχωρίσιμο υπεράνω του $F$.
	\end{enumerate}
\end{cor}

\begin{prop}
	\label{myprop}
	Έστω $K/F$ διαχωρίσιμη και $K/E/F$ ενδιάμεση επέκταση. Τότε η επέκταση $K/E$ είναι και αυτή διαχωρίσιμη.
\end{prop}
\begin{proof}
	Έστω $a \in K$. Έχουμε ότι το $Irr(a,F)$ έχει ρίζα το $a$ και συνεπώς έχουμε:
	$$Irr(a,E) | Irr(a,F) \quad \text{ στο } E[x]$$
	επίσης, το $Irr(a,F)$ είναι διαχωρίσιμο υπεράνω του $F$ λόγω ότι ολόκληρη η επέκταση $K/F$ είναι διαχωρίσιμη. Από το λήμμα $\ref{tessera tria}$, το τέταρτο επιχείρημα μας δίνει ότι το $Irr(a,F)$ είναι διαχωρίσιμο υπεράνω του $E$. Μαζί με αυτό, το δεύτερο επιχείρημα του λήμματος μας δίνει ότι το $Irr(a,E)$ είναι διαχωρίσιμο υπεράνω του $E$.
\end{proof}

\pagebreak
\subsection{Θεμελιώδες Θεώρημα της Θεωρίας \tl{Galois}}

Είμαστε τώρα σε θέση να αποδείξουμε το θεμελιώδες θεώρημα της θεωρίας \tl{Galois} που περιγράφει την σχέση μεταξύ των ενδιάμεσων επεκτάσεων μιας επέκτασης \tl{Galois} $K/F$ με τις υποομάδες της $Gal(K/F)$.

\begin{theorem}[Θεμελιώδες Θεώρημα της Θεωρίας \tl{Galois}]
	Έστω $K$ μια πεπερασμένη επέκταση \tl{Galois} ενός σώματος $F$ και $G=Gal(K/F)$. Τότε υπάρχει μια 1-1 αντιστοιχία που αντιστρέφει την φορά μεταξύ των ενδιάμεσων επεκτάσεων της $K/F$ και των υποομάδων της $G$. Αυτή η αντιστοιχία δίνεται από τις απεικονίσεις $L \mapsto Gal(K/L)$ και $H \mapsto F^H$. Επιπλέον, αν $L\leftrightarrow H$ τότε $[K:L] = |H|$ και $[L:F] = [G:H]$. Μαζί με αυτό, η $H$ είναι κανονική υποομάδα της $G$ αν και μόνο αν η επέκταση $L/F$ είναι \tl{Galois}. Όταν αυτό συμβαίνει έχουμε $Gal(L/F) \cong G/H$.
\end{theorem}
Έχουμε δηλαδή τις αντοιστιχίες:
\begin{center}
	\begin{tikzcd}
		K \arrow[d, no head]                                        &  &  & 1 \arrow[d, no head]          & 1 \arrow[d, no head]                                 &  &  & K \arrow[d, no head]   \\
		E \arrow[d, no head] \arrow[rrr, "{E\longmapsto Gal(K,E)}"] &  &  & {Gal(K,E)} \arrow[d, no head] & H \arrow[d, no head] \arrow[rrr, "F\longmapsto F^H"] &  &  & F^H \arrow[d, no head] \\
		F                                                           &  &  & G                             & G                                                    &  &  & F                     
		\end{tikzcd}
\end{center}
όπου μεταξύ των πύργων που συνδέονται με μια από τις απεικονίσεις, οι βαθμοί που βρίσκονται στο ίδιο ύψος ταυτίζονται.
\begin{proof}$ $
	
	$ $\newline
	Έχουμε δείξει το πρώτο μέρος της απόδειξης με το λήμμα $\ref{duo ennia}$. Έστω $L$ ένα υπόσωμα του $K$ που περιέχει το $F$. Καθώς το $K$ είναι \tl{Galois} υπεράνω του $F$ έχουμε ότι είναι κανονική και διαχωρίσιμη επέκταση του $F$. Από τις προτάσεις $\ref{exa3.25},\ref{myprop}$ έχουμε ότι το $K$ είναι διαχωρίσιμη και κανονική επέκταση υπεράνω του $L$. Δηλαδή η επέκταση $K/L$ είναι \tl{Galois}. Άρα, $L = F^{Gal(K/L)}$ και συνεπώς κάθε ενδιάμεσο σώμα είναι ένα σταθερό σώμα υποομάδας. Επιπλέον, αν $H\leq G$, τότε η $H$ είναι πεπερασμένη και $H = Gal(K/F^H)$, από την πρόταση $\ref{duo dekatessera}$. Κάθε υποομάδα της $G$ είναι λοιπόν μια ομάδα \tl{Galois} και οι απεικονίσεις της εκφώνησης δίνουν την ζητούμενη αντιστοιχία. Για την \tl{Galois} επέκταση $K/F$ έχουμε $|Gal(K/F)| = [K:F]$. Αν λοιπόν $L \leftrightarrow H$ τότε $|H| = [K:L]$, καθώς το $K$ είναι \tl{Galois} επέκταση του $L$ και $H = Gal(K/L)$. Συνεπώς:
	
	$$[G:H] = |G|/|H| = [K:F] / [K:L] = [L:F]$$

	Έστω $H\unlhd G$ και $L = F^H$. Έστω $a \in L$ και $b$ μια ρίζα του $Irr(a,F)$ που ανήκει στο $K$. Από το θεώρημα επέκτασης ισομορφισμών, υπάρχει $\sigma \in G$ με $\sigma (a) = b$. Αν $\tau \in H$, τότε $\tau (b) = \sigma(\sigma^{-1} \tau \sigma (a))$. Ωστόσο, έχουμε ότι $\sigma^{-1} \tau \sigma \in H$ αφού είναι κανονική υποομάδα. Άρα $\sigma^{-1} \tau \sigma (a) = a$. Συνεπώς, $\tau (b) = \sigma (a) = b$, δηλαδή $b \in F^H = L$. Καθώς το $Irr(a,F)$ διασπάται πλήρως στο $K$, με αυτό δείξαμε ότι διασπάται πλήρως στο $L$. Από την πρόταση $\ref{3.28}$ έχουμε ότι η επέκταση $L/F$ είναι κανονική. Πράγματι, αν θεωρήσουμε τυχόν ανάγωγο $p(x) \in F[x]$ με ρίζα $a \in K$ αυτό θα ταυτίζεται με το $Irr(a,F)$ που διασπάται πλήρως στο $K$ και όπως δείξαμε στο $L$. Επιπλέον, καθώς $K/F$ είναι διαχωρίσιμη και $L \subseteq K$ έχουμε $L/F$ διαχωρίσιμη από τον ορισμό. Άρα η επέκταση $L/F$ είναι \tl{Galois}.

	Αντίστροφα, έστω ότι η επέκταση $L/F$ είναι \tl{Galois}. Έστω $\theta : G \rightarrow Gal(L/F)$ με τον κανόνα $\theta (\sigma) = \sigma|_L$. Η κανονικότητα της $L/F$ μας δίνει ότι $\sigma|_L \in Gal(L/F)$ από την πρόταση $\ref{3.28}$, άρα ο $\theta$ είναι καλά ορισμένος ομομορφισμός ομάδων. Έχουμε:
	$$ker \theta = \{\sigma \in K: \quad \sigma|_L = id\} = Gal(K/L) = H$$
	και άρα η ομάδα $H$ είναι κανονική στην $G$. Η απεικόνιση $\theta$ είναι επί, καθώς αν $\tau \in Gal(L/F)$ το θεώρημα επέκτασης ισομορφισμών μας δίνει $\sigma \in G$ με $\sigma|_L = \tau$. Άρα $Gal(L/F) \cong G/H$ από το πρώτο θεώρημα ισομορφισμών. 
\end{proof}
\begin{example}
θα μελετήσουμε τις αντιστοιχίες των υποσωμάτων του $K$ και των υποομάδων του $G=Gal(K/ \Q )$, όπου $K \subseteq \mathbb C$ είναι το σώμα ριζών του $f(x) = x^4 - 2 \in \Q (x)$.
\end{example}

Το $f(x)$ γράφεται στο $\mathbb C$ ως $(x-\rho )(x+\rho )(x-i\rho )(x+i\rho )$ όπου $\rho = \sqrt[4]2$. Δηλαδή $K = \Q (i,\rho )$ και η επέκταση $K/\Q$ είναι κανονική και διαχωρίσιμη, δηλαδή \tl{Galois}. Έχουμε:
$$[\Q (\rho ): \Q] = 4$$
$$[\Q (\rho, i) : \Q(\rho )]=2 $$
επειδή $x^2 +1$ ανάγωγο στο $\Q (\rho) \subseteq \mathbb R$ και άρα από θεώρημα πύργων μαζί με το ότι η επέκταση είναι \tl{Galois} παίρνουμε:
$$[K:\Q ]  = 8 = |Gal(K/\Q )|$$

Αν $\sigma \in G$ τότε από τις σχέσεις $\rho^4 = 2$ και $i^2 = -1$ έχουμε ότι $\sigma (\rho) \in \{ \rho, -\rho , i\rho , -i\rho \}$ και $\sigma (i) \in \{i,-i\}$ και για κάθε επιλογή έχουμε τους $4\cdot 2 = 8$ $\Q$-αυτομορφισμούς του $K$.

Έστω $\sigma , \tau$ που ορίζονται από:
$$\sigma (i) = i, \quad \sigma (\rho ) = i\rho $$
$$\tau (i) = -i, \quad \tau(\rho ) = \rho $$

Τότε με πράξεις έχουμε ότι $\sigma ^4 = \tau^2 = 1_G$ και $(\sigma \tau )^2 = 1_G$. Δηλαδή έχουμε: 
$$ Gal(K, \Q) = \{1, \sigma , \sigma^2 , \sigma ^3 , \tau , \tau\sigma , \tau\sigma^2 , \tau\sigma^3 \} \cong D_4$$
όπου $D_4$ είναι η διεδρική ομάδα 8 στοιχείων. 

$ $\newline
Οι υποομάδες της $D_4$ εκτός από την τετριμμένη και την ίδια είναι οι:
$$\{1,\tau \}, \{1,\tau\sigma^2 \} , \{1,\sigma^2 \} , \{1,\tau\sigma \} , \{1,\tau\sigma^3\}$$
$$\{1,\tau,\sigma^2,\tau\sigma^2\},\{1,\sigma,\sigma^2,\sigma^3\},\{1,\sigma^2 , \tau\sigma , \tau\sigma^3\}$$

Στην συνέχεια υπολογίζοντας τα σταθερά σώματα των υποομάδων παίρνουμε το διάγραμμα υποσωμάτων από το διάγραμμα υποομάδων.
Υπολογίζουμε ενδεικτικά τα σταθερά σώματα των $\{1,\sigma^2\}$ και $\{1,\tau\sigma \}$.

$ $\newline
Έχουμε $\sigma^2 (i) = i, \sigma^2 (\rho) = \sigma(i\rho) = i(i\rho) = -\rho$ και άρα $\sigma^2 (\rho^2) = \rho^2$. Συνεπώς $\Q (i, \rho^2) \subseteq F^{\{1,\sigma^2\}}$ και από θεμελιώδες θεώρημα $[F^{\{1,\sigma^2\}} : \Q ] = [G:\{1,\sigma^2\}] = 4$. Καθώς $i \not\in \Q (\rho^2 )$ και $\rho^2 \not\in \Q$ έχουμε $[\Q (i, \rho^2) : \Q] = 4$. Δηλαδή $F^{\{1,\sigma^2\}} = \Q (i, \rho^2)$.

$ $\newline Επιπλέον, $\tau\sigma ((1-i)\rho) = \tau ( \sigma (1-i) \sigma(\rho)) = \tau (1-i) \tau (i) \tau (r) = (1+i)(-i)\rho = (1-i)\rho$. Δηλαδή όμοια με πριν $F^{\{1,\tau\sigma \}} = \Q ((1-i)\rho)$.
Κάνοντας όλους τους υπολογισμούς έχουμε το διάγραμμα υποομάδων της $D_4$:


\begin{center}
	\begin{tikzcd}
		&                                                                             & 1                                                                           &                                                                 &                                                                \\
{\{1,\tau \}} \arrow[rru, no head] \arrow[rd, no head] & {\{1,\tau\sigma^2 \}} \arrow[ru, no head] \arrow[d, no head]                & {\{1,\sigma^2 \}} \arrow[u, no head] \arrow[d, no head] \arrow[rd, no head] & \{1.\tau\sigma\} \arrow[lu, no head] \arrow[d, no head]         & {\{1,\tau\sigma^3 \}} \arrow[llu, no head] \arrow[ld, no head] \\
		& {\{1,\tau,\sigma^2,\tau\sigma^2 \}} \arrow[ru, no head] \arrow[rd, no head] & {\{1,\sigma,\sigma^2,\sigma^3 \}} \arrow[d, no head]                        & {\{1,\sigma^2 ,\tau\sigma ,\tau\sigma^3 \}} \arrow[ld, no head] &                                                                \\
		&                                                                             & G                                                                           &                                                                 &                                                               
\end{tikzcd}
\end{center}

το οποίο το αντιστοιχούμε σε διάγραμμα υποσωμάτων του $K$:

\begin{center}
	\begin{tikzcd}
		&                                                      & K                                                                           &                                                       &                                                           \\
\Q (\rho) \arrow[rru, no head] \arrow[rd, no head] & \Q (i\rho ) \arrow[ru, no head] \arrow[d, no head]   & {\Q (\rho^2 ,i )} \arrow[u, no head] \arrow[d, no head] \arrow[rd, no head] & \Q ((1-i)\rho) \arrow[lu, no head] \arrow[d, no head] & \Q ( (1+i)\rho ) \arrow[llu, no head] \arrow[ld, no head] \\
		& \Q (\rho^2 ) \arrow[ru, no head] \arrow[rd, no head] & \Q (i) \arrow[d, no head]                                                   & \Q (i\rho^2 ) \arrow[ld, no head]                     &                                                           \\
		&                                                      & F                                                                           &                                                       &                                                          
\end{tikzcd}
\end{center}




















\pagebreak
\section{Γενική Τοπολογία}


%https://eclass.uoa.gr/modules/document/?course=MATH491


$ $\newline
Θα υπενθυμίσουμε όρους της τοπολογίας χωρίς να εμβαθύνουμε ιδιαίτερα, τους οποίους θα χρειαστούμε στο επόμενο κεφάλαιο.

\begin{defn}[Τοπολογία] Έστω $X \neq \varnothing$ σύνολο. Μια τοπολογία $\mathcal T$ στο $X$ είναι ένα υποσύνολο $\mathcal T \subseteq \mathcal P (X)$ ώστε:
	\begin{enumerate}
		\item $\varnothing, X \in \mathcal T$.
		\item Αν $\{A_i: i \in I\} \subseteq \mathcal T$, όπου $I$ τυχόν σύνολο δεικτών, τότε $\cup_{i \in I} A_i \in \mathcal T$.
		\item Αν $n\in \mathbb N$ και $A_1, \ldots A_n \in \mathcal T$ τότε $\cap_{k=1}^n A_k \in \mathcal T$.
	\end{enumerate}
\end{defn}

$ $\newline
\noindent Το ζεύγος $(X,\mathcal T)$ ονομάζεται τοπολογικός χώρος και τα μέλη της $\mathcal T$ ονομάζονται ανοικτά υποσύνολα του $X$. Επιπλέον, ονομάζουμε ένα υποσύνολο $F\subseteq X$ κλειστό αν $F^C$ ανοικτό.

$ $\newline
\noindent Αν $\mathcal P (x) = \mathcal T$ ονομάζουμε αυτήν την τοπολογία διακριτή, καθώς όλα τα υποσύνολα είναι ανοικτά και αυτό προέρχεται από την διακριτή μετρική.


\begin{defn}[Κλειστή θήκη]
	Έστω $(X,\mathcal T)$ τοπολογικός χώρος και $A \subseteq X$. Η κλειστή θήκη ή κλειστότητα του $A$ είναι το σύνολο:
	$$\overline{A} = \cap \{B: B \text{ κλειστό και } A\subseteq B\}$$
\end{defn}
\vspace{0.1cm}
\begin{prop}
	Έστω $(X,\mathcal T)$ και $A \subseteq X$. Τότε $x\in \overline A \iff$ για κάθε ανοικτό $U \subseteq X$ με $x \in U$ ισχύει $U\cap A \neq \varnothing$.
\end{prop}

\vspace{0.1cm}

\begin{defn}[Εσωτερικό]
	Έστω $(X,\mathcal T)$ τ.χ. και $A \subseteq X$. Τότε ορίζουμε ως εσωτερικό του $A$ το:
	$$A^{\circ} = \cup \{B: B\subseteq A \text{ και } B \in \mathcal T\}$$
	το οποίο είναι ανοικτό και τα σημεία του λέγονται εσωτερικά σημεία του $A$.
\end{defn}

$ $\newline
\noindent Επιπλέον, λέμε ότι το $A$ είναι περιοχή του $x \in X$ αν $x \in A^{\circ}$. Όμοια λέμε ανοικτή περιοχή αν $A=A^{\circ}$ δηλαδή αν $A \in \mathcal T$.

\begin{defn}[Βάση]
	Έστω $(X,\mathcal T)$ τοπολογικός χώρος. Μια οικογένεια $\mathcal B \subseteq \mathcal T$ λέμε ότι έιναι βάση για την $\mathcal T$ αν κάθε ανοικτό μη κενό σύνολο είναι ένωση στοιχείων της $\mathcal B$.
\end{defn}

\vspace{0.1cm}

\begin{defn}[Τοπολογία περιορισμός]
	Έστω $(X,\mathcal T)$ τ.χ. και $A \subseteq X$. Τότε η οικογένεια
	$$\mathcal{T}_A = \{A\cap U: \quad U \in \mathcal{T} \}$$
	ονομάζεται τοπολογία περιορισμός στο $A$ ή σχετική τοπολογία του $A$. Ο χώρος $(A, \mathcal T_A)$ είναι τοπολογικός υπόχωρος του $X$.
\end{defn}

\vspace{0.1cm}

\begin{defn}[Συνέχεια]
	Εστω $X,Y$ τοπολογικοί χώροι και $f: X \rightarrow Y$ συνάρτηση και $a \in X$. Λέμε ότι η $f$ είναι συνεχής στο $a$ αν για κάθε περιοχή $V$ του $f(a)$ υπάρχει περιοχή $U$ του $a$ στον $X$ ώστε $f(U) \subseteq V$.
\end{defn}

\vspace{0.1cm}

\begin{prop}[Χαρακτηρισμός συνέχειας] Έστω $X,Y$ τ.χ. και $f:X \rightarrow Y$ συνάρτηση. Τα ακόλουθα είναι ισοδύναμα:
	\begin{enumerate}
		\item Η $f$ είναι συνεχής.
		\item Για κάθε ανοικτό $V \subseteq Y$, το $f^{-1} (V) $ είναι ανοικτό στον $X$.
		\item Για κάθε κλειστό $F \subseteq Y$, το $f^{-1}(F)$ είναι κλειστό στον $X$.
		\item Για κάθε $A \subseteq X$ ισχύει ότι $f(\overline{A}) \subseteq \overline{f(A)}$.
	\end{enumerate}
\end{prop}

\vspace{0.1cm}

\begin{defn}[Ομοιομορφισμός] Μια συνάρτηση $f$ μεταξύ δύο τοπολογικών χώρων $X,Y$ λέγεται ομοιομορφισμός αν είναι 1-1, επί και αν οι $f, f^{-1}$ είναι συνεχείς. Τότε οι χώροι $X,Y$ λέγονται ομοιομορφικοί και στην ουσία ταυτίζονται ως τοπολογικοί χώροι.
\end{defn}

\vspace{0.1cm}

\begin{defn}[Καρτεσιανό Γινόμενο]
	Έστω $\{X_i: i \in I\}$ οικογένεια συνόλων. Το καρτεσιανό γινόμενο $\prod\limits_{i \in I} X_i$ είναι το σύνολο όλων των συναρτήσεων $f: I \rightarrow \cup_{i \in I} X_i$ με $f(i) \in X_i$ για κάθε $i \in I$. Αν $X_i = X$ για κάθε $i \in I$ τότε συμβολίζεται το καρτεσιανό γινόμενο και ως $X^I$. Ένα στοιχείο $f \in \prod\limits_{i \in I} X_i$ γράφεται ως $(x_i)_{i \in I}$ ή $(x_i)$, δηλαδή $f(i) = x_i$ για κάθε $i \in I$. Με αυτόν τον συμβολισμό το $x_j \in X_j$ ονομάζεται $j$-συντεταγμένη του $(x_i)$ και το $X_j$ είναι ο $j$-παράγοντας του $\prod\limits_{i \in I} X_i$. Επιπλέον, για κάθε $j \in I$ η απεικόνιση $\pi_j : \prod\limits_{i \in I} X_i \rightarrow X_j$ με κανόνα $\pi_j ( (xi)) = x_j$ ονομάζεται η προβολή στην $j$-συντεταγμένη.
\end{defn}


$ $\newline
\noindent Υπενθυμίζουμε και το αξιώμα επιλογής, στο οποίο βασιζόμαστε για να ορίσουμε την τοπολογία γινόμενο.

\begin{prop}[Αξίωμα Επιλογής]
	Έστω $\{X_i: i \in I\}$ μη κενή οικογένεια μη κενών συνόλων, τότε $\prod\limits_{i \in I} X_i \neq \varnothing$. Δηλαδή, $I\neq \varnothing$ και $X_i \neq \varnothing$ για κάθε $i \in I \implies \prod\limits_{i \in I} X_i \neq \varnothing$.
\end{prop}

\vspace{0.1cm}

\begin{defn}[Τοπολογία γινόμενο]
	Έστω $(X_i, \mathcal{T}_i)$ οικογένεια τοπολογικών χώρων, η τοπολογία γινόμενο ή καρτεσιανή τοπολογία επί του $X = \prod\limits_{i \in I} X_i$ είναι εκείνη η (μοναδική) τοπολογία η οποία έχει σαν βάση την οικογένεια:
	$$B = \big\{\prod\limits_{i \in I} U_i: \quad U_i \in \mathcal{T}_i \text{ και } \{i \in I: U_i \neq X_i\} \text{ πεπερασμένο } \big\}$$
\end{defn}

\vspace{0.1cm}

\begin{prop}
	Η προβολή $\pi_j : \prod\limits_{i \in I} X_i \rightarrow X_j$ είναι συνεχής απεικόνιση και επιπλέον στέλνει τα ανοικτά σε ανοικτά, για κάθε $j \in I$. 
\end{prop}

\vspace{0.1cm}

\begin{defn}[Τοπολογία πηλίκο]
	Έστω $Y \neq \varnothing$ σύνολο, $X$ τοπολογικός χώρος και $f : X \rightarrow Y$ μια απεικόνιση του $X$ στο $Y$. Η τοπολογία πηλίκο στο $Y$ η οποία ορίζεται από την $f$ είναι η
	$$\mathcal T(f) = \{ U \subseteq Y: f^{-1} (U) \text{ ανοικτό στον } X\}$$
	μπορούμε και να ερμηνεύσουμε την $f$ ως την κανονική απεικόνιση $ x\in X \mapsto [x] \in X/ \sim$ της σχέσης ισοδυναμίας του $X$: $x \sim y \iff f(x) = f(y)$ ταυτίζοντας το $Y$ με τον χώρο των κλάσεων ισοδυναμίας $X/\sim$. Οι κλάσεις έχουν μορφή $[x] = f^{-1}( f(x))$ και ένα ανοικτο υποσύνολο του $X/ \sim$ είναι μια οικογένεια κλάσεων ισοδυναμίας των οποίων η ένωση είναι ανοικτό υποσύνολο του $X$.
\end{defn}

\vspace{0.1cm}

\begin{defn}[\tl{Hausdorff}]
	Έστω $X$ τοπολογικός χώρος. Ο $X$ λέγεται \tl{Hausdorff} ή $T_2$ αν για κάθε ζεύγος $x\neq y$ του $X$ υπάρχουν ξένα ανοικτά σύνολα $U,V$ τέτοια ώστε $x \in U$ και $y \in V$.
\end{defn}

\vspace{0.1cm}

\begin{defn}[Συμπάγεια]
	Εστω $X$ τ.χ. και $K \subseteq X$. Ο $X$ λέγεται συμπαγής, αν κάθε ανοικτό κάλυμμα $\{U_i: i \in I\}$ του $X$ έχει πεπερασμένο υποκάλλυμα. Δηλαδή αν
	$$X = \bigcup\limits_{i \in I} U_i \implies \exists F \subseteq I \text{ πεπερασμένο: } X = \bigcup\limits_{i \in F} U_i$$

	Το $K$ λέγεται συμπαγές υποσύνολο του $X$ αν είναι συμπαγής υπόχωρος του $X$ με την σχετική τοπολογία.
\end{defn}

\vspace{0.1cm}

\begin{prop} Έστω $\{X_i : i \in I\}$ οικογένεια τοπολογικών χώρων. Τότε  ο $X = \prod\limits_{i \in I} X_i$ είναι \tl{Hausdorff} αν και μόνο αν κάθε $X_i$ είναι \tl{Hausdorff}.
\end{prop}

\vspace{0.1cm}

\begin{prop}[Θ. \tl{Tychonoff}] Έστω $\{X_i : i \in I\}$ οικογένεια τοπολογικών χώρων. Τότε  ο $X = \prod\limits_{i \in I} X_i$ είναι συμπαγής αν και μόνο αν κάθε $X_i$ είναι συμπαγής.
\end{prop}

\vspace{0.1cm}

\begin{defn}[Συνεκτικότητα]
	Ένας τοπολογικός χώρος $X$ λέγεται συνεκτικός αν δεν είναι ένωση δύο ξένων ανοικτών μη κενών υποσυνόλων του. Δηλαδή, δεν υπάρχουν ανοικτά $U,V \neq \varnothing$ ώστε $U\cap V = \varnothing$ και $X = U\cup V$.

	$ $\newline
	Ένα υποσύνολο $A\subseteq X$ λέγεται συνεκτικό υποσύνολο αν είναι συνεκτικό ως υπόχωρος του $X$ με την σχετική τοπολογία.
\end{defn}

\vspace{0.1cm}

\begin{defn}[\tl{Totally Disconnected}]
	Ένας τοπολογικός χώρος $X$ λέγεται \tl{totally disconnected} αν τα μόνα συνεκτικά υποσύνολα του $X$ είναι τα μονοσύνολα.
\end{defn}


















\pagebreak
\section{Άπειρη θεωρία \tl{Galois}}
\vspace{1cm}

Εδώ ξεκινάμε να χτίζουμε την θεωρία για τις άπειρες επεκτάσεις που θα μας απασχολήσουν. Για όλη την ενότητα θα βασιστούμε αρκέτα στους ακόλουθους συμβολισμούς.

$ $\newline
\noindent Έστω $K/F$ \tl{Galois} επέκταση, τότε συμβολίζουμε:

$$G=Gal(K/F)$$
$$\mathcal{I} = \{ E: K/E/F , \quad [E:F]<\infty , \quad E/F \quad\text{ \tl{Galois} } \}$$
$$\mathcal{N} = \{N \subseteq G: N = Gal(K/E) \quad\text{ για κάποιo } E \in \mathcal{I} \}$$

Υπενθύμιση: Από λήμμα \ref{3.28}, αν $K/F$ κανονική επέκταση και $N/K/L/F$ σώματα με $\tau :L \mapsto N$ ένας $F$-ομομορφισμός, τότε $\tau(L) \subseteq K$ και υπάρχει $\sigma \in Gal(K/F)$ με $\sigma|_L = \tau$.
% proposition 3.28!


\begin{lemma} \label{17.1} Aν $a_1 , \ldots a_n \in K$ τότε υπάρχει $E \in \mathcal{I}$ με $a_i \in E$ για κάθε $i \in \{1,\dots, n\}$.
\end{lemma}
\begin{proof} $ $%Απόδειξη


%(Σχόλια: μεταβατικότητα κανονικότητας? μπορώ να θεωρήσω κατευθείαν το F(a_1,... a_n) ???)
$ $\newline
Έστω $E\subseteq K$ το σώμα ριζών των ελαχίστων πολυωνύμων των $a_i$ υπεράνω του $F$, δηλαδή το σώμα ριζών του γινομένου τους. Kάθε $a_i$ είναι διαχωρίσιμο υπεράνω του $F$, εφόσον είναι στοιχέιο της διαχωρίσιμης επέκτασης $K/F$, αρα το $E$ είναι κανονική επέκταση του $F$ και διαχωρίσιμη, επομένως η επέκταση $E/F$ είναι \tl{Galois}. Καθώς έχουμε πεπερασμένα $a_i$ τότε $[E:F]< \infty$, επομένως $E \in \mathcal{I}$.

\end{proof}

\begin{lemma} \label{17.2} Αν $N \in \mathcal{N}$ με $N = Gal(K/E)$, $E \in \mathcal{I}$ τότε $E=F^N$ και $N\unlhd G$. Τότε έχουμε τον ισομορφισμό $G/N \cong Gal(E/F)$ και επιπλέον $|G/N| = |Gal(E/F)| = [E:F] < \infty$.
\end{lemma}  
\begin{proof} $ $
 
$ $\newline
Το σώμα $K$ είναι κανονική και διαχωρίσιμη επέκταση υπεράνω του $F$ το οποίο συνεπάγεται ότι είναι και υπεράνω του $E$. Δηλαδή $K/E$ \tl{Galois} και συνεπώς $E = F^N$. Όπως στην απόδειξη του θεμελιώδους θεωρήματος της θεωρίας \tl{Galois}, η απεικόνιση $\theta : G \mapsto Gal(E/F)$ με κανόνα $\sigma \mapsto \sigma|_E$ είναι ένας ομομορφισμός ομάδων με πυρήνα $Gal(K/E) = N$. Από την υπενθύμιση της πρότασης \ref{3.28} έχουμε ότι το $\theta$ είναι επιμορφισμός. Τα υπόλοιπα έπονται από το 1ο θεώρημα ισομορφισμών ομάδων και ότι η επέκταση $E/F$ είναι \tl{Galois}.
\end{proof}

\begin{lemma} \label{17.3}  $\bigcap_{N \in \mathcal{N}} N= \{1_G \} = \{ id: K \mapsto K \}$. Επιπλέον, $\bigcap_{N \in \mathcal{N}} \sigma N = \{\sigma \}$ για κάθε $\sigma \in G$.
\end{lemma}
\begin{proof} $ $

$ $\newline	
Έστω $\tau \in \bigcap_{N \in \mathcal{N}} N$ και $a \in K$. Από το λήμμα \ref{17.1} υπάρχει  $E \in \mathcal{I}$ με $a \in E$. Έχουμε $N := Gal(K/E) \in \mathcal{N}$ εφόσον $E \in \mathcal{I}$. Ο αυτομορφισμός $\tau$ κρατάει σταθερό το $E$ καθώς $\tau \in N$, επομένως $\tau (a) = a$ για το τυχόν $a \in K$. Συνεπώς $\tau = id_K$ και άρα αυτό είναι το μοναδικό στοιχείο της τομής. Για το δεύτερο επιχείρημα, αν $\tau \in \sigma N$ για κάθε $N$ τότε $\sigma^{-1} \tau \in N$ για κάθε $N$, επομένως $\sigma^{-1} \tau = id_K$ και άρα $\tau = \sigma$ για το τυχόν $\tau \in \bigcap_{N \in \mathcal{N}} \sigma N$.
\end{proof}

\begin{lemma} \label{17.4} Αν $N_1 , N_2 \in \mathcal{N}$ τότε $N_1 \cap N_2 \in \mathcal{N}$.
\end{lemma}
\begin{proof} $ $
%the last condition is true iff \sigma in Gal(K/E1E2)

$ $\newline
Έστω $N_i = Gal(K/E_i)$ με $E_i \in \mathcal{I}$. Κάθε $E_i$ είναι πεπερασμένη επέκταση \tl{Galois} του $F$, επομένως το σώμα $E_1 E_2$ είναι και αυτό πεπερασμένη επέκταση \tl{Galois} του $F$, άρα $E_1 E_2 \in \mathcal{I}$. Ωστόσο, έχουμε ότι $Gal(K/E_1 E_2) = N_1 \cap N_2$. Πράγματι,
$$\sigma \in N_1 \cap N_2 \iff \sigma|_{E_1} = id \text{ και } \sigma|_{E_2}=id \iff E_1 \subseteq F^{(\sigma)} \text{ και } E_2 \subseteq F^{(\sigma)}$$
$$\iff E_1 E_2 \subseteq F^{(\sigma)}$$
όπου η τελευταία σχέση είναι ισοδύναμη με την $\sigma \in Gal(K/E_1 E_2)$. Επομένως $N_1 \cap N_2 = Gal(K/E_1 E_2) \in \mathcal{N}$.
\end{proof}

\noindent Τώρα θα ορίσουμε την τοπολογία στην ομάδα \tl{Galois} $G$.

\begin{defn}[Τοπολογία \tl{Krull}] $(G,\mathcal{T})$ είναι τοπολογικός χώρος όπου $\mathcal{T}$ είναι η τοπολογία \tl{Krull} που ορίζεται ως εξής:
Ένα υποσύνολο $X$ του $G$ είναι ανοιχτό αν $X=\varnothing$ ή $X= \cup_i \sigma_i N_i$ για κάποια $\sigma_i \in G$ και $N_i \in \mathcal{N}$.
\end{defn}

\noindent Βέβαια πρέπει να δείξουμε ότι πράγματι έχουμε μια τοπολογία. Από τον ορισμό τo $\varnothing$ είναι ανοιχτό και οι ενώσεις ανοιχτών είναι ανοιχτό σύνολο. Έχουμε ότι $F \in \mathcal{I}$ και άρα $G \in \mathcal{N}$, δηλαδή το $G$ μπορεί να γραφτεί ως ένωση εφόσον κάποιο $N_i = G$. Μένει να δείξουμε την κλειστότητα στις πεπερασμένες τομές.

$ $\newline
Έχουμε ότι: 
$$\left(\bigcup_i \sigma_i N_i \right) \cap \left(\bigcup_j \sigma_j N_j \right) = \bigcup_{i, j} \left( \sigma_i N_i \cap \sigma_j N_j \right)$$
και άρα αρκεί να δείξουμε ότι το $\tau_1 N_1 \cap \tau_2 N_2$ είναι ανοιχτό για κάθε $N_1, N_2 \in \mathcal{N}$. Πράγματι, έστω $\sigma \in \tau_1 N_1 \cap \tau_2 N_2$, τότε :

$$\tau_1 N_1 \cap \tau_2 N_2 = \sigma N_1 \cap \sigma N_2 = \sigma (N_1 \cap N_2)$$
και το $\sigma (N_1 \cap N_2)$ είναι ανοιχτό εφόσον $N_1 \cap N_2 \in \mathcal{N}$ από το λήμμα \ref{17.4}.

\subsection{Ιδιότητες της τοπολογίας \tl{Krull}}

$ $\newline
\noindent Εφόσον κάθε μη κενό ανοιχτό υποσύνολο του $G$ έχει οριστεί ως ένωση τότε το σύνολο:
$$\{\sigma N : \sigma \in G , N \in \mathcal{N} \}$$
είναι βάση της τοπολογίας.

$ $\newline
\noindent Αν τώρα $N \in \mathcal{N}$ τότε $|G:N|<\infty$ οπότε αν $S$ είναι ένα σύνολο αντιπροσώπων των συμπλόκων του $N$ τότε έχουμε:
$$G-\sigma N = \bigcup\limits_{a\in S, a\not\in \sigma N} aN$$
δηλαδή, το $G-\sigma N$ είναι πεπερασμένων ένωση συμπλόκων του $N$. Επομένως, το $\sigma N$ είναι και ανοιχτό και κλειστό (\tl{clopen}).
Καταλήξαμε στο ότι αυτή η τοπολογία έχει βάση από ανοιχτά κλειστά σύνολα.


\begin{prop} Ο τοπολογικός χώρος $(G,\mathcal{T})$ είναι \tl{Hausdorff}.
\end{prop}
\begin{proof}


Έστω $\sigma , \tau \in G, \sigma \neq \tau$.  Από το λήμμα \ref{17.3} έχουμε ότι 
$$\{\sigma \} = \bigcap_N \sigma N$$
δηλαδή υπάρχει $N \in \mathcal{N}$ έτσι ώστε $\tau \notin N \implies \tau \in G-\sigma N$. Τα $\sigma N , G-\sigma N$ είναι ανοιχτά και διαχωρίζουν τα $\sigma , \tau$.
\end{proof} 
\begin{prop} Ο τοπολογικός χώρος $(G, \mathcal{T})$ είναι \tl{totally disconnected}.
\end{prop}
\begin{proof} Έστω $X\subseteq G$ που περιέχει τουλάχιστον δύο στοιχεία $\sigma ,\tau$. Όμοια με την προηγούμενη απόδειξη, υπάρχει $\sigma N$ ανοιχτή περιοχή του $\sigma$ που δεν περιέχει το $\tau$. Συνεπώς:

$$X = \left( \sigma N \cap X \right) \bigcup \left( \left( G-\sigma N\right) \cap X\right)$$

\noindent δηλαδή το $X$ γράφεται ως ένωση ξένων, μη κενών ανοιχτών (της $\mathcal{T}_X$ ). Άρα τα μοναδικά συνεκτικά υποσύνολα του $G$ είναι μονοσύνολα.
\end{proof}

\noindent Στην συνέχεια ακολουθεί και η πιο σημαντική ιδιότητα της τοπολογίας \tl{Krull}, η οποία είναι και αρκετά πιο δύσκολη να αποδειχθεί.
\begin{prop}O τοπολογικός χώρος $(G, \mathcal{T})$ είναι συμπαγής.
\end{prop}
\begin{proof} $ $


$ $\newline
Θα δείξουμε ότι το $G$ μπορεί να κατασκευαστεί από πεπερασμένες \tl{Galois} ομάδες. Θεωρούμε τις ομάδες πηλίκο $G/N$ οι οποίες είναι πεπερασμένες (από το λήμμα \ref{17.2}) και θέτουμε

$$P = \prod\limits_{N \in \mathcal{N}} G/N$$
το ευθύ γινόμενό των ομάδων.

$ $\newline
\noindent Αν Θεωρήσουμε τους τοπολογικούς χώρους $(G/N, \mathcal{T}_{\delta})$, όπου $\mathcal{T}_{\delta}$ η διακριτή τοπολογία, μπορούμε να κάνουμε το $P$ τοπολογικό χώρο δίνοντάς του την τοπολογία γινόμενο. Στην συνέχεια, τα $G/N$ είναι πεπερασμένα και άρα συμπαγή. Άρα, από το θεώρημα \tl{Tychonoff} το $P$ είναι συμπαγής τοπολογικός χώρος. Επιπλέον, κάθε $G/N$ είναι \tl{Hausdorff} ως πεπερασμένο με διακριτή τοπολογία και η ιδιότητα \tl{Hausdorff} διατηρείται στο γινόμενο, άρα ο $P$ είναι επίσης \tl{Hausdorff}.

$ $\newline
\noindent Υπάρχει τώρα ένας φυσικός ομομορφισμός ομάδων:
$$f:G \longrightarrow P$$
$$\sigma \longmapsto \{\sigma N \} = \prod\limits_{N \in \mathcal{N}} \sigma N$$
Είναι πράγματι ομομορφισμός ομάδων εφόσον:
$$\sigma \circ \tau \longmapsto \prod\limits_{N \in \mathcal{N}} (\sigma \circ \tau) N$$

και 
$$f(\sigma)f(\tau) = \left( \prod\limits_{N \in \mathcal{N}} \sigma N \right) \left(\prod\limits_{N \in \mathcal{N}}\tau N \right) = \prod\limits_{N \in \mathcal{N}} (\sigma N)(\tau N) = \prod\limits_{N \in \mathcal{N}} (\sigma \circ \tau)N$$
όπου στην δεύτερη ισότητα ή πράξη γίνεται στο ευθύ γινόμενο ομάδων ΄κατά συντεταγμένη΄ και στην επόμενη ισότητα είναι η πράξη εξ΄ορισμού της ομάδας πηλίκο $G/N$.

$ $\newline
\noindent Στην συνέχεια θα δείξουμε ότι η $f$ είναι ομοιομορφισμός, αν θεωρήσουμε ως σύνολο άφιξης
την εικόνα της, και ότι η εικόνα της είναι κλειστό υποσύνολο του $P$. Από εκεί θα έπεται ότι η εικόνα θα είναι συμπαγής. Συνεπώς, μέσω του ομοιομορφισμού $f$ θα έχουμε δείξει το ζητούμενο.

$ $\newline
\noindent Έστω $f: G \rightarrow imf$ όπως παραπάνω και $\sigma \in G$ τέτοιο ώστε $\{\sigma N \} = \{ N\}$.

$$\sigma \in ker(f) \iff \{\sigma N\} = \{N \} \iff \sigma \in \bigcap_{N \in \mathcal{N}}N = \{id\}$$
όπου η τελευταία ισότητα ισχύει από το λήμμα \ref{17.3}. Συνεπώς, η $f$ είναι 1-1 και εξ΄ορισμού επί. 

$ $\newline
\noindent Έστω $\pi_N : P \rightarrow G/N$ η προβολή στον $N$-παράγοντα. Τότε $\pi_N (f(\sigma)) = \sigma N$ για κάθε $\sigma \in G$. Στη διακριτή τοπολογία στα $G/N$ η βάση αποτελείται από μονοσύνολα, δηλαδή στοιχεία της μορφής $\tau N$. Κάθε ανοιχτό υποσύνολο του $P$ είναι ένωση βασικών και από τον ορισμό της τοπολογίας γινόμενο, κάθε βασικό στοιχείο είναι πεπερασμένη τομή συνόλων της μορφής $\pi^{-1}_N (\tau N)$ για διάφορα $\tau \in G$ και $N \in \mathcal{N}$. 








$ $\newline
\noindent Θα δείξουμε πρώτα ότι η $f^{-1}$ είναι συνεχής, αρκεί η $f$ να είναι ανοιχτή, δηλαδή να στέλνει ανοιχτά σε ανοιχτά. Έστω $\sigma H$ ενα βασικό ανοιχτό, έχουμε $\sigma \in G , H \in N$ και άρα υπάρχει $E \in \mathcal{I}$ τέτοιο ώστε $H = Gal(K/E)$. Τότε:
$$f(\sigma H) = \{\left( \sigma h N\right)_{N \in \mathcal{N}}| \quad h \in H, h|_E = 1_E\} = \{\left( \sigma h N\right)_{N \in \mathcal{N}}| \quad h \in H, \sigma h|_E = \sigma|_E\}$$
$$= \{\left( \tau N\right)_{N \in \mathcal{N}}| \quad \tau|_E = \sigma|_E\} = \pi^{-1}_H (\sigma H)$$

όπου η τελευταία ισότητα ισχύει εφόσον:

$ $\newline 
Αν $(\tau N)_{N \in \mathcal{N}}$ με $\tau |_E = \sigma |_E$ τότε έστω $x \in E$, έχουμε: $\sigma^{-1} \tau (x) = \sigma^{-1} \sigma (x) = x$ δηλαδή το $\sigma^{-1} \tau$ κρατάει σταθερό το $E$ αν και μόνο αν $\sigma^{-1} \tau \in H \iff \sigma^{-1} \tau H = H \iff \sigma H = \tau H$. Άρα αν πάρουμε την προβολή $\pi_H \left( \left(\tau N \right)_{N \in \mathcal{N}}\right) = \tau H = \sigma H$. Έχουμε συνεπώς την μια σχέση του περιέχεσθαι.

$ $\newline
\noindent Αντίστροφα, αν $(\tau N)_{N \in \mathcal{N}}$ τέτοιο ώστε:
$$\tau H = \pi_H \left( \left(\tau N \right)_{N \in \mathcal{N}}\right) = \sigma H$$
$$\tau H = \sigma H$$
και $x \in E$ τότε $\sigma h_1 (x) = \tau h_2 (x) \implies \sigma (x) = \tau (x)$ και άρα $\sigma|_E = \tau|_E$.
Έχουμε από ορισμό της τοπολογίας γινόμενο ότι $\pi^{-1}_H (\sigma H)$ ανοιχτό (στο $P$) και $f(\sigma H) \subseteq imf$ άρα $f(\sigma H) = \pi^{-1}_H (\sigma H) \cap imf$ ανοιχτό στο $imf$.

$ $\newline
\noindent Με βάση τα προηγούμενα, για να δείξουμε ότι η $f$ αντιστρέφει ανοιχτά σε ανοιχτά αρκεί να ισχύει ότι το $f^{-1} (\pi^{-1}_N (\sigma H))$ είναι ανοιχτό στο $G$ για κάθε $\sigma H$. Πράγματι:
$$f^{-1} (\pi^{-1}_H (\sigma N)) = f^{-1} (\{(\tau N)_{N \in \mathcal{N}} | \quad \tau|_E = \sigma|_E \}) = \sigma H$$
το οποίο είναι ανοιχτό.

$ $\newline
\noindent Μένει να δείξουμε ότι η εικόνα $imf$ είναι κλειστή στο $P$. Εδώ αντί για $G/N$ θα χρησιμοποιούμε το ισόμορφό του $Gal(E_N / F)$ με $E_N = F^N$ με βάση το λήμμα \ref{17.2}. Έτσι, θα αναγνωρίζουμε το σύμπλοκο $\tau N$ ως $\tau|_{E_N}$. Με αυτή τη σύμβαση, αν $p \in P$ δηλαδή $p = (\tau_N N)_{N}$ τότε $\pi_N (p) = \tau_N N = \tau_N |_{E_N}$ είναι ένας αυτομορφισμός του $E_N$. Θέτουμε:
$$C = \{ p \in P: \forall N,M \in \mathcal{N} ,\quad \pi_N (p) |_{E_N \cap E_M} = \pi_M(p)|_{E_N \cap E_M}\}$$

$ $\newline
\noindent Θα δείξουμε ότι $C=imf$ . Για την κατεύθυνση $imf \subseteq C$ έχουμε ότι:
$$\pi_N (f(\tau))|_{E_N} =\pi_N [(\tau N)_{N \in \mathcal{N}}]|_{E_N} = (\tau N) |_{E_N} = (\tau|_{E_N})|_{E_N} = \tau|_{E_N}$$
για κάθε $\tau \in G$ . Άρα:
$$\pi_N (f(\tau))|_{E_N \cap E_M}=(\tau|_{E_N})|_{E_N \cap E_M} = \tau|_{E_N \cap E_M} = (\tau|_{E_M})|_{E_N \cap E_M} =\pi_M (f(\tau))|_{E_N \cap E_M}$$
δηλαδή για κάθε $\tau \in G$ ισχύει ότι $f(\tau) \in C$.

$ $\newline
\noindent Αντίστροφα, έστω $p \in C$. Ορίζουμε $\tau : K \rightarrow K$ τέτοια ώστε αν $a \in K$ διαλέγουμε ένα $E_N \in \mathcal{I}$ με $a \in E_N$, γνωρίζουμε ότι υπάρχει τέτοιο από το λήμμα $\ref{17.1}$, έτσι ώστε $a \mapsto \pi_N (p) (a)$. Για να είναι καλά ορισμένη απεικόνιση πρέπει να μην εξαρτάται από την επιλογή του $E_N$ και αυτό ακριβώς μας παρέχει η συνθήκη του $p \in C$.  Δηλαδή, διαλέγουμε $E_N, E_M$ τέτοια ώστε $a \in E_N, E_M \implies a \in E_N \cap E_M$ και άρα εφόσον $p \in C$ ισχύει ότι:
$$\pi_N (p) (a) =  \pi_M (p) (a)$$
 
$ $\newline
\noindent Το $\tau$ είναι και ομομορφισμός δακτυλίων. Πράγματι, αν $a,b \in K$ και έστω $E_N \in I$ με $a,b \in E_N$ τότε το $\tau$ δρα κατάλληλα στα $a,b$ μέσω του ομομορφισμού $\tau|_{E_N} = \pi_N (p)$.

$ $\newline
\noindent Επιπλέον είναι 1-1 και επί εφόσον μπορούμε μέσω του $p^-1$ να κατασκευάσουμε το $\tau^{-1}$, δηλαδή:
$$\pi_N (p^{-1}) (a) = (\pi_N (p))^{-1} (a) = \tau^{-1} (a)$$


$ $\newline
\noindent Στην συνέχεια, αν το $x$ ανήκει στο στο αρχικό σώμα $F$ που έχουμε θεωρήσει στην αρχή του κεφαλαίου, διαλέγουμε $E_N \in \mathcal{I}$ με $x \in E_N$ όμοια με πριν και άρα το $\pi_N (p)$ είναι εξ ορισμού στοιχείο του $G$. Δηλαδή είναι $F$-ισομορφισμός του $K$ περιορισμένος στο $E_N$. Άρα έχουμε ότι $\pi_N (p) \in Gal(E_N/F)$ και συνεπώς $\tau \in G$.

$ $\newline
\noindent Έτσι καθώς έχουμε $\tau |_{E_N}=\pi_N (p)$ ισχύει ότι:
$$f(\tau) = (\tau N)_{N \in \mathcal{N}} = (\tau|_{E_N})_{N \in \mathcal{N}} = (\pi_N (p))_{N \in \mathcal{N}} = p$$
δηλαδή $p \in imf \implies C=imf$.

$ $\newline
\noindent Για την κλειστότητα, έστω $p \in P\setminus C$. Δηλαδή, υπάρχουν $N,M \in \mathcal{N}$ τέτοια ώστε:
$$\pi_N (p) |_{E_N \cap E_M} \neq \pi_M (p) |_{E_N \cap E_M}$$ 

\noindent Για το σύνολο:

$$X = \pi^{-1}_N \left(\pi_N \left( p \right) \right) \bigcap \pi^{-1}_M \left(\pi_M \left( p \right) \right)$$
έχουμε ότι περιέχει το $p$ και ότι είναι ανοιχτό υποσύνολο του $P$ ως πεπερασμένη τομή ανοιχτών, από ορισμό προβολών στην τοπολογία γινόμενο. Αν $x \in X$, τότε παίρνουμε τις προβολές $\pi_N(x) = \pi_N (p)$ και $\pi_M (x) = \pi_M (p)$ τα οποία δεν είναι ίσα καθώς παραπάνω φαίνεται ότι δεν ταυτίζονται στον περιορισμό στο $E_N \cap E_M$. Δηλαδή το $X$ περιέχεται εξ'ολοκλήρου στο $P$ και συνεπώς είναι ανοιχτή περιοχή του τυχαίου $p \in P\setminus C$. Καταλήξαμε στο ότι $P\setminus C$ ανοιχτό, ισοδύναμα $C$ κλειστό.
\end{proof}

$ $\newline
\noindent Το επόμενο θεώρημα είναι το τελευταίο βήμα που χρειαζόμαστε για να επεκτείνουμε το θεμελιώδης θεώρημα σε άπειρες επεκτάσεις \tl{Galois}. Εδώ θα φανεί πως χρησιμοποιείται η τοπολογία στο $G$ και έρχεται σε αναλογία με την πρόταση ότι αν $G$ είναι μια πεπερασμένη ομάδα αυτομορφισμών του $K$ τότε $G = Gal(K/F^G)$.

\begin{theorem}Έστω $H$ υποομάδα της $G$ και έστω $H^{\prime} = Gal(K/F^H)$. Τότε $H^{\prime} = \overline{H}$, η κλειστή θήκη του $H$ στην τοπολογία του $G$.
\end{theorem}
\begin{proof} $ $


$ $\newline
Από τον ορισμό του σταθερού σώματος έχουμε ότι $H \subseteq H^{\prime}$. Αρκεί να δείξουμε ότι το $H^{\prime}$ είναι κλειστό και ότι $H^{\prime} \subseteq \overline H$. 

$ $\newline
\noindent Έστω $\sigma \in G - H^{\prime}$. Τότε υπάρχει $a \in F^H$ τέτοιο ώστε $\sigma (a) \neq a$. Παίρνουμε $E \in \mathcal{I}$ με $a \in E$ και θεωρούμε την ομάδα $N = Gal(K/E) \in \mathcal{N}$. Για κάθε $\tau \in N$ έχουμε $\tau (a) = a$ εφόσον κρατάνε οι ισομορφισμοί σταθερό το $E$ και έτσι $\sigma \tau (a) = \sigma (a) \neq a$. Δηλαδή, το $\sigma N$ είναι ανοιχτή περιοχή του $\sigma$ και ξένη με το $H^{\prime}$. Συνεπώς το $G - H^{\prime}$ είναι ανοιχτό και άρα το $H^{\prime}$ κλειστό. 

$ $\newline
\noindent Για να δείξουμε ότι $H^{\prime} \subseteq \overline H$, έστω $\sigma \in H^{\prime}$ με $N \in \mathcal{N}$ και $E = F^N \in \mathcal{I}$. Ορίζουμε:

$$H_0 = \{p|_E : p \in H\} \leq Gal(E/F)$$
όπου είναι πράγματι υποομάδα της πεπερασμένης $Gal(E/F)$ εφόσον οι αυτομορφισμοί της είναι αυτομορφισμοί της $H \subseteq G$ που κρατάνε σταθερό το $F$ και είναι περιορισμένοι στο $E$.
Έχουμε:

$$F^{H_0} = \{ a \in K : p|_E (a) = a \quad\forall p|_E \in H_0\}= E \cap \{a 
\in K : p (a) = a \quad\forall  p \in H\}= E \cap F^H$$

$ $\newline
Από αντιστοιχία \tl{Galois} για την πεπερασμένη $Gal(E/F)$ έχουμε $H_0 = Gal(E/(E\cap F^H))$.
\begin{center}
	\begin{tikzcd}
		1 \arrow[d, no head]                                    &  &  & E \arrow[d, no head]                                            &  &  & 1 \arrow[d, no head]   \\
		H_0 \arrow[d, no head] \arrow[rrr, "H \longmapsto F^H"] &  &  & F^{H_0} \arrow[d, no head] \arrow[rrr, "L\longmapsto Gal(E/L)"] &  &  & H_0 \arrow[d, no head] \\
		Gal(E/F)                                                &  &  & F                                                               &  &  & Gal(E/F)              
		\end{tikzcd}
\end{center}
%(αντιστοιχία  σχήμα 1-$H_0$ - $Gal(E/F)$ μέσω της $Gal(E, \_ )$ στον πύργο $E$ - $F^{H_0} - F$ )

$ $\newline
\noindent Αν $\sigma \in Gal(K/F^H)$, τότε $\sigma|_{F^H} = id$ δηλαδή το $\sigma$ κρατάει σταθερό το $E \cap F^H \subseteq F^H$ και άρα αν το περιορίσουμε στο $E$ έχουμε:
$$\sigma|_E \in Gal(E/ (E\cap F^H)) = H_0$$

$ $\newline
\noindent Από ορισμό $H_0$ υπάρχει $p \in H$ με $p|_E = \sigma|_E$. Δηλαδή $\sigma^{-1} p|_E = 1_E$.
Άρα έχουμε:

$$\sigma^{-1} p \in Gal(K/E) = N \implies p \in \sigma N \cap H$$

$ $\newline
\noindent Δηλαδή, αφού το $N$ ήταν τυχόν, κάθε βασική ανοιχτή περιοχή $\sigma N$ του $\sigma \in H^{\prime}$  τέμνει το $H$, το οποίο είναι ισοδύναμο από χαρακτηρισμό κλειστής θήκης ότι $\sigma \in \overline H$.

\end{proof}

\subsection{Θεμελιώδες Θεώρημα της Άπειρης Θεωρίας \tl{Galois}}
\vspace{0.3truecm}

\begin{theorem}[Θεμελιώδες Θεώρημα της Άπειρης Θεωρίας \tl{Galois}]

Έστω $K/F$ \tl{Galois} επέκταση και $G=Gal(K/F)$. Με την \tl{Krull} τοπολογία στο $G$
οι απεικονίσεις $L \mapsto Gal(K/L)$ και $H \mapsto F^H$ είναι 1-1 και εμφυτεύουν τα σύνολα:

$$\{L : K/L/F \} \longleftrightarrow \{H \leq G: H = \overline H\}$$
το ένα στο άλλο με την ανάποδη αντιστοιχία. 
Δηλαδή αν $H$ κλειστό και $K/L/F$ έχουμε:
\begin{center}
	\begin{tikzcd}
		K \arrow[d, no head]                                        &  &  & 1 \arrow[d, no head]          & 1 \arrow[d, no head]                                 &  &  & K \arrow[d, no head]   \\
		L \arrow[d, no head] \arrow[rrr, "{L\longmapsto Gal(K,L)}"] &  &  & {Gal(K,L)} \arrow[d, no head] & H \arrow[d, no head] \arrow[rrr, "F\longmapsto F^H"] &  &  & F^H \arrow[d, no head] \\
		F                                                           &  &  & G                             & G                                                    &  &  & F                     
		\end{tikzcd}
\end{center}
\vspace{0.3truecm}
% (αντιστοχία σχήμα)

$ $\newline
Επιπλέον, αν $L \longleftrightarrow H$ τότε $|G:H| < \infty \iff [L:F] < \infty$, αν και μόνο αν το $H$ είναι ανοιχτό στην τοπολογία. Όταν αυτό συμβαίνει, ισχύει $|G:H| = [L:F]$. Ακόμα, $H \unlhd G$ αν και μόνο αν η επέκταση $L/F$ είναι \tl{Galois}. Όταν αυτό συμβαίνει έχουμε τον ισομορφισμό ομάδων $Gal(L/F) \cong G/H$. Αν εμπλουτίσουμε την ομάδα πηλίκο $G/H$ με την τοπολογία πηλίκο, τότε αυτός ο ισομορφισμός έιναι και ομοιομορφισμός.
\end{theorem}
\begin{proof} $ $


$ $\newline	
Έστω $L$ υπόσωμα του $K$ που περιέχει το $F$, τότε εφόσον το $K$ είναι κανονική και διαχωρίσιμη επέκταση του $F$ θα ισχύουν και τα ίδια υπεράνω του $L$. Έτσι έχουμε ότι η επέκταση $K/L$ είναι \tl{Galois} και άρα $L = F^{Gal(K/L)}$. Αν $H \leq G$ τότε από το προηγούμενο θεώρημα έχουμε ότι $H=Gal(K/F^H)$  αν και μόνο αν το $H$ είναι κλειστό. Άρα έχουμε την ζητούμενη αντιστοιχία.

$ $\newline
Έστω $L$ ενδιάμεσο σώμα της $K/F$ και έστω $H = Gal(K/L)$, δηλαδή $H$ κλειστό από το προηγούμενο θεώρημα. Αν υποθέσουμε ότι $|G : H| < \infty$ έχουμε την ξένη ένωση:
$$G = H \cup a_1 \cup \ldots \cup a_n H$$

$ $\newline
Αυτό σημαίνει ότι το $G-H$ είναι πεπερασμένη ένωση συμπλόκων του $H$. Ωστόσο, επειδή το $H$ είναι κλειστό θα είναι και κάθε σύμπλοκο του κλειστό, δηλαδή θα είναι το $G-H$ κλειστό και συνεπώς το $H$ ανοιχτό.
Πράγματι, έστω $x \in \overline{aH}$. Τότε:

$$xN\cap aH \neq \varnothing \quad \forall N \in \mathcal N$$
$$\iff a^{-1} x N \cap H \neq \varnothing \quad\forall N \in \mathcal N$$
$$\iff a^{-1} x \in \overline H = H \implies x\in aH$$

Αντίστροφα, αν το $H$ είναι ανοιχτό τότε περιέχει μια βασική περιοχή του $id$. Δηλαδή υπάρχει $N \in \mathcal{N}$ τέτοιο ώστε:
$$id N = N \subseteq H \implies F^N \supseteq F^N$$
δηλαδή $L \subseteq E$, αν θεωρήσουμε $E = F^N$. Επειδή $N \in \mathcal N$, έχουμε ότι $E \in \mathcal I$ και άρα $[E:F] < \infty$. Από κανόνα πύργων παίρνουμε:
$$[E:F]=[E:L][L:F]$$
και άρα $[L:F] < \infty$.

$ $\newline
Για την τελευταία κατεύθυνση, αν $[L:F] <\infty$ τότε $L=F(a_1,\ldots a_n)$ με $a_i \in K$ και για αυτά τα $a_i$ το λήμμα $\ref{17.1}$ μας λέει ότι υπάρχει $E \in \mathcal{I}$ με $a_i \in E$ για κάθε $i$ και συνεπώς $L \subseteq E$. Έστω τώρα $N = Gal(K/E)$ τότε:
$$L \subseteq H \implies Gal(K/L) \geq Gal(K/H)$$
δηλαδή $N \leq H$ και $| G:H| \leq |G:N| < \infty$.

$ $\newline
Από το λήμμα $\ref{17.2}$ έχουμε ότι $G/N \cong Gal(E/F)$ μέσω της απεικόνισης $\sigma N \mapsto \sigma|_E$. Επομένως, η ομάδα πηλίκο $H/N$ απεικονίζεται στο $\{p|_E : p \in H\}=H_0$, το οποίο είναι υποομάδα της $Gal(E/F)$ και έχουμε δείξει προηγουμένως ότι αυτό έχει σταθερό σώμα $L \cap E = L$. Από το θεμελιώδες θεώρημα για πεπερασμένες επεκτάσεις έχουμε ότι $|H_0 | = [E:L]$. Από αυτό έπεται ότι:

$$|G:H| = |G/N : H/N | = \frac{|G/N|}{|H/N|} = \frac{[E:F]}{[E:L]} = [L:F]$$

$ $\newline
Υποθέτουμε τώρα ότι η $H = Gal(K/L)$ είναι κανονική υποομάδα της $G$. Έστω $a \in L$ και $f(x) = Irr(a,F)$. Αν $b \in K$ είναι ρίζα του $f(x)$ τότε από το θεώρημα επέκτασης ισομορφισμών υπάρχει $\sigma \in G$ με $\sigma (a) = b$. Θα δείξουμε ότι $b \in L$. Έστω $\tau \in H$, τότε:
$$\tau (b) = \sigma^{-1} (\sigma \tau \sigma^{-1} (a)) = \sigma ^{-1} (a) = b$$
εφόσον $H \unlhd G$ και άρα $\sigma \tau \sigma^{-1} \in H$. Συνεπώς το $b$ ανήκει στο σταθερό σώμα της $H$, δηλαδή στο $L$. 
Δείξαμε ότι το $f(x)$ διασπάται πλήρως στο $L$. Αυτό αποδεικνύει την κανονικότητα της επέκτασης $L/F$ και η διαχωρισιμότητα της επέκτασης έπεται από την διαχωρισιμότητα της $K/F$. Άρα η επέκταση $L/F$ είναι \tl{Galois}.

$ $\newline
Αντίστροφα, αν $L/F$ \tl{Galois} επέκταση τότε από υπενθύμιση έχουμε ότι:
$$\theta: G \longrightarrow Gal(L/F)$$
$$\sigma \longmapsto \sigma|_L$$

$ $\newline
Το $\theta$ είναι καλά ορισμένος ομομορφισμός ομάδων με πυρήνα το $H = Gal(K/L)$ αφού αν
$$\theta ( \sigma ) = 1_L \implies \sigma |_L = 1_L \implies \sigma \in Gal(K/L)$$

συνεπώς έχουμε $H \unlhd G$ ως πυρήνα ομομορφισμού. Επιπλέον ο $\theta$ είναι επί αφού αν έχουμε ένα τυχόν $\tau \in Gal(L/F)$ τότε το επεκτείνουμε μέσω του θεωρήματος επέκτασης ισομορφισμών σε $\tau^{\prime} \in G$ και έτσι $\tau^{\prime}|_L = \tau$. Από το πρώτο θεώρημα ισομορφισμών ομάδων έχουμε ότι $G/H \cong Gal(L/F)$.

$ $\newline %wstoso !!!!! topology
Το τελευταίο βήμα της απόδειξης είναι να δείξουμε ότι ο ισομορφισμός αυτός είναι και ομοιομορφισμός, ωστόσο, η συνέχεια και η κλειστότητα διατηρούνται στην τοπολογία πηλίκο. Άρα αρκεί να δείξουμε ότι η $\theta$ είναι συνεχής και κλειστή. Τότε η επαγόμενη απεικόνιση:
$$ \nu  :G/H \longrightarrow Gal(L/F)$$
θα είναι ομοιμορφισμός.

$ $\newline
Όμοια με την \tl{Galois} επέκταση $K/F$, στην \tl{Galois} επέκταση $L/F$ τα βασικά ανοιχτά υποσύνολα της $Gal(L/F)$ είναι της μορφής $\rho Gal(L/E)$ για πεπερασμένες \tl{Galois} επεκτάσεις $E/F$ όπου $E\subseteq L$. Έστω $N =Gal(K/E) \in \mathcal N$. Το σύνολο $\theta^{-1} (Gal(L/E))$ περιέχει όλους τους ισομορφισμούς $\sigma \in G$ που αφού τους περιορίσουμε στο $L$ μέσω της $\theta$ κρατάνε σταθερό το $E$, δηλαδή:
$$\theta^{-1} (Gal(L/E)) = N$$

όμοια: 
$$\theta^{-1} (\rho Gal(L/E)) = \tau N$$
για κάθε $\tau \in G$ τέτοιο ώστε $\theta (\tau) = \tau|_L = \rho$. 

$ $\newline
Τα $\tau N$ είναι βασικά ανοιχτά υποσύνολα του $G$, συνεπώς δείξαμε ότι η $\theta$ είναι συνεχής. Επιπλέον, η εικόνα μέσω συνεχούς απεικόνισης ενός συμπαγούς συνόλου παραμένει συμπαγές σύνολο. H $G$ είναι συμπαγής και άρα είναι και η $Gal(L/F)$. Αντίστοιχα με την απόδειξη για την $G$, η $Gal(L/F)$ είναι \tl{Hausdorff} και κάθε συμπαγές υποσύνολο χώρου \tl{Hausdorff} είναι κλειστό. Δηλαδή, αν θεωρήσουμε ένα κλειστό υποσύνολο της $G$ αυτό θα είναι συμπαγές και μέσω της $\theta$ θα απεικονίζεται σε κλειστό υποσύνολο της $Gal(L/F)$. Έτσι, δείξαμε ότι και η $\theta^{-1}$ είναι συνεχής και άρα ο ισομορφισμός που επάγεται από την $\theta$ είναι αμφισυνεχής όταν δωθεί η τοπολογία πηλίκο στο $G/H$. Δηλαδή, είναι και ομοιμορφισμός.
\end{proof}

\begin{example}
έστω $K/F$ πεπερασμένη \tl{Galois} επέκταση. Τότε η \tl{Krull} τοπολογία στο $Gal(K/F)$ είναι η διακριτή. Πράγματι αν $\sigma \in G$, έχουμε $K\in \mathcal I$ αφού $[K:F]<\infty$ και άρα το $\sigma N = \sigma Gal(K/K) = \sigma \{1_K\} = \{ \sigma\}$ είναι ανοιχτή περιοχή του $\sigma$. Ετσι, κάθε υποομάδα $H\leq G$ είναι κλειστή και βρισκόμαστε ξανά στο αρχικό θεμελιώδες θεώρημα της θεωρίας \tl{Galois}.
\end{example}

$ $\newline Πριν δώσουμε δύο ακόμα παραδείγματα, το θεμελιώδες θεώρημα της άπειρης θεωρίας \tl{Galois} μαζί με την υπενθύμιση μας δίνουν ένα 'όμορφο' αποτέλεσμα. Έστω $K/L/F$, όπου οι επεκτάσεις $K/F$ και $L/F$ είναι \tl{Galois}. Αν $f(\sigma) = \sigma$ και $g(\sigma) = \sigma|_L$ τότε παρακάτω έχουμε μια βραχεία ακριβή ακολουθία ομάδων: (ή \tl{profinite} τοπολογικών ομάδων με βάση το επόμενο κεφάλαιο.)

\begin{center}
	\begin{tikzcd}
		1 \arrow[r] & Gal(K/L) \arrow[r, "f"] & Gal(K/F) \arrow[r, "g"] & Gal(L/F) \arrow[r] & 1
		\end{tikzcd}
\end{center}

\vspace{0.3truecm}

\begin{example}
	Έστω $K = \mathbb{Q}(\zeta_{2^{\infty}}) = \cup_n \mathbb{Q} (\zeta_{2^n})$ και $K_n = \mathbb{Q}(\zeta_{2^n})$. Έχουμε ότι: 
	$$Gal(K_n,\Q ) \cong (\Z / 2^n \Z )^*$$
	$$\sigma_a (\zeta_{2^n}) = \zeta^a_{2^n}$$
	για τα αντιστρέψιμα $a$  $(mod2^n)$.

	$ $\newline %να αλλάξω ίσως το σημείο με το άτοπο, ίσως 13^k = 5 mod2^n για fixed k?
	\noindent Θεωρούμε τις κυκλικές υποομάδες $H = (\sigma_5)$ και $H^{\prime} = (\sigma_{13})$ της $Gal(K/ \Q)$. Έχουμε ότι $H \neq H^{\prime}$, διαφορετικά αν απεικονίζαμε έναν γεννήτορα της μιας ομάδας σε έναν γεννήτορα της άλλης θα είχαμε $\zeta^5_{2^n} = \zeta^{13^k}_{2^n}$ το οποίο είναι ισοδύναμο με το άτοπο $5 = 13^k \quad (mod2^n)$ για κάθε φυσικό $n$ και σταθερό $k$.
	Ωστόσο, θα δείξουμε ότι ισχύει $K^H = K^{H^{\prime}}$! Θεωρούμε επίσης $H_n, H^{\prime}_n$ τις κυκλικές υποομάδες $(\sigma_5|_{K_n} )$ και $(\sigma_{13}|_{K_n})$ της $Gal(K_n / \Q)$. Αυτές είναι ισόμορφες καθώς $<5 \quad mod2^n> = < 13 \quad mod2^n>$ για κάθε $n\geq 2$. Καθώς $13,5 = 1\quad (mod4)$ έχουμε ότι τα $\sigma_5, \sigma_{13}$ κρατάνε σταθερό το $i$, δηλαδή $\Q (i) \subseteq K^{H_n}_n , K^{H^{\prime}_n}_n $. Από πεπερασμένη αντιστοιχία \tl{Galois} έχουμε ότι $K^{H_n}_n = \Q (i) = K^{H^{\prime}_n}_n$. Αυτό είναι για τυχόν $n \in \mathbb N$. Συνεπώς,  παρόλο που $H \neq H^{\prime}$ ισχύει ότι:
	
		$$K^H = \{a \in K : \quad \sigma (a) = a \quad\forall \sigma \in H\} = \cup_n \{a\in K: \sigma|_{K_n} (a) = a \quad\forall \sigma \in H_n\} = $$
		$$\cup_n K^{H_n}_n = \cup_n \Q (i) = \cup_n K^{H^{\prime}_n}_n = K^{H^{\prime}}$$
%μπορώ νδο ότι δεν είναι κλειστή?	
\end{example}

\vspace{0.3truecm}
%On groups with uncountably many subgroupsof  nite index Daniel S. Silver
%https://math.stackexchange.com/questions/39895/the-direct-sum-oplus-versus-the-cartesian-product-times
\begin{example}
	Αν θεωρήσουμε την \tl{Galois} επέκταση $\Q (i, \sqrt 2 , \sqrt 3 , \sqrt 5 , \sqrt 7 , \ldots)$ του $\Q$ τότε επειδή οι αυτομορφισμοί του θα απεικονίζουν κάθε $\sqrt p$ (ή το $i$) στα $ \pm \sqrt p$ (ή $\pm i$) έχουμε ότι:
	$$G \cong \prod\limits_{i=1}^{\infty} \Z / 2 \Z$$
	Αυτή η ομάδα έχει υπεραριθμήσιμες υποομάδες με δείκτη 2, ενώ οι επεκτάσεις διάστασης 2 του $\Q$ είναι αριθμήσιμες. 

	$ $\newline
	Πράγματι, κάθε επέκταση διάστασης 2 του $\Q$ θα είναι μια επισύναψη του $\sqrt{q}$ για κάθε $q\in \mathbb{Q}$ (με την σύμβαση ότι για $a>0, \sqrt{-a}=i\sqrt{a}$) όπου το $|q|$ δεν είναι τέλειο τετράγωνο. Δηλαδή, έχουμε αριθμήσιμες επιλογές για τις επεκτάσεις διάστασης 2.

	$ $\newline
	Για το άλλο επιχείρημα, μπορούμε να δούμε την ομάδα $G$ ως έναν διανυσματικό χώρο υπεράνω του $\mathbb Z / 2\mathbb Z$ άπειρης αριθμήσιμης διάστασης και να θεωρήσουμε τον δυϊκό χώρο:
	$$G^* = \{\phi: G\rightarrow \Z / 2\Z | \quad \phi \text{ γραμμική απεικόνιση }\}$$
	Ένα γνωστό θεώρημα της συναρτησιακής ανάλυσης μας λέει ότι κάθε διανυσματικός χώρος έχει \tl{Hamel} βάση. Αν θεωρήσουμε μια \tl{Hamel} βάση $\{e_n\}_{n \in \mathbb N}$ του διανυσματικού χώρου του $G$ υπεράνω του $\Z / 2 \Z$, τότε ένα τυχόν στοιχείο του χώρου γράφεται ως:
	$$x = \sum\limits_{n=1}^{\infty} \lambda_n e_n$$
	με $\lambda_n \in \Z / 2\Z$. Αν θεωρήσουμε και μια απεικόνιση $\phi \in G^*$ τότε:
	$$\phi (x) = \sum\limits_{n=0}^{\infty} \lambda_n \phi (e_n)$$
	δηλαδή η $\phi$ καθορίζεται πλήρως σαν απεικόνιση από τις επιλογές $\phi (e_n) \in \Z / 2 \Z$. Με άλλα λόγια ο χώρος $G^*$ έχει την ίδια πληθικότητα με το σύνολο $\{0,1\}^{\mathbb N}$, δηλαδή άπειρη υπεραριθμήσιμη.
	
	$ $\newline
	Από αυτό έπεται ότι και ο υπόχωρος $\{ ker\phi : \phi \in G^*\}$ θα είναι υπεραριθμήσιμος. Πράγματι, για κάθε $x \in G, \phi \in G^*$ έχουμε $\phi (x) = 0$ ή $1$ και άρα ο πυρήνας $ker\phi$ μας ορίζει μονοσήμαντα την $\phi$ για κάθε $x \in G$. Δηλαδή, αν ξέρουμε ποια $x$ πάει η $\phi$ στο $0$, ξέρουμε ακριβώς ότι τα $x \in G-ker\phi$ τα πάει στο $1$. Με άλλα λόγια, μια επιλογή ενός συνόλου $ker\phi$ είναι μια επιλογή μιας $\phi$ και αντίστροφα. Έτσι καταλήγουμε στην υπεραριθμησιμότητα του υπόχωρου.

	$ $\newline Η κάθε $\phi$ ως γραμμική είναι και ομομορφισμός ομάδων και έτσι χρησιμοποιώντας το 1ο θεώρημα ισομορφισμών ομάδων για τις $\phi$, έχουμε υπεραριθμήσιμους πυρήνες, δηλαδή υπεραριθμήσιμες υποομάδες $H$ τέτοιες ώστε:
	$$|G:H| = |G:ker\phi| = |\Z / 2\Z |  = 2$$

\end{example}

Ουσιαστικά, οι υποομάδες της ομάδας \tl{Galois} μιας άπειρης επέκτασης είναι `πάρα πολλές` σε σχέση με τις ενδιάμεσες πεπερασμένες επεκτάσεις. Για αυτό και αποτυγχάνει η αντιστοιχία \tl{Galois} χωρίς τον περιορισμό της κλειστότητας των υποομάδων.


\pagebreak
\section{Περαιτέρω Μελέτη}


Στην προσπάθεια να γενικεύσει κανείς τα προηγούμενα επιχειρήματα μπορεί να φτάσει στους ακόλουθους ορισμούς:

\begin{defn}Τοπολογική ομάδα $G$ είναι ένας τοπολογικός χώρος $(G,\mathcal T )$ όπου η $G$ είναι ομάδα με τις ιδιότητες ότι η απεικόνιση πολλαπλασιασμού $(a,b) \mapsto ab$ και η αντιστροφή $a \mapsto a^{-1}$ είναι συνεχείς. Αντίστοιχα ζητάμε οι ομομορφισμοί μεταξύ των ομάδων να είναι και συνεχείς για να τους λέμε ομομορφισμούς τοπολογικών ομάδων.
\end{defn}

\noindent Όπως κάναμε και πριν δηλαδή που απαιτούσαμε ο ισομορφισμός ομάδων που προέκυπτε να είναι και ομοιομορφισμός.

\begin{defn} Αν $\Lambda\neq \varnothing$ ένα σύνολο και $\leq$ είναι μια διμελής σχέση στο $\Lambda \times \Lambda$ τότε το $(\Lambda , \leq )$ λέγεται κατευθυνόμενο σύνολο αν ικανοποιούνται οι δύο σχέσεις της προδιάταξης:

	$1)$ Αυτοπαθής $\lambda \leq \lambda \quad\forall \lambda \in \Lambda$
	
	$2)$ Μεταβατική $\lambda_1 \leq \lambda_2$ και  $\lambda_2 \leq \lambda_3 \implies \lambda_1$ 
	
	μαζί με την :
	
	$3)$ Για κάθε $\lambda_1 , \lambda_2 \in \Lambda$ υπάρχει $\lambda_3 \in \Lambda$ τέτοιο ώστε $\lambda_1, \lambda_2 \leq \lambda_3$.
	\end{defn}

$ $\newline
\noindent Για παράδειγμα, αν σκεφτόμαστε υποσύνολα $A,B$ ενός μη κενού συνόλου $X$ τότε η σχέση $A\leq B \iff A\supseteq B$ καθιστά το $X$ κατευθυνόμενο εφόσον $A, B \leq A\cap B$.

$ $\newline
\noindent Στην συνέχεια, τα επόμενα είναι συνήθως ορισμένα στην θεωρία των κατηγοριών αλλά εδώ θα τα ορίσουμε περιορισμένοι στις ομάδες.

\begin{defn}[\tl{Inverse System}] Ένα αντίστροφο σύστημα αποτελείται από ένα κατευθυνόμενο σύνολο $(J,\leq )$ και μια συλλογή πεπερασμένων ομάδων $\mathcal{G} = \{G_i : i \in J\}$ οι οποίες είναι τοπολογικές ομάδες εφοδιασμένες με την διακριτή τοπολογία. Επιπλέον απαιτούμε και μια συλλογή ομομορφισμών $\{f^j_i : G_j \rightarrow G_i | i,j \in J \quad\forall i\leq j\}$ οι οποίοι ικανοποιούν τις εξής σχέσεις:
$$f^i_i = id(G_i)$$
$$f^j_i \circ f^k_j = f^k_i$$
\end{defn}

\begin{defn}[\tl{Inverse Limit}] Αντίστροφο όριο ενός συστήματος όπως παραπάνω θα λέμε μια ομάδα $G$ μαζί με τους ομομορφισμούς $f_i : G \rightarrow G_i$ που ικανοποιούν $f^j_i \circ f_j = f_i$ για κάθε ζεύγος $i \leq j$ , εφόσον η ομάδα $G$ ικανοποιεί την παρακάτω καθολική ιδιότητα:

$ $\newline
Αν $H$ είναι μια ομάδα μαζί με ομομορφισμούς $\tau_i : H \rightarrow G_i$ που ικανοποιούν $f^j_i \circ \tau_j = \tau_i$ για κάθε ζεύγος $i\leq j$ τότε υπάρχει μοναδικός ομομορφισμός $\tau : H  \rightarrow G$ με $\tau_i = f_i \circ \tau$ για κάθε $i$. Δηλαδή το παρακάτω διάγραμμα μετατίθεται:
\end{defn}
\begin{center}
	\begin{tikzcd}
		H \arrow[r, "\tau_i"] \arrow[d, "\tau"'] & G_i \\
		G \arrow[ru, "f_i"]                      &    
		\end{tikzcd}
\end{center}



$ $\newline
Έτσι μπορεί να δειχθεί ότι το αντίστροφο όριο ενός συστήματος υπάρχει, είναι μοναδικό ως προς ισομορφισμό και είναι το 
$$\varprojlim G_i  = \{(g_i)_{i \in J} \in \prod\limits_{i \in J} G_i : \quad f^j_i (g_j) = g_i \quad\forall i\leq j\}$$

\noindent Σαν ομάδα, το αντίστροφο όριο είναι υποομάδα της $\prod G_i$ και είναι τοπολογική ομάδα που παίρνει την επαγόμενη τοπολογία περιορισμό, εφόσον στην $\prod G_i$ δίνεται η τοπολογία γινόμενο.

$ $\newline
\noindent Στην συνέχεια θα δώσουμε έναν τελευταίο ορισμό που θα δέσει με το προηγούμενο κεφάλαιο:

\begin{defn}[\tl{Profinite}]Μια τοπολογική ομάδα λέγεται \tl{profinite (projective + finite)} αν είναι ισόμορφη με το αντίστροφο όριο ενός αντιστρόφου συστήματος πεπερασμένων ομάδων. 
\end{defn}

$ $\newline
\noindent Τα αποτελέσματα του προηγούμενο κεφαλαίου θα μπορούσαν να παραπέμψουν κάποιον ότι ένας ισοδύναμος ορισμός είναι ακριβώς η τοπολογική ομάδα να έχει τις ιδιότητες: συμπάγεια, \tl{Hausdorff} και \tl{totally disconnected}.

$ $\newline
\noindent Έτσι, ένα παράδειγμα χωρίς ιδιαίτερο ενδιαφέρον είναι ότι κάθε πεπερασμένη ομάδα μαζί με την διακριτή τοπολογία είναι \tl{profinite}.

$ $\newline
\noindent Το παράδειγμα που μας ενδιαφέρει είναι ότι για κάθε άπειρη επέκταση \tl{Galois}, η ομάδα \tl{Galois} που προκύπτει είναι \tl{profinite}. Αν ακολουθήσουμε τους ορισμούς του προηγούμενο κεφαλαίου και θεωρήσουμε την συλλογή πεπερασμένων ομάδων με την διακριτή τοπολογία:
$$\{G/N: \quad N \in \mathcal{N}\}$$
και ως ομομορφισμούς:
$$f^j_i : G/N_i \longrightarrow G/N_j$$
τις κανονικές προβολές, όπου $N_i \geq N_j \iff N_i \subseteq N_j$
δηλαδή τις απεικονίσεις:

$$G/Gal(K/E_i) \cong Gal(E_i /F) \longrightarrow Gal(E_j /F) \cong G/Gal(K/E_j)$$
$$\sigma \longmapsto \sigma|_{E_j}$$

\noindent τότε τα παραπάνω αποτελούν αντίστροφο σύστημα και μάλιστα έχουμε τον ομοιομορφισμό:

$$G \cong \varprojlim G/N$$

\noindent δηλαδή, η τοπολογία που προκύπτει στο αντίστροφο όριο ως τοπολογία περιορισμός δεν είναι άλλη από την τοπολογία \tl{Krull}.

\noindent Αξίζει να αναφερθούμε και σε ένα αποτέλεσμα που βλέπει τα πράγματα από αντίθετη σκοπιά, το οποίο είναι ότι κάθε \tl{profinite} ομάδα είναι ομάδα \tl{Galois} για κάποια επέκταση σωμάτων. (βλέπε \tl{Profinite groups are Galois groups, William C. Waterhouse, 1974})

$ $\newline
Ένα άλλο παράδειγμα άξιο μελέτης είναι ο ορισμός της προσθετικής ομάδας των \tl{p-adic} ακεραίων. Είναι η profinite ομάδα $\varprojlim \mathbb{Z}/p^n \mathbb{Z}$ όπου το $n$ διατρέχει τους φυσικούς μαζί με τις φυσικές απεικονίσεις $\mathbb{Z}/p^n \mathbb{Z} \rightarrow \mathbb{Z}/p^m \mathbb{Z}$ για όλα τα $n\geq m$. Αναμενόμενο είναι και η τοπολογία που προκύπτει στο αντίστροφο όριονα ταυτίζεται με την τοπολογία που έχουν οι \tl{p-adic} ακέραιοι μέσω του συνήθους ορισμού τους από την ανάλυση.

\pagebreak

\begin{thebibliography}{9}
	\bibitem{pmorandi}
	\tl{Patrick Morandi.}
	\textit{\tl{Fields and Galois Theory.}}
	\tl{Springer-Verlag, New York, 1996.}

	\bibitem{jsmilne}
	\tl{James S. Milne.}
	\textit{\tl{Fields and Galois Theory.}}
	\tl{Available at www.jmilne.org/math/, 2020.}

	\bibitem{fbutler}
	\tl{Frederick Butler.}
	\textit{\tl{Infinite Galois Theory, Master Thesis, University of Pennsylvania, 2001.}}
\end{thebibliography}

%
\end{document}



	


