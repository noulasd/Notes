\documentclass{beamer}
\usepackage[utf8]{inputenc}
\usepackage[greek]{babel}
\newcommand {\tl}{\textlatin}
\usepackage{tikz-cd}

\newtheorem{thrm}{Θεώρημα}
\newtheorem{lhmma}{Λήμμα}
\newtheorem{exa}{Παράδειγμα}
\newtheorem*{defn}{Ορισμός}
\newtheorem{prop}{Πρόταση}
\newtheorem{cor}{Πόρισμα}


\title{Άπειρη Θεωρία \tl{Galois}}
\author{Δημήτριος Νούλας }
\date{Δεκέμβρης 2020}


\mode<presentation>
{
  \usetheme{Copenhagen}
  % or ...

  \setbeamercovered{transparent}
  % or whatever (possibly just delete it)
}

\begin{document}

\maketitle

\AtBeginSection[]
{
  \begin{frame}<beamer>
    \frametitle{\tl{Outline}}
    \tableofcontents[currentsection]%,currentsubsection]
  \end{frame}
}
%στην τάξη κάπως απλοιποιήσαμε το ΘΘΘΓ, μιλήσαμε μόνο για σ.ρ. Το θεώρημα είναι κάπως πιο γενικό. θα πω αυτήν την γενική έκδοση που παραμένει πεπερασμένη και μετα θα μιλησω και για την απειρη
\section{Κλασική Θεωρία \tl{Galois}}
\setbeamercovered{invisible}
\begin{frame}
    
    \frametitle{Σώμα ριζών}
    \begin{defn}
    Έστω μια επέκταση σωμάτων $K/F$ και $f(x) \in F[x]$. Τότε το $K$ λέγεται σώμα ριζών του $f(x)$ υπεράνω του $F$ αν $K=F(a_1,\ldots , a_n)$ και $f(x) = a(x-a_1)\cdots (x-a_n) \in K[x]$
    \end{defn}
    \pause
    \begin{block}{Για ένα σύνολο πολυωνύμων}
    Αν $S \subseteq F[x]$ τότε επεκτείνουμε τον ορισμό και λέμε ότι το $K$ είναι σώμα ριζών του $S$ αν καθένα από τα πολυώνυμα του $S$ διασπάται πλήρως στο $K$ και $K = F(X)$, όπου $X$ το σύνολο ριζών των πολυωνύμων. Αν $S= \{f_1, f_2 ,\ldots f_n\}$ τότε το $K$ είναι σώμα ριζών του $S \iff K$ σώμα ριζών του $f_1 f_2 \cdots f_n$.
    $ $\newline
    Ύπαρξη? Αν το $S$ είναι άπειρο?
    %Υπενθυμίζω ύπαρξη για ένα πολυώνυμο και πεπερασμένο σύνολο: παίρνουμε έναν ανάγωγο παράγοντα του πολυωνύμου και παίρνουμε για επέκταση τον δακτύλιο πηλίκο modulo τον παράγοντα αυτόν, σε αυτή την επέκταση το πολυώνυμο έχει ρίζα και κάνουμε αυτή τη διαδικασία διαδοχικά μέχρι να επισυνάψουμε όλες τις ρίζες
    
    \end{block}
    
\end{frame}

%Για την ύπαρξη χρειαζόμαστε μια έννοια κλειστότητας
\begin{frame}
    \frametitle{Αλγεβρική Κλειστότητα}
    \begin{defn}
    Έστω επέκταση $K/F$, τότε ονομάζουμε αλγεβρική κλειστότητα του $F$ στο $K$ το σώμα $\overline{F} = \{a \in K: a \text{ αλγεβρικό υπεράνω του } F\}$
    \end{defn}
    \pause
    \begin{block}{Αλγεβρικά Κλειστό Σώμα}
    %σαν τετριμμένη επέκταση $K/K$
    Ισοδύναμα, έχουμε ότι ένα σώμα $K$ είναι αλγεβρικά κλειστό αν κάθε πολυώνυμο στο $K[x]$ έχει ρίζα στο $K$. 
    \end{block}
    \pause
    \begin{thrm}
        Κάθε σώμα $F$ έχει αλγεβρική κλειστότητα.
        Έπεται ως πόρισμα η ύπαρξη σώματος ριζών ενός τυχαίου συνόλου πολυωνύμων $\subseteq F[x]$. %προυποθέτει λήμμα του Zorn όπως πχ βάση σε Δ.Χ. . Γενικά εφόσον έχουμε αυθαίρετες επεκτάσεις θα "διαλέγουμε" υπό μια έννοια που θα στέλνονται τα νεα στοιχεία με τους ισομορφισμούς
    \end{thrm}
\end{frame}

\begin{frame}
    \frametitle{Θεώρημα Επέκτασης Ισομορφισμών}
    \begin{thrm}
    	Έστω $\sigma : F \rightarrow F^{\prime}$ ένας ισομορφισμός σωμάτων. Έστω $S = \{f_i (x)\}$ ένα σύνολο πολυωνύμων με συντελεστές από το $F$ και $S^{\prime} = \{\sigma (f_i)\}$. Έστω $K$ ένα σώμα ριζών του $S$ υπεράνω του $F$ και $K^{\prime}$ ένα σώμα ριζών του $S^{\prime}$ υπεράνω του $F^{\prime}$. Τότε υπάρχει ισομορφισμός $\tau : K \rightarrow K^{\prime}$ με $\tau|_F = \sigma$. Επιπλέον, αν $a \in K$ και το $a^{\prime}$ είναι οποιαδήποτε ρίζα του $\sigma(Irr(a,F))$ στο $K^{\prime}$ τότε ο ισομορφισμός $\tau$ μπορεί να επιλεγεί έτσι ώστε $\tau(a) = a^{\prime}$.

    \end{thrm}
    
\end{frame}

\begin{frame}
    \frametitle{Κανονική Επέκταση}
    \begin{defn}
    Έστω επέκταση $K/F$. Λέμε την επέκταση του $F$ κανονική αν το $K$ είναι σώμα ριζών ενός συνόλου πολυωνύμων με συντελεστές από το $F$.
    \end{defn}
    \pause
    \begin{block}{Ισοδύναμος ορισμός}
    Στη βιβλιογραφία πολλές φορές ορίζεται και ως: Κάθε ανάγωγο πολυώνυμο του $F[x]$ που έχει ρίζα στο $K$ διασπάται πλήρως στο $K$. Ο όρος κανονική επέκταση δεν είναι τυχαίος καθώς όπως θα δούμε η κανονικότητα της επέκτασης έχει να κάνει με την κανονικότητα υποομάδας.
    \end{block}
\end{frame}

\begin{frame}
    \frametitle{Κανονική Επέκταση}
    \begin{defn}
    Αν $K/F$ επέκταση τότε ορίζουμε την ομάδα \tl{Galois} να είναι το σύνολο των αυτομορφισμών του $K$ που κρατάνε σταθερό το $F$ και την συμβολίζουμε με $Gal(K/F)$
    \end{defn}
    \begin{block}{Ένα σημαντικό αποτέλεσμα}
        Αν $K/L/F$ διαδοχικές επεκτάσεις και $K/F$ κανονική τότε η απεικόνιση:
        $$Gal(K/F) \longrightarrow Gal(L/F)$$
        $$ \sigma \longmapsto \sigma|_L$$
        είναι επί. %ουσιαστικά θεωρούμε ένα στοιχείο της δεξιά ομάδας και "διαλέγουμε" που θα πηγαίνει τα στοιχεία K\L μέσω του Θ.επέκτασης Ισομορφμισμών
    \end{block}
\end{frame}
\begin{frame}
    \frametitle{Διαχωρίσιμη Επέκταση}
    Χρειαζόμαστε ακόμα μια έννοια για την αντιστοιχία \tl{Galois}!
    \begin{defn}[Διαχωρίσιμο Πολυώνυμο]
    	Έστω $F$ ένα σώμα. Ένα ανάγωγο πολυώνυμο $f(x) \in F[x]$ είναι διαχωρίσιμο υπεράνω του $F$ αν οι ρίζες του είναι απλές σε οποιοδήποτε σώμα ριζών. Ένα πολυώνυμο $g(x) \in F[x]$ είναι διαχωρίσιμο υπεράνω του $F$ αν όλοι οι ανάγωγοι παράγοντες του είναι διαχωρίσιμοι υπεράνω του $F$.
    \end{defn}
    \pause
    \begin{defn}[Διαχωρίσιμο Στοιχείο και Επέκταση]
	Έστω $K/F$ επέκταση και $a \in K$. Τότε το $a$ είναι διαχωρίσιμο υπεράνω του $F$ αν το $Irr(a,F)$ είναι διαχωρίσιμο υπεράνω του $F$. Αν αυτό ισχύει για κάθε $a \in K$ λέμε την επέκταση $K/F$ διαχωρίσιμη.
    \end{defn}
\end{frame}
\begin{frame}
    \frametitle{Διαχωρίσιμη Επέκταση}
    \begin{exa}
    Τα πολυώνυμα $x^2 - 2$ και $(x-1)^5$ είναι διαχωρίσιμα υπεράνω του $\mathbb{Q}$. Γενικότερα κάθε επέκταση σώματος χαρακτηριστικής $0$ είναι διαχωρίσιμη. Αν $F$ σώμα χαρακτηριστικής $p>0$ και $a \not\in F^p$ τότε το $x^p - a \in F[x]$ δεν είναι διαχωρίσιμο.
    \end{exa}
\end{frame}

\begin{frame}
    \frametitle{Επέκταση \tl{Galois}}
    \begin{defn}[Επέκταση \tl{Galois}]
        Για μια αλγεβρική επέκταση:
        $ $\newline
        Διαχωρίσιμη $+$ Κανονική $:=$ \tl{Galois} !
    \end{defn}    
    \pause
    \begin{block}{Ισοδύναμος ορισμός}
    Στην βιβλιογραφία ορίζεται και ως: μια αλγεβρική επέκταση $K/F$ είναι \tl{Galois} αν $F = F^{Gal(K/F)}$, όπου $F^H =\{a \in K: \sigma (a) = a \quad\forall \sigma \in H\}$ το σταθερό σώμα της υποομάδας $H\leq Gal(K/F)$. Αυτός ο ορισμός δείχνει και ποια θέλουμε να είναι η αντιστοιχία μεταξύ ενδιάμεσων σωμάτων και υποομάδων της ομάδας \tl{Galois}. Επιπλέον, αν η επέκταση $K/F$ είναι \tl{Galois} και πεπερασμένη τότε παίρνουμε την ισότητα από την γενική σχέση: $|Gal(K/F)| \leq [K:F] $
    \end{block}
\end{frame}

\begin{frame}
    \frametitle{Θεμελιώδες Θεώρημα της Θεωρίας \tl{Galois}}
    \begin{thrm}
    	Έστω $K$ μια πεπερασμένη επέκταση \tl{Galois} ενός σώματος $F$ και $G=Gal(K/F)$. Τότε υπάρχει μια 1-1 αντιστοιχία που αντιστρέφει την φορά μεταξύ των ενδιάμεσων επεκτάσεων της $K/F$ και των υποομάδων της $G$. Αυτή η αντιστοιχία δίνεται από τις απεικονίσεις $L \mapsto Gal(K/L)$ και $H \mapsto F^H$. Επιπλέον, αν $L\leftrightarrow H$ τότε $[K:L] = |H|$ και $[L:F] = [G:H]$. Μαζί με αυτό, η $H$ είναι κανονική υποομάδα της $G$ αν και μόνο αν η επέκταση $L/F$ είναι \tl{Galois}. Όταν αυτό συμβαίνει έχουμε $Gal(L/F) \cong G/H$.

    \end{thrm}
\end{frame}

\begin{frame}
    \frametitle{Θεμελιώδες Θεώρημα της Θεωρίας \tl{Galois}}
    \begin{center}
        \includegraphics[width=\linewidth,height=\textheight,keepaspectratio]{θθθγ.PNG}
    \end{center}
\end{frame}

\begin{frame}
    \includegraphics[width=\linewidth,height=\textheight,keepaspectratio]{διάγραμμα1.PNG}
\end{frame}
\begin{frame}
    \includegraphics[width=\linewidth,height=\textheight,keepaspectratio]{διάγραμμα2.PNG}
\end{frame}


\section{Άπειρη Θεωρία \tl{Galois}}


\begin{frame}
    Ισχύει το ίδιο αν δεν υποθέσουμε ότι η \tl{Galois} επέκταση $K/F$ είναι πεπερασμένη?
    \pause
    
    
    $ $\newline
    Η απάντηση είναι πως όχι! χρειαζόμαστε κάποιον περιορισμό... %κοιταμε που ισχυει αυτο που θελουμε οπως με αντιστροφο lagrange και sylow υποομαδες
\end{frame}

\begin{frame}
    \frametitle{Άπειρες επεκτάσεις \tl{Galois}}
    \begin{exa}
    Έστω $K = \mathbb{Q}(\zeta_{2^{\infty}}) = \cup_n \mathbb{Q} (\zeta_{2^n})$ και $K_n = \mathbb{Q}(\zeta_{2^n})$.
    $$Gal(K_n,\mathbb{Q} ) \cong (\mathbb{Z} / 2^n \mathbb{Z} )^*$$
	$$\sigma_a (\zeta_{2^n}) = \zeta^a_{2^n}$$
	για τα αντιστρέψιμα $a$  $(mod2^n)$. Θεωρούμε τις κυκλικές υποομάδες $H = (\sigma_5)$ και $H^{\prime} = (\sigma_{13})$ της $Gal(K/ \mathbb Q)$. Έχουμε ότι $H \neq H^{\prime}$ διαφορετικά αν απεικονίζαμε έναν γεννήτορα της μιας ομάδας σε έναν γεννήτορα της άλλης θα είχαμε $\zeta^5_{2^n} = \zeta^{13^k}_{2^n}$ το οποίο είναι ισοδύναμο με το άτοπο $5 = 13^k \quad (mod2^n)$ για κάθε φυσικό $n$ και σταθερό $k$.
    \end{exa}
\end{frame}

\begin{frame}
    \begin{block}{Συνέχεια παραδείγματος}
        %τρέχει μόνο το k στην παραπάνω σχέση
        Αν περιοριστούμε σε $H_n = (\sigma_5|_{K_n}) $ και $H^{\prime}_n = (\sigma_{13}|_{K_n})$ έχουμε ότι αυτές οι δύο ταυτίζονται ως υποομάδες της $Gal(K_n, \mathbb{Q})$. Εδώ η αντιστοιχία \tl{Galois} για την πεπερασμένη επέκταση μας δίνει ότι $K^{H_n}_n = \mathbb Q (i) = K^{H^{\prime}_n}_n$. Καθώς, ενώνουμε συνέχεια το σώμα $\mathbb Q (i)$ καταλήγουμε στο ότι $K^H = K^{H^{\prime}}$, παρόλο που $H \neq H^{\prime}$!
    \end{block}
    καταρρέει η αντιστοιχία...
\end{frame}
%On groups with uncountably many subgroupsof  nite index Daniel S. Silver
%https://math.stackexchange.com/questions/39895/the-direct-sum-oplus-versus-the-cartesian-product-times
%https://math.stackexchange.com/questions/971232/can-a-countable-group-have-uncountably-many-subgroups/971237#971237
\begin{frame}
\begin{block}{Ένα δεύτερο παράδειγμα}
    Αν θεωρήσουμε την \tl{Galois} επέκταση $K = \mathbb Q (i, \sqrt 2 , \sqrt 3 , \sqrt 5 , \sqrt 7 , \ldots)$ του $\mathbb Q$ τότε:
	$$Gal(K/\mathbb{Q}) \cong \prod \limits_{i=1}^{\infty} \mathbb Z / 2 \mathbb Z$$
\end{block}
\pause
    Αυτή η ομάδα έχει υπεραριθμήσιμες υποομάδες με δείκτη 2, ενώ οι επεκτάσεις διάστασης 2 του $\mathbb{Q}$ είναι αριθμήσιμες. Ουσιαστικά, οι υποομάδες της ομάδας \tl{Galois} μιας άπειρης επέκτασης είναι `πάρα πολλές` σε σχέση με τις ενδιάμεσες πεπερασμένες επεκτάσεις, ώστε να δουλεύει η αντιστοιχία όπως πριν.
\end{frame}
%έχει υποομάδα το ευθύ γινόμενο, είναι αβελιανή, ουσιαστικα περιεχει τις δυαδικες λεξεις καθε μηκους, κάθε υποομάδα κανονική οπότε κατασκευάζοντας ομομορφισμους στο ζ/2ζ οι πυρήνες που παιρνουμε εχουν δεικτη 2



\begin{frame}
    \frametitle{Μερικοί Συμβολισμοί}
    Έστω $K/F$ \tl{Galois} επέκταση, συμβολίζουμε:
    $$G=Gal(K/F)$$
    $$\mathcal{I} = \{ E: K/E/F , \quad [E:F]<\infty , \quad E/F \quad\text{ \tl{Galois} } \}$$
    $$\mathcal{N} = \{N \subseteq G: N = Gal(K/E) \quad\text{ για κάποιo } E \in \mathcal{I} \}$$
\end{frame}

\begin{frame}
    \begin{lhmma}
     Aν $a_1 , \ldots a_n \in K$ τότε υπάρχει $E \in \mathcal{I}$ με $a_i \in E$ για κάθε $i \in \{1,\dots, n\}$ .
    \end{lhmma}
    \pause %τα στοιχεια ειναι διαχωρισιμα άρα και τα ελάχιστα πολυωνυμά τους, οριζουμε Ε να ειναι σ.ρ. αυτων των πολυωνυμων αρα Ε κανονική και διαχωρίσιμη, πεπερασμενη εφοσον πεπ στοιχεια
    \begin{lhmma}
    Αν $N \in \mathcal{N}$ με $N = Gal(K/E)$, $E \in \mathcal{I}$ τότε $E=F^N$ και $N\unlhd G$. Τότε έχουμε τον ισομορφισμό $G/N \cong Gal(E/F)$ και επιπλέον $|G/N| = |Gal(E/F)| = [E:F] < \infty$.

    \end{lhmma} %K/E κανονική αφου Κ σ.ρ. over F implies σ.ρ. over E και διαχωρισιμη επειδη εφοσον K/F διαχωρισιμη αν θεωρησουμε ενα στοιχειο a in K το Irr(a,E) διαιρει το διαχωιρισιμο Irr(a,F) 
    %μετα το οτι εχουμε F^N και οχι E^N είναι ιδιο αν γραψω το συνολο: a in K: s(a) = a forall s in Gal(K/E)
    %στην συνεχεια μιμουμαστε την αποδειξη για το θθθγ, η απεικονιση theta G \mapsto Gal(E/F) με κανονα περιορισμο εχει πυρήνα το N και απο την κανονικοτητα της επεκτασης K/F ειναι επι
    \pause
    \begin{lhmma}
        $\bigcap_{N \in \mathcal{N}} N= \{1_G \} = \{ id: K \mapsto K \}$. Επιπλέον, $\bigcap_{N \in \mathcal{N}} \sigma N = \{\sigma \}$ για κάθε $\sigma \in G$.
    \end{lhmma}% παιρνουμε τ και α in K και θελουμε να το πηγαινει στο α, εφοσον a in K υπάρχει E in I και άρα N = Gal(K/E) με τ \in N αρα κρατάει το a in E σταθερο. μετα σ^-1 τ = 1
    
    
\end{frame}

\begin{frame}
    \begin{lhmma}
    Αν $N_1, N_2 \in \mathcal{N}$ τότε $N_1 \cap N_2 \in \mathcal{N}$.
    \end{lhmma}
\end{frame}% θεωρόυμε Ni = Gal(K/Ei) Ei in I, E1E2 pep galois epektash ara E1E2 in I kai Gal(K/E1E2) = N1 cap N2
% efoson s \in N1 \cap N2 stous periorismous einai to id an kai mono an E1,E2 subset tou F^(s) an kai mono an E1E2 subset tou F^(s) an kai mono an s krataei stathero to E1E2 dhladh s in Gal(K/E1E2)


\begin{frame}
    \frametitle{Τοπολογία \tl{Krull}}
    \begin{defn}[Τοπολογία \tl{Krull}] Το $(G,\mathcal{T})$ είναι τοπολογικός χώρος όπου $\mathcal{T}$ είναι η τοπολογία \tl{Krull} που ορίζεται ως εξής:
Ένα υποσύνολο $X$ του $G$ είναι ανοιχτό αν $X=\varnothing$ ή $X= \cup_i \sigma_i N_i$ για κάποια $\sigma_i \in G$ και $N_i \in \mathcal{N}$.
\end{defn}
\end{frame}

\begin{frame}
    \frametitle{Ιδιότητες της τοπολογίας \tl{Krull}}
    Το σύνολο $\{\sigma N: \quad \sigma \in G, N \in \mathcal N\}$ είναι βάση της τοπολογίας.
    \pause
    
    $ $\newline
    Έστω $N \in \mathcal N$, τότε $[G:N]<\infty$ και το $\sigma N$ είναι κλειστό ώς πεπερασμένη ένωση συμπλόκων. Άρα η τοπολογία έχει βάση από ανοιχτά κλειστά.
\end{frame}


\begin{frame}
    \frametitle{Ιδιότητες της τοπολογίας \tl{Krull}}
    \begin{prop}
    Ο τοπολογικός χώρος $(G,\mathcal T)$ είναι \tl{Hausdorff}.
    \end{prop}
    \pause
    \begin{block}{Απόδειξη}
    Έστω $\sigma , \tau \in G, \sigma \neq \tau$.
$$\{\sigma \} = \bigcap_N \sigma N$$
δηλαδή υπάρχει $N \in \mathcal{N}$ έτσι ώστε $\tau \notin N \implies \tau \in G-\sigma N$. Τα $\sigma N , G-\sigma N$ είναι ανοιχτά και διαχωρίζουν τα $\sigma , \tau$.
    \end{block}
\end{frame}




\begin{frame}
    \frametitle{Ιδιότητες της τοπολογίας \tl{Krull}}
    \begin{prop}
     Ο τοπολογικός χώρος $(G, \mathcal{T})$ είναι \tl{totally disconnected}.
    \end{prop}
    \pause
    \begin{block}{Απόδειξη}
     Έστω $X\subseteq G$ που περιέχει τουλάχιστον δύο στοιχεία $\sigma ,\tau$. Όμοια με την προηγούμενη απόδειξη, υπάρχει $\sigma N$ ανοιχτή περιοχή του $\sigma$ που δεν περιέχει το $\tau$. Συνεπώς:

    $$X = \left( \sigma N \cap X \right) \bigcup \left( \left( G-\sigma N\right) \cap X\right)$$
    δηλαδή το $X$ γράφεται ως ένωση ξένων, μη κενών ανοιχτών (της $\mathcal{T}_X$). Άρα τα μοναδικά συνεκτικά υποσύνολα του $G$ είναι μονοσύνολα.
    \end{block}
\end{frame}

\begin{frame}
    \frametitle{Ιδιότητες της τοπολογίας \tl{Krull}}
    \begin{prop}O τοπολογικός χώρος $(G, \mathcal{T})$ είναι συμπαγής.
    \end{prop}
    \pause
    \begin{block}{Ιδέα}
    Δείχνουμε ότι η ομάδα $G$ είναι ισόμορφη και ταυτόχρονα, με την ίδια απεικόνιση, ομοιομορφική με το ευθύ γινόμενο:
    $$P = \prod\limits_{N \in \mathcal{N}} G/N$$
    Τις ομάδες $G/N$ τις εμπλουτίζουμε με την διακριτή τοπολογία και το $P$ παίρνει την τοπολογία γινόμενο. Επίσης το $P$ είναι χώρος \tl{Hausdorff}  και επικαλούμενοι το θεώρημα \tl{Tychonoff} είναι και συμπαγής.
    \end{block}

\end{frame}


\begin{frame}
\begin{thrm}
    Ο τοπολογικός χώρος $(G,\mathcal{T})$ είναι συμπαγής, \tl{Hausdorff} και \tl{totally disconnected}.
\end{thrm}
    \pause
    το επόμενο θεώρημα θα 'διορθώσει' την αντιστοιχία:
    \begin{thrm}
    Έστω $H$ υποομάδα της $G$ και έστω $H^{\prime} = Gal(K/F^H)$. Τότε $H^{\prime} = \overline{H}$, η κλειστή θήκη του $H$ στην τοπολογία του $G$.
%να κάνω απόδειξη????????????
    \end{thrm}
\end{frame}
\begin{frame}
Ψάχνοντας πηγές ανακαλύπτει κανείς νέες αποδεικτικές μεθόδους:
$ $\newline

    \includegraphics[width=\linewidth,height=\textheight,keepaspectratio]{proof.PNG}
\end{frame}

\begin{frame}
    \frametitle{Θεμελιώδες Θεώρημα της Άπειρης Θεωρίας \tl{Galois}}
    \begin{thrm}
    Έστω $K/F$ \tl{Galois} επέκταση και $G=Gal(K/F)$. Με την \tl{Krull} τοπολογία στο $G$
οι απεικονίσεις $L \mapsto Gal(K/L)$ και $H \mapsto F^H$ είναι 1-1 και εμφυτεύουν τα σύνολα:
$$\{L : K/L/F \} \longleftrightarrow \{H \leq G: H = \overline H\}$$
το ένα στο άλλο με την ανάποδη αντιστοιχία. Επιπλέον, αν $L \longleftrightarrow H$ τότε $|G:H| < \infty \iff [L:F] < \infty$, αν και μόνο αν το $H$ είναι ανοιχτό στην τοπολογία. Όταν αυτό συμβαίνει, ισχύει $[G:H] = [L:F]$. Ακόμα, $H \unlhd G$ αν και μόνο αν η επέκταση $L/F$ είναι \tl{Galois}. Όταν αυτό συμβαίνει έχουμε τον ισομορφισμό ομάδων $Gal(L/F) \cong G/H$. Αν εμπλουτίσουμε την ομάδα πηλίκο $G/H$ με την τοπολογία πηλίκο, τότε αυτός ο ισομορφισμός έιναι και ομοιομορφισμός.

    \end{thrm}
\end{frame}

\begin{frame}
\includegraphics[width=\linewidth,height=\textheight,keepaspectratio]{θθθγ2.PNG}
    
\end{frame}

\begin{frame}
    \begin{block}{Γενίκευση}
    Έστω $K/F$ πεπερασμένη \tl{Galois} επέκταση. Τότε η \tl{Krull} τοπολογία στο $Gal(K/F)$ είναι η διακριτή. Έτσι κάθε υποομάδα της $Gal(K/F)$ είναι κλειστή.
    \end{block}
    \pause
    \begin{block}{}
    Aν $\sigma \in G$, έχουμε $K\in \mathcal I$ αφού $[K:F]<\infty$ και άρα το $\sigma N = \sigma Gal(K/K) = \sigma \{1_K\} = \{ \sigma\}$ είναι ανοιχτή περιοχή του $\sigma$.
    \end{block}
\end{frame}

\begin{frame}
\frametitle{Ένα όμορφο αποτέλεσμα}
    Έστω $K/L/F$ όπου οι επεκτάσεις $K/F$ και $L/F$ είναι \tl{Galois}, τότε παρακάτω έχουμε μια βραχεία ακριβή ακολουθία ομάδων:
    
    $ $\newline
    \includegraphics[width=\linewidth,height=\textheight,keepaspectratio]{βαα.PNG}
\end{frame}



\section{Μια κατηγορική οπτική}


\begin{frame}
    \frametitle{Τοπολογικές Ομάδες}
    
    \begin{defn}Τοπολογική ομάδα $G$ είναι ένας τοπολογικός χώρος $(G,\mathcal T )$, όπου η $G$ είναι ομάδα με τις ιδιότητες ότι η απεικόνιση πολλαπλασιασμού $(a,b) \mapsto ab$ και η αντιστροφή $a \mapsto a^{-1}$ είναι συνεχείς. Αντίστοιχα, ζητάμε οι ομομορφισμοί μεταξύ των ομάδων να είναι συνεχείς για να τους λέμε ομομορφισμούς τοπολογικών ομάδων.
\end{defn}
\end{frame}

\begin{frame}
\frametitle{Κατευθυνόμενο Σύνολο}
\begin{defn} Αν $\Lambda\neq \varnothing$ ένα σύνολο και $\leq$ είναι μια διμελής σχέση στο $\Lambda \times \Lambda$ τότε το $(\Lambda , \leq )$ λέγεται κατευθυνόμενο σύνολο αν ικανοποιούνται οι δύο σχέσεις της προδιάταξης:

$1)$ Αυτοπαθής $\lambda \leq \lambda \quad\forall \lambda \in \Lambda$

$2)$ Μεταβατική $\lambda_1 \leq \lambda_2$ και  $\lambda_2 \leq \lambda_3 \implies \lambda_1$ 

μαζί με την :

$3)$ Για κάθε $\lambda_1 , \lambda_2 \in \Lambda$ υπάρχει $\lambda_3 \in \Lambda$ τέτοιο ώστε $\lambda_1, \lambda_2 \leq \lambda_3$.
\end{defn}
\end{frame}



\begin{frame}
\frametitle{Αντίστροφο Σύστημα}
\begin{defn}[\tl{Inverse System}] Ένα αντίστροφο σύστημα αποτελείται από ένα κατευθυνόμενο σύνολο $(J,\leq )$ και μια συλλογή πεπερασμένων ομάδων $\mathcal{G} = \{G_i : i \in J\}$ οι οποίες είναι τοπολογικές ομάδες εφοδιασμένες με την διακριτή τοπολογία. Επιπλέον απαιτούμε και μια συλλογή ομομορφισμών $\{f^j_i : G_j \rightarrow G_i | i,j \in J \quad\forall i\leq j\}$ οι οποίοι ικανοποιούν τις εξής σχέσεις:
$$f^i_i = id(G_i)$$
$$f^j_i \circ f^k_j = f^k_i$$
\end{defn}
\end{frame}

\begin{frame}
\frametitle{Αντίστροφο Όριο}
    \begin{defn}[\tl{Inverse Limit}] Αντίστροφο όριο ενός συστήματος όπως παραπάνω θα λέμε μια ομάδα $G$ μαζί με τους ομομορφισμούς $f_i : G \rightarrow G_i$ που ικανοποιούν $f^j_i \circ f_j = f_i$ για κάθε ζεύγος $i \leq j$, εφόσον η ομάδα $G$ ικανοποιεί την παρακάτω καθολική ιδιότητα:
\end{defn}
\end{frame}
\begin{frame}
\begin{block}{Καθολική ιδιότητα αντιστρόφων ορίων}
Αν $H$ είναι μια ομάδα μαζί με ομομορφισμούς $\tau_i : H \rightarrow G_i$ που ικανοποιούν $f^j_i \circ \tau_j = \tau_i$ για κάθε ζεύγος $i\leq j$ τότε υπάρχει μοναδικός ομομορφισμός $\tau : H  \rightarrow G$ με $\tau_i = f_i \circ \tau$ για κάθε $i$. Δηλαδή το παρακάτω διάγραμμα μετατίθεται:
\end{block}
\begin{center}
\includegraphics[width=5cm,clip]{διαγρ.PNG}
\end{center}
\end{frame}

\begin{frame}
\frametitle{Αντίστροφο Όριο}
    Έτσι μπορεί να δειχθεί ότι το αντίστροφο όριο ενός συστήματος υπάρχει, είναι μοναδικό ως προς ισομορφισμό και είναι το 
$$\varprojlim G_i  = \{(g_i)_{i \in J} \in \prod\limits_{i \in J} G_i : \quad f^j_i (g_j) = g_i \quad\forall i\leq j\}$$

Σαν ομάδα, το αντίστροφο όριο είναι υποομάδα της $\prod G_i$ και είναι τοπολογική ομάδα που παίρνει την επαγόμενη τοπολογία περιορισμό, εφόσον στην $\prod G_i$ δίνεται η τοπολογία γινόμενο.
\end{frame}


\begin{frame}
    \frametitle{\tl{Profinite} Ομάδες}
    \begin{defn}[\tl{Profinite}]Μια τοπολογική ομάδα λέγεται αν είναι ισόμορφη με το αντίστροφο όριο ενός αντιστρόφου συστήματος πεπερασμένων ομάδων. 
\end{defn}
\pause
\begin{block}{Ισοδύναμος ορισμός}
Τα προηγούμενα αποτελέσματα θα μπορούσαν να παραπέμψουν κάποιον ότι ένας ισοδύναμος ορισμός είναι ακριβώς η τοπολογική ομάδα να έχει τις ιδιότητες: συμπάγεια, \tl{Hausdorff} και \tl{totally disconnected}.
\end{block}

\pause
Ένα παράδειγμα χωρίς ιδιαίτερο ενδιαφέρον είναι ότι κάθε πεπερασμένη ομάδα μαζί με την διακριτή τοπολογία είναι \tl{profinite}.
\end{frame}

\begin{frame}
    \frametitle{\tl{Profinite} Ομάδες}
    Το παράδειγμα που μας ενδιαφέρει είναι ότι για κάθε άπειρη επέκταση \tl{Galois}, η ομάδα \tl{Galois} που προκύπτει είναι \tl{profinite}. Με τους προηγούμενους ορισμούς, αν θεωρήσουμε την συλλογή πεπερασμένων ομάδων με την διακριτή τοπολογία:
$$\{G/N: \quad N \in \mathcal{N}\}$$
και ως ομομορφισμούς:
$$f^j_i : G/N_i \longrightarrow G/N_j$$
τις κανονικές προβολές, όπου $N_i \geq N_j \iff N_i \subseteq N_j$
δηλαδή τις απεικονίσεις:
$$G/Gal(K/E_i) \cong Gal(E_i /F) \longrightarrow Gal(E_j /F) \cong G/Gal(K/E_j)$$
$$\sigma \longmapsto \sigma|_{E_j}$$
\end{frame}

\begin{frame}
Τότε τα παραπάνω αποτελούν αντίστροφο σύστημα και μάλιστα έχουμε τον ομοιομορφισμό τοπολογικών ομάδων:

$$G \cong \varprojlim G/N$$

\noindent δηλαδή, η τοπολογία που προκύπτει στο αντίστροφο όριο ως τοπολογία περιορισμός δεν είναι άλλη από την τοπολογία \tl{Krull}.
\end{frame}




\begin{frame}
...ή μάλλον βραχεία ακριβής ακολουθία \tl{profinite} τοπολογικών ομάδων!
     $ $\newline
    \includegraphics[width=\linewidth,height=\textheight,keepaspectratio]{βαα.PNG}
\end{frame}

\begin{frame}
    \begin{block}{Κάπου εδώ ένας αγαπημένος καθηγητής μας θα ρώταγε:}
    Ισχύει το αντίστροφο?
    \end{block}
    \pause
    
    \includegraphics[width=\linewidth,height=\textheight,keepaspectratio]{antistrofo.PNG}
\end{frame}


\begin{frame}
    Σε προηγούμενο παράδειγμα είδαμε ότι αν $K = \mathbb Q (i, \sqrt 2 , \sqrt 3 , \sqrt 5 , \sqrt 7 , \ldots)$ τότε:
	$$Gal(K/\mathbb{Q}) \cong \prod \limits_{i=1}^{\infty} \mathbb Z / 2 \mathbb Z$$
	Τα ενδιάμεσα σώματα του $K$ θα είναι της μορφής: $E = \mathbb{Q} (\sqrt{p_1},\ldots \sqrt{p_n})$ και έτσι αν πάρουμε $\leq \iff \subseteq$
	$$E_1 = \mathbb{Q} (\sqrt{p_1},\ldots \sqrt{p_n}) \leq  E_2 = \mathbb{Q} (\sqrt{p_1},\ldots \sqrt{p_n},\sqrt{q_1},\ldots ,\sqrt{p_n})$$
	με τις απεικονίσεις $Gal(E_2/ \mathbb{Q}) \rightarrow Gal(E_1 / \mathbb{Q})$ με κανόνα τον περιορισμό $\sigma \mapsto \sigma|_{E_1}$ τότε οι παραπάνω ομάδες είναι ισόμορφες με το αντίστροφο όριο για το σύστημα που μόλις ορίσαμε.
	$$\varprojlim\limits_{p \text{\tl{ prime}}} Gal(\mathbb{Q}(\sqrt{p}) / \mathbb{Q})$$
\end{frame}


\begin{frame}
    \frametitle{\tl{P-adic Numbers}}
    Ο ορισμός της προσθετικής ομάδας των \tl{p-adic} ακεραίων είναι η \tl{profinite} ομάδα $\varprojlim \mathbb{Z}/p^n \mathbb{Z}$ όπου το $n$ διατρέχει τους φυσικούς μαζί με τις φυσικές απεικονίσεις $\mathbb{Z}/p^n \mathbb{Z} \rightarrow \mathbb{Z}/p^m \mathbb{Z}$ για όλα τα $n\geq m$.
    
    $ $\newline
    H τοπολογία που προκύπτει στο αντίστροφο όριο ταυτίζεται με την τοπολογία που έχουν οι \tl{p-adic} ακέραιοι μέσω του συνήθους ορισμού τους από την ανάλυση.

\end{frame}


\begin{frame}
\frametitle{Βιβλιογραφία}
\begin{thebibliography}{9}
	\bibitem{pmorandi}
	\tl{Patrick Morandi.}
	\textit{\tl{Fields and Galois Theory.}}
	\tl{Springer-Verlag, New York, 1996.}

	\bibitem{jsmilne}
	\tl{James S. Milne.}
	\textit{\tl{Fields and Galois Theory.}}
	\tl{Available at www.jmilne.org/math/, 2020.}
	
	\bibitem{fbutler}
	\tl{Frederick Butler.}
	\textit{\tl{Infinite Galois Theory, Master Thesis, University of Pennsylvania, 2001}}
\end{thebibliography}
    
\end{frame}
\end{document}
