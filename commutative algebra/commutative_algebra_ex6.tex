\documentclass[oneside,a4paper]{article}

%%%%%%%%%%%%%%%%%%%%%%%%%%%%
\usepackage{amsthm}
\usepackage{amsmath}
\usepackage{amssymb}
%%%%%%%%%%%%%%%%%%%%%%%%%%%%%
\usepackage[greek]{babel}
\usepackage[utf8]{inputenc}
\usepackage{mathtools}
\usepackage{blindtext}
\usepackage[T1]{fontenc}
\usepackage{titlesec}
\usepackage{sectsty}
\usepackage{verbatim}
\usepackage{multirow}
\chapternumberfont{\tiny} 
\chaptertitlefont{\Huge}
%ελληνικοι χαρακτηρες σε μαθ pdf utf-8
%%%%%%%%%%%%%%%%%%%%%%%%%%%%%%%%%
\usepackage{tikz-cd}

\usepackage{xcolor}
\usepackage{framed}%frames

\usepackage{array}
\usepackage{pbox}
\usepackage{comment}
%%%%%%%%%%%%%%%%%%%%%%%%
\usepackage{tikz}
%%%%%%%%%%%%%%%%%%%%%%%%%%

%%%%%%%%περιθώρια%%%%%%%%%%%%
\usepackage[a4paper,margin=3.5cm]{geometry}


%%%%%%%%συντομευσεις%%%%%%%%%%
\newtheorem{theorem}{Θεώρημα}
\newtheorem{lemma}{Λήμμα}
\newtheorem{example}{Παράδειγμα}
\newtheorem*{defn}{Ορισμός}
\newtheorem{prop}{Πρόταση}
\newtheorem{cor}{Πόρισμα}

\newcommand {\tl}{\textlatin}
%%%%%%%%%αριθμηση%%%%%%%%%%%%%%
\renewcommand{\theenumi}{\arabic{enumi}}
\renewcommand{\labelenumi}{{\rm(\theenumi)}}
\renewcommand{\labelenumii}{\roman{enumii}) }
%%%%%%%%%%%% New theorems %%%%%%%%%%%%%%%%%%%%%%%%

%%%%%%%%%%%%%%%%%%%%%%%%%%%%%%%%%%%%%%%%%%%%%%%%%%%
\newcommand{\Z}{\mathbb{Z}}
\newcommand{\Q}{\mathbb{Q}}
\newcommand{\Co}{\mathbb{C}}
%%%%%%%%%%%%%%%%%%%%% Document starts %%%%%%%%%%%%
\begin{document}
	
	%%%%%%%%%%%%%%%%%%%%%%%%%%%%%%%%%%%%%%%%%%%%%%%%%%
	\selectlanguage{greek}
	%%%%%%%%%%%%%%%%%%%%%%% Start Roman numbering %%%% vbbnn
	%\pagenumbering{roman}
	%%%%%%%%%%%%%%%%%%%%%%%%%%%%%%%%%%%%%%%%%%%%%%%%%%
	
	\begin{framed}	
		%\vspace{0.3truecm}
		\begin{center}
			\huge Μεταθετική Άλγεβρα
		\end{center}
		%\vspace{0.3truecm}
		\begin{center}
			\huge Εργασία 6
		\end{center}
		\vspace{0.3truecm}
		\begin{center}
			Ονομ/νο: Νούλας Δημήτριος\\
			ΑΜ: 1112201800377\\
			\tl{email}: \tl{dimitriosnoulas@gmail.com}
		\end{center}
		\vspace{0.3truecm}
	\end{framed}
	\vspace*{\fill}
	\begin{center}
	\includegraphics[width=0.5\textwidth]{/mnt/c/Users/dimit/Desktop/TeX/uoa_logo}
	%\includegraphics[width=0.5\textwidth]{C:/Users/dimit/Desktop/TeX/uoa_logo}
	\end{center}
\vspace{1cm}
\pagebreak

\noindent Άσκηση $8.4)$ Ποιες από τις ακόλουθες επεκτάσεις δακτυλίων του $\Z$ είναι ακέραιες?
\begin{enumerate}
	\item $\Z[\sqrt{2} + \sqrt[3]{3} + \sqrt[5]{5}]$.
	\item $\Z[1/\sqrt{2}]$.
	\item $\Z[x]/(x^3)$.
\end{enumerate}
\vspace*{0.1cm}
\begin{proof} $ $

	$ $\newline
	$(1)$ Έχουμε ότι τα $\sqrt{2},\sqrt[3]{3},\sqrt[5]{5}$ είναι ακέραια πάνω από το $\Z$ εφόσον είναι ρίζες των μονικών πολυωνύμων $x^2 - 2, x^3 -3, x^5 -5$ αντίστοιχα. Άρα το $\Z[\sqrt{2},\sqrt[3]{3},\sqrt[5]{5}]$ είναι πεπερασμένα παραγόμενο $\Z$-πρότυπο και έτσι κάθε στοιχείο του είναι ακέραιο πάνω από το $\Z$. Ειδικότερα, περιέχει τον δακτύλιο $\Z[\sqrt{2} + \sqrt[3]{3} +\sqrt[5]{5}]$ και άρα αυτός είναι μια ακέραια επέκταση του $\Z$.

	\vspace*{0.5cm}
	$ $\newline
	$(2)$ Αν το $\Z[1/\sqrt{2}]$ ήταν ακέραια επέκταση τότε θα ήταν ακέραιο πάνω από το $\Z$ και το στοιχείο $$\frac1{\sqrt{2}}\cdot \frac1{\sqrt{2}} = \frac12 \in \Q$$ Ωστόσο, το $\Z$ είναι ακέραια κλειστό στο $\Q$ όπως είναι κάθε περιοχή μοναδικής παραγοντοποίησης στο σώμα πηλίκων της. Άρα έχουμε άτοπο αφού $\frac{1}{2}$ ακέραιο και δεν ανήκει στο $\Z$.

	\vspace*{0.5cm}
	$ $\newline
	$(3)$ Καθώς το $x^3$ είναι μονικό, δηλαδή ο μεγιστοβάθμιος συντελεστής είναι αντιστρέψιμος, έχουμε αλγόριθμο διαίρεσης στο $\Z[x]$. Έτσι, η εικόνα ενός $f(x) \in \Z[x]$ μέσω του φυσικού επιμορφισμού θα είναι $f_0 + f_1 x + f_2 x^2$ στο $\Z[x]/(x^3)$. Δηλαδή, το $\Z[x]/(x^3)$ είναι πεπερασμένα παραγόμενο $\Z$-πρότυπο ίσο με $(1,x,x^2)$. Άρα είναι και ακέραια επέκταση του $\Z$.
\end{proof}


\pagebreak
\noindent Καθώς χρησιμοποιήσαμε την άσκηση $8.1)$, δηλαδή ότι κάθε περιοχή μοναδικής παραγοντοποίησης είναι ακέραια κλειστή στο σώμα πηλίκων της, θα την αποδείξουμε.
\vspace*{1cm}
\begin{proof} $ $

	$ $\newline
	Αν το $R$ είναι σώμα θα ταυτίζεται με το σώμα πηλίκων του άρα δεν έχουμε κάτι να δείξουμε. Έστω ότι το $R$ δεν είναι σώμα και $k$ το σώμα πηλίκων του. Αν $\frac{x}{y} \in k$ είναι ακέραιο πάνω από το $R$, με $x,y \in R$ θα δείξουμε ότι $\frac{x}{y} \in R$. Εφόσον είμαστε σε περιοχή μοναδικής παραγοντοποίησης μπορούμε να υποθέσουμε ότι δεν υπάρχει ανάγωγο στοιχείο $p \in R$ που να διαιρεί ταυτόχρονα τα $x$ και $y$, διαφορετικά από την μοναδική παραγοντοποίηση το διαγράφουμε από αριθμητή και παρονομαστή και παίρνουμε νέο κλάσμα χωρίς κοινό ανάγωγο διαιρέτη. Εφόσον το $\frac{x}{y}$ είναι ακέραιο πάνω από το $R$ έχουμε ότι υπάρχουν $a_i \in R$ τέτοια ώστε 
	$$\left(\frac{x}{y}\right)^n + a_{n-1}\left(\frac{x}{y}\right)^{n-1} + \ldots + a_1\left(\frac{x}{y}\right) + a_0 = 0$$ την οποία σχέση πολλαπλασιάζουμε με το $y^n$. Έτσι έχουμε $$x^n + a_{n-1}x^{n-1}y + \ldots + a_1 x y^{n-1} + a_0 y^n = 0$$

	$ $\newline
	Αν τώρα ισχύει ότι $y \not\in U(R)$, τότε λόγω μοναδικής παραγοντοποίησης έχουμε ότι υπάρχει ανάγωγο $p$ τέτοιο ώστε $p | y$. Εφόσον το $y$ είναι σε όλους τους όρους εκτός από τον πρώτο και στο δεύτερο μέλος της σχέσης έχουμε 0, έπεται ότι $p | x^n$. Καθώς το $p$ είναι ανάγωγο σε περιοχή μοναδικής παραγοντοποίησης έχουμε ότι $p |x$ εφόσον
	$$\text{Αν } x = u q^{m_1}_1 \cdots q^{m_s}_s, \quad p|x^n \implies pk = x^n = u^n q^{nm_1}_1 \cdots q^{nm_s}_s$$ και λόγω μοναδικής παραγοντοποίησης το ανάγωγο $p$ θα ταυτίζεται με κάποιο από τα $q_i$ και άρα $p|x$. Αυτό είναι άτοπο καθώς έχουμε ότι δεν υπάρχει ανάγωγο $p$ να διαιρεί ταυτόχρονα τα $x,y$. Συνεπώς $y \in U(R)$ και έτσι έχουμε
	$$\frac{x}{y} = \frac{x}{y}\cdot 1 = \frac{x}{y} \frac{y^{-1}}{y^{-1}} = xy^{-1} \in R$$ δηλαδή το $R$ είναι ακέραια κλειστό στο $k$. 
\end{proof}
\pagebreak

\noindent Άσκηση $8.5)$ Αν $R\subseteq S$ είναι δακτύλιοι με $S$ ακέραιο πάνω από το $R$, τότε ο $S[x]$ είναι ακέραιος πάνω από τον $R[x]$.
\vspace*{1cm}
\begin{proof} $ $

	$ $\newline
	Θα χρησιμοποιήσουμε την παρατήρηση ότι αν έχουμε για τυχαίους δακτυλίους $R\subseteq S$ ότι αν το $S$ είναι πεπερασμένα παραγόμενο $R$-πρότυπο, τότε και το $S[x]$ θα είναι πεπερασμένα παραγόμενο $R[x]$-πρότυπο. Πράγματι, έστω ότι έχουμε $$S = (s_1,\ldots,s_n)$$ ως $R$-πρότυπο και $f(x) \in S[x]$ με $$f(x) = f_n x^n + \ldots f_1 x + f_0, \quad f_i \in S$$ εφόσον $f_i \in S$ γράφουμε $$f_i = \sum\limits_{j=1}^n y_{ij}s_j, \quad y_{ij} \in R$$

	$ $\newline Αντικαθιστώντας τα $f_i$ στο πολυώνυμο $f(x)$ και αναδιατάσσοντάς το σαν να είναι πολυώνυμο των $s_1,\ldots,s_n$ έχουμε $$f(x) = g_1(x) s_1 + g_2 (x) s_2 + \ldots + g_n (x) s_n$$ όπου τα $g_i$ είναι πολυώνυμα του $x$ με συντελεστές πράξεις των $y_{ij}$. Άρα $g_i(x) \in R[x]$ για κάθε $i=1,\ldots,n$. Συνεπώς τα $s_1,\ldots,s_n$ παράγουν το $S[x]$ ως $R[x]$-πρότυπο.

	$ $\newline
	Τώρα θεωρούμε ένα πολυώνυμο $s(x) \in S[x]$ και θα δείξουμε ότι είναι ακέραιο πάνω από το $R[x]$. Έστω ότι οι συντελεστές του είναι οι $s_0,s_1,\ldots,s_n \in S$. Εφόσον αυτά είναι στοιχεία της ακέραιας επέκτασης $S$ έχουμε ότι το $R$-πρότυπο $R[s_0,s_1,\ldots,s_n]$ είναι πεπερασμένα παραγόμενο. Με βάση την παρατήρηση, το $R[s_0,s_1,\ldots,s_n][x]$ είναι πεπερασμένα παραγόμενο $R[x]$-πρότυπο. Έτσι, έχουμε ότι κάθε στοιχείο του $R[s_0,s_1,\ldots,s_n,x]$ είναι ακέραιο πάνω από το $R[x]$. Ειδικότερα, το $s(x)$ με το οποίο ξεκινήσαμε ανήκει σε αυτόν τον δακτύλιο, άρα το $s(x)$ είναι ακέραιο πάνω από το $R[x]$.

\end{proof}

\pagebreak

\noindent Άσκηση $8.6)$. Έστω $k$ σώμα, $R=k[x^2]$ και $S = k[x]$.
\begin{enumerate}
	\item Αληθεύει ότι το $S$ είναι ακέραιο πάνω από το $R$?
	\item Αληθεύει ότι το $S$ είναι πεπερασμένα παραγόμενο $R$-πρότυπο? Ελεύθερο $R$-πρότυπο?
	\item Δείξτε ότι κάθε $f \in S$ είναι ρίζα πολυωνύμου βαθμού 2 με συντελεστές στο $R$.
\end{enumerate}
\vspace*{1cm}
\begin{proof} $ $

	$ $\newline
	$(1)$ Το $S$ είναι πράγματι ακέραιο πάνω από το $R$ καθώς μπορούμε να γράψουμε $S = k[x] = k[x,x^2] = (k[x^2])[x] = R[x]$ και αρκεί να δείξουμε ότι το $x$ είναι ακέραιο πάνω από το $R$. Αυτό ισχύει εφόσον είναι ρίζα του μονικού $t^2 - x^2 \in R[t]$.
	
	\vspace*{0.5cm}
	$ $\newline
	$(2)$ Ισχύει ότι είναι και πεπερασμένα παραγόμενο $R$-πρότυπο και ελεύθερο. Έστω $f(x) \in k[x]$. Χωρίζουμε τις δυνάμεις σε άρτιες και περιττές, δηλαδή $$f(x) = \left(f_0 + f_2 x^2 + \ldots \right) + \left(f_1 x + f_3 x^3 + \ldots \right) = $$
	$$ = \left(f_0 + f_2 x^2 + \ldots \right) + x(f_1 + f_3 x^2 + \ldots) = 1 \cdot g_1(x^2) + x \cdot g_2(x^2)$$ με $g_1(x^2), g_2(x^2) \in k[x^2]$. Άρα τα $1,x$ παράγουν το $k[x]$ ως $k[x^2]$-πρότυπο. Αυτά τα δύο είναι και $k[x^2]$-γραμμικά ανεξάρτητα, άρα το πρότυπο είναι ελεύθερο. Πράγματι, αν $$g_1(x^2) + g_2(x^2)x = 0$$ τότε το πρώτο πολυώνυμο θα αποτελείται μόνο από άρτιες δυνάμεις και το δεύτερο μόνο από περιτές, δηλαδή αυτά τα δύο πολυώνυμα θα αθροίζουν στο μηδενικό πολυώνυμο αν είναι και τα δύο τα μηδενικά πολυώνυμα. Άρα το σύνολο ${1,x}$ είναι μια βάση του $R$-προτύπου $S$.

	\vspace*{0.5cm}
	$ $\newline
	$(3)$ Έστω $f(x) \in k[x]$. Το αποτέλεσμα είναι μια απλή εφαρμογή της πρότασης $1$ της παραγράφου $8.1$ των σημειώσεων. Πράγματι, για $M = S =  k[x]$ και για $I = k[x^2]$ ιδεώδες του $k[x^2]$ έχουμε ότι το $M$ ως $R$-πρότυπο παράγεται από δύο στοιχεία και $f(x) k[x] \subseteq k[x^2]k[x] = k[x]$ (αφού το $k[x^2]k[x]$ ως ιδεώδες του $k[x]$ περιέχει το 1). Άρα υπάρχουν $a_1 \in I, a_2 \in I^2 \subseteq k[x^2]$ τέτοια ώστε $$\left(f(x)\right)^2 + a_1 f(x) + a_2 \in Ann_S M$$ και φυσικά $$Ann_S M = \{s(x) \in k[x]: \quad s(x)h(x) = 0 \quad \forall h(x) \in k[x]\} = \{0\}$$

\end{proof}
\pagebreak



\noindent Άσκηση $9.3)$ Έστω $f \in k[x_1,...,x_n]$ που δεν διαιρείται με το τετράγωνο αναγώγου πολυωνύμου. Αληθεύει ότι $I(V(f))=(f)$?

\vspace*{1cm}
\begin{proof} $ $

	$ $\newline
	Από το \tl{Nullstellensatz} αρκεί να δείξουμε ότι το ιδεώδες $(f)$ είναι ριζικό και θα αληθεύει η παραπάνω σχέση. Έχουμε ότι το $k[x_1,\ldots,x_n]$ είναι περιοχή μοναδικής παραγοντοποίησης και μαζί με το γεγονός ότι το $f$ είναι ελεύθερο τετραγώνων παίρνουμε ότι:
	$$f = u f_1 f_2 \cdots f_n$$ με $f_i$ ανάγωγα και $u \in U(k[x_1,\ldots,x_n]) = \{\pm 1\}$. Χωρίς την υπόθεση δηλαδή θα είχαμε εκθέτες μεγαλύτερου ίσου του $1$ στα $f_i$. Ισχυριζόμαστε ότι αυτό σε σχέση ιδεωδών γίνεται: 
	$$(f) = (f_1 \cdots f_n) = (f_1)\cap (f_2) \cap \ldots \cap (f_n)$$ και τα $(f_i)$ είναι πρώτα ιδεώδη. Πράγματι, αν έχουμε $$ab \in (f_i) \implies f_i h_i = ab \implies f_i| a \text{ ή } f_i|b $$ αφού είμαστε σε περιοχή μοναδικής παραγοντοποίησης και ο ανάγωγος παράγοντας $f_i$ θα βρίσκεται αναγκαστικά και στο δεξί μέλος, δηλαδή θα είναι παράγοντας σε τουλάχιστον κάποιο από τα δύο $a,b$.

	$ $\newline
	Με βάση τον ισχυρισμό, έχουμε ότι
	$$\sqrt{(f)} = \sqrt{(f_1)\cap \ldots \cap (f_n)} = \sqrt{(f_1)}\cap \ldots \cap \sqrt{(f_n)} = (f_1)\cap\ldots\cap (f_n) = (f)$$ όπου φυσικά $\sqrt{(f_i)} = (f_i)$ αφού είναι πρώτα ιδεώδη. Άρα ισχύει $$I(V(f)) = \sqrt{(f)} = (f)$$

	$ $\newline
	Θα αποδείξουμε τώρα τον ισχυρισμό. Η σχέση $(\subseteq)$ είναι προφανής καθώς κάθε $f_i$ διαιρεί το $f$ και άρα $(f) \subseteq (f_i)$ για κάθε $i$. Αντίστροφα, έστω $g \in (f_1)\cap\ldots\cap (f_n)$. Έχουμε $$g \in (f_1) \implies g = h_1 f_1$$ και στην συνέχεια $$g = h_1 f_1 \in (f_2) \implies f_2 | h_1 f_1 \implies f_2 | h_1 $$ εφόσον είμαστε σε περιοχή μοναδικής παραγοντοποίησης και τα $f_1,f_2$ είναι ανάγωγα. Δηλαδή, $g = h_2 f_2 f_1$. Επαναλαμβάνοντας αυτή τη διαδικασία σε πεπερασμένα $n$ βήματα παίρνουμε $$g = h_n f_n \cdots f_2 f_1 = h_n f \in (f)$$ και άρα αποδείξαμε τον ισχυρισμό. 
\end{proof}

\pagebreak
\noindent \textbf{Σημείωση για το (3) της άσκησης 8.6 από την βιβλιογραφία.}

$ $\newline
Αν συμβουλευτούμε την λύση που υπάρχει στους \tl{Altman-Kleinman} θα δούμε ότι αφενός κρύβεται το θεώρημα \tl{Cayley-Hamilton} και αφετέρου μπορούμε να βρούμε αυτά τα $a_1,a_2$. Αρχικά, αντί για την πρόταση 1 μπορούμε να χρησιμοποιήσουμε την πιο γενική πρόταση $2.4$ των \tl{Atiyah-Macdonald} η οποία λέει το εξής:

$ $\newline
Έστω $M$ ένα πεπερασμένα παραγόμενο $R$-πρότυπο, $I$ ένα ιδεώδες του $R$ και $\phi : M \rightarrow M$ ένας ομομορφισμός $R$-προτύπων ή αλλιώς ενδομορφισμός του $M$ τέτοιος ώστε $\phi(M) \subseteq IM$. Τότε ο ομομορφισμός $\phi$ ικανοποιεί μια σχέση της μορφής $$\phi^n + a_1 \phi^{n-1} + \ldots + a_n = 0$$ όπου το $0$ εδώ είναι ο μηδενικός ομομορφισμός $M \rightarrow M$ και το $n$ είναι ο αριθμός των γεννητόρων του $M$. Ουσιαστικά, η πρόταση 1 είναι ειδική περίπτωση αυτής της πρότασης για τον ομομορφισμό $m \mapsto s\cdot m$. Η απόδειξη αυτής της πρότασης είναι εντελώς όμοια με την απόδειξη της πρότασης 1 και βασίζεται στο τέχνασμα της ορίζουσας. Με βάση αυτήν την πρόταση έχουμε και το θεώρημα \tl{Cayley-Hamilton} για πίνακες με στοιχεία από μεταθετικό δακτύλιο με μονάδα. 

$ $\newline
Φυσικά, η πρόταση αναφέρεται για ομομορφισμούς $R$-προτύπων και όχι συγκεκριμένα για $k$-διανυσματικούς χώρους που έχουμε συνηθίσει από τα μαθήματα γραμμικής άλγεβρας. Έτσι, αν θεωρήσουμε πίνακα $\mathcal{A}$ που προκύπτει ως πίνακας $R$-ομομορφισμού προτύπων ως προς συγκεκριμένη βάση ελευθέρου προτύπου, όμοια με τους πίνακες των γραμμικών απεικονίσεων της γραμμικής άλγεβρας, αυτός ο πίνακας $\mathcal{A}$ θα ικανοποιεί το χαρακτηριστικό του πολυώνυμο. Αυτό το έχουμε από την πρόταση των \tl{Atiyah-Macdonald} αν θεωρήσουμε το $R^n$ ως $R$-πρότυπο με την συνήθη βάση $e_1,\ldots,e_n$ και θεωρήσουμε ως $\phi$ τον ενδομορφισμό του $R^n$ που προκύπτει ως προς την συνήθη βάση και τον πίνακα $\mathcal{A}$. Υπενθυμίζουμε ότι ο ενδομορφισμός $\phi$ θα καθορίζεται πλήρως από τις εικόνες του στα $e_i$ και τις εικόνες αυτές τις παίρνουμε από τον πολλαπλασιασμό $\mathcal{A} \cdot e^{T}_i$.

$ $\newline
Επιπλέον, οι \tl{Altman-Kleinman} αποδεικνύουν ότι μπορούμε να φτάσουμε σε αυτή την πρόταση των \tl{Atiyah-Macdonald} ξεκινώντας απο το θεώρημα \tl{Cayley-Hamilton}, δηλαδή το θεώρημα \tl{Cayley-Hamilton} και το τέχνασμα της ορίζουσας είναι ισοδύναμα.


$ $\newline
Έτσι αν θεωρήσουμε στην συγκεκριμένη άσκηση τον ενδομορφισμό $$\phi : k[x] \rightarrow k[x]$$
$$g(x) \mapsto f(x)g(x)$$ και γράψουμε $f(x) = f_e (x) + f_o(x)$ όπου ξεχωρίζουμε τις άρτιες και περιττές δυνάμεις, τότε για την βάση $\{1,x\}$ έχουμε ότι 
$$\phi(1) = f(x) = f_e(x) + f_o(x) = f_e (x) + G \cdot x $$
όπου $G$ είναι το πηλίκο της διαίρεσης του $f_o(x)$ με το $x$ και $G \in k[x^2]$ αφού το $f_o(x)$ περιέχει τις περιττές δυνάμεις. Επιπλέον

$$\phi(x) = f(x)x = f_e(x) x + Gx^2 = 1 \cdot Gx^2 + x \cdot f_e(x)$$

$ $\newline
Δηλαδή, ο πίνακας που ορίζεται από την $\phi$ ως προς την βάση $\{1,x\}$ (και του πεδίου ορισμού και του πεδίου άφιξης) είναι ο ακόλουθος: $$\begin{pmatrix}
f_e & Gx^2 \\
G & f_e
\end{pmatrix} \in M_2 (k[x^2])$$ με χαρακτηριστικό πολυώνυμο $T^2 -2 f_e(x)T + \left(f_e(x)\right)^2 - \left(f_o(x)\right)^2$. Άρα από την ισοδυναμία του θεωρήματος \tl{Cayley-Hamilton} με το τέχνασμα της ορίζουσας, έχουμε ότι ο $\phi$ ικανοποιεί αυτή τη σχέση. Δηλαδή:
$$\left(f(x)\right)^2 - f_e(x)f(x) + \left(f_e(x)\right)^2 - \left(f_o(x)\right)^2 = 0$$ και έτσι μπορούμε να υπολογίσουμε κατευθείαν από το δοσμένο πολυώνυμο τα $a_1,a_2$.
\pagebreak


\end{document}
