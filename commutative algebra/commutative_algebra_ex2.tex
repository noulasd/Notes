\documentclass[oneside,a4paper]{article}

%%%%%%%%%%%%%%%%%%%%%%%%%%%%
\usepackage{amsthm}
\usepackage{amsmath}
\usepackage{amssymb}
%%%%%%%%%%%%%%%%%%%%%%%%%%%%%
\usepackage[greek]{babel}
\usepackage[utf8]{inputenc}
\usepackage{mathtools}
\usepackage{blindtext}
\usepackage[T1]{fontenc}
\usepackage{titlesec}
\usepackage{sectsty}
\usepackage{verbatim}
\usepackage{multirow}
\chapternumberfont{\tiny} 
\chaptertitlefont{\Huge}
%ελληνικοι χαρακτηρες σε μαθ pdf utf-8
%%%%%%%%%%%%%%%%%%%%%%%%%%%%%%%%%
\usepackage{tikz-cd}

\usepackage{xcolor}
\usepackage{framed}%frames

\usepackage{array}
\usepackage{pbox}

%%%%%%%%%%%%%%%%%%%%%%%%
\usepackage{tikz}
%%%%%%%%%%%%%%%%%%%%%%%%%%

%%%%%%%%περιθώρια%%%%%%%%%%%%
\usepackage[a4paper,margin=3.5cm]{geometry}


%%%%%%%%συντομευσεις%%%%%%%%%%
\newtheorem{theorem}{Θεώρημα}
\newtheorem{lemma}{Λήμμα}
\newtheorem{example}{Παράδειγμα}
\newtheorem*{defn}{Ορισμός}
\newtheorem{prop}{Πρόταση}
\newtheorem{cor}{Πόρισμα}

\newcommand {\tl}{\textlatin}
%%%%%%%%%αριθμηση%%%%%%%%%%%%%%
\renewcommand{\theenumi}{\arabic{enumi}}
\renewcommand{\labelenumi}{{\rm(\theenumi)}}
\renewcommand{\labelenumii}{\roman{enumii}) }
%%%%%%%%%%%% New theorems %%%%%%%%%%%%%%%%%%%%%%%%

%%%%%%%%%%%%%%%%%%%%%%%%%%%%%%%%%%%%%%%%%%%%%%%%%%%
\newcommand{\Z}{\mathbb{Z}}
\newcommand{\Q}{\mathbb{Q}}
\newcommand{\Co}{\mathbb{C}}
%%%%%%%%%%%%%%%%%%%%% Document starts %%%%%%%%%%%%
\begin{document}
	
	%%%%%%%%%%%%%%%%%%%%%%%%%%%%%%%%%%%%%%%%%%%%%%%%%%
	\selectlanguage{greek}
	%%%%%%%%%%%%%%%%%%%%%%% Start Roman numbering %%%% vbbnn
	%\pagenumbering{roman}
	%%%%%%%%%%%%%%%%%%%%%%%%%%%%%%%%%%%%%%%%%%%%%%%%%%
	
	\begin{framed}	
		%\vspace{0.3truecm}
		\begin{center}
			\huge Μεταθετική Άλγεβρα
		\end{center}
		%\vspace{0.3truecm}
		\begin{center}
			\huge Εργασία 2
		\end{center}
		\vspace{0.3truecm}
		\begin{center}
			Ονομ/νο: Νούλας Δημήτριος\\
			ΑΜ: 1112201800377\\
			\tl{email}: \tl{dimitriosnoulas@gmail.com}
		\end{center}
		\vspace{0.3truecm}
	\end{framed}
	\vspace*{\fill}
	\begin{center}
	\includegraphics[width=0.5\textwidth]{C:/Users/dimit/Desktop/TeX/uoa_logo}
	\end{center}
\vspace{1cm}
\pagebreak

\begin{center}
	\includegraphics[keepaspectratio]{C:/Users/dimit/Desktop/TeX/template_nullstellensatz.jpg}
	\end{center}
\pagebreak

\noindent Άσκηση $2.1)$
\quad Βρείτε τα πρώτα και μέγιστα ιδεώδη του $R$ καθώς για το $nil(R)$ και $Jac(R)$ στις ακόλουθες περιπτώσεις.
\begin{enumerate}
    \item $R = \Z$.
    \item $R = \Z_n, n=p^2q^3, p,q$ διακεκριμένοι πρώτοι.
    \item $R = \mathbb{R}[x]$.
    \item $R = \mathbb{C}[x]$.
    \item $R = \mathbb{Q}[x]/(x^2(x-1))$.
\end{enumerate}
\begin{proof} $ $

	$ $\newline

	\begin{enumerate}
		\item Για το $R=\Z$ έχουμε ότι το $\Z / p\Z$ είναι σώμα και ακέραια περιοχή για $p$ πρώτο. Αν έχουμε σύνθετο $n$ το $\Z / n \Z$ έχει μηδενοδιαιρέτες και άρα δεν είναι καν περιοχή. Επιπλέον το $(0)$ είναι πρώτο ιδεώδες καθώς $\Z / (0) \simeq \Z$ είναι περιοχή.
		Άρα τα πρώτα ιδεώδη είναι:
		$$(p) \quad p \text{ πρώτος, } (0) $$
		Τα μέγιστα είναι $(p)$ με $p$ πρώτο και
		$$nil(\Z) = \bigcap\limits_{p \text{ πρώτος}} (p) = (0)$$
		καθώς για $p\neq q, p \not\in (q)$. Διαφορετικά αν $x \in \Z$ με $x^n = 0 \implies x = 0$.

		Επιπλέον
		$$Jac(\Z) = (0) \cap \left(\bigcap\limits_{p \text{ πρώτος}} (p)\right) = (0)$$

		\item Για το $R=\Z_n$ με $n = p^2 q^3, p,q$ διακεκριμένοι πρώτοι έχουμε ότι τα ιδεώδη του $\Z_m$ είναι σε 1-1 και επί αντιστοιχία με τα ιδεώδη του $\Z$ που περιέχουν το $(m)$. Άρα το διάγραμμα ιδεωδών του $\Z_n$ είναι
		\begin{center}
		\begin{tikzcd}
			&                                                 & \mathbb{Z}_{p^2q^3} \arrow[ld, no head] \arrow[rd, no head] &                                                 &                               \\
			& {([p])} \arrow[rd, no head] \arrow[ld, no head] &                                                             & {([q])} \arrow[ld, no head] \arrow[rd, no head] &                               \\
{([p^2])} \arrow[rd, no head] &                                                 & {([pq])} \arrow[ld, no head] \arrow[rd, no head]            &                                                 & {([q^2])} \arrow[ld, no head] \\
			& {([p^2q])} \arrow[rd]                           &                                                             & {([pq^2])} \arrow[ld, no head]                  &                               \\
			&                                                 & {([p^2q^2])} \arrow[d, no head]                             &                                                 &                               \\
			&                                                 & (0)                                                         &                                                 &                              
\end{tikzcd}
\end{center}

και τα μέγιστα ιδεώδη όπως φαίνεται στο διάγραμμα είναι τα $([p]),([q])$. Μπορούμε να το δούμε διαφορετικά ως:
$$\frac{\Z_{p^2 q^3}}{([p])} = \frac{\Z / p^2q^3\Z }{p\Z / p^2q^3 \Z} \simeq \Z / p\Z$$
όπου ο ισομορφισμός είναι από το 3ο θεώρημα ισομορφισμών δακτυλίων και αυτό είναι σώμα, άρα το $([p])$ είναι μέγιστο. Όμοια και το $([q])$.

$ $\newline
καθώς σε έναν πεπερασμένο δακτύλιο θα έχουμε πεπερασμένα πηλίκα και κάθε πεπερασμένη περιοχή είναι σώμα, τα πρώτα με τα μέγιστα ιδεώδη ταυτίζονται. Άρα
$$nil(\Z_n) = Jac(\Z_n) = ([p])\cap ([q]) = ([pq])$$


\item Έχουμε το αποτέλεσμα ότι στα σώματα $k$ τα ιδεώδη του $k[x]$ είναι κύρια. Για $k=\mathbb{R}$ από το θεμελιώδες θεώρημα της άλγεβρας, αν $deg(f(x)) \geq 3$ τότε υπάρχει $z \in \mathbb{C}$ έτσι ώστε
$$x^2 - (z + \overline{z})x + z\overline{z} \in \mathbb{R}[x]$$
$$(x^2 - (z + \overline{z})x + z\overline{z}) | f(x)$$
άρα τα ανάγωγα $f(x) \in \mathbb{R}[x]$ είναι τα πολυώνυμα βαθμού $1$ και βαθμού $2$ με αρνητική διακρίνουσα. Αν τώρα ένα $g(x)$ έχει παραγοντοποίηση τότε ο δακτύλιος $\mathbb{R}[x]/(g(x))$ θα έχει μηδενοδιαιρέτες και δεν θα είναι περιοχή.

$ $\newline 
Άρα τα μέγιστα ιδεώδη του $\mathbb{R}[x]$ είναι τα $(x - a), (x^2 + ax + b)$ με $a^2-4b <0$ και τα πρώτα είναι τα ίδια μαζί με το $(0)$ εφόσον $\mathbb{R}[x]$ περιοχή. Επιπλέον, ένα μη μηδενικό πολυώνυμο υπεράνω των πραγματικών θα έχει μη μηδενικό μεγιστοβάθμιο συντελεστή $a_n$ και στο πολυώνυμο υψωμένο σε κάποια δύναμη $m$ θα εμφανίζεται ο συντελεστής $(a_n)^m \neq 0$. Άρα δεν υπάρχουν μηδενοδύναμα στοιχεία, δηλαδή $nil(\mathbb{R}[x]) = (0)$.

$ $\newline 
Αν $f(x) \in Jac(\mathbb{R}[x])$ με $f(x)$ όχι το μηδενικό πολυώνυμο, τότε από την πρόταση $2.4.3$ έχουμε ότι για κάθε $g(x) \in \mathbb{R}[x]$ ότι
$$1-f(x)g(x) \in U(\mathbb{R}[x]) = \mathbb{R} \setminus \{0\}$$
άρα για $g(x) = x$ παίρνουμε το συμπέρασμα ότι το $f(x)$ πρέπει να είναι βαθμού $0$ για να ισχύει ότι $deg(1-f(x)g(x))=0$ ώστε να είναι αντιστρέψιμο. Αν $f(x)=c \in \mathbb{R}$, τότε για $g(x) = \frac1c$ παίρνουμε $0 \in \mathbb{R}\setminus \{0\}$ το οποίο είναι άτοπο. Άρα $Jac(\mathbb{R}[x]) = (0)$.

\item Για $R = \mathbb{C}[x]$ όμοια με πριν, με το θεμελιώδες θεώρημα της άλγεβρας έχουμε ότι τα μόνα ανάγωγα πολυώνυμα είναι τα $x-a \in \mathbb{C}[x]$ και άρα τα μέγιστα ιδεώδη είναι τα $(x-a)$ και τα πρώτα τα ίδια μαζί με το $(0)$. Με τα ίδια επιχειρήματα με το $(3)$ έχουμε $nil(\mathbb{C}[x]) = Jac(\mathbb{C}[x]) = (0)$.


\item Όμοια με το $\mathbb{Z}_n$ και την ανάλυση σε διαιρέτες, έχουμε το διάγραμμα ιδεωδών


\begin{center}
\begin{tikzcd}
	&                                                                     & {\mathbb{Q}[x]/(x^2(x-1))} \arrow[ld, no head] \arrow[d, no head] \\
	& {x\mathbb{Q}[x]/(x^2(x-1))} \arrow[rd, no head] \arrow[ld, no head] & {(x-1)\mathbb{Q}[x]/(x^2(x-1))} \arrow[d, no head]                \\
{x^2\mathbb{Q}[x]/(x^2(x-1))} \arrow[rd, no head] &                                                                     & {x(x-1)\mathbb{Q}[x]/(x^2(x-1))} \arrow[ld, no head]              \\
	& (0)                                                                 &                                                                  
\end{tikzcd}
\end{center}
	\end{enumerate}
\end{proof}

Όπως φαίνεται στο διάγραμμα, τα μέγιστα και πρώτα ιδεώδη είναι τα $([x]),([x-1])$ (με αγκύλες εννούμε την κλάση \tl{modulo} $x^2(x-1)$).
Πράγματι έχουμε
$$R/([x]) = \frac{\Q[x]/(x^2(x-1))}{(x)/(x^2(x-1))} \simeq \Q[x]/(x) \simeq \Q$$
το οποίο είναι σώμα και όμοια το $([x-1])$.

$ $\newline
Τα υπόλοιπα ιδεώδη δεν είναι πρώτα, το $(0)$ δεν είναι καθώς ο $R$ δεν είναι περιοχή. Έχουμε:
$$R/([x^2]) = \frac{\Q[x]/(x^2(x-1))}{(x^2)/(x^2(x-1))} \simeq \Q[x]/(x^2) $$
$$R/([x(x-1)]) = \frac{\Q[x]/(x^2(x-1))}{(x(x-1))/(x^2(x-1))} \simeq \Q[x]/(x(x-1))$$
τα οποία δεν είναι περιοχές αφού $x\cdot x= 0$ και $x(x-1) = 0$ αντίστοιχα.

$ $\newline
Άρα
$$nil(R) = Jac(R) = ([x])\cap ([x-1]) = ([x^2 - x]) = \frac{(x^2 - x)\Q [x]}{x^2(x-1)}$$

\pagebreak

\noindent Άσκηση $2.2)$
\quad Υπολογίστε το $\sqrt I$ στις ακόλουθες περιπτώσεις.
\begin{enumerate}
    \item $R = k[x,y], I = ((x-1)^3,y^4)$, όπου $k$ σώμα.
    \item $R = k[x,y], I = (x-1,y^2 - 4y -xy + y + 4)$, όπου $k$ σώμα.
    \item $R = \Z [x], I = (5,x^2 + 2)$. 
\end{enumerate}
\begin{proof} $ $

	$ $\newline

	\begin{enumerate}
		\item 
		$$\sqrt I = \sqrt{\left((x-1)^3,y^4 \right)} = \sqrt{ \left((x-1)^3\right) + (y^4)} = \sqrt{ \sqrt{\left((x-1)^3\right)} + \sqrt{(y^4)}  } =$$
		$$ = \sqrt{\sqrt{\left(x-1\right)^3}  + \sqrt{(y)^4}   } = \sqrt{ (x-1) + (y)} = \sqrt{ (x-1,y) } = (x-1,y)$$
		όπου η τρίτη ισότητα είναι η ιδιότητα του ριζικού $\sqrt{I + J} = \sqrt{\sqrt I + \sqrt J}$. Στην τέταρτη ισότητα χρησιμοποιούμε ότι το κύριο ιδεώδες $((x-1)^3)$ είναι ίσο με το $I^3$ όπου $I=(x-1)$ και όμοια για το $(y)$. Έπειτα χρησιμοποιούμε την ιδιότητα που φεύγουν οι ρίζες και οι δυνάμεις καθώς τα $(x-1),(y)$ είναι πρώτα ιδεώδη εφόσον:

		$$ k[x,y]/(x-1) \simeq k[y] \quad \quad k[x,y]/(y) \simeq k[x]$$
		τα οποία είναι περιοχές. Στο τέλος εφαρμόζουμε την ίδια ιδιότητα καθώς το $(x-1,y)$ είναι μέγιστο και άρα πρώτο εφόσον
		$$k[x,y]/(x-1,y) \simeq k$$

		\item Έχουμε $y^2 -4y -xy + y +4 = (y-2)^2 + y(1-x)$. Άρα 
		$$I = (x-1,y^2 -4y -xy +y +4) = (x-1,(y-2)^2 +y(1-x)) = (x-1,(y-2)^2)$$
		Άρα με τις ίδιες ιδιότητες με πριν
		$$\sqrt{I} = \sqrt{\sqrt{x-1} + \sqrt{\left((y-2)^2\right)}} = \sqrt{ (x-1) + \sqrt{(y-2)^2}} = $$
		$$ = \sqrt{(x-1,y-2)} = (x-1,y-2)$$

		\item 
		$$\frac{\Z[x]}{(5,x^2+2)} \simeq \frac{\Z [x] /(5)}{(5,x^2+2) /(5)} \simeq \frac{\Z_5 [x]}{(x^2+2)} \simeq \mathbb{F}_{5^2}$$
		άρα το ιδεώδες είναι μέγιστο και συνεπώς $\sqrt{I} = I = (5,x^2+2)$.

		$ $\newline
		Ο πρώτος ισομορφισμός προκύπτει από το 3ο θεώρημα ισομορφισμών. Για τον δεύτερο, από το 1ο θεώρημα ισομορφισμών για τον ομομορφισμό:
		$$\pi : \Z [x] \longrightarrow \Z_5[x]$$
		$$\sum\limits_{i=0}^n a_i x^i \longmapsto \sum\limits_{i=0}^n [a_i]_5 x^i$$
		με πυρήνα το ιδεώδες $(5)$ παίρνουμε ότι
		$$\Z[x]/(5) \simeq \Z_5[x]$$
		και ο περιορισμός του $\pi$ στο ιδεώδες $(5,x^2+2)$ δίνει $(x^2+2) \simeq (5,x^2+2)/(5)$ και επειδή είναι ο περιορισμός του ισομορφισμού του \tl{``}αριθμητή\tl{``} έχουμε τον δεύτερο ισομορφισμό. Ο τελευταίος προκύπτει επειδή το $x^2+2$ είναι βαθμού $2$ χωρίς ρίζες στο $Z_5$ και άρα το πηλίκο είναι σώμα.
	\end{enumerate}


\end{proof}



\pagebreak

\noindent Άσκηση $2.3)$
\quad Έστω $I,J$ ιδεώδη του δακτυλίου $R$. Δείξτε τις εξής σχέσεις.
\begin{enumerate}
    \item $\sqrt{\sqrt I} = \sqrt I$.
    \item $\sqrt{I} = R \iff I = R$.
    \item $\sqrt{\sqrt I + \sqrt J} = \sqrt{I+J}$.
\end{enumerate}

\begin{proof} $ $
	$ $\newline
	\begin{enumerate}
        \item Έχουμε από ορισμό ριζικού $ \sqrt I \subseteq \sqrt{\sqrt I}$. Για την άλλη κατεύθυνση, έστω $x \in \sqrt{\sqrt I}$. Τότε υπάρχει $n \in \Z_{>0}$ τέτοιο ώστε $x^n \in \sqrt I$. Δηλαδή, υπάρχει $k \in \mathbb{Z}_{>0}$ τέτοιο ώστε $(x^n)^k \in I$. Άρα $x^{nk} \in I$, δηλαδή $x \in \sqrt I$.
        \item Αν $I = R$ επειδή $I \subseteq \sqrt I$ παίρνουμε $\sqrt I = R$. Αν $\sqrt I = R$, αυτό σημαίνει ότι $1 \in \sqrt I$ δηλαδή $1^n \in I$ για κάποιο θετικό $n$. Δηλαδή, $1 \in I \implies I = R$.
        \item Έστω $x \in \sqrt{I+J}$. Τότε υπάρχει $n \in \Z_{>0}$ τέτοιο ώστε $x^n \in I + J \subseteq \sqrt I + \sqrt J$. Δηλαδή, $x^n \in \sqrt I + \sqrt J \implies x \in \sqrt{\sqrt I + \sqrt J}$.
        
        $ $\newline
        Αντίστροφα, έστω $x \in \sqrt{\sqrt I + \sqrt J}$. Υπάρχει $n \in \Z_{>0}$ τέτοιο ώστε $x^n = a + b \in \sqrt I + \sqrt J$ και επιπλέον υπάρχουν $k,\lambda \in \Z_{>0}$ τέτοια ώστε $a^k \in I, b^{\lambda} \in J$. Έχουμε:
        $$(x^n)^{k+\lambda} = (a+b)^{k+\lambda} = \sum\limits_{i=0}^{k+\lambda } \binom{k+\lambda}{i} a^i b^{k+\lambda - i} = $$
        $$ = a^k \left(\sum\limits_{i = k}^{k + \lambda} \binom{k+\lambda}{i}a^{i-k}b^{k+\lambda - i}\right) + b^{\lambda} \left(\sum\limits_{i = 0}^{k -1 } \binom{k+\lambda}{i}a^{i}b^{k - i}\right) \in I + J$$
        δηλαδή $x^{n(k+\lambda)} \in I + J \implies x \in \sqrt{I + J}$.
    \end{enumerate}
\end{proof}
\pagebreak


\noindent Άσκηση $2.7)$
\quad Έστω $R,S$ δακτύλιοι. Δείξτε ότι κάθε πρώτο ιδεώδες του $R\times S$ είναι της μορφής $\mathfrak{p} \times S$ ή $R \times \mathfrak{q}$, όπου $\mathfrak{p}$ (αντίστοιχα $\mathfrak{q}$) είναι πρώτο ιδεώδες του $R$ (αντίστοιχα $S$).

\noindent Αληθεύει ότι $nil(R \times S) = nil(R) \times nil(S)$?

\noindent Αληθεύει ότι $Jac(R\times S) = Jac(R) \times Jac(S)$?

\begin{proof} $ $
	
	$ $\newline
	Θα δείξουμε κάποια αποτελέσματα πριν απαντήσουμε. Αρχικά, κάθε ιδεώδες $I$ ενός δακτυλίου $R = R_1 \times R_2$ είναι της μορφής $I = I_1 \times I_2$ με $I_1$ ιδεώδες του $R_1$ και $I_2$ ιδεώδες του $R_2$.

	Πράγματι, έστω $I$ ιδεώδες του $R = R_1 \times R_2$. Θέτουμε:
	$$I_1 = \{x_1 : \quad (x_1,0) \in I \} \subseteq R_1 $$
	$$I_2 = \{x_2 : \quad (0,x_2) \in I \} \subseteq R_2 $$

	τα οποία είναι ιδεώδη των $R_1, R_2$ αντίστοιχα. 
	
	$ $\newline
	Πράγματι, αν $x,y \in I_1$ τότε $(x,0),(y,0) \in I$ και το $I$ είναι ιδεώδες. Άρα $(x,0)-(y,0) = (x-y,0) \in I$, δηλαδή $x-y \in I_1$. Επιπλέον αν $r \in R_1$, τότε $(r,r^{\prime}) (x,0) = (rx,0) \in I$. Δηλαδή $rx \in I_1$ και έτσι το $I_1$ είναι ιδεώδες του $R_1$ και όμοια το $I_2$ είναι ιδεώδες του $R_2$.

	$ $\newline
	Τώρα ισχυριζόμαστε ότι $I = I_1 \times I_2$. Έχουμε ότι $I_1 \times 0, 0 \times I_2 \subseteq I$ και άρα 
	$$I \supseteq (I_1 \times 0) + (0 \times I_2) = I_1 \times I_2$$

	Αντίστροφα, έστω $(x,y) \in I$. Τότε $(x,0) = (1,0)(x,y) \in I$ αφού το $I$ είναι ιδεώδες, δηλαδή $x \in I_1$. Όμοια $y \in I_2$ και άρα $(x,y) \in I_1 \times I_2$.

	$ $\newline Έχουμε επιπλέον τον ισομορφισμό:
	$$R/I \simeq R_1 / I_1 \times R_2 / I_2$$
	$$(r_1,r_2) + I \longmapsto (r_1 + I_1, r_2 + I_2)$$
	Είναι πράγματι ισομορφισμός, το ομομορφισμός και το επί είναι προφανή. Για τον πυρήνα της απεικόνισης έχουμε:
	$$(r_1 + I_1, r_2 + I_2) = (0,0) \iff r_1 + I_1 = 0, r_2 + I_2 = 0 \iff r_1 \in I_1, r_2 \in I_2$$
	$$ \iff (r_1,0),(0,r_2) \in I \implies (r_1,r_2) \in I = 0_{R/I}$$
	άρα ο πυρήνας είναι τετριμμένος.

	$ $\newline
	Ένα άλλο αποτέλεσμα είναι ότι ο δακτύλιος $R = R_1 \times R_2$ είναι περιοχή αν και μόνο αν ένα από τα $R_1$ ή $R_2$ είναι περιοχή και το άλλο είναι $0$. Πράγματι, αν $R$ περιοχή τότε:
	$$(1,0)(0,1) = (0,0) \implies (1,0) = (0,0) \text{ ή } (0,1) = (0,0)$$
	άρα για κάποιον δακτύλιο θα ισχύει $1=0$, δηλαδή θα είναι ο τετριμμένος. Έτσι δίχως βλάβη γενικότητας, αν $R_2 = 0$ τότε:
	$$xy = 0 \implies (xy,0) = (0,0) \iff (x,0)(y,0) = (0,0)$$
	$$ \implies (x,0) = (0,0) \text{ ή } (y,0) = (0,0) \iff x = 0 \text{ ή } y =0$$
	
	Αντίστροφα, αν $R_1$ περιοχή και $R_2 = 0$ τότε:
	$$(x,0)(y,0) = (0,0) \iff (xy,0) = (0,0) \iff xy = 0$$
	$$ \implies x = 0 \text{ ή } y = 0 \implies (x,0) = (0,0) \text{ ή } (y,0) = (0,0) $$


	$ $\newline
	Έστω τώρα ο δακτύλιος $R \times S$ και $P=(P_1,P_2)$ ένα πρώτο ιδεώδες του. Τότε ο δακτύλιος πηλίκο $(R\times S)/P$ είναι περιοχή. Δηλαδή μέσω του ισομορφισμού έχουμε την περιοχή $R/P_1 \times S/P_2$.
	
	$ $\newline
	Με βάση τα προηγούμενα αποτελέσματα, στην πρώτη περίπτωση αν $S/P_2 = 0$ έχουμε $S=P_2$ και $R/P_1$ περιοχή, δηλαδή $P_1$ πρώτο ιδεώδες του $R$. Άρα $P = (\mathfrak{p}, S)$ στην πρώτη περίπτωση και $P = (R,\mathfrak{q})$ όμοια στην άλλη περίπτωση όπου $R/P_1 = 0$.


	$$nil(R \times S) = \bigcap\limits_{P \unlhd R\times S \text{ πρώτο }} P = \left(\bigcap\limits_{\mathfrak{p} \unlhd R \text{ πρώτο }} \mathfrak{p} \times S \right) \bigcap \left(\bigcap\limits_{\mathfrak{q} \unlhd S \text{ πρώτο }} R \times \mathfrak{q} \right) = $$
	$$ = \left( \left( \bigcap\limits_{\mathfrak{p} \unlhd R \text{ πρώτο } } \mathfrak{p} \right) \times S \right) \bigcap \left(R \times \left( \bigcap\limits_{\mathfrak{q} \unlhd S \text{ πρώτο }} \mathfrak{q} \right)\right) = $$
	$$ = \left(nil(R) \times S\right) \bigcap \left( R \times nil(S) \right) = nil(R) \times nil(S)$$

	$ $\newline
	Για το $Jac(R\times S)$, αν $\mathfrak{m}$ μεγιστικό ιδεώδες τότε θα είναι και πρώτο. Άρα $\mathfrak{m} = \mathfrak{p} \times S$ ή $R \times \mathfrak{q}$ με $\mathfrak{p}$ ή αντίστοιχα $\mathfrak{q}$ πρώτο. Ωστόσο, θα έχουμε ότι $(R\times S)/m$ σώμα ισόμορφο με $R/\mathfrak{p} \times 0$ αν βρισκόμαστε στην πρώτη περίπτωση. Δηλαδή, το $R/\mathfrak{p}$ είναι σώμα και άρα $\mathfrak{p}$ μεγιστικό. Συνεπώς έχουμε τις ίδιες πράξεις με παραπάνω:

	$$Jac(R \times S) = \bigcap\limits_{P \unlhd R\times S \text{ μέγιστο }} P = \left(\bigcap\limits_{\mathfrak{p} \unlhd R \text{ μέγιστο }} \mathfrak{p} \times S \right) \bigcap \left(\bigcap\limits_{\mathfrak{q} \unlhd S \text{ μέγιστο }} R \times \mathfrak{q} \right) = $$
	$$ = \left( \left( \bigcap\limits_{\mathfrak{p} \unlhd R \text{ μέγιστο } } \mathfrak{p} \right) \times S \right) \bigcap \left(R \times \left( \bigcap\limits_{\mathfrak{q} \unlhd S \text{ μέγιστο }} \mathfrak{q} \right)\right) = $$
	$$ = \left(Jac(R) \times S\right) \bigcap \left( R \times Jac(S) \right) = Jac(R) \times Jac(S)$$

	$ $\newline
	Αν επιχειρηματολογήσουμε χωρίς να βασιστούμε στα προηγούμενα, έστω $(x,y) \in nil(R\times S)$. Τότε για κάποιο $n$ ισχύει $(0,0) = (x,y)^n = (x^n,y^n)$ και άρα $x \in nil(R), y \in nil(S) \implies (x,y) \in nil(R\times S)$.
	Αντίστροφα, αν $x \in nil(R), y \in nil(S)$ τότε υπάρχουν $n,m \in \Z_{>0}$ έτσι ώστε $x^n = 0_R, y^m = 0_S$. Αν $k = \max\{n,m\}$ τότε
	$$ (x,y)^k = (x^k,y^k) = (0,0) $$
	άρα $(x,y) \in nil(R\times S)$.

	$ $\newline
	Όμοια, αν $(x,y) \in Jac(R\times S)$ τότε για κάθε $(r,s) \in R\times S$ έχουμε
	$$(1,1) - (x,y)(r,s) \in U(R\times S) = U(R) \times U(S)$$
	$$\iff (1-xr,1-ys) \in U(R)\times U(S)$$
	$$\iff x \in Jac(R), \quad y \in Jac(S)$$
\end{proof}
\pagebreak


\noindent Άσκηση $2.10)$
\quad Τα μόνα ταυτοδύναμα στοιχεία τοπικού δακτυλίου είναι τα $0,1$.

\begin{proof} $ $
	
	$ $\newline
	Έστω $e \neq 0,1$. Τότε $e^2 = e \implies 0 = e(1-e)$. Δηλαδή, τα $e,(1-e) \neq 0$ είναι μηδενοδιαιρέτες και άρα όχι αντριστρέψιμα. Έτσι το ιδεώδες $(e)$ περιέχεται σε κάποιο μέγιστο ιδεώδες $\mathfrak{m}$. Αυτό είναι μοναδικό καθώς βρισκόμαστε σε τοπικό δακτύλιο και άρα το $e$ ανήκει στην τομή όλων των μεγίστων ιδεωδών, δηλαδή $e \in Jac(R)$. Με την πρόταση $2.4.3$ αυτό είναι ισοδύναμο με:
	$$e \in Jac(R) \iff (1-er) \in U(R) \quad \forall r \in R$$
	και για $r=1$ παίρνουμε ότι το $1-e$ είναι αντριστρέψιμο, το οποίο είναι άτοπο.

\end{proof}
\pagebreak

\noindent Άσκηση $2.12)$
\quad Έστω $V \subseteq k^n$ μη κενό αλγεβρικό σύνολο.
\begin{enumerate}
	\item Δείξτε ότι το $I(V)$ είναι ιδεώδες του $k[x_1 ,\ldots , x_n ]$ και
	$$ k[x_1, \ldots , x_n ] / I(V) \simeq k[V]$$
	\item Δείξτε ότι $nil(k[V])=0$.
	\item Για κάθε $P \in V$, έστω $\mathfrak{m}_P \in k[V]$ τό σύνολο των πολυωνυμικών συναρτήσεων του $k[V]$ που μηδενίζονται στο $P$. Δείξτε ότι το $\mathfrak{m}_P$ είναι μέγιστο ιδεώδες του $k[V]$.
	\item Σύμφωνα με το $(3)$ έχουμε μια απεικόνιση
	$$P \mapsto \mathfrak{m}_P$$
	από το $V$ στα ιδεώδη του $k[V]$. Δείξτε ότι αυτή η απεικόνιση είναι $1-1$.
\end{enumerate}

\begin{proof} $ $
	
	$ $\newline
	Όπου $x$ θα συμβολίζουμε το $(x_1, x_2, \ldots, x_n)$ στα παρακάτω
	\begin{enumerate}
		\item Έστω $f(x),g(x) \in I(V)$ και $h(x) \in k[x_1,\ldots ,x_n]$. Τότε για τυχόν $P \in V$ έχουμε:
		$$0 = f(P) - g(P) = (f-g)(P)$$
		άρα το πολυώνυμο $f(x)-g(x) = (f-g)(x)$ ανήκει στο $I(V)$. Επιπλέον
		$$(hf)(P) = h(P)f(P) = h(P)\cdot 0 = 0$$
		άρα και το πολυώνυμο $h(x)f(x) = (hf)(x)$ ανήκει στο $I(V)$, δηλαδή το $I(V)$ είναι ιδεώδες του $k[V]$.

		$ $\newline
		Ορίζουμε την απεικόνιση
		$$\phi : k[x_1, \ldots, x_n] \longrightarrow k[V]$$
		$$f \longmapsto f_V:V\rightarrow k$$
		$$\quad \quad \quad P\mapsto f(P)$$

		Η απεικόνιση $\phi$ είναι επί εφόσον περιορίζουμε κάθε πολυώνυμο στο υποσύνολο $V$ του $k^n$. Είναι επιπλέον ομομορφισμός καθώς:
		$$\phi(f+g)(P) = (f+g)_V (P) = (f+g)(P) = f(P) + g(P) = f_V (P) + g_V (P) =$$
		$$ = (f_V + g_V) (P) = [\phi(f) + \phi(g)](P)$$
		$$\phi(fg) (P)= (fg)_V (P) = (fg)(P) = f(P)g(P) = f_v(P) g_v(P) = [\phi(f)\phi(g)](P)$$
		
		$ $\newline
		Αν τώρα $f(x) \in I(V)$ τότε $f_V (P) = f(P) = 0$ για κάθε $P \in V$ δηλαδή η $f_V$ είναι η μηδενική απεικόνιση από το $V$ στο $k$ και άρα $f(x) \in ker\phi$.
		
		$ $\newline
		Αντίστροφα, αν
		$$\phi(f) = 0_{k[V]} : V \rightarrow k$$
		$$\quad \quad \quad \quad \quad  \quad \quad \quad \quad  \quad \quad \quad \quad \quad P \mapsto 0 = f(P) \quad \forall P \in V$$
		τότε $f(x) \in I(V)$. Άρα $ker\phi = I(V)$ και από το 1ο θεώρημα ισομορφισμών δακτυλίων παίρνουμε:
		$$ k[x_1, \ldots , x_n ] / I(V) \simeq k[V]$$

		\item Έστω $f_V \in nil(k[V])$. Τότε υπάρχει $n \in \Z_{>0}$ τέτοιο ώστε $(f_V)^n = 0_{k[V]} : V \rightarrow k$. Το $0_{k[V]}$ είναι ως συνάρτηση το μηδενικό πολυώνυμο περιορισμένο στο $V$. 
		
		$ $\newline
		Αν υποθέσουμε ότι το $f_V$ δεν είναι περιορισμός του μηδενικού πολυωνύμου τότε θα έχει μεγιστοβάθμιο συντελεστή (ως προς όλες τις μεταβλητές $x_1,\ldots x_n$) $a_m \neq 0  \in k$.
		
		$ $\newline
		Τότε στο $(f_V)^n$ θα εμφανίζεται ο συντελεστής $(a_m)^n$ ο οποίος από την υπόθεση θα είναι $0$. Δηλαδή, το στοιχείο $a_m$ του σώματος $k$ θα είναι $0$ το οποίο είναι άτοπο.
		

		\item Έστω $P \in V$, το $\mathfrak{m}_P = \{f_V \in k[V]: \quad f_V (P) = 0 \}$ είναι ιδεώδες εφόσον:
		$$(f_V - g_V)(P) = f_V(P) - g_V (P) = 0 - 0 = 0$$
		$$(g_V f_V)(P) = g_V (P) f_V (P) = g_V (P) \cdot 0 = 0$$

		$ $\newline
		Ορίζουμε τον ομομορφισμό εκτίμησης $\phi_P$ στο $P$
		$$k[V]\longrightarrow k$$
		$$f_V \longmapsto f_V(P) = f(P)$$
		ο οποίος είναι επί εφόσον μπορούμε να θεωρήσουμε τα σταθερά πολυώνυμα $f(x) = a$ για κάθε όρο $a \in k$ και να τα περιορίσουμε στο $V$.

		$ $\newline
		Αν $f_V \in \mathfrak{m}_P$ τότε $\phi_P (f_V) = f_V (P) = 0$, άρα $\mathfrak{m} \subseteq ker\phi_P$.

		$ $\newline
		Αντίστροφα, αν $f_v \in ker\phi_P$ τότε $\phi_P (f_V) = 0$, όμως $\phi_P (f_V) = f_V(P)$. Άρα $f_V(P) = 0$, δηλαδή $f_V \in \mathfrak{m}_P$.

		$ $\newline
		Επομένως, από το 1ο θεώρημα ισομορφισμών δακτυλίων παίρνουμε
		$$\frac{k[V]}{\mathfrak{m}_P} \simeq k$$
		και άρα εφόσον το $k$ είναι σώμα, το $\mathfrak{m}_P$ είναι μέγιστο ιδεώδες.

		\item Έστω $P,Q \in V$ και ότι
		$$\mathfrak{m}_P = \mathfrak{m}_Q$$
		$$\{f_V \in k[V]: \quad f_V(P) = 0\} = \{f_V \in k[V]: \quad f_V(Q) = 0\}$$

		Αν $P = (P_1 ,\ldots, P_n), Q = (Q_1 ,\ldots, Q_n)$ τότε για $i=1,\ldots,n$ θεωρούμε τα πολυώνυμα:
		$$f_i (x) = x_i - P_i$$
		και ισχύει ότι $(f_i)_V \in \mathfrak{m}_P$ αφού $(f_i)_V (P) = f_i(P) = P_i - P_i = 0$. Από την ισότητα των συνόλων έχουμε ότι τα $(f_i)_V$ ανήκουν στο $\mathfrak{m}_Q$ δηλαδή μηδενίζονται στο $Q$. Αυτό σημαίνει ότι $Q_i - P_i = 0$ για κάθε $i=1,\ldots,n$. Άρα $P=Q$.
	\end{enumerate} 

\end{proof}



\pagebreak


\noindent Άσκηση $2.14)$
\quad Έστω $\phi : R \rightarrow S$ ένας ομομορφισμός δακτυλίων.
\begin{enumerate}
	\item Δείξτε ότι αν $\mathfrak{q}$ είναι πρώτο ιδεώδες του $S$, τότε το σύνολο $\phi^{-1} (\mathfrak{q})$ είναι πρώτο ιδεώδες του $R$.
	\item Αληθεύει ότι αν $\mathfrak{m}$ είναι μέγιστο ιδεώδες του $S$, τότε το σύνολο $\phi^{-1}(\mathfrak{m})$ είναι μέγιστο ιδεώδες του $R$?
\end{enumerate}

\begin{proof} $ $

	$ $\newline
	\begin{enumerate}
		\item Έστω $a,b \in R$ με $ab \in \phi^{-1} (\mathfrak{q})$. Δηλαδή, υπάρχει $y \in \mathfrak{q}$ τέτοιο ώστε $y = \phi (ab) = \phi(a) \phi(b)$. Άρα $\phi(a) \phi(b) \in \mathfrak{q}$ το οποίο είναι πρώτο. Έπεται ότι
		$$\phi(a) \in \mathfrak{q} \quad \text{ ή }\quad \phi(b) \in \mathfrak{q} \implies a \in \phi^{-1} (\mathfrak{q}) \quad\text{ ή }\quad b \in \phi^{-1} (\mathfrak{q})$$
		άρα το $\phi^{-1} (\mathfrak{q})$ είναι πρώτο ιδεώδες του $R$.

		\item Δεν αληθεύει. Αν θεωρήσουμε την εμφύτευση $i: \mathbb{Z} \xhookrightarrow{} \mathbb{Q}$ και το μοναδικό μέγιστο ιδεώδες $\{0\}$ του $\Q$, είναι μέγιστο αφού $\Q / \{0\} \simeq \Q$ σώμα, έχουμε:
		$$i^{-1} ( \{0\} ) = \{0\}$$
		το οποίο φυσικά είναι πρώτο ιδεώδες του $\mathbb{Z}$ αλλά όχι μέγιστο, διαφορετικά θα είχαμε $\mathbb{Z}/\{0\} \simeq \mathbb{Z}$ σώμα που δεν ισχύει.
	\end{enumerate}
\end{proof}
\pagebreak

\noindent Άσκηση $2.16)$
\quad Δείξτε ότι ο δακτύλιος
$$\Z [\sqrt{-3}] = \{a + b\sqrt{-3} \in \mathbb{C}: a,b\in \Z \}$$
δεν είναι περιοχή μοναδικής παραγοντοποίησης.

\begin{proof} $ $
	
	$ $\newline
	Έχουμε $2\cdot 2 = (1-\sqrt{-3})(1+ \sqrt{-3}) = 4$. Θα δείξουμε ότι κανένα από τα $2,1\pm \sqrt{-3}$ δεν είναι αντιστρέψιμο και ότι όλα είναι ανάγωγα.
	
	$ $\newline
	Ορίζουμε το μέτρο του μιγαδικού 
	$$N : \Z [\sqrt{-3}] \longrightarrow \mathbb{N}$$
	$$a+b\sqrt{-3} \longmapsto |a+b\sqrt{-3}|^2 = a^2 + 3b^2$$
	και έχουμε $N(xy) = N(x)N(y)$. 

	$ $\newline
	Αν $u \in U(\Z[ \sqrt{-3}])$ τότε $uv=1$ και άρα $1 = N(u)N(v) \implies N(u)=1$. Δηλαδή $a^2 + 3b^2 = 1$ με $a,b \in \Z$ το οποίο συμβαίνει μόνο όταν $a=\pm 1, b=0$. Άρα $U(\Z[ \sqrt{-3}]) = \{\pm 1\}$.

	$ $\newline
	Δείχνουμε ότι το $2$ είναι ανάγωγο. Αν $2 = xy$ με $x,y \in \Z [\sqrt{-3}]$ τότε
	$$4 = N(2) = N(xy) = N(x) N(y)$$
	
	αν $N(x) = a^2 + 3b^2$ έχουμε:
	$$N(x) = 
	\begin{cases}
		1, \quad x \in U(\Z [\sqrt{-3}]) \\
		2, \quad a^2 + 3b^2 = 2, \text{ δεν μπορεί να ισχύει } \\
		4, \quad N(y) = 1 \implies y \in U(\Z [\sqrt{-3}])
	\end{cases}
	$$
	άρα το $2$ είναι ανάγωγο σε κάθε δυνατή περίπτωση του $N(x)$. Όμοια είναι και τα υπόλοιπα ανάγωγα αφού $4 = N(1\pm \sqrt{-3})$.
	
	$ $\newline
	Άρα το $\Z [\sqrt{-3}]$ δεν είναι περιοχή μοναδικής παραγοντοποίησης καθώς τα $2,1\pm \sqrt{-3}$ είναι όλα ανάγωγα και έτσι η σχέση $2\cdot 2 = (1-\sqrt{-3})(1+ \sqrt{-3})$ θα έπρεπε να μας δίνει ότι $2 = u (1 \pm \sqrt{-3})$ το οποίο δεν ισχύει για κανένα αντιστρέψιμο $u = \pm 1$.
\end{proof}
\end{document}