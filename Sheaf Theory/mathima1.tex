
$ $\newline
\tl{Sheaf Theory}:

\begin{defn} Μια κατηγορία είναι μια τριάδα $(\mathcal{C}, \mathcal{M}, \circ) \equiv \mathcal{C}$, όπου
    \begin{enumerate}
        \item $\mathcal{C}$ κλάση από \underline{αντικείμενα}.
        \item Για κάθε $A, B \in \mathcal{C}$ υπάρχει μοναδικό σύνολο $Mor_{\mathcal{C}}(A,B)$ από \underline{μορφισμούς} από το $A$ στο $B$ και
        $$\mathcal{M} = \bigcup\limits_{A,B \in \mathcal{C}} Mor_{\mathcal{C}} (A,B)$$

        \item Για κάθε $A,B,C$ αντικείμενα υπάρχει απεικόνιση:
        $$\circ : Mor(A,B) \times Mor(B,C) \longrightarrow Mor(A,C)$$
        $$(f,g) \longmapsto \circ(f,g) \equiv g\circ f$$ όπου λέγεται \underline{σύνθεση}, που ικανοποιούν τα αξιώματα:
    \end{enumerate}

    \begin{enumerate}
        \item $(A_1,B_1) \neq (A_2, B_2) \implies Mor_{\mathcal{C}} (A_1, B_1) \cap Mor_{\mathcal{C}}(A_2,B_2) = \varnothing$.
        \item Για κάθε $A,B \in \mathcal{C}$ και $f \in Mor_{\mathcal{C}}(A,B)$ και για κάθε $g \in Mor_{\mathcal{C}}(B,C)$ και για κάθε $h \in Mor_{\mathcal{C}} (C,D)$ μπορούμε:
        $$(h\circ g)\circ f = h\circ(g\circ f)$$ δηλαδή προσεταιριστική

        \item Για κάθε $A \in \mathcal{C}$ υπάρχει $1_A \in Mor_{\mathcal{C}}(A,A):$
        $$1_A \circ f = f, \quad \forall f: B \rightarrow A, \quad \forall B \in \mathcal{C}$$
        $$g\circ 1_A = g, \quad \forall g : A\rightarrow C, \quad \forall C \in \mathcal{C}$$

    \end{enumerate}

\end{defn}

$ $\newline
\underline{Παραδείγματα:}
\begin{enumerate}
    \item $\mathcal{S} =$ κατηγορία συνόλων με απεικονίσεις και συνήθη σύνθεση.
    \item $ \mathcal{T} = $ κατηγορία τοπολογικών χώρων με συνεχείς απεικονίσεις (και συνήθη σύνθεση, θα εννοείται στο εξής εκτός αν πούμε διαφορετικά).
    \item $\mathcal{G} = $ ομάδες με μορφισμούς ομάδων.
    \item $\mathcal{A}b = $ αβελιανές ομάδες με μορφισμούς ομάδων.
    \item $\mathcal{S}_0 = $ κατηγορία σημειωμένων συνόλων, δηλαδή τα αντικείμενα είναι ζεύγη $(X,x)$ με $X$ σύνολο και $x \in X$ και μορφισμοί να είναι απεικονίσεις:
    $$f:(X,x) \longrightarrow (Y,y)$$ με $f(x) = y$.
    \item $\mathcal{T}_0 = $ σημειωμένοι τοπολογικοί χώροι.
    \item $\mathcal{V_F} = $ διανυσματικοί χώροι πάνω από ένα σώμα $F$ με γραμμικές απεικονίσεις.
    \item $\mathcal{R} = $ δακτύλιοι με ομομορφισμούς δακτυλίων.
    \item $\mathcal{R}_1 = $ μοναδιαίοι δακτύλιοι με μορφισμούς τους ομομορφισμούς δακτυλίων που διατηρούν την μονάδα.
    \item $\mathcal{M}^L_R = $ αριστερά $R$-πρότυπα με $R$-γραμμικές απεικονίσεις.
    \item $\mathcal{T}g = $ τοπολογικές ομάδες με συνεχείς μορφισμούς ομάδων.
    \item $\mathcal{E}q = $ αντικείμενα: $(X,R)$ με $X$ σύνολο, $R$ σχέση ισοδυναμίας στο $X$ και μορφισμοί απεικονίσεις:
    $$ (X,R) \longrightarrow (Y,S)$$ είναι 
    απεικόνιση: $$f: X \rightarrow Y$$ με $$x_1 R x_2 \implies f(x_1) S f(x_2)$$ 
    \item $\mathcal{O}rd = $ αντικείμενα είναι $(X,\leq)$ όπου $X$ σύνολο και $\leq$ σχέση διάταξης στο $X$ και μορφισμοί:
    $$(X,\leq_1) \longrightarrow (Y,\leq_2)$$ με απεικόνιση:
    $$f: X\rightarrow Y$$ όπου ισχύει $a \leq_1 b \implies f(a)\leq_2 f(b)$.


\end{enumerate}

\vspace{0.3truecm}
    
\begin{defn} Μια κατηγορία $(\mathcal{C},\mathcal{M},\circ)$ λέγεται \underline{μικρή} αν $\mathcal{C}$ είναι σύνολο.
\end{defn}

\vspace{0.3truecm}

$ $\newline
\underline{Παραδείγματα:}
\begin{enumerate}
    \setcounter{enumi}{13}
    \item $(G,*)$ ομάδα ($\mathcal{C} = \{G\}$, $\mathcal{M} = G, \circ = *$)
    \item $(X,\leq)$ διατεταγμένο σύνολο με:
    $$\left( \mathcal{C} = X, \quad M = \bigcup\limits_{x,y \in X} Mor(x,y), \quad \circ \right) $$ όπου:
    $$Mor(x,y) = \begin{cases} \{(x,y)\}, \quad \text{ αν } x \leq y \\
        \varnothing, \quad \text{ διαφορετικά }
    \end{cases}$$ και η σύνθεση του $\{(x,y)\}$ με το $\{(y,z)\}$ δίνει το $\{(x,z)\}$, ενώ η σύνθεση με $\varnothing$ δίνει $\varnothing$.

    %σύνθεση a,b με b,c δίνει το ζεύγος, σύνθεση με κενό = κενό
    \item Ομοίως για $(X,R)$ σύνολο με σχέση ισοδυναμίας.
\end{enumerate}


\begin{defn} $\mathcal{C} \equiv (\mathcal{C},\mathcal{M},\circ)$ και $\mathcal{C}_0 \equiv (\mathcal{C}_0, \mathcal{M}_0, *)$ κατηγορίες, θα λέμε $\mathcal{C}_0$ είναι \underline{υποκατηγορία} της $\mathcal{C} \iff \mathcal{C}_0 \subseteq \mathcal{C}$ και για κάθε $A,B \in \mathcal{C}_0$ να ισχύει:
    $$Mor_{\mathcal{C}_0} (A,B) \subseteq Mor_{\mathcal{C}}(A,B)$$ και $*$ είναι ο περιορισμός της $\circ$.
    
    Αν για κάθε $A,B \in \mathcal{C}_0$ ισχύει ότι $Mor_{\mathcal{C}_0}(A,B) = Mor_{\mathcal{C}}(A,B)$ τότε λέγεται \underline{πλήρης} \underline{υποκατηγορία} της $\mathcal{C}$.
\end{defn}

π.χ.

\begin{enumerate}
    \item $\mathcal{A}b$ πλήρης υποκατηγορία της $\mathcal{G}$.
    \item $\mathcal{R}_1$ υποκατηγορία της $\mathcal{R}$, όχι πλήρης.
\end{enumerate}


\begin{defn} $\mathcal{C},\mathcal{D}$ κατηγορίες. Ένας (συναλλοίωτος) \underline{συναρτητής} $F : \mathcal{C} \rightarrow \mathcal{D}$ (όχι απεικόνιση) είναι ένα ζεύγος $(F_1, F_2):$
    $$F_1 : \mathcal{C} \longrightarrow \mathcal{D}$$
    $$F_2: \mathcal{M}_{\mathcal{C}} \longrightarrow \mathcal{M}_{\mathcal{D}}$$ με
    \begin{enumerate}
        \item Για κάθε $A,B \in \mathcal{C}$ και $f \in Mor_{\mathcal{C}}(A,B)$:
        $$F_2(f) \in Mor_{\mathcal{D}}(F_1(A),F_1(B))$$
        %figure
        \item Για κάθε $A \in \mathcal{C}$:
        $$F_2(1_A) = 1_{F_1(A)}$$
        \item Για κάθε $A,B,C \in \mathcal{C}$ και $f :A \rightarrow B, \quad g: B \rightarrow C$ ισχύει ότι:
        $$F_2(g\circ f) = F_2(g) \circ F_2(f) $$
    \end{enumerate}
\end{defn}

π.χ.
\begin{enumerate}
    \item Ο ελεύθερος συναρτητής $\mathcal{S} \longrightarrow \mathcal{V}_{F}$ με
    $$S \longrightarrow \langle S \rangle = \text{ διανυσματικός χώρος }$$ των τυπικών γραμμικών συνδυασμών με βάση $S$ και πάει μια $f: S\rightarrow T$ σε γραμμική επέκτασή της.

    \item Επιλήσμων συναρτητής (\tl{forgetful functor}): 
    $$F: \mathcal{T} \rightarrow \mathcal{S}$$
    $$F: \mathcal{G} \rightarrow \mathcal{S}$$
    $$F: \mathcal{T}g \rightarrow \mathcal{G}$$
    ξεχνάει μέρος της δομής.

    \item $\mathcal{C}$ έχει αντικείμενα τα $U \subseteq \mathbb{R}^n$ ανοιχτά, $n \in \mathbb{N}$, σημειωμένα $(U,x)$ με $x \in U$ και μορφισμοί
    $$f: (U,x) \rightarrow (V,y)$$ διαφορίσιμη με $f(x) = y$.

    Θεωρούμε την $\mathcal{D} = \mathcal{V}_{\mathbb{R}}$, τότε ορίζουμε:
    $$F_1 (U,x) = \mathbb{R}^m, \quad U \subseteq \mathbb{R}^m $$
    $$f: (U,x) \longrightarrow (Y,y)$$
    $$F_2 (f) = Df(x): \mathbb{R}^m \rightarrow \mathbb{R}^m$$
\end{enumerate}