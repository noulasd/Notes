\vspace{0.3truecm}
\begin{remark} Έστω $U \subseteq X$ (όχι απαραίτητα ανοιχτό) και $s: U \rightarrow \So$ τυχαία (όχι κατ ανάγκη συνεχής) με $\pi \circ s = id_U$. Έπεται ότι η $s:U \rightarrow s(U)$ είναι 1-1 και επί. Δηλαδή, υπάρχει $s^{-1}: s(U) \rightarrow U$ και από την σχέση $\pi \circ s = id_U$ παίρνουμε $\pi|_{s(U)} = s^{-1}$ από την μοναδικότητα αριστερής και δεξιάς αντίστροφης. Ισοδύναμα $s = \left( \pi|_{s(U)}\right)$. Εδώ έχουμε χρησιμοποιήσει μόνο συνολοθεωρητικά επιχειρήματα και δεν έχουμε απαιτήσει ανοιχτά σύνολα και συνέχεια.
\end{remark}

\begin{figure}[H]
    \centering
\begin{tikzcd}
    \So \arrow[d, "\pi"] & \supseteq & s(U) \arrow[d, "\pi|_{s(U)}"']   \\
    X                    & \supseteq & U \arrow[u, "s"', bend right=49] \\
    \pi \circ s = id_U & \implies & \pi|_{s(U)} \circ s = id_U
    \end{tikzcd}
\end{figure}


$ $\newline
Παρακάτω θα θεωρούμε (συνεχείς) τομές σε \underline{ανοιχτά} $U\subseteq X$.

$ $\newline
Συμβολίζουμε: $\So (U)$ ή $\Gamma(U, \So)$ για το σύνολο
$$\{s: U \longrightarrow \So| \quad s \text{ τομή }\}$$

\begin{remark}
    Έστω $z \in A \subseteq \So$, το $A$ να είναι ανοιχτό. Μπορούμε να θεωρήσουμε ότι υπάρχει $V \subseteq X$ ανοιχτό με $z \in V, \pi(V)$ ανοιχτό και $\pi|_V: V\rightarrow \pi(V)$ ομοιομορφισμό με $V\subseteq A$ (Υπενθύμιση: τα $V$ του ορισμού του δράγματος είναι βάση της τοπολογίας του $\So$).
\end{remark}

\begin{lemma} Έστω $\Sheaf$, $U \subseteq X$ ανοιχτό και $s \in \So (U)$ τομή. Τότε $s(U)\subseteq \So$ ανοιχτό και $s: U \rightarrow s(U)$ ομοιομορφισμός.
\end{lemma}

$ $\newline
Ουσιαστικά με αυτό το λήμμα έχουμε ότι οι τομές $s$ είναι τοπικές αντίστροφες της $\pi$ πάνω στα $s(U)$.
%τομές τοπικές αντίστροφες της π πάνω στα s(U)
\begin{proof} $(1)$ Θα δείξουμε ότι $s(U)$ ανοιχτό υποσύνολο του $\So$.

    $ $\newline
    Έστω $z = s(x) \in s(U), x\in U$. Υπάρχει $V$ ανοιχτό υποσύνολο του $\So$ με $z \in V$ έτσι ώστε το $\pi (V)$ να είναι ανοιχτό υποσύνολο του $X$ και $\pi |_V:V\rightarrow \pi (V)$ ομοιομορφισμός. Επιπλέον, η $s$ είναι συνεχής στο $x$ και $V \in \mathcal{N}_{s(x)}$, $V$ ανοιχτό. Λόγω συνέχειας παίρνουμε ότι το $A$ είναι ανοιχτή περιοχή του $x$ με $A \subseteq U$ και $s(A) \subseteq V$. Εφαρμόζοντας την $\pi$ παίρνουμε:
    $$\pi (s(A)) = A \subseteq \pi (V)$$ και $\pi(V)$ ανοιχτό. Άρα $s(A) \subseteq V$, δηλαδή το $s(A)$ είναι ανοιχτό μέσα στο ανοιχτό $V$ και άρα είναι και σε όλο το $\So$.

    $ $\newline
    $s(A) \subseteq \So$ ανοιχτό και $x \in A \subseteq U \implies z = s(x) \in s(A) \subseteq s(U)$, άρα $s(U)$ ανοιχτό.

    $ $\newline
    $(2)$ $V\supseteq s(A)$

    \begin{figure}[H]
        \centering
    \begin{tikzcd}
        V \arrow[d, "\pi_V"'] & \supseteq & s(A) \arrow[d, "\pi|_{S(A)} = \text{ομοιομ.} = s^{-1}|_{s(A)}"] \\
        \pi(V)                & \supseteq & A \subseteq U                                                         
        \end{tikzcd}
    \end{figure}
    $ $\newline
    και έχουμε $s$ 1-1, επί, συνεχής, και τοπικός ομοιομορφισμός, άρα η $s$ είναι ομοιομορφισμός.

\end{proof}

%οποιοδήποτε σημείο του ολικού χώρου το πιάνουμε σαν εικόνα τομής
\begin{lemma}
    Έστω $\Sheaf$ δράγμα, $z \in \So$. Τότε υπάρχει $U\subseteq X$ ανοιχτό και $s \in \So (U)$ με $z \in s(U)$.
\end{lemma}

\begin{proof} $z \in \So \implies$ υπάρχει $V \subseteq \So$ ανοιχτό με $z \in V, \pi(V)$ ανοιχτό, $\pi |_V:V\rightarrow \pi (V)$ ομοιομορφισμός. Άρα υπάρχει:
    $$s:= \left(\pi |_V\right)^{-1} : \pi (V) \longrightarrow V$$ συνεχής, με $\pi (V) \subseteq U$ και $\pi \circ s = id_U$. Άρα για $x = \pi (z)$ ισχύει $s(x) = s(\pi(z)) = z$.
\end{proof}

\begin{lemma}
    Έστω $\Sheaf$ δράγμα, τότε το σύνολο:
    $$\{s(U): \quad U \text{ ανοιχτό } \subseteq X, s \in \So(U)\}$$ αποτελεί βάση της τοπολογίας του $\So$.
\end{lemma}

$ $\newline
Το επόμενο λήμμα είναι σημαντικό:

\begin{lemma}
    Έστω $\Sheaf$ δράγμα, $A,B \subseteq X$ ανοιχτά με $x \in A\cap B \neq \varnothing$ και $s \in \So (A), \sigma \in \So (B)$ με $s(x) = \sigma (x)$. Τότε υπάρχει $U \subseteq A\cap B$ ανοιχτό με $x \in U$ και $s|_U = \sigma |_U$.
    %λύσεις εξισώσεων από ένα σημείο ισότητα παντού σε μια περιοχή
\end{lemma}

$ $\newline
Αν σκεφτεί κανείς ότι ο \tl{Leray} που ασχολήθηκε με τα δράγματα με σκοπό την μελέτη διαφορικών εξισώσεων, αυτό το λήμμα είναι η κατάλληλη περίπτωση για αυτό που ήθελε μιας και αν οι αρχικές συνθήκες δύο λύσεων ταυτίζονται τότε ταυτίζονται και ολόκληρες οι λύσεις σε μια ανοιχτή περιοχή.

\begin{proof}
    $ z= s(x) = \sigma (x) \in \So$. Από ορισμό δράγματος υπάρχει $V$ ανοιχτό $\subseteq X: \pi(V)$ ανοιχτό και $\pi|_V:\rightarrow \pi (V)$ ομοιομορφισμός. 
    
    $$s(x) = z \in s(A)\subseteq \So \text{ ανοιχτό }$$ 
    $$\sigma (x) = z \in \sigma (B) \subseteq \So \text{ ανοιχτό }$$ Θεωρούμε στην θέση του $V$ το $V= V\cap \sigma (B) \cap s(A)$, μικραίνουμε το ανοιχτό δηλαδή να είναι μέσα στα άλλα δύο και άρα έχουμε:

    $$\pi |_{s(A)} = s^{-1}: s(A)\longrightarrow A$$
    $$\pi |_{\sigma (B)} = \sigma^{-1}: \sigma (B) \longrightarrow B$$

    $$\pi|_V = s^{-1}|_V = \sigma^{-1} |_V \text{ μαζί με } \sigma|_U = s|_U \implies $$
    $$s = \sigma |_{\pi(V) = U}$$


\end{proof}

$ $\newline
Σε αυτό το σημείο έχουμε εξαντλήσει όσα προκύπτουν με τον ορισμό, έχουμε μόνο τον τοπικό ομοιομορφισμό για να δουλέψουμε, μένει να δούμε πώς συμπεριφέρονται οι μορφισμοί δραγμάτων στις τομές:
%πως συμπεριφέρονται οι μορφισμοί δραγμάτων στις τομές

\begin{prop}
    Έστω $\Sheaf, (\mathcal {T},\rho , X)$ δράγματα και $f : \So \rightarrow \mathcal {T}$ συνεχής. Τα ακόλουθα είναι ισοδύναμα:
    \begin{enumerate}
        \item $f$ μορφισμός δραγμάτων.
        \item Για κάθε $U \subseteq X$ ανοιχτό, για κάθε $s \in \So (U)$ ισχύει $f\circ s \in \mathcal{T}(U)$.
        \item Για κάθε $z \in \So$ υπάρχει ανοιχτό $U \subseteq X$ και τομή $s \in \So (U)$ τέτοια ώστε $ z \in s(U)$ και $f\circ s \in \mathcal{T}(U)$.
    \end{enumerate}
\end{prop}

\begin{figure}[H]
    \centering
\begin{tikzcd}
    \So \arrow[r, "f"] \arrow[rd, "\pi"', shift right=2] & \mathcal{T} \arrow[d, "\rho"] \\
                                                         & X \arrow[lu, "s"']           
    \end{tikzcd}
\end{figure}


\begin{proof}
    $ $\newline
    $(1)\implies (2)$ $f$ συνεχής και $f$ μορφισμός, άρα $\rho \circ f = \pi$. Έστω $U \subseteq X$  ανοιχτό, $s \in \So (U)$. Τότε $f\circ s : U \rightarrow \mathcal{T}$ συνεχής σαν σύνθεση συνεχών. Επιπλέον:
    $$\rho \circ (f\circ s) = (\rho \circ f)\circ s = \pi \circ s = id_U$$ άρα $f\circ s \in \mathcal{T} (U)$.

    $ $\newline
    $(2) \implies (3)$ Για κάθε $z \in \So$ υπάρχει $U$ με $x \in U$ και $s \in \So (U)$ με $s(x) = z$ και από προηγούμενο λήμμα έχουμε $f \circ s \in \mathcal{T}(U)$.

    $ $\newline
    $(3) \implies (1)$ Θα δείξουμε ότι $\rho \circ f = \pi$. Έστω $z \in \So$. Από $(3)$ έχουμε $U \subseteq X$ ανοιχτό με $x \in U$ και $s \in \So (U)$ με $s(x) = z \in s(U)$. Έχουμε $f\circ s \in \mathcal{T}(U) \implies \rho \circ (f\circ s) = id_U$. Δηλαδή, $\rho \circ f \circ s (x) = \rho \circ f(z)$ αλλά και $\rho \circ f \circ s(x) = id(x) = x = \pi (z)$. Άρα $\rho \circ f = \pi$ αφού το $z$ τυχαίο, δηλαδή πράγματι $f$ μορφισμός δραγμάτων.

    %μορφισμοί δραγμάτων μεταφέρουν τομές σε τομές (to remember)
\end{proof}

$ $\newline
Άρα σαν μνημονικό κανόνα έχουμε ότι οι μορφισμοί δραγμάτων μεταφέρουν τομές σε τομές. Θα δούμε στην συνέχεια την αλγεβρική εικόνα των δραγμάτων, κάτι που αρχικά μοιάζει με τελείως διαφορετικό αντικείμενο.

%Θα δούμε τώρα την αλγεβρική εικόνα, αρχικά φαίνεται κάτι τελείως διαφορετικό

$ $\newline
Έστω $X$ τοπολογικός χώρος, $\tau_X$ η τοπολογία του $X$. Ένα \underline{προδράγμα συνόλων} πάνω από το $X$ είναι ένα ζεύγος:
$$\left( \{P(U)\}_{U \in \tau_X}, \{\rho^U_V\}_{V\subseteq U \in \tau_X}\right)$$ όπου $\{P(U)\}_{U \in \tau_X}$ οικογένεια συνόλων, με σύνολο δεικτών την $\tau_X$ και 
$$\rho^U_V : P(U) \longrightarrow P(V)$$ απεικονίσεις: (περιορισμού)
\begin{enumerate}
    \item $\rho^U_U = id_{P(U)}: P(U) \rightarrow P(U) \quad \forall U \in \tau_X$.
    \item Για κάθε $ W\subseteq V \subseteq U$ στην $\tau_X$, τότε ισχύει:
    $$\rho^V_W \circ \rho^U_V = \rho^U_W$$ ή αλλιώς το τρίγωνο είναι μεταθετικό:

    \begin{figure}[H]
        \centering
    \begin{tikzcd}
        P(U) \arrow[rd, "P^V_W"'] \arrow[r, "P^U_V"] & P(V) \arrow[d, "P^V_W"] \\
                                                     & P(W)                   
        \end{tikzcd}
    \end{figure}
 
\end{enumerate}
\underline{Παραδείγματα:}
\begin{enumerate}
    \item $(X,\tau_X)$ τοπολογικός χώρος, για κάθε $U$ ανοιχτό συμβολίζουμε:
    $$\mathcal{C}(U) = \{f: U \rightarrow \mathbb{R}| \quad \text{ συνεχής }\}$$ και για κάθε $V\subseteq U$ με $ V,U \in \tau_X$ θέτουμε:
    $$\rho^U_V : \mathcal{C} (U) \rightarrow \mathcal{C}(V)$$
    $$f \longmapsto f|_V$$ Τότε το ζεύγος των οικογενειών $(\{\mathcal{C}(U)\}, \{\rho^U_V\}\})$ είναι ένα προδράγμα πάνω από το $X$ ($:=$ το προδράγμα των συνεχών). Αρκεί να ελέγξουμε τις συνθήκες:

    $ $\newline
    (1)$\rho^U_U (f) = f|_U = f$ όπου $f: U \rightarrow \mathbb{R}$
    $ $\newline
    (2) Για κάθε $ W \subseteq V \subseteq U$ και $f:U \rightarrow \mathbb{R} \implies f|_W = (f|_V)|_W$

    \item $P(U) = \{f: U \rightarrow \mathbb{R}| \quad \text{ σταθερές }\}$ (οι συνήθεις περιορισμοί θα εννοούνται σε τέτοια παραδείγματα)
    \item $B(U) = \{f: U \rightarrow \mathbb{R}| \quad \text{ φραγμένες }\}$
    \item $(M = \text{ πολλαπλότητα}, \tau_M)$ και 
    $$\mathcal{C}^{\infty}(U) = \{f: U \rightarrow \mathbb{R}| \quad \mathcal{C}^{\infty}\text{-διαφ}\}$$

    \item $X = \mathbb{C}$,
    $$\mathcal{O}(U) = \{f: U \rightarrow \mathbb{C}| \quad \text{ ολόμορφη }\}$$

    \item $\Sheaf$ δράγμα, τότε το ζεύγος:
    $$\left( \{\So (U)\}_{U \in \tau_X}, \{\rho^U_V = \text{ συνήθεις περιορισμοί } \}_{V\subseteq U \in \tau_X}\right)$$ ονομάζεται το προδράγμα των τομών του δράγματος.


\end{enumerate}

\begin{defn} $(X,\tau_X)$ τοπολογικός χώρος και 
    $$S \equiv (S(U),\rho^U_V)_{V\subseteq U \in \tau_X}, \quad T \equiv (T(U),\lambda^U_V)_{V\subseteq U \in \tau_X}$$ προδράγματα πάνω από το $X$. Ένας μορφισμός προδραγμάτων $f: S \rightarrow T$ είναι μια οικογένεια απεικονίσεων:
    $$f_U: S(U) \longrightarrow T(U), \quad U \in \tau_X$$ έτσι ώστε για κάθε $V\subseteq U \in \tau_X$ να είναι μεταθετικό το τετράγωνο: 

    \begin{figure}[H]
        \centering
    \begin{tikzcd}
        S(U) \arrow[r, "f_U"] \arrow[d, "\rho^U_V"'] & T(U) \arrow[d, "\lambda^U_V"] \\
        S(V) \arrow[r, "f_V"]                        & T(V)                         
        \end{tikzcd}
    \end{figure}
\end{defn}

$ $\newline
Για κάθε προδράγμα $(P(U),\rho^U_V) \equiv P$ η οικογένεια $1 = \{1_U\}: P \rightarrow P$ με
$$1_U = id_U: P(U) \rightarrow P(U)$$ είναι μορφισμός προδραγμάτων.

$ $\newline
Θέλουμε να ορίσουμε τώρα μια σύνθεση μορφισμών προδραγμάτων ώστε να φτιάξουμε μια νέα κατηγορία. Αν $f:S \rightarrow T$ και $g:T \rightarrow P$ μορφισμοί προδραγμάτων, όπου $f \equiv \{f_U\}_{U \in \tau_X}$ και $g \equiv \{g_U\}_{U \in \tau_X}$ τότε ορίζουμε την σύνθεση:
$$g \circ f \equiv \{ (g\circ f) := g_U \circ f_U \}_{U \in \tau_X}$$ και για κάθε $V\subseteq U \in \tau_X$ έχουμε το μεταθετικό διάγραμμα:

\begin{figure}[H]
    \centering
    \begin{tikzcd}
        S(U) \arrow[r, "f_U"] \arrow[rr, "g_U \circ f_U", bend left=49] \arrow[d, "\rho^U_V"'] & T(U) \arrow[r, "g_U"] \arrow[d, "\lambda^U_V"'] & P(U) \arrow[d, "\mu^U_V"] \\
        S(V) \arrow[rr, "g_V \circ f_V", bend right=49] \arrow[r, "f_V"]                       & T(V) \arrow[r, "g_V"]                           & P(V)                     
        \end{tikzcd}
\end{figure}
$$\mu^U_V \circ (g_U \circ f_U) = (g_V \circ f_V)\circ \rho^U_V$$
$$\mu^U_V \circ (g_V \circ f_V) = (g_V \circ \lambda^U_V) \circ f_U$$
$$= g_V \circ f_V \circ \rho^U_V$$ άρα πράγματι $g\circ f$ είναι μορφισμός προδραγμάτων $S \rightarrow P$.

$ $\newline
Συνεπώς τα προδράγματα και μορφισμοί προδραγμάτων από μόνα τους αποτελούν μια κατηγορία $\mathcal{P}\mathcal{S}h_X$.