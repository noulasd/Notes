\vspace*{0.1cm} 

\noindent Έστω $(X,\tau_X),(Y,\tau_Y)$ τοπολογικοί χώροι.

\begin{figure}[H]
    \centering
    \begin{tikzcd}
        \mathcal{C}_X \arrow[d] &  & \mathcal{C}_Y \arrow[d] \\
        X \arrow[rr]            &  & Y                      
        \end{tikzcd}
\end{figure}
\noindent Όπου $\mathcal{C}_X $ είναι το δράγμα που παράγεται από το (πλήρες) προδράγμα $(C_X(U),r^U_{U^{\prime}})_{U^{\prime}\subseteq U \in \tau_X}$:
$$C_X(U) = \{h:U\longrightarrow \mathbb{R} \text{ συνεχείς }\}$$ με απεικονίσεις τους συνήθεις περιορισμούς. Ένας φυσιολογικός τρόπος να τα συνδέσουμε αυτά είναι:
\begin{figure}[H]
    \centering
    \begin{tikzcd}
        X \supseteq f^{-1}(V) \arrow[rd] \arrow[r, "f"] & Y\supseteq V \arrow[d, "h"] \\
                                                        & \mathbb{R}                 
        \end{tikzcd}
\end{figure}

$$f_V: C_Y(V) \longrightarrow C_X(f^{-1}(V))$$ δηλαδή υπάρχει οικογένεια απεικονίσεων:
$$(f_V: C_Y(V) \longrightarrow C_X(f^{-1}(V)) = f_*(C_X)(V))_{V\in \tau_Y}$$ δηλαδή το $(f_V)$ είναι μορφισμός προδραγμάτων $C_Y \longrightarrow f_*(C_X)$. Άρα ορίζεται μορφισμός δραγμάτων $\tilde{f}$:

$$\tilde{f}:\mathcal{C}_Y \longrightarrow f_*(\mathcal{C}_X)$$ και έτσι κάναμε την εξής σύνδεση στο αρχικό πρόβλημα:

\begin{figure}[H]
    \centering
    \begin{tikzcd}
        \mathcal{C}_X \arrow[d] &  & \mathcal{C}_Y \arrow[d] \arrow[r, "\tilde{f}"] & f_*(\mathcal{C}_X) \arrow[ld] \\
        X \arrow[rr]            &  & Y                                              &                              
        \end{tikzcd}
\end{figure}

\noindent Αν το δούμε αλλιώς:
$$f^{-1}(V) \subseteq X \overset{f}{\longrightarrow} V \subseteq Y \overset{h}{\longrightarrow} \mathbb{R}$$

\noindent Για κάθε $x \in f^{-1}(V)$ και $h \in C_Y(V)$ το $[h]_{f(x)}$ είναι μέσα στο νήμα  $\mathcal{C}_{Y,f(x)}$. Ισοδύναμα:
$$(x,[h]_{f(x)}) \in f^*(\mathcal{C}_Y)_x$$ στο νήμα του \tl{pullback}.

$ $\newline
Κατασκευάζουμε το $[h\circ f]_x \in \mathcal{C}_{X,x}$ και έτσι έχουμε απεικόνιση:
$$\hat{f} : f^*(\mathcal{C}_Y) \longrightarrow X$$ άρα γυρνώντας στο αρχικό πρόβλημα έχουμε:

\begin{figure}[H]
    \centering
    \begin{tikzcd}
        f^*(\mathcal{C}_Y) \arrow[rd] \arrow[r, "\hat f"] & \mathcal{C}_X \arrow[d] &  & \mathcal{C}_Y \arrow[d] \\
                                                          & X \arrow[rr]            &  & Y                      
        \end{tikzcd}
\end{figure}

\noindent Εναλλακτικά, για κάθε $x \in X$ υπάρχει απεικόνιση νημάτων
(θα θέλαμε να τις κολλήσουμε και να πάρουμε απεικόνιση αλλά μπορεί δύο $x_1, x_2$ να έχουν ίδια εικόνα)
$$\overline{f}_x : \mathcal{C}_{Y,f(x)} \longrightarrow \mathcal{C}_{X,x}$$ και άρα και οικογένεια
$$(\overline{f}_x : \mathcal{C}_{Y,f(x)} \longrightarrow \mathcal{C}_{X,x})_{x \in X}$$ (όχι απεικόνιση)
\begin{figure}[H]
    \centering
    \begin{tikzcd}
        \mathcal{C}_X \arrow[d] &  & \mathcal{C}_Y \arrow[d] \arrow[ll, "(\overline{f}_x)_{x\in X}", dashed] \\
        X \arrow[rr]            &  & Y                                                                      
        \end{tikzcd}
\end{figure}
\noindent Θα ασχοληθούμε κυρίως με τις δύο πρώτες προσεγγίσεις.

$ $\newline
Έστω $f:X\longrightarrow Y$ συνεχής και $\mathcal A\in \So h_X, \mathcal B \in \So h_Y$.

\begin{defn} Ένας μορφισμός μεταξύ των $\mathcal A$ και $\mathcal B$ πάνω από την $f$ είναι ένας μορφισμός δραγμάτων:
    
    $$\text{ 1η κατάσταση: } g : \mathcal{B} \longrightarrow f_*(\mathcal{A})$$
    \begin{figure}[H]
        \centering
        \begin{tikzcd}
            \mathcal{A} \arrow[d] &  & \mathcal{B} \arrow[d] \arrow[r, "g"] & f_*(\mathcal{A}) \arrow[ld] \\
            X \arrow[rr, "f"]     &  & Y                                    &                            
            \end{tikzcd}
    \end{figure}

    
    $$\text{ 2η κατάσταση: } g: f^*(\mathcal B)\longrightarrow \mathcal A$$

    \begin{figure}[H]
        \centering
        \begin{tikzcd}
            f^*(B) \arrow[r, "g"] \arrow[rd] & \mathcal{A} \arrow[d] &  & \mathcal{B} \arrow[d] \\
                                             & X \arrow[rr, "f"]     &  & Y                    
            \end{tikzcd}
    \end{figure}
\end{defn}
\noindent Ουσιαστικά και στις δύο περιπτώσεις ξεκινάμε από το $\mathcal B$ και πάμε στο $\mathcal{A}$. Θα δείξουμε ότι οι ορισμοί είναι ισοδύναμοι.

\begin{theorem} Έστω $f:X\longrightarrow Y$ συνεχής. Για κάθε $\mathcal{A} \in \So h_X$ και $\mathcal B \in \So h_Y$ υπάρχει αμφιμονοσήμαντη αντιστοιχία:
    $$\Phi: Mor(f^*(\mathcal{B}),\mathcal{A}) \longrightarrow Mor(\mathcal{B},f_*(\mathcal{A}))$$
\end{theorem}

$ $\newline
Υπενθύμιση (1): για $f:X\longrightarrow Y$ συνεχής υπάρχει φυσικός μετασχηματισμός $\phi^f : I\longrightarrow f_*f^*$ για τους συναρτητές $I,f_*f^* : \So h_Y\longrightarrow \So h_Y$. Επιπλέον, για κάθε $\mathcal B \in \So h_Y$ έχουμε:
$$\phi^f_{\mathcal B} : \mathcal{B} \longrightarrow f_*f^*(\mathcal{B})$$ έτσι ώστε για κάθε $V\in \tau_Y$:
$$\phi^f_{\mathcal{B}V} : \mathcal{B}(V) \longrightarrow f_*f^*(\mathcal{B})(V) = f^*(\mathcal{B}(f^{-1}(V))) = \Gamma_f(f^{-1}(V),\mathcal{B})$$ (Η πρώτη ισότητα ως \tl{pushout} και η δεύτερη ως \tl{pullback})

$ $\newline
Αν $\beta : V\subseteq Y\longrightarrow \mathcal{B}$ τομή, τότε έχουμε $\phi^f_{\mathcal{B}V}(\beta) = \beta \circ f$

\begin{figure}[H]
    \centering
    \begin{tikzcd}
        & \mathcal{B}                      \\
f^{-1}(V) \arrow[ru, "\beta\circ f"] \arrow[r, "f"'] & V\subseteq Y \arrow[u, "\beta"']
\end{tikzcd}
\end{figure}

\noindent Εφόσον $\phi^f$ είναι φυσικός μετασχηματισμός, έχουμε για κάθε $\mathcal{B},\mathcal{T} \in \So h_Y$ και για κάθε μορφισμό δραγμάτων $g:\mathcal{B}\longrightarrow \mathcal{T}$ τα μεταθετικά τετράγωνα:

\begin{figure}[H]
    \centering
    \begin{tikzcd}
        \mathcal{B} \arrow[rr, "\phi^f_{\mathcal B}"] \arrow[d, "g"'] &  & f_* f^*(\mathcal{B}) \arrow[d, "f_*f^*(g)"] \\
        \mathcal{T} \arrow[rr, "\phi^f_{\mathcal{T}}"']               &  & f_* f^*(\mathcal{T})                       
        \end{tikzcd}
\end{figure}

$ $\newline
Υπενθύμιση (2): Έχουμε φυσικό μετασχηματισμό $\psi^f:f^*f_* \rightarrow I$ για τους συναρτητές $f^*f_*, I : \So h_X \longrightarrow \So h_X$, όπου για $\mathcal{A} \in \So h_X$:
$$\psi^f_{\mathcal{A}}: f^*f_*(\mathcal{A})\longrightarrow \mathcal{A}$$
$$(x,a)\in f^*(f_*(\mathcal{A}))_x \implies a \in f_*(\mathcal{A})_{f(x)}$$ Προέρχεεται από $a \in f_*(\mathcal{A})(V) = A(f^{-1}(V))$ με $V \in \mathcal{N}^0_{f(x)}$ 

$ $\newline
Άρα το $(x,a)$ προέρχεται (από το όριο) της $a \circ f$. Δηλαδή:
$$\psi^f_{\mathcal{A}}(x,a) = [a]_x$$

$ $\newline
Οι διαδικασίες είναι ανάποδες η μία της άλλης. Η μία συνθέτει με $f$ ενώ η άλλη διώχνει την $f$, χωρίς να είναι η μία αντίστροφη της άλλης αφού δεν συμπίπτουν τα πεδία ορισμού και τιμών. Το ότι η $\psi^f$ είναι φυσικός μετασχηματισμός έπεται ότι για κάθε $\mathcal{A},\So \in \So h_X$ και για κάθε μορφισμό δραγμάτων $g:\mathcal{A}\longrightarrow \So$ έχουμε μεταθετικό τετράγωνο:

\begin{figure}[H]
    \centering
    \begin{tikzcd}
        f^*f_*(\mathcal{A}) \arrow[rr, "\psi^f_{\mathcal{A}}"] \arrow[d, "f^*f_*(g)"'] &  & \mathcal{A} \arrow[d, "g"] \\
        f^*f_*(\So) \arrow[rr, "\psi^f_{\So}"']                                        &  & \So                       
        \end{tikzcd}
\end{figure}

\begin{lemma} Έστω $f:X\longrightarrow Y$ συνεχής, $\mathcal{A} \in \So h_X$ και $\mathcal{B} = f_*(\mathcal{A})\in \So h_Y$. Τότε:
    $$\phi^f_{\mathcal{B}} = \phi^f_{f_*(A)} \ \text{ ισομορφισμός }$$ και
    $$\left(\phi^f_{f_*(\mathcal A)} \right)^{-1} = f_*(\psi^f_{\mathcal{A}})$$
\end{lemma}

\begin{proof}

    \begin{figure}[H]
        \centering
        \begin{tikzcd}
            f^*f_*(\mathcal{A}) \arrow[rd] \arrow[rr, "\psi^f_{\mathcal{A}}"]         &   & \mathcal{A} \arrow[ld]                                              &  &                                     \\
                                                                                      & X &                                                                     &  &                                     \\
            f_*f^*f_*(\mathcal{A}) \arrow[rd] \arrow[rr, "f_*(\psi^f_{\mathcal{A}})"] &   & f_*(\mathcal{A}) \arrow[ld] \arrow[rr, "\phi^f_{f_*(\mathcal{A})}"] &  & f_*f^*f_*(\mathcal{A}) \arrow[llld] \\
                                                                                      & Y &                                                                     &  &                                    
            \end{tikzcd}
    \end{figure}
    \noindent Αρκεί να δείξω ότι $$f_*(\psi^f_{\mathcal{A}}) \circ \phi^f_{f_*(\mathcal{A})} = id_{f_*(\mathcal{A})}$$ και 
    $$\phi^f_{f_*(\mathcal{A})} \circ f_*(\psi^f_{\mathcal{A}}) = id_{f_*f^*f_*}$$


    \begin{align*}
        f_*(\psi^f_{\mathcal{A}})_V \circ \phi^f_{f_*(\mathcal{A})V}(a) & =  f_*(\psi^f_{\mathcal{A}})_V (a\circ f) \\
        &= \psi^f_{\mathcal{A}f^{-1}(V)}(a\circ f) \\
        &= a 
    \end{align*}

    \noindent Για το δεύτερο:
    $$f_*f^*f_*(\mathcal{A})(V) = f^*f_*(\mathcal{A})(f^{-1}(V)) = \Gamma_f(f^{-1}(V),f_*(\mathcal{A}))$$ το οποίο περιέχει στοιχεία $\beta \circ f$:

    \begin{align*}
        [\phi^f_{f_*(\mathcal{A})} \circ f_*(\psi^f_{\mathcal{A}})]_V (\beta\circ f) & = \phi^f_{f_*(\mathcal{A})V} \circ \psi^f_{\mathcal{A}f^{-1}(V)} (\beta \circ f) \\
        & = \phi^f_{f_*(\mathcal{A})V} (\beta) = \beta \circ f
    \end{align*}
\end{proof}

\begin{lemma} Έστω $f:X\longrightarrow Y$ συνεχής, $\mathcal{B} \in \So h_Y$ και $A = f^*(\mathcal{B}) \in \So h_X$. Τότε:
    $$\psi^f_{\mathcal{A}} = \psi^f_{f^*(\mathcal{B})} \text{ αντιστρέψιμος }$$ και 
    $$\left(\psi^f_{f^*(\mathcal{B})}\right)^{-1} = f^*(\phi^f_{\mathcal{B}})$$
\end{lemma}

\begin{proof}
    \begin{figure}[H]
        \centering
        \begin{tikzcd}
            f^*f_*(\mathcal{A})  \arrow[rr, "\psi^f_{\mathcal{A}}"]       &  & \mathcal{A}                                             \\
            f^*f_*f^*(\mathcal{B}) \arrow[rr, "\psi^f_{f^*(\mathcal B)}"] &  & f^*(\mathcal{B}) \arrow[d, "f^*(\phi^f_{\mathcal{B}})"] \\
                                                                          &  & f^*f_*f^*(\mathcal{B})                                 
            \end{tikzcd}
    \end{figure}

    $$(x,b) \in f^*(\mathcal{B}) \ b \in \mathcal{B}_{f(x)}$$ δηλαδή 
    $$(x,b) \in f^*(\mathcal{B})_x$$ με $b = [\beta]_{f(x)}$

    \begin{align*}
        f^*(\phi^f_{\mathcal{B}})(x,b) &= (x,\phi^f_{\mathcal{B}}(b)) \\
        &= (x, [\phi^f_{\mathcal{B}}(\beta)]_{f(x)}) \\
        &= (x, [\beta \circ f]_{f(x)})  \implies \\
        \\
        \psi^f_{f^*(\mathcal{B})} \circ f^*(\phi^f_{\mathcal{B}})(x,b) &= \psi^f_{f^*(\mathcal{B})}(x,[\beta\circ f]_{f(x)}) \\
        &= (x,[\beta]_{f(x)}) \\
        &= (x,b)
    \end{align*}

    \noindent Ανάποδα: Αν $(x,c) \in f^*f_*f^*(\mathcal{B})$ τότε
    $$c \in f_*f^*(\mathcal{B})_{f(x)} \implies c = [\gamma]_{f(x)}$$ με $\gamma \in f_*f^*(\mathcal{B})(V)$ για $V \in \mathcal{N}^0_{f(x)}$, συνεπώς:
    $$\gamma \in f_*(\mathcal{B})(f^{-1}(V)) = \Gamma_f (f^{-1}(V),\mathcal{B}) \implies $$
    $$\gamma = \gamma_1 \circ f \ \gamma_1 \in \mathcal B(V)$$ Άρα:

    \begin{align*}
        f^*(\phi^f_{\mathcal{B}})\circ \psi^f_{f^*(\mathcal{B})} (x,c) & = f^*(\phi^f_{\mathcal{B}})(x,[\gamma_1]) \\
        &= (x,\phi^f_{\mathcal{B}}([\gamma_1])) \\
        &= (x,[\gamma_1 \circ f]_{f(x)}) \\
        &= (x,c)
    \end{align*}
\end{proof}

\noindent Γυρνάμε στο θεώρημα:
\begin{theorem} Έστω $f:X\rightarrow Y$ συνεχής, $\mathcal{A} \in \So h_X$ και $\mathcal{B} \in \So h_Y$. Τότε υπάρχει ισομορφισμός:
    $$\Phi: Mor(f^*(\mathcal{B},\mathcal{A})) \longrightarrow Mor(\mathcal{B},f_*(\mathcal{A}))$$
\end{theorem}

\begin{figure}[H]\
    \centering
    % https://tikzcd.yichuanshen.de/#N4Igdg9gJgpgziAXAbVABwnAlgFyxMJZABgBpiBdUkANwEMAbAVxiRADMA9AKgAoAdfgFs6OABYBjRsABCAXwCUIOaXSZc+QigCM5KrUYs2gkeKkNgAQTnLVIDNjwEiAZj3V6zVohAnRk6XlbNUdNIl1tfU8jHwANYPt1Jy1kN0iPQ28QAE0Ehw1nFAAmUiKozON+LCE0Bix4PKSwlDIXcq9K0wCLIJUQgpSStoyOn3YAfW4uPj8zQMVG0MLkABZSYYNRjkmBYX9zKwW5fRgoAHMGlFB2ACcIISQyEBwIJF1NmJAxBNv7x+oXkgXH0OHcHoh3oDEMC7L9wW5nq9EGsPll2D8wUgSoikCsQXCkAA2AFIgDsI0+gjQYiwnAmwFm3Vkchs+MxiHJOMQAA4KWidmIlMc5EA
\begin{tikzcd}
    f^*(\mathcal{B}) \arrow[r, "h"] \arrow[rd]     & \mathcal{A} \arrow[d] &                                          & \mathcal{B} \arrow[d] &                  \\
                                                   & X \arrow[rr, "f"]     &                                          & Y                     &                  \\
                                                   &                       & \implies                                 &                       &                  \\
    \mathcal{B} \arrow[rr, "\phi^f_{\mathcal{B}}"] &                       & f_*f^*(\mathcal{B}) \arrow[rr, "f_*(h)"] &                       & f_*(\mathcal{A})
    \end{tikzcd}
\end{figure}
\noindent Οπότε θέτουμε:
$$\Phi(h) = f_*(h)\circ \phi^f_{\mathcal{B}}$$ Απόδειξη στην επόμενη διάλεξη.