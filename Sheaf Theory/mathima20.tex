\vspace*{0.3cm}

Έστω $\mathcal{E},\mathcal{F}$ δράγματα $\mathbb{R}$-διανυσματικών χώρων και $f:\mathcal{E}\longrightarrow \mathcal{F}$ μορφισμός δραγμάτων διανυσματικών χώρων, δηλαδή είναι μορφισμός δραγμάτων και για κάθε $x \in X$:
$$f_x : \mathcal{E}_x \longrightarrow \mathcal{F}_x \quad \mathbb{R}-\text{γραμμική}$$

$$\mathcal{K}= \ker f:= \bigsqcup\limits_{x \in X}\ker f \subseteq \mathcal{E}$$ Αν $\pi, \mathfrak{p}$ είναι οι προβολές στον $X$ των $\mathcal{E},\mathcal{F}$ αντίστοιχα, τότε δίνουμε στο $\mathcal{K}$ την σχετική τοπολογία και έχουμε προβολή:
$$\pi|_{\mathcal{K}} : \mathcal{K} \longrightarrow X$$ ισχύει ότι $\pi|_{\mathcal{K}}$ είναι τοπικός ομοιμορφισμός? Αν ναι έχουμε φτιάξει νέο δράγμα.

$ $\newline
Αν αντίστοιχα στα προδράγματα έχουμε $E\equiv (E(U),\rho^U_V), F\equiv(F(U),\lambda^U_V)$ προδράγματα $\mathbb{R}$-διανυσματικών χώρων και
$$f\equiv (f_U: E(U)\longrightarrow F(U))_{U \in \tau_X}$$ ένας μορφισμός προδραγμάτων, τότε για κάθε $U\in\tau_X$ υπάρχει ο αντίστοιχος πυρήνας που είναι διανσυματικός υπόχωρος:
$$\ker f_U \leq E(U)$$ και η ερώτηση εδώ είναι περιορίζοντας τον συνήθη περιορισμό $\rho^U_V|_{\ker f_U}$ μπορούμε να φτιάξουμε προδράγμα? Ισχύει δηλαδή ότι $\rho^U_V(\ker f_U) \subseteq \ker f_V$ για κάθε $V\subseteq U$? Θέτουμε $\rho^U_V := \rho^U_V|_{\ker f_U}$.

\begin{figure}[H]
    \centering
    \begin{tikzcd}
        \ker f_U \arrow[rr, "i_U", hook] \arrow[d, "r^U_V"'] &  & E(U) \arrow[d, "\rho^U_V"] \arrow[rr, "f_U"] &  & F(U) \arrow[d, "\lambda^U_V"] \\
        \ker f_V \arrow[rr, "i_V", hook]                     &  & E(V) \arrow[rr, "f_V"']                      &  & F(V)                         
        \end{tikzcd}
\end{figure}

\noindent Έστω $u \in \ker f_U$, ισχύει ότι $\rho^U_V(u) \in \ker f_V$? Αυτό είναι ισοδύναμο με $f_V(\rho^U_V(u)) = 0$ και από μεταθετικότητα του διαγράμματος ισοδύναμα $\lambda^U_V(f_U(u)) =0$ το οποίο προφανώς ισχύει αφού $u \in \ker f_U$.

$ $\newline
\underline{Συμπεράσματα}:

$$(\ker f_U, r^U_V) \quad \text{ προδράγμα δ.χ.}$$ και $$(i_U: \ker f_U\longrightarrow E(U))_{U \in \tau_X}$$ είναι μορφισμός προδραγμάτων διανυσματικών χώρων.


$ $\newline
Με τις ίδιες υποθέσεις στα προδράγματα, μπορούμε να κάνουμε το αντίστοιχο στις εικόνες, δηλαδή για κάθε $U \in \tau_X$ έχουμε:
$$f(E(U))\leq F(U)$$ γραμμικός υπόχωρος. Θέτουμε $\ell^U_V := \lambda^U_V |_{f_U(E(U))}$ και έχουμε μεταθετικό διάγραμμα:


\begin{figure}[H]
    \centering
    \begin{tikzcd}
        E(U) \arrow[d, "\rho^U_V"'] \arrow[rr, "f_U"] &  & f_U(E(U))\subseteq F(U) \arrow[d, "\ell^U_V"', shift right=6] \arrow[d, "\lambda^U_V", shift left=10] \\
        E(V) \arrow[rr, "f_V"]                        &  & f_V(E(V)) \subseteq F(V)                                                                            
        \end{tikzcd}
\end{figure}


$ $\newline
Είναι $\ell^U_V (f_U(E(U))) \subseteq f_V(E(V))$. Έστω $w \in f_U(E(U))$ τότε υπάρχει $u \in E(U)$ με $f_U(u) = w$. Είναι $\ell^U_V(w) = \lambda^U_V(w) \in f_V(E(V))$ και από μεταθετικότητα διαγράμματος:
$$\lambda^U_V(w) = \lambda^U_V(f_U(u)) = f_V(\rho^U_V(u)) \in f_V(E(V))$$ Άρα αντίστοιχα έχουμε προδράγμα διανυσματικών χώρων 
$$(f_U(E(U)), \ell^U_V)$$ και μορφισμό προδραγμάτων διανυσματικών χώρων
$$(i_U: f_U(E(U))\longrightarrow F(U))_{U\in\tau_X}$$

\vspace*{0.3cm}
\noindent Γυρνώντας στα δράγματα, το $\pi|_{\mathcal{K}}$ είναι τοπικός ομοιομορφισμός αν και μόνο αν για κάθε $u \in \mathcal{K}$ υπάρχει $V$ ανοιχτό στο $\mathcal{E}$ και $U$ ανοιχτό στο $X$ με $\pi(u)=x \in U$ έτσι ώστε η απεικόνιση:
$$\pi|_{V\cap \mathcal{K}}: V\cap \mathcal{K}\longrightarrow U$$ να είναι ομοιομορφισμός. Ισοδύναμα να υπάρχει $U\in\openx$ και $s\in \Gamma(U,\mathcal{E})$ με $s(x) = u$ και $s(U)\subseteq \mathcal{K}$.

$ $\newline
\underline{Γνωστό}: Αν $u \in \mathcal{K} \subseteq \mathcal{E}$ τότε υπάρχει $U\in \openx, \ x= \pi(u), s \in \Gamma(U,\mathcal{E})$ με $s(x) = u$.

$ $\newline
\underline{Γνωστό}: $f\circ s:U\longrightarrow \mathcal{F}$ (συνεχής) τομή του $\mathcal{F}$ αφού οι μορφισμοί δραγμάτων μεταφέρουν τομές σε τομές. Δηλαδή, $f\circ s \in \Gamma(U,\mathcal{F})$ και
$$(f \circ s)(x) = f(U)  = 0_x \in \mathcal{F}_x$$ Θέλουμε $f\circ s(y) = 0_y$ για κάθε $y \in U$ για να είναι $s \in \Gamma(U,\mathcal{K})$. Θέτουμε:

$$\omega: U \longrightarrow \mathcal{F}$$
$$\omega(x) = 0_x \in \mathcal{F}_x$$ είναι τομή? είναι συνεχής?

$ $\newline
Έστω τυχαία $\sigma \in \Gamma(U,\mathcal{F})$, οι πράξεις στον διανυσματικό χώρο είναι συνεχείς και άρα:
$$-\sigma = (-1) \cdot \sigma $$ συνεχής τομή. Συνεπώς:
$$\sigma + (-\sigma) = \omega$$ συνεχής τομή. Έχουμε δηλαδή δύο τομές που ταυτίζονται σε ένα σημείο, άρα και σε μια περιοχή:

$$f\circ s(x) = \omega(x) \implies$$
$$\exists \ V \in \openx, V\subseteq U \quad f\circ s|_V = \omega|_V \implies $$

$$f(s(y)) = 0_y \quad \forall \ y \in V \implies s|_V(V)\subseteq \mathcal{K}$$ και άρα έχουμε το δράγμα διανυσματικών χώρων:
$$(\mathcal{K},\pi|_{\mathcal{K}}, X)$$ και τον μορφισμό δραγμάτων διανυσματικών χώρων:
$$i:\mathcal{K}\longrightarrow \mathcal{E}$$

\vspace*{0.3cm}
$ $\newline
Όμοια στις εικόνες, αν έχουμε μορφισμό δραγμάτων διανσυματικών χώρων $f:(\mathcal{E},\pi,X)\longrightarrow (\mathcal{F},\mathfrak{p},X)$ και δώσουμε την σχετική τοπολογία στο $f(\mathcal{E})\subseteq \mathcal{F}$ είναι τότε ο $\mathfrak{p}|_{f(\mathcal{E})}$ τοπικός ομοιομορφισμός?

$ $\newline
Έστω $w \in f(\mathcal{E}) \implies \exists \ u \in \mathcal{E}_x$ με $f_x(u) = w$. Υπάρχει περιοχή $U \in \openx$ και τομή $s \in \Gamma(U,\mathcal{E}): s(x) = u$. Άρα έχουμε:

$$f(s(x)) = w$$ και 
$$f\circ s \in \Gamma(U,\mathcal{F})$$ με $f \circ s (U)\subseteq f(\mathcal{E})$.

\vspace*{0.3cm}
$ $\newline
Στους διανυσματικούς χώρους κάθε γραμμική $f:E\longrightarrow F$ επάγει μια σύντομη ( ή βραχεία) ακριβή ακολουθία:
\begin{figure}[H]
    \centering
    \begin{tikzcd}
        0 \arrow[r] & \ker f \arrow[r, "i"] & E \arrow[r, "f"] & f(E) \arrow[r] & 0
        \end{tikzcd}
\end{figure}
\noindent και μια λιγότερο σύντομη:
\begin{figure}[H]
    \centering
    \begin{tikzcd}
        0 \arrow[r] & \ker f \arrow[r, "i"] & E \arrow[r, "f"] & F \arrow[r] & F/f(E) \arrow[r] & 0
        \end{tikzcd}
\end{figure}

$ $\newline
\textbf{Ερωτήματα:}
\begin{enumerate}
    \item Ένας μορφισμός προδραγμάτων διανυσματικών χώρων $(f_U:E(U)\longrightarrow F(U))_{U \in \tau_X}$ ορίζει την σύντομη ακριβή ακολουθία?
    \item Όμοια ορίζει την αντίστοιχη λιγότερο σύντομη?
    \item Ένας μορφισμός δραγμάτων $f:\mathcal{E}\longrightarrow \mathcal{F}$ ορίζει την σύντομη ακριβή ακολουθία?
    \item Όμοια ορίζει την αντίστοιχη λιγότερο σύντομη? (πρέπει να οριστεί έννοια του πηλίκου ως $\bigsqcup\limits_{x\in X} \mathcal{F}_x/f_x(\mathcal{E}_x)$ και ποια τοπολογία το κάνει δράγμα?)
    \item Αν έχουμε μια σύντομη ακριβή ακολουθία δραγμάτων και εφαρμόσουμε τον συναρτητή τομή $\Gamma$ παίρνουμε σύντομη ακριβή ακολουθία στα προδράγματα?
    \item Ανάποδα, αν έχουμε μια σύντομη ακριβή ακολουθία προδραγμάτων και εφαρμόσουμε τον συναρτητή δραγματοποίησης $\mathbb{S}$ παίρνουμε σύντομη ακριβή ακολουθία στα δράγματα?
\end{enumerate}

$ $\newline
\textbf{Καλό Καλοκαίρι!} 